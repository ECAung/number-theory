%%%%%%%%%%%%%%%%%%%
\chapter{Height functions}\label{height-functions}
\cite{Ko84}, \cite{Si86}, \cite{HS00}, \cite{Si07}
\index{height functions}
\section{Heights on projective space}
Let $K$ be a number field. We aim to define a function $h$ on $\Pj^n(K)$ with the following properties.
\begin{enumerate}
\item There is a bounded number of points with small height. 
\item The height encases nice arithmetical and geometrical information about the point, and behaves well under rational maps.
\end{enumerate}
It is natural to define the height in terms of the absolute values, or places, on $K$. The finite places will capture how divisible the coordinates of a point $P$ are by various primes, while the infinite places captures the more geometrical notion of distance. 
We will thus define the height as a product over all places on $K$.
\begin{df}
Let $K$ be a number field, and $P=(x_0,\ldots, x_n)\in \Pj^n(K)$. Define the \textbf{multiplicative height} and \textbf{logarithmic height} of $P$ to be
\begin{align*}
H_K(P)&=\prod_{v\in M_K} \max\{\ve{x_0}_v,\ldots, \ve{x_n}_v\}^{n_v}\\
h_K(P)&=\log H_K(P)=\sum_{v\in M_K} -n_v\min\{v(x_0),\ldots, v(x_n)\}
\end{align*}
where $n_v=[K_v:\Q_v]$. (Recall that the normalized absolute value has $\ve{x}_v=|x|_v^{n_v}$.)
\end{df}
Note that the value of $H_K(P)$ is independent of the choice of homogeneous coordinates for $P$, because by the Product Formula~\ref{global-fields}.\ref{product-formula}, for any $c\in K^{\times}$ we have
\begin{align*}
\prod_{v\in M_K} \max\{\ve{cx_0}_v,\ldots, \ve{cx_n}_v\}
&=
\prod_{v\in M_K} \ve{c}_v\prod_{v\in M_K} \max\{\ve{x_0}_v,\ldots, \ve{x_n}_v\}\\
&=\prod_{v\in M_K} \max\{\ve{x_0}_v,\ldots, \ve{x_n}_v\}.
\end{align*}

Note that for the case $n=1$, we will often write $H_K(x)$ to mean $H_K(1:x)$, and likewise for $h_K$ and the other height functions to be defined.

\begin{ex}\label{qheight}
Suppose that $P\in \Pj^n(\Q)$, and write $P=(x_0:\ldots :x_n)$ where $\gcd(x_0,\ldots, x_n)=1$. Then   
\[
H(P)=\max\{|x_0|,\ldots, |x_n|\}.
\]
Indeed, for each prime $p$, one of $x_0,\ldots, x_n$ is not divisible by $p$, so $\max\{|x_0|_p,\ldots, |x_n|_p\}=1$. The only factor that contributes is from the real place.

For the special case $n=1$, if $\frac ab$ is such that $\gcd(a,b)=1$, then we simply have 
\[
H\pf{a}{b}=H(a:b)=\max\{|a|,|b|\}.
\]
\end{ex}
\begin{pr}[Elementary properties of height]$\,$
\begin{enumerate}\label{elem-height}
\item $H_K(P)\ge 1$ for all $P\in \Pj^n(K)$.
\item If $L/K$ is a finite extension, then
\[
H_{L}(P)=H_K(P)^{[L:K]}.
\]
\item The action of the Galois group on $\Pj^n(\ol{Q})$ leaves height invariant, i.e. for any $\si\in G(\ol{\Q}/{\Q})$ and $P\in \Pj^n(\ol{\Q})$, 
\[H(\si(P))=H(P).\]
\end{enumerate}
\begin{proof}$\,$
\begin{enumerate}
\item
Scale the coordinates of $P$ so that one of them equals 1. Then by definition, $H_K(P)\ge 1$.
\item
Use formula (??) by Lemma~\ref{norm-val-ext}.
\item
The Galois group permutes the places.
\qedhere
\end{enumerate}
\end{proof}
\end{pr}
In light of item 2, we can define an absolute height on $\Pj^n$.
\begin{df}
Let $P\in \Pj(\ol{\Q})$. Let $K$ be any finite extension of $\Q$ containing the coordinates of $P$. Define the \textbf{absolute multiplicative/logarithmic height} of $P$ to be
\begin{align*}
H(P)&=H_K(P)^{\rc{[K:\Q]}}\\
h(P)&=\log H(P)=\rc{[K:\Q]}h_K(P).
\end{align*}
\end{df}
Define the \textbf{field of definition} of $P=(x_0:\cdots:x_n)$ to be the smallest field $K$ such that $P\in \Pj(K)$. We have that
\[
\Q(P)=\Q\pa{\frac{x_0}{x_j},\ldots, \frac{x_n}{x_j}}
\]
where $j$ is any index such that $x_j\ne 0$.
%The following theorem says that there are a finite num
\begin{thm}\label{height-finite}
For any $B$ and $D$, the set
\[
\set{P\in \Pj^n(\ol{\Q})}{H(P)\le B\text{ and }[\Q(P):\Q]\le D}
\]
is finite. In particular, the number of points with height bounded by $B$ in any fixed number field $K$ is finite.
\end{thm}
\begin{proof}$\,$
\noindent\underline{Step 1:} First note the theorem holds if we only consider points in $\Q$, i.e. the set
\[
\set{P\in \Pj^n(\Q)}{H(P)\le B}
\]
is finite. Indeed, this follows from the characterization of the height on $\Q$ in Example~\ref{qheight} and the fact that there are finitely many points in $(\Z\cap [-B,B])^n$.\\

\noindent\underline{Step 2:}  Next, we reduce to the case $n=1$, as follows. Choose coordinates of $P$ so that $x_j=1$ for some $j$. Then for any $i$, we have
\[
H(P)=\prod_{v\in M_\Q(P)} \max_{1\le j\le n}\{\ve{x_j}_v\}
\ge\prod_{v\in M_\Q(P)} \max\{\ve{x_j}_v,1\}\ge H(x_j).
\]
Hence it suffices to show that 
\begin{equation}\label{height-finite-p1}
\set{x\in \ol{\Q}}{H(x)\le B\text{ and } [\Q[x]:\Q]\le D}
\end{equation}
is finite. It will follow from this that there are finitely many choices for each $x_j$, and hence a finite number of possibilities for $P$.\\

\noindent\underline{Step 3:} We would like to work with $\Q$. To do so, we consider the minimal polynomial $f$ of $x$. The lemma below shows that the height of the point formed from the coefficients is bounded in terms of the roots of the polynomial. A finite number of possibilities for $f$ will mean a finite number of possibilities for $x$.
\begin{lem}
Let 
\[f(X)=a_dX^d+a_{d-1}X^{d-1}+\cdots +a_0=(X-r_1)\cdots (X-r_d)\in \Q[X]\]
be a monic polynomial of degree $d$. Then\footnote{A closely related quantity to the RHS is the {\it Mahler measure} of a polynomial, defined as $M(f)=|a_d|\prod_{i=1}^n \max(1,|x_i|)$.}
\[
H(a_0:\cdots :a_d)\le 2^{d-1}\prod_{j=1}^d H(r_j).
\]
\end{lem} 
\begin{proof}
%Let $s_k$ be the $k$th symmetric polynomial in the $\al_j$.  
%By Vieta's formula, $a_{d-k}=(-1)^ks_k=\sum_{j_1<\cdots <j_k} \al_{j_1}\cdots \al_{j_k}$. Thus by the triangle inequality,
%\[
%|a_{d-k}|_v\le 
%\binom dk\max_{j_1<\cdots <j_k} |\al_{j_1}\cdots \al_{j_k}|_v
%\le
%\binom dk \max_{1\le j\le d} |\al_j|^k_v
%\le 2^d \prod_{j=1}^d\max\{|\al_j|_v,1\}^k.
%\]
%Hence
%\[
%\max\{|a_0|_v,\ldots, |a_d|_v\}\le 2^d \prod_{j=1}^d \max_{1\le j\le d}\{|\al_j|_v,1\}^d.
%\]
%Multiplying over all $v\in M_K$ and taking the $[K:\Q]$th root gives
%\[
%H(a_0:\cdots :a_d)=2^d \prod_{j=1}^{d}H(\al_j)^d.
%\]
We prove this by induction on $d$. The base case $d=1$ holds by definition of $H(\al)$. Suppose the lemma proved for polynomials of degree $d-1$. Let 
\[g(X)=b_{d-1}X^{d-1}+\cdots +b_0=(X-r_1)\cdots (X-r_{d-1}).\]
Then
\[
a_k=r_{d}b_k+b_{k-1},
\]
where for convenience $b_{-1}=0$. 

Let $K$ be the field of definition for $(a_0:\ldots:a_n)$ and define
\begin{equation}\label{epv}
\ep_v(m):=\begin{cases} 1,&v\in M_K^{0}\text{ (i.e. $v$ nonarchimedean)}\\
m,&v\in M_K^{\iy}\text{ (i.e. $v$ archimedean)}.\end{cases}
\end{equation}
By the triangle inequality,
\begin{align*}
|a_k|_v&\le \ep_v(2)\max\{|r_db_k|_v ,|b_{k-1}|_v\}\\
&\le \ep_v(2)\max\{|r_d|_v,1\}\max\{|b_k|_v,|b_{k-1}|_{v}\}.
\end{align*}
Hence
\[
\max_{0\le k\le d}(|a_k|_v)\le \ep_v(2)\max\{|r_d|_v,1\}\max_{0\le k\le d-1}|b_k|_v.
\]
Take the product over all $v\in M_K$ and noting that there are at most $[K:\Q]$ archimedean places (since each corresponds to a real embedding or a pair of complex conjugate embeddings), we get
\[
\prod_{v\in M_K}\max_{0\le k\le d}(|a_k|_v) \le 2^{[K:\Q]} \prod_{v\in M_K} \max_{0\le k\le d-1}\{|b_k|_v,1\}.
\]
Raising each side to the power $\rc{[K:\Q]}$ gives
\[
H_K(a_0:\ldots:a_n)%\stackrel{\text{def}}{=}\pa{\prod_{i=1}^ \max\{|a_i|_v,1\}}^{\rc{[K:\Q]}}
\le 2H(r_k)H(b_k)\le 2^{d-1}\prod_{j=1}^d H(\al_j)
%\end{align*}
\]
where the last step follows from the induction hypothesis.
\end{proof}
Suppose $x$ is in the set~(\ref{height-finite-p1}). 
Let $f(X)=a_dX^d+\cdots +a_0$ be the minimal polynomial of $x$, and $x_1,\ldots, x_d$ be the conjugates of $x$. Note $d\le D$. Further noting that all conjugates of $x$ have the same height (Proposition~\ref{elem-height}(3)), we have by the lemma that 
\[
H(a_d:\ldots:a_0)\le 2^{d-1}\prod_{j=1}^e H(x_j)=2^{d-1}H(x)^d\le 2^{D-1}B^D.
\]
This means all the coefficients $a_k$ have absolute value at most $2^{D-1}B^D$. This shows there are a finite number of possibilities for $f$ and hence a finite number of possibilities for $x$.
%\end{enumerate}
\end{proof}
%Kronecker
As a first application, we prove the following famous theorem of Kronecker.
\index{Kronecker's Theorem}
\begin{thm}[Kronecker]\label{kronecker-conj-unit-circ}
Suppose $\al\in \ol{\Q}$ has all conjugates lying on the unit circle. Then $\al$ is a root of unity.
\end{thm}
\begin{proof}
First we show that $H(\al)=1$. To this end, let $K=\Q(\al)$. If $v$ is a finite place of $K$, then $|\al|_v=1$ since $\al$ is a unit. If $v$ is an infinite place of $K$, then it is determined by a real or complex embedding, and $|\al|_v=1$ by assumption. This proves our claim.

It is easy to see from the definition of $H$ that $H(\al^n)=1$ for all $n$. Furthermore $\al^n\in \Q(\al)$ for each $\al$. 
However, by Theorem~(\ref{height-finite-p1}) there are a finite number of $x\in \ol{\Q}$ such that $x\in \Q(\al)$ and $H(x)=1$. Hence $\al^j=\al^k$ for some $j\ne k$, and $\al$ is a $(k-j)$th root of unity.
\end{proof}
\begin{rem}
It is informative to unravel the arguments leading to the theorem above. For each $\al^j$, we have that the minimal polynomial of $f_j$ has bounded degree; moreover it has bounded coefficients, simply because all conjugates of $\al^j$ have absolute value 1. (This is essentially the argument in Theorem~\ref{height-finite}.) Hence there are a finite number of $f_j$, and $\al^j=\al^k$ for some $j\ne k$.
\end{rem}
\section{Height functions and rational maps}
Next we consider how height transforms under rational functions.
\begin{thm}\label{height-under-rational-maps}
Let $\phi:\Pj^n\to \Pj^m$ be a rational map over $\ol{\Q}$. Write $\phi=(f_0,\ldots, f_m)$, where the $f_j$ are homogeneous of degree $d$. Let $Z=Z(f_0,\ldots, f_m)$, the subset of common zeros of the $f_j$ and $D=\Pj^n(\ol{\Q})\bs Z$. Then
\[
h(\phi(P))\le dh(P)+O(1)\quad \text{for all }P\in \Pj^n(\ol{\Q}).
\]
Moreover, if $X$ is a closed variety contained in $D$ (so $\phi$ defines a morphism $X\to \Pj^m$), then 
\begin{equation}\label{height-degree}
h(\phi(P))= dh(P)+O(1)\quad \text{for all }P\in X(\ol{\Q}).
\end{equation}
\end{thm}
In particular, if $\ph$ is a morphism then $h(\phi(P))=dh(P)+O(1)$ for all $P\in \Pj(\ol{\Q})$.
\begin{proof}
Let $K/\Q$ be a finite extension contain the field of definition for $\phi$ and $P$. To obtain the upper bound on $h(\phi(P))$ we calculate the valuations of the $f_j(P)$ and use the triangle inequality. Each $f_j$ can be written in the form
\[
f_j(x)=\sum_{|e|=d}a_ex^e.
\]
%where $e=(e_1,\ldots, e_n)$ and $x^e
Note there are $\binom{n+d}{d}$ terms in the above sum. 
Defining $\ep_v(t)$ as in~(\ref{epv}), we get that by the triangle inequality that
\[
|f_j(x)|_v\le \ep_v\binom{n+d}{d}\max_{e}(|a_e|_v)\max_{1\le j\le n}(|x_j|_v)^d
\]
and hence
\[
\max_{1\le j\le m}|f_j(x)|_v\le \ep_v\binom{n+d}{d}\max_{e}(|a_e|_v)\max_{1\le j\le n}(|x_j|_v)^d.
\]
Multiplying over all $v\in M_K$, taking the $[K:\Q]$th root, and noting that there are at most $[K:\Q]$ archimedean valuations, we get
\[
H(\phi(P))=H(f_0(P):\ldots :f_n(P))\le \binom{n+d}{d} H((a_e)) H(P)^d
\]
where $(a_e)$ is the point with coordinates equal to the $a_e$; note $H((a_e))$ is a constant depending on $\phi$. Taking the logarithm gives the first part.

For the second part, we will relate the height of $P$ with the height of $\phi(P)$ by writing powers of $x_i$ in terms of the $f_i$ by the Nullstellensatz. Let $X=Z(g_1,\ldots, g_{n'})$. 
Since $Z(f_1,\ldots, f_m,g_1,\ldots, g_{m'})=X\cap Z=\phi$, by the Nullstellensatz, \[\sqrt{(f_1,\ldots, f_m,g_1,\ldots, g_{m'})}=I(Z(f_1,\ldots, f_m,g_1,\ldots, g_{m'}))=(x_1,\ldots, x_m).\]
Hence there are polynomials $p_{k,1},\ldots, p_{k,m},q_{k,1},\ldots, q_{k,m'}$ and $e\in \N$ such that such that
\[
p_{k,1}f_1+\cdots +p_{k,m}f_m+q_{k,1}g_1+\cdots +q_{k,m}g_m=x_k^e.
\]
By taking the terms of highest degree we may assume the $p_j$ and $q_j$ are homogeneous.
For any point $P\in X$, we have $g_j(P)=0$ so the above becomes
\[
p_{k,1}(P)f_1(P)+\cdots +p_{k,m}(P)f_m(P)=x_k^e.
\]
Let $G$ be the point with coordinates equal to $b$ where $b$ is the coefficient of some $p_{k,j}$ or $q_{k,j}$. 
Since the $p_{k,j}$ have degree $d$, we see that $|p_{k,j}(P)|_v\le |G|_v\max_{1\le j\le n}(|x_j|_v)^{e-d}$. 
Taking the valuation and using the triangle inequality, %(INTRODUCE NOTATION FOR HEIGHT OF POLYNOMIAL)
\begin{align*}
|x_k|_v^m&\le \ep_v(m)|G|_v\max_{1\le j\le n}(|x_j|_v)^{m-d}\max_{1\le j\le n}(|f_j(P)|_v).\\
\implies
\max_{1\le j\le n}(|x_j|_v)^d&\le \ep_v(n)|G|_v\max_{1\le j\le n}(|f_j(P)|_v).
\end{align*}
Taking the product over all $v\in M_K$ and taking the $[K:\Q]$th root gives
\begin{align*}
H(P)^d&\le mH(G)H(\phi(P)).
\end{align*}
Taking logarithms gives the desired result.
\end{proof}
This theorem has an immediate application to the dynamics of rational maps on number fields. Define a \textbf{preperiodic point} of a function $f$ to be a point $P$ such that there exist $m\ne n$ with $f^m(P)=f^n(P)$.
\begin{thm}[Northcott]
Let $\phi:\Pj^N(K)\to \Pj^N(K)$ be a morphism of degree $d\ge 2$ over a number field $K$. Then the set $\PrePer(\phi)\sub \Pj^N(\ol{K})$ is of bounded height. 

In particular, the set of preperiodic points of $\phi$ in $K$ is finite.
\end{thm}
\begin{cor}
Let $\phi$ be a rational function on $\Pj^1(K)$. There are a finite number of points $P$ such that $\phi^m(P)=\phi^n(P)$ for some $m\ne n$.
\end{cor}
\begin{proof}
Theorem~\ref{height-under-rational-maps} gives us the lower bound
\begin{equation}\label{height-low-bd}
h(\phi(Q))\ge dh(Q)-C\text{ for all }Q\in \Pj^N(K).
\end{equation}
Suppose $\phi^m(P)=\phi^{m+k}(P)$. Then repeated application of the above gives
\[
h(\phi^m(P))=h(\phi^{m+k}(P))\ge dh(\phi^{m+k-1}(P))-C\ge\cdots \ge d^kh(\phi^m(P))-C(1+d+\cdots +d^{k-1}).
\]
Hence we get
\[
h(\phi^m(P))\le \frac{C}{d-1}.
\]
On the other hand,~(\ref{height-low-bd}) also gives
\[
h(\phi^m(P))\ge d^mh(P)-C(1+d+\cdots +d^{m-1}).
\]
Putting these two bounds together gives
\[
h(P)\le \frac{C}{(d-1)d^m}+\frac{C}{d-1}\le 2C.
\]

The second part now follows from Theorem~\ref{height-finite}.
\end{proof}