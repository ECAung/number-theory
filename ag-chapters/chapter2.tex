\chapter{Diophantine approximation}
\section{Approximation theorems}
Any real number can be approximated to an arbitrary degree by rational numbers. However, we would like these approximations to be ``efficient," that is, have good approximations without having denominators that are too large. Dirichlet's theorem gives a measure of how well we can be guaranteed to do this.
\begin{thm}[Dirichlet]
Given $\al\in \R$, there are infinitely many rational numbers $\frac pq\in\Q$ such that
\[
\ab{\frac pq-\al}\le \rc{q^2}.
\]
\end{thm}
In the other direction, it turns out that algebraic numbers cannot be approximated too closely by rationals.
\begin{thm}[Liouville]($\dagger$)
Let $\al\in \ol{\Q}$. There is a constant $C:=C(\al)$ such that for every $\frac{p}{q}\in \Q$, 
%For every $C$, there are only finitely many $\frac pq\in \Q$ such that
\[
\ab{\frac pq-\al}\ge \frac{C}{q^d}.
\]
(Equivalently, for every $\ep>0$, there are only finitely many $\frac pq \in \Q$ such that $\ab{\frac pq-\al}\le \frac{C}{q^d}$.)
\end{thm}
\begin{proof}
Assume $\al\nin \ol{\Q}$. 
Let $f$ be the minimal polynomial of $\al$. 

Note that $q^nf\pf pq$ is a nonzero integer, so
\[
\ab{q^nf \pf pq}\ge 1\implies \ab{f\pf pq}\ge \rc{q^n}.
\]
On the other hand, by the Intermediate Value Theorem there exists $x$ between $\frac pq$ and $\al$ such that 
\[
\ab{f\pf pq}=\ab{f\pf pq-f(\al)}=f'(x) \ab{\frac pq-\al}.
\]
Assuming $\ab{\frac pq-\al}<1$, there is a constant $C$ such that this is at most $C \ab{\frac pq-\al}$. Combining the above two inequalities gives
\[
\ab{\frac pq-\al}\ge \rc{Cq^n}
\]
for all $\frac pq$ with $\rc{\frac pq-\al}<1$, as needed.
\end{proof}
In fact, Liouville's Theorem can be made much stronger: $d$ can be replaced by $2+\ep$ for any $\ep>0$. This is the Thue-Siegel-Roth Theorem. We will state it for arbitrary number fields, keeping in mind that the case for $\Q$ is that described above. Recall that the natural measure of arithmetic complexity on $K$ is the height function $H_K$ (which in the case of $\Q$ is related to the numerator and denominator of the fraction).
\begin{thm}[Thue-Siegel-Roth]
Let $K$ be a number field, and $\al\in \ol{K}$. 
For every $C$, there are only finitely many $\frac pq\in \Q$ such that
\[
\ab{\frac pq-\al}\le \frac{C}{q^{2+\ep}}.
\]
\end{thm}
Remark on effectivity.
\section{Thue-Siegel-Roth Theorem}
\begin{lem}[Siegel's lemma]
For a $m\times n$ matrix $M$ let $|M|=\max_{\scriptsize\begin{array}{c} 1\le i\le m\\1\le j\le n\end{array}}|m_{ij}|$

Suppose $A\in \Mat_{m\times n}(\Z)$, with $n>m$. Let the row sums be
\[
A_i=\sum_{j=1}^n|a_{ij}|.
\]
Then there exists a nonzero solution $T=(t_1,\ldots, t_n)^T$ of 
$AT=0$ 
such that
\[
|T|\le (C_1\cdots C_m)^{\rc{n-m}}\le (N|A|)^{\frac{m}{n-m}}.
\]
\end{lem}
\begin{proof}
The key idea is to use the pigeonhole principle: 
Consider a set $S$ of $T$ with $|T|$ small, say 
\[S=\set{T}{0\le t_i\le M}.\]
When 
\begin{equation}\label{siegel-lemma-idea}
|S|>|\set{AT}{T\in S}|,
\end{equation}
then there must be $T_1$ and $T_2$ so that $AT_1=AT_2$, or $A(T_1-T_2)=0$. We can choose $M$ large enough so that~(\ref{siegel-lemma-idea}) holds: because there are more unknowns than equations, the LHS grows faster in $M$. This value of $M$ will give our bound.
%: There are approximately $B^n$ values of $T\in \Z^n$ with $|T|<B$, and for these $T$, the quantity $AT$ varies among approximately $(|A|B)^M$ values. When $B^N\ge (|A|B)^M$, then the Pigeonhole Principle gives $T_1\ne T_2$ with $AT_1=AT_2$, or $A(T_1-T_2)=0$.

Let $R_i$ be the $i$th row of $A$. Note that fixing $i$,
\[
\pa{\sum_{j\mid a_{ij}<0}a_{ij}}|T| \le R_iT \le\pa{\sum_{j\mid a_{ij}>0}a_{ij}}|T|,
\]
so there are at most $A_i=\ce{M}\sum_{j=1}^n|a_{ij}|$ possibilities for $R_iT$. Thus we have
\begin{align*}
|S|&=(M+1)^n\\
|\set{AT}{T\in S}|&=(1+\fl{M}A_1)\cdots (1+\fl{M}A_m)\le A_1\cdots A_m (1+\fl{B})^n.
\end{align*}
Taking $M=(A_1\cdots A_m)^{\rc{n-m}}$ gives~(\ref{siegel-lemma-idea}). As noted, using the Pigeonhole Principle gives the existence of $T_1$ and $T_2$ with $AT_1=AT_2$; take the vector $T_1-T_2$.
\end{proof}
\section{$S$-unit equation}
\begin{thm}[S-unit equation]
Let $S\subeq M_K$ be a finite set of places, and $a,b\in K^{\times}$. Then the equation
\[
ax+by=1
\]
has a finite number of solutions in $S$-units $x,y\in U(S)^{\times}$.%\sO_K(S)^{\times}
\end{thm}
\begin{proof}
Let $m$ be a large integer, to be chosen.
Every solution is in the form $x=\al X^m$ and $y=\be Y^m$ for $\al,\be$ coset representatives in $U(S)^{\times}/U(S)^{\times m}$. There are a finite number of cosets since by Dirichlet's $S$-unit theorem~\ref{units}.\ref{dsut} $U(S)$ is finitely generated. Thus it suffices to show that each equation $a\al X^m+b \be Y^m=1$ has finitely many solutions. Let $A=a\al$ and $B=b\be$. Then
\[
AX^m+BY^m=1.
\]
Write this as
\[
\prod_{\zeta^m=1}\pa{\frac XY -\zeta \ga}=\frac{1}{AY^m}.
\]
where $\ga$ is a $m$th root of $-\frac BA$. 

Assume by way of contradiction that there are infinitely many solutions. We have
\[
\prod_{\zeta^m=1}\ab{\frac XY-\zeta \ga}_v=\ab{\frac{1}{AY^m}}_v;
\]
we show that for some solution, this forces $\frac XY$ to be too close to $\zeta \ga$. Since $H_K(Y)=\prod_{v\in S}\max\{1,|Y|_v^{n_v}\}$, we get $|Y|_v\ge H_K(Y)^{\rc{|S[K:\Q]}}$ for some $v$. (Why?)

\end{proof}