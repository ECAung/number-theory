\chapter{Zeta and $L$-functions in number fields}\llabel{l-nf}
In this chapter we will define zeta and $L$-functions in number fields, to obtain density theorems for primes in those fields, in particular:
\begin{enumerate}
\item
Prime Number Theorem for number fields, and
\item
Chebotarev Density Theorem.
\end{enumerate}
To define $L$-functions, we will have to generalize our definition of characters.

As in the previous two chapters, we need a functional equation and analytic continuation of the $L$-function in order to get good asymptotic estimates. This presents a significant challenge. There are two approaches:
\begin{enumerate}
\item (Hecke) Generalize the proof for the $L$-functions over $\Q$. Namely, use a higher-dimensional analogue of theta functions.
\item (Tate) This is an illustration of the local-to-global principle. First define $L$-functions over local (complete) fields. This is easier because there is only a single prime to work with.
Then put these $L$-functions together to get a $L$-function for the global field.
\end{enumerate}
Note that $L$-functions over complete fields are much simpler---provided that you have the background in measure theory and functional analysis. We will give the required background in Section~\ref{measure-theory}.

As an illustration, note that the functional equation for $\ze$ (and similarly $L$) becomes more transparent ($\xi(s)=\xi(1-s)$) after we define $\xi$:
\[
\xi(s)=\pi^{-\fc s2}\Ga\pf s2\ze(s)=\underbrace{\pi^{-\fc s2} \Ga\pf s2}_{?} \prod_{p\text{ prime}}\rc{1-p^{-s}}.
\]
The presence of the term in front seems quite mysterious. However, we can think of it as coming from the infinite place; so instead of thinking of $\xi$ as a product over primes we should think of it as coming from a product over {\it places}.  We will define the zeta-function over a local field $K$ by
\[
\ze(f,s)=\int_{K^{\times}} f(a)\ve{a}_v^s da
\]
where we choose for $f$ a function that is its own Fourier transform (to get a good transformation law). 
Note that the measure here is the Haar measure on $K^{\times}$. For the case $K=\Q_p$, $f$ is a characteristic function; by calculating this integral on the sets $\set{a}{v_p(a)=n}$, $n\in \Z$ and summing, we get a geometric series which becomes the factor $\rc{1-p^{-s}}$ up to a constant. For the real place, we choose $f(x)=\rc{2\pi} e^{-2\pi x^2}$ and get $\pi^{-\fc s2} \Ga\pf s2$ out. Magic.
\section{Zeta and $L$-functions}
\section{Class number formulas}
\section{Density theorems (weak form)}
\section{Analytic continuation: Hecke's proof}
\section{Measure theory and functional analysis}\llabel{measure-theory}
\subsection{Measure theory}
For a set $E\neq \phi$ define the power set
\[\mathcal P(E)=2^E=\{\Ga:\Ga\subeq E\}.\]
\begin{df}\label{salgdf}
A subset $\mathcal B\subeq \mathcal P(E)$ is a $\sigma$\textbf{-algebra} it satisfies the following properties:
\begin{enumerate}
\item $E\in \mathcal B$.
\item $\mathcal B$ is closed under complementation: $\Ga\in \mathcal B$ implies $\Ga^c=E\backslash \Ga\in \cal B$.
\item $\{\Ga_n:n\geq 1\}\subeq \cal B$ implies $\bigcup_{n=1}^{\infty} \Ga_n\in \cal B$.
\end{enumerate}
%(If item 2 is satisfied just for finite instead of countable unions then we call $\mathcal B$ an algebra.)
\end{df}
Note that items 2 and 3 imply that a countable intersection of elements in $\mathcal B$ is in $\mathcal B$, and a difference of sets in $\mathcal B$ is in $\mathcal B$.
\begin{df}
We call 
$(E, \cal B)$ is a measurable space. A \textbf{measure} on $(E,\cal B)$ is a map $\mu:\cal B\to [0,\infty]$ such that 
\begin{enumerate}
\item $\mu(\phi)=0$.
\item (Countable additivity) If $\{\Ga_n:n\geq 1\}$ is a family of pairwise disjoint subsets of $E$, then
\[
\mu\pa{\bigcup_{n=1}^{\infty}\Ga_n}=\sum_{n=1}^{\infty} \mu(\Ga_n),
\]
i.e. the volume of the whole is the sum of the volume of the parts.
\end{enumerate}
\end{df}
Compare this to the definition of a  topological space---measurable spaces have measureable sets while topologies have open sets.

\begin{ex}
Define a measure $\mu$ on the integers $\Z$ by associating some $\mu_i\geq 0$ for each integer $i$, and setting
\[
\mu(\Ga)=\sum_{i\in \Ga}\mu_i.
\]
\end{ex}
%Lebesgue provided a measure for the reals---a more complicated set---that agrees with our previous definition of volume for intervals.

Our strategy is to start with some class of nice, well-defined subsets, and generate more.
\begin{df}
For a family of subsets $\cal C\subeq \cal P(E)$, define the
$\sigma$-algebra generated by $\cal C$, denoted by $\sigma(\cal C)$, to be the smallest $\sigma$-algebra containing $\cal C$. In other words it is the intersection of all $\sigma$-algebras containing $\cal C$. (This is well-defined since the power set is a $\sigma$-algebra containing $\cal C$.)

If $E$ is a topological space and $\cal C=\{\Ga\subeq E:\Ga\text{ open}\}$ then $\sigma(\cal C)=\cal B_E$ is called the \textbf{Borel }$\sigma$\textbf{-algebra}. (The sets are called Borel sets.)
\end{df}
Lebesgue showed that there exists a unique measure on $\cal B_{\R_N}$ such that $\mu_{\R^N}(I)=\vol(I)$ for rectangles $I$. 

DEFINE integrals given a measure... DEFINE $L^r$...

The following shows that given one measure, ``essentially" all other measures can be written in terms of an integral.
\begin{thm}[Riesz representation]
Suppose that $(E,\cal B,\nu)$ is a $\si$-finite measure space and $\mu$ is a finite measure on $(E,\cal B)$ with $\mu\le \nu$. Then there is a unique $\ph\in L^1(\nu;\R)$ such that 
\[
\mu(\Ga)=\int_{\Ga} \ph\,d\nu
\]
for all $\Ga\in \cal B$. %Moreover, $\ph$ can be chosen so it takes values in $[0,1]$.
\end{thm}
\begin{proof}
Stroock [add reference], 8.1.2.
\end{proof}
\begin{df}
Let $\mu$ be a Borel measure on a locally compact Hausdorff space $X$ and $E$ be a subset. $\mu$ is \textbf{outer regular} on $E$ if $\mu(E)=\inf\set{\mu(U)}{U\supeq E,U\text{ open}}$ and \textbf{inner regular} on $E$ if $\mu(E)=\sup\set{\mu(K)}{K\subeq E,K\text{ compact}}$.

A \textbf{Radon measure} on $X$ is a Borel measure that is finite on compact sets, regular on all Borel sets, and inner regular on all open sets.
\end{df}
\subsection{Haar measure}
\begin{df}
Let $G$ be a topological group and $\mu$ a Borel measure. $\mu$ is \textbf{left translation invariant} if for all Borel subsets $E$ of $G$, $\mu(sE)=\mu(E)$. Ditto for right translation invariant.

Let $G$ be a locally compact topological group. A \textbf{left} (right) \textbf{Haar measure} on $G$ is a nonzero Radon measure $\mu$ on $G$ that is left (right) translation-invariant. A bi-invariant Haar measure is a Haar measure that is both left and right invariant.
\end{df}
\begin{thm}
Let $G$ be a locally compact group. Then there exists a left/right Haar measure, unique up to scalar multiple.
\end{thm}
\begin{proof}
\cite{RV99}, Theorem 1.8.
\end{proof}
\subsection{Fourier inversion and Pontryagin duality}
\begin{df}
Let $G$ be an abelain topological group. 
A \textbf{continuous complex character} on $G$ is continuous homomorphism $G\to S^1$, where $S^1=\set{z\in \C}{|z|=1}$.\footnote{Alternatively, $G\to \R/\Z$, thought of additively.}

Under multiplication, the continuous complex characters form a group $\wh{G}$, called the \textbf{Pontryagin dual} of $G$. Give it the compact-open topology, i.e. the topology such that 
\[
W(K,V)=\set{\chi\in \wh{G}}{\chi(K)\subeq V}, \quad K\text{ compact, }V\text{ open}
\]
is a neighborhood base for the trivial character.
\end{df}
\begin{df}
Let $G$ be a locally compact topological group. A Haar-measurable function $\ph:G\to \C$ in $L^{\iy}(G)$ is of \textbf{positive type} if for any $f\in \cal C_c(G)$ (continuous, compact support),
\[
\iint_{G\times G} \ph(s^{-1}t)f(s)\,ds \ol{f(t)} \,dt\ge 0.
\]
\end{df}
\begin{df}
Let $f\in L^1(G)$. The \textbf{Fourier transform} of $f$ is the function $\wh f:\wh G\to \C$ defined by
\[
\wh f(\chi)=\int_G f(y)\ol{\chi}(y)\,dy.
\]
\end{df}
\begin{df}
Define $V(G)$ to be the complex span of continuous functions of positive type ib $G$ and $V^1(G)=V(G)\cap L^1(G)$.
\end{df}
\begin{thm}[Fourier inversion]
There exists a Haar measure on $\wh G$ such that for all $f\in V^1(G)$, 
\[
f(y)=\int_{\wh G} \wh f(\chi)\chi(y)\,d\chi.
\]
\end{thm}
The Fourier transform $f\mapsto \wh f$ identifies $V^1(G)$ with $V^1(\wh G)$.
\begin{ex}
The Pontryagin dual of $\R$ is $\R$, via the identification $y\mapsto e^{2\pi ixy}$. The Fourier transform is
\[
\wh f(y)=\int_{\R} f(x)e^{-ixy}\,dx.
\]
The Fourier inversion formula reads
\[
f(x)=CONSTANT\int_{\R}\wh f(y)e^{ixy}\,dx
\]

The Pontryagin dual of $\R/\Z$ is $\Z$, via the identification $e^{2\pi inx}$. The Fourier transform is 
\[
\wh f(y)=\int_{\R/\Z} f(x)e^{-2\pi ixy}\,dx
\]
and the Fourier inversion formula reads
\[
f(y)=CONSTANT\sum_{n\in \Z} \wh f(y)e^{2\pi iny}.
\]

The Pontryagin dual of an abelain group $G$ can be identified (noncanonically) with $G$ itself. Fourier inversion formula gives character formula! Connect with stuff in chapter on characters.
\end{ex}
\begin{thm}[Pontryagin duality]
The map $\al:G\to \wh{\wh{G}}$ defined by
\[
\al(y)(\chi)=\chi(y)
\]
is an isomorphism of topological groups. Hence $G$ and $\wh G$ are mutually dual.
\end{thm}
Measure on local fields. Relate to metric. Ostrowski's theorem again.

\begin{thm}\llabel{thm:rest-prod}
Suppose
\[
G=\prod_v'(G_v,H_v)
\]
is a restricted direct product of locally compact abelian groups $G_v$ with respect to open subgroups $H_v$. Then
\[
\wh G\cong \prod' \wh{G_v}.
\]
\end{thm}
APPLY TO IDELES/ADELES!
\section{Analytic continuation: Tate's thesis}
The main steps of the proof are as follows.
\begin{enumerate}
\item Define an additive and multiplicative measure on local fields, and classify all characters on these fields. We divide into three cases: real, complex, and $\mfp$-adic.
\item
Define local $L$-functions and prove a functional equation for them. This functional equation comes directly from the Fourier inversion formula applied to the local fields. Compute the functional equation in each of the three cases. 
\item
Show that the adele ring---a restricted direct product of local fields---behaves nicely as a product. That is, the following hold.
\begin{enumerate}
\item
The measure is the product of local measures.
\item
Products of nice (continuous, $L^1$) functions on the $K_v$ give nice functions on $K$.
\item
The Fourier transform of a product is the product of the Fourier transforms.
\end{enumerate}
%measure is just the product of local measures, a Fourier transroam
%Show that the adele ring---a restricted direct product of local fields---
Moreover, the adele is self-dual, because it is a restricted product of self-dual spaces.
%, using Theorem~\ref{rest-prod}. C
\item
Establish the Poisson formula and Riemann-Roch Theorem. Embed $K$ into $\A_K$ and think of $K$ as a ``lattice" in $\A_K$ to apply the Riemann-Roch Theorem. The local functional equations plus the Riemann-Roch Theorem give the analytic continuation and functional equation for the global $L$-function. This formula gives a relationship between a character and its dual, but we know that $\A_K$ is self-dual.
\item
Specialize to the case of Hecke characters to obtain the classical functional equation.
\end{enumerate}
We now carry out this program.
\subsection{Haar measure on local fields}
\subsection{Local functional equation}
\begin{df}
Let $f$ be a NICE function. Define the \textbf{local $L$-function} of $f$ to be the function on quasi-characters with positive exponent given by
\[
L(f,c)=\int_{K} f(x)c(x)\,d^{\times}x.
\]
\end{df}
Traditionally, we think of $L$ functions as functions of a complex variable. We recover this viewpoint if we write $c$ in the form
\[
c(x)=c_0(x)|x|^{s}=c_0(x)|x|^{\si+it},
\]
where $c_0(x)$ is a character in the same equivalence class as $c(x)$. Then fixing $c_0$, we can think of $L(f,c)$ as a function in $s$:
\[
L(f,c_0,s):=L(f,c_0|\cdot |^s).
\]
\begin{lem}
For any $f,g$ NICE and any quasi-character $c$ with exponent in $(0,1)$,
%be a quasi-character with $
\[
L(f,c)L(\wh g,\wh c)=L(\wh f,\wh c)L(g,c).
\]
Here $\wh c(x)=|x|c(x)^{-1}$.
\end{lem}
In other words, where it is defined $\fc{L(f,c)}{L(\wh f,\wh c)}$ is a function determined only by $c$. Thus we get the following.
\begin{thm}[Local functional equation for $L$]\llabel{thm:local-l-feq}
A local $L$-function has analytic continuation to the domain of all quasi-characters given by a functional equation
\[
L(f,c)=\rh(c)L(\wh f,\wh c),
\]
where $\rh(c)$ is a function independent of $f$.
\end{thm}
We now calculate the functional equations for $K$ real, complex, and $\mfp$-adic. To calculate $\rh(c)$, it suffices to choose a nice $f$ and compute
\[
\rh(c)=\fc{L(f,c)}{L(\wh f,\wh c)}
\]
since this function is independent of $f$.
The results are summarized in the following table.
\begin{thm}
The quasi-characters for $K$ are given in the top row of the table. Defining the corresponding functions $f$ as in the second row, th Fourier transforms of those functions $\wh f$ are those given in the third row, the $\ze$-functions are given in the fourth row, and the functions $\rh(c)$ are given in the fifth row.\\

\noindent
\begin{tabular}{|l|>{\raggedright}p{1.9in}|>{\raggedright}p{1.8in}|>{\raggedright}p{2.4in}|}
\hline
& $\R$ & $\C$ & $K_{\mfp}$ \tabularnewline
\hline
$c$ & $|x|^s$\\ $\sign(x)|x|^s$ & $c_n(\al)|x|^s$\\
where $c_n(re^{i\te})=e^{in\te}$ & $c_n(\al)$ character\\ of conductor $\mathfrak f=\mfp^n$\tabularnewline
\hline
$f$ & $f(s)=e^{-\pi s^2}$ \\ $f_{\pm}(s)=se^{-\pi s^2}$ & $f_n(s)=\begin{cases}\ol{s}^{|n|}e^{-2\pi |s|^2},&n\ge 0\\ s^{|n|} e^{-2\pi |s|^2},&n\le 0\end{cases}$ & $f_n=%=\begin{cases}
e^{2\pi i \la(s)}\one_{(\mfd\mf)^{-1}}$%,&s\in (\mfd\mf)^{-1}\\
\tabularnewline
\hline
$\hat{f}$
&
$\wh f(y)=f(y)$\\
$\wh {f_{\pm}}(y)=if_{\pm}(y)$ &
$\wh{f_n}(y)=i^{|n|}f_{-n}(y)$ &
$\wh{f_n}=(\fN \mfd)^{\rc 2}\fN \mf \one_{1+\mf}$\tabularnewline
\hline
$L$&
$L(f,\ad^s)= \pi^{-\fc s2} \Ga\pf s2$\\
$L(f_{\pm},\ad^s)=\pi^{-\fc{s+1}2} \Ga\pf{s+1}2$\\
$L(\wh f,\wh{\ad^s})=\pi^{\fc{s-1}2}\Ga\pf{1-s}2$\\
$L(\wh{f_{\pm}}, \wh{\pm\ad^s})=i\pi^{\fc{s-2}2}\Ga\pf{2-s}2$
& 
$L(f_n,c_n\ad^s)=(2\pi)^{1-s+\fc{|n|}2} \Ga\pa{s+\fc{|n|}2}$\\
$L(\wh{f_n},\wh{c_n\ad^s})=i^{|n|}(2\pi)^{s+\fc{|n|}2} \Ga\pa{1-s+\fc{|n|}2}$
& 
%$L(f,\ad^s)= \pi^{-\fc s2} \Ga\pf s2$\\
$L(f_{n},c_n\ad^s)=\fN\mfd^{-s}\fN\mfd^{s}G\pa{c_ne^{2\pi i\pf{\bullet}{\pi^{\ord_{\mfp}(\mfd\mf)}}}}$\\
%\pi^{-\fc{s+1}2} \Ga\pf{s+1}2$\\
%$L(\wh f,\wh{\ad^s})=\pi^{\fc{s-1}2}\Ga\pf{1-s}2$\\
$L(\wh{f_n}, \wh{c_n\ad^s})=\fN\mfd^{\rc2}\mu^{\times}(1+\mf)$
 \tabularnewline
\hline
$\rho$ &
$\rh(\ad^s)=2^{1-s}\pi^{-s}\cos\pf{\pi s}2 \Ga(s)$\\
$\rh(\pm\ad^s)=-i2^{1-s}\pi^{-s}\sin\pf{\pi s}2 \Ga(s)$
 &
$\rh(c_n\ad^s)=(-i)^n\fc{(2\pi)^{1-s}\Ga\pa{s+\fc{|n|}2}}{(2\pi)^s\Ga\pa{(1-s)+\fc{|n|}2}}$ 
 & $\rh(\ad^s) = \fN\mfd^{s-\rc 2} \fc{1-\fN\mfp^{s-1}}{1-\fN \mfp^{-s}}$\\
$\rh(\ad^s) = \fN(\mfd\mf)^{s-\rc2} \fN \mf^{-\rc2}G\pa{c,e^{2\pi i\pf{\bullet}{\pi^{\ord_{\mfp}(\mfd\mf)}}}}$
 \tabularnewline
\hline
\end{tabular}
\end{thm}

\section{Density theorems (strong form)}






