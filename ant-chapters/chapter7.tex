
%%%%%%%%%%%%%%
\chapter{Ramification}
\llabel{ramification}
We seek to generalize the definition of discriminant over Dedekind domains $A$ which are not PID's. To do this we will first define the {\it different}, which measure how much we can enlarge $B$ so that the image of the trace map is still in $A$, then define the discriminant as the discrepancy between $B$ and the enlarged $B$, using $\chi_A$. We will find that the different is the (ideal) norm of the discriminant.

We will see that our definition coincides with our previous definition when $A$ is a PID. Fortunately, we don't have to prove everything from scratch again: by localization we can always reduce to the DVR/PID case.

The main use of the discriminant is to measure ramification: The primes dividing the discriminant are those that ramify. On a deeper level, the exponents measure the degree of ramification.
%\section{Different/Discriminant}
%prime ideal ramified iff divide disc, diff in suitable sense
\section{Lattices and $\chi$}
\begin{df}
Let $A$ be a Dedekind domain, $K=\Frac(A)$, and $V$ a finite dimensional $K$-vector space. An $A$-submodule $X\subeq V$ is a \textbf{lattice} if it is finitely generated $A$-module and $\spn_K(X)=V$.
\end{df}
The most basic example of a latice is a fractional ideal of $K$.
%\begin{ex}
%$\ma$ fractional ideal of $K$ is an $A$-lattice.
%\end{ex}

We would like to measure the discrepancy between two lattices---like the norm, but measured by an {\it ideal} instead. To do this, we first need some facts from commutative algebra.
\subsection{Filtrations of modules}
\begin{df}
A module is \textbf{simple} if it is nonzero and has no nonzero proper submodule. A \text{composition series} of length $m$ is a chain of submodules
\[
M=M_0\supset M_1\supset\cdots \supset M_m=0
\]
where $M_{i-1}/M_i$ is simple for each $i$. $M$ has \textbf{finite length} if it has a finite composition series.
\end{df}
\begin{pr}
The simple modules are exactly those in the form $R/\mm$ where $\mm$ is a maximal ideal of $R$. If $M$ is simple, $M=R/\mm$ where $\mm=\Ann(M)$.
\end{pr}
The main theorem on filtrations is the following.
\begin{thm}[Jordan-H\"older]
Suppose $M$ has a composition series.
\begin{enumerate}
\item (Existence) Any chain of submodules of $M$ can be refined to a composition series.
\item (Uniqueness) Any composition series of $M$ has the same length; moreover the number of times $R/\mm$ appears as a quotient $M_{i-1}/M_i$ in the filtration is invariant.
% and the same multiset
%\[
%\set{\mm}{R/\mm\cong M_{i-1}/M_i}.
%\]
\end{enumerate}
\end{thm}
We will be applying this when $R$ is a Dedekind domain, so the maximal ideals are simply the nonzero prime ideals.

We also need the following.
\begin{pr}
If $M/M'$ and $M'$ have finite length, then so does $M$.
\end{pr}
\subsection{The function $\chi_A$}
\begin{df}
%Two equivalent definitions of $\chi_A$.
%\begin{enumerate}
%\item
Let $A$ be a Dedekind domain. 
Define
\[
\chi_A:\{\text{$A$-module of finite length}\}\to \{\text{ideals of $A$}\}
\]
as follows: Given $M$ of finite length, with composition series
\[
M=M_0\supset M_1\supset \cdots \supset M_m=0
\]
and $A/\mfp_i\cong M_{i-1}/M_i$, define
%\end{enumerate}
%Finite length means there exists a filtration
%\[
%0=M_0\subeq M_1\subeq \cdots \subeq M_l=M
%\]
%such that $M_i/M_{i-1}$ is a simple $A$-module.
%There exists filtration $0=M_0\subeq M_1\subeq \cdots \subeq M_l=M$ such that $M_i/M_{i-1}$ is a simple $A$-module ($\cong A/\mfp_i$ for some prime $0\ne \mfp_i$). Jordan-Holder theorem says $M_i/M_{i-1}$ are inependent of choice of filtration. 
%pos char w/o problem
%$0\ne m\in M_{\text{sim}}$ then $A\tra M_{\text{sim}}$ $a\mapsto am$.
\[
\chi_A(M)=\prod_{i=1}^m \mfp_i.
\]
\end{df}
\begin{ex}
The primes appearing in the filtration of an ideal $\ma\sub A$ are just the primes dividing $\ma$ with multiplicity, so 
\[\chi_{A}(\ma)=(\ma).\]
\end{ex}
\begin{pr}\llabel{chi-exact}
If $M'$ and $M''$ have finite length and $0\to M'\to M\to M''\to 0$ exact sequence of $A$-modules, then
\[
\chi_A(M)=\chi_A(M')\chi_A(M'').
\]
%%%
\end{pr}
%Let $A$ be a Dedekind domain, $K=\Frac(A)$, $V$ a finite-dimensional $K$-vector space, $X\subeq V$ an $A$-lattice.
\begin{df}
Let $A$ be a Dedekind domain, $K=\Frac(A)$, %$V$ a finite-dimensional $K$-vector space
and $X_1,X_2\subeq V$ be $A$-lattices. Choose $X_3\subeq X_1\cap X_2$ any $A$-lattice and define
\[
\chi_A(X_1,X_2):=\chi_A(X_1/X_3)\chi_A(X_2/X_3)^{-1}
\]
as fractional ideals of $K$.
\end{df}
%X_1/X_3, X_2/X_3 finite length
\begin{proof}[Proof of well-definedness]
We show this is independent of choice of $X_3$. %It suffices to show we get the same answer

Observe $\chi_A(X_1,X_2)\chi_A(X_2,X_1)=(1)$.
Note this is independent of choice of $X_3$. It suffices to show that
\[
\chi_A(X_1/X_3)\chi_A(X_2/X_3)^{-1}
=\chi_A(X_1/X_4)\chi_A(X_2/X_4)^{-1}
\]
when $X_4\subeq X_3$. This follows by the exact sequence
\[
0\to X_3/X_4\to X_1/X_4\to X_1/X_3.
\]
and Proposition~\ref{chi-exact}.
\end{proof}
%\chi(X_1/X_3)\chi(X_3/X_4)\chi(X_2/X_3)^{-1}\chi(\chi_3,\chi_4)^{-1}
%Existence: X_1\cap X_2 is A lattice of finite length: a^{-1}X_1\subeq X_2\subeq X_1. Use finite generation.
%\subsection{Localization technique}
%{\color{red} Start of unedited stuff.}
\subsection{$\chi$ and localization}
It is easier to study $\chi_A$ when $A$ is local; in this case $\chi_A(X)$ is simply a power of the maximal ideal. To understand $\chi_A$ (and hence the discriminant) for general $A$, we thus consider the localization of $A$ at all primes. The following says that $\chi_A$ is well-behaved under localization.
%As we will be using $\chi_A$ to define the discriminant, in light of the beginning comments, we study the behavior of $\chi_A$ under localization.
%Problems get easier!
%Then $X_{\mfp}=A_{\mfp}\cdot X=A_{\mfp}\ot_A X$ is an $A_{\mfp}$-lattice of $V$.
\begin{pr}%can detect exponent
%DVR much easier
\llabel{chi-exponent}
Let $A$ be a Dedekind domain and $\mfp\sub A$ be a nonzero prime. Then %and $A_{\mfp}$ the localization at $\mfp$; write $K=\Frac(A_{\mfp})$. 
\[
v_{\mfp}(\chi_A(\chi_1,\chi_2))
=v_{\mfp A_{\mfp}}(\chi_{A_{\mfp}}((X_{1})_{\mfp},(X_2)_{\mfp})).
\]
\end{pr}
\begin{proof}
Note $X_{\mfp}=A_{\mfp}\cdot X=A_{\mfp}\ot_A X$ is an $A_{\mfp}$-lattice of $V$.

Localization is exact, so preserves quotients. Suppose $M\supeq N$ are adjacent terms in the filtration of $A$. If $M/N=A/\mfp$ then \[M_{\mfp}/N_{\mfp}=(M/N)_{\mfp}=(A/\mfp)_{\mfp} =A_{\mfp}/\mfp A_{\mfp}\]
while if $M/N=A/\mq$, $\mq\ne\mfp$, then $M_{\mfp}/N_{\mfp}=0$. Only the quotients with $A/\mfp$ remain; the result follows.
\end{proof}
\begin{pr}
Let $A$ be a Dedekind domain with fraction field $K$, $X$ an $A$-lattice in $V$, and $\si\in \Aut_K(V)$. Then  
\[
\chi_A(X,\si X)=(\det\si).
\]
%in $I_K$, the group of fractional ideals.
\end{pr}
\begin{proof}
%By Proposition to see that for all nonzero $\mfp\sub A$ prime, both sides at the same order of $\mfp$. %To do this we localize (check after localization at $\mfp$).
%smith can. form only works with PID
It suffices to check both sides have the same $\mfp$-valuation for every prime $\mfp$ of $A$; by Proposition~\ref{chi-exponent} this is equivalent to
\[
\chi_{A_{\mfp}}(X_{\mfp},\si_{\mfp} X_{\mfp})=(\det\si_{\mfp}).
\]

Thus we only need to check the proposition for the case where $A$ is a DVR, hence a PID. For all nonzero $\al\in A$, 
\[
\chi(X,\al \si X)=\al^n \chi(X,\si X)=\det[\al]\cdot\chi(X,\si X);
\]
note we used $\chi(uX,\al uX)=\al^n$ since $X$ is free over $A$, and that the matrix of the transformation $[\al]$ is simply $\al I$.
%(Every lattice over $A$ is free; w
Thus by choosing $\al$ such that $\al \si X\subeq X$ we may assume $\si(X)\subeq X$.

%Choosing a basis $u$ can be represented by a diagonal matrix $\al_1,\ldot,s \al_n$, and $X
By the structure theorem for modules, $X/\si X\cong A/\al_1\times \cdots \times A/\al_n$ for some $\al_j$, giving
\[
\chi_A(X,\si X)=(\al_1\cdots \al_n)=(\det \si).
\]
\end{proof}
\subsection{Discriminant of bilinear forms}
In this section we will define the discriminant of a bilinear form $T$ on a lattice $X$ over $K$, the fraction field of a Dedekind domain $A$. When we specialize to the case that $X$ is a extension of $A$ and $T=\tr$, then we get a generalization of our original definition~\ref{disc-df}, in the case where $X$ is not necessarily free over $A$.
\begin{df}
%Keep the same assumptions, setting $K=\Frac(A)$. 
Keep the above assumptions. 
Let $V$ be a finite-dimensional $K$-vector space and $T:(V,V)\to K$ be a nondegenerate $K$-bilinear form. Thinking of $T$ as a map $V\ot_K V\to K$, 
%for all $v\in V$, $T(v,v')=0$ and for all $v'$ implies $v=0$; $T(v',v)=0$ etc.
we get a map
\[
\wedge^nT:\wedge^nV\ot_K \wedge^n V\to K
\]
defined by 
\begin{equation}\llabel{wedge-T-def}
\wedge^n T(v_1\wedge\cdots \wedge v_n, w_1\wedge \cdots \wedge w_n)=\sum_{\pi\in S_n}(-1)^{\sign(\pi)} T(v_1\wedge w_{\pi(1)})\cdots T(v_n\wedge w_{\pi(n)}).
\end{equation} 
Note $\wedge^n T\ot\wedge^n T$ is a 1-dimensional vector space over $K$, with lattice $\wedge^n X\ot_K\wedge^n X$.
%X free mod over A - explicit
Define the \textbf{discriminant} of $T$ on $X$ to be 
\[\mfd_{X,T}:=\chi_A(\wedge^nT,\wedge^nX\ot \wedge^nX).\]
\end{df}
%The following schows that this coincides with our previous definition~\ref{}. (ADD LINK)
The main reason for defining the discriminant as above is because the ``$\wedge$" construction is natural and makes it easy to prove a few basic properties. 
\begin{pr}\llabel{chi-det}
If $X$ is free over $A$ with basis $(e_1,\ldots, e_n)$, then%, $X\cong \bigoplus_{i=1}^n Ae_i$, then 
\[\mfd_{X,T}=(\det(T(e_i, e_j))).\]
\end{pr}
\begin{proof}
%exercise in ext alg!
%Unravel exterior algebra. Consider 
%By the characterization of determinant in terms of exterior products,
Note that $X\ot_K X$ is generated by $\wedge^n T(e_1\wedge \cdots \wedge e_n, e_1\wedge \cdots \wedge e_n)$. By~(\ref{wedge-T-def}), this is exactly $(\det(T(e_i, e_j)))$.
\end{proof}
%After we establish the properties of the discriminant, we'll only need its discription in Remark~\ref{two-def-disc}.

%We'll only really need that taking exterior product commutes with localization. 
%Whenever possible we reduce to the free case by localization.
%\subsection{Different}
%We now define a related quantity called the different.
We now give an alternative characterization of the discriminant, in terms of the {\it dual lattice}.
\begin{df}
Define the \textbf{dual} of $X$ with respect to $T$ by
\[
X_T^*:=\set{y\in V}{T(x,y)\in A\text{ for all }x\in X}.
\]
This is an $A$-lattice of $V$.
\end{df}
We first need the following.
\begin{pr}\llabel{dual-basis}
If $e_1,\ldots, e_n$ is a basis for $X$ over $A$, and $e_1^*,\ldots, e_n^*$ is a {\it dual basis}, i.e. $T(e_i,e_j^*)=\de_{ij}$ for each $j$, then $e_1^*,\ldots, e_n^*$ is a basis for $X^*$ over $A$.
\end{pr}
\begin{proof}
%The $e_j^*$ are linearly independent so form a basis for $V$ over $K$. 
Note $y\in X^*$ iff $T(e_j,y)\in A$ for each $j$. Writing $y=\sum_{j=1}^n a_je_j^*$, we find $T(e_j,y)=a_j$, so $y\in X^*$ iff $a_j\in A$ for each $j$, i.e. $y\in \spn_A(e_1^*,\ldots ,e_n^*)$.
\end{proof}
\begin{pr}
We have
\[
\chi_A(X_T^*,X)=\mfd_{X,T}.
\]
\end{pr}
%Free case, explicit computation (see beginning of course, A=Z, usual disc of num field); this is more general
\begin{proof}
We use the fact that a fractional ideal is determined by its localizations at all primes (this follows since the exponent of $\mfp$ in $\ma$ is the same as that of $\mfp A_{\mfp}$ in $\ma A_{\mfp}$, Proposition~\ref{uf-dedekind-cor}).

By using Proposition~\ref{chi-exponent}, we may localize at nonzero $\mfp\sub A$. Hence it suffices to prove may assume $A$ is DVR, i.e. free over $A$.

%Write $X=\bigoplus_{i=1}^n Ae_i$, check $X_T^*=\bigoplus_{i=1}^n A\cdot e_i^*$ where $\{e_i^*\}$ is a dual basis given by $T(e_i\ot e_j^*)=\de_{ij}$.
Write
\[
\colthree{e_1}{\vdots}{e_n}
=
B
\colthree{e_1^*}{\vdots}{e_n^*}
%(e_i)=(b_{i,j})(e_j^*)
\]
where $B=(b_{i,j})$ is a $n\times n$ matrix.
%($X=(b_{i,j}) X_T^*$ ($b_{i,j}$ is the $u$ in the proposition))
Then by Proposition~\ref{chi-det},
%relate orig and dual basis
\[
\mfd_{X,T}=(\det(T(e_i,e_j)))=(\det(b_{i,j}))=\chi(X_T^*, BX_T^*)=\chi(X_T^*,X),
\]
as needed.
\end{proof}
\section{Discriminant and different}
For the AKLB setup with $L/K$ finite separable, consider the nondegenerate $K$-bilinear map
\begin{align*}
\tr:L\times L&\to K\\
(x,y)&\mapsto \tr_{L/K}(xy).
\end{align*}
\begin{df}
Define the \textbf{codifferent}
\[
B^*:=B_{\tr}^*=\set{y\in L}{\tr(xy)\in A\text{ for all }x\in B}
\]
and the \textbf{different} and \textbf{discriminant} by
\begin{align*}
\mathfrak \mD_{B/A}=\mathfrak \mD_{L/K}&:=(B^*)^{-1}\\
\mfd_{B/A}=\mfd_{L/K}&:=\mfd_{B,\tr}.
\end{align*}
\end{df}
(These are fractional ideals of $K$.)

Observe that $B\subeq B^*$ (in light of $\tr_{L/K}(B)\subeq A$) so $B\supeq \mathfrak \mfd_{B/A}$.

%Interlude: norm of ideal $\nm_{L/K}:J_L\to J_K$ group homomorphism, $\mP\mapsto \mfp^{f_{\mP/\mfp}}$. Another view: $\nm_{L/K}(\mb)$ is the ideal generated by $\nm_{L/K}(b)$ with $b\in \mb$. Remark: when $K=\Q$, $A=\Z$ gives usual thing. $\nm_{L/\Q}(\mb)=\mathfrak N(\mb):=|\sO_L/\mb|$.
%Example $\mb=\mP$. $|\sO_L/\mP|=p^f$. Dimension $f$ over $p$.
%more general, between any fin sep ext.
The following gives the precise relationship between the discriminant and different.
\begin{pr}
$\nm_{L/K}(\mathfrak \mD_{B/A})=\mfd_{B/A}$.
\end{pr}
\begin{proof}
We have
\[
\mfd_{B/A}=\chi_A(B^*,B)=\chi_A(B^*/B)
\]
and
\[
\mD_{B/A} = (B^*)^{-1} = \chi_B(B^*/B).
\]
The result thus follows from commutativity of the following diagram.
\[
\xymatrixcolsep{5pc}
\xymatrix{
\{\text{finite length $B$-module}\}\ar[r]^{\chi_B} \ar[rd]^{\chi_A}&
I_B\ar[d]^{\nm_{L/K}}\\
&I_K.
}
\]
We have commutativity since if $B/\mP$ is a quotient of adjacent terms in the $B$-filtration of $M$, then when we refine it to a $A$-filtration, since $B/\mP=(A/\mfp)^{f(\mP/\mfp)}$ as vector spaces, we get $f(\mP/\mfp)$ copies of $A/\mfp$.
%decompose into simple module.
%Reduce to $M\cong B/\mP$, $\mP$ nonzero prime. $f$-dimensional vector space over $A/\mfp$. We have $\chi_B(M)=\mP$ and $\chi_A(M)=\mfp^f=\nm_{L/K}(\mfp)$.
\end{proof}
\fixme{Note: Grothendieck group.}
\subsection{Basic properties}
First, a slightly cleaner characterization of the codifferent.
\begin{lem}\llabel{codiff-lem}
$\ma\in I_K$ and $\mb\in I_L$. Then
\[
\tr_{L/K}(\mb)\subeq \ma\iff \mb\subeq \ma \mathfrak D^{-1}_{B/A}.
\]
\end{lem}
\begin{proof}
%\begin{align*}
%\tr(b)\sub \ma&\iff \ma^{-1} \tr(\mb)\subeq A
%\end{align*}
We check $\tr(\ma^{-1}\mb)\subeq A$ iff $\ma^{-1}\mb\subeq \mathfrak \mD_{L/K}^{-1}$. 

The reverse direction is clear. For the forward direction, note that if $x\in \ma^{-1}\mb$ and $y\in B$, then $xy\in \ma^{-1}\mb$ and hence $\tr(xy)\in A$. This shows $x\in \mD_{B/A}^{-1}$.
%for all $x\in \ma^{-1}\mb$, $y\in \mb$, we have $\tr(xy)\in A$ ($\ma^{-1}\mb$) and $x\in \mathfrak \mfd_{L/K}^{-1}$.
\end{proof}
\begin{pr}
\begin{enumerate}$\,$
\item (Transitivity) Let $M/L$ be a finite separable extension, % of degree $n$, 
with $C$ the integral closure of $A$ in $M$. Then
\[
\mathfrak D_{C/A} =\mathfrak D_{C/B}\mathfrak D_{B/A}.
\]
\item (Localization) For $S\subeq A$ a multiplicative subset,
\[
S^{-1}\mathfrak D_{B/A} = \mathfrak D_{S^{-1}B/S^{-1}A}.
\]
\item (Completion) %go from AKLB to \hat{A}_{\mfp}, etc.
\[
\mathfrak D_{B/A} \cdot \hat{B}_{\mP} = \mathfrak D_{\hat{B}_{\mP}/\hat{A}_{\mfp}}.
\]
\end{enumerate}
\end{pr}
\begin{proof}
\begin{enumerate}
\item
We have
\begin{align*}
&e\in \mathfrak D_{C/B}^{-1} \mathfrak D_{B/A}^{-1}\\
\iff & \mathfrak D_{B/A} e\subeq \mathfrak D_{C/B}^{-1}\\
{\iff}& \tr_{M/L}(\mathfrak D_{B/A}e)\subeq B&\text{Lemma~\ref{codiff-lem} with $M/L$}\\
\iff & \mathfrak D_{B/A} \tr_{M/L}(e)\subeq B\\
\iff & \tr_{M/L}(e)\in \mathfrak D_{B/A}^{-1}\\
\iff & \tr_{L/K}(\tr_{M/L}(e))\subeq A&\text{Lemma~\ref{codiff-lem} with $L/K$}\\
\iff & e\in \mathfrak D_{C/A}^{-1}.
\end{align*}
\item Omit.
\item Localize at $\mfp$. May assume $A$ is a DVR. ($B$ may not be a DVR.) Consider
\[
\xymatrix{
\prod_{\mP\mid \mfp} \hat{B}_{\mP} \ar[d]^{\cong} \ar@{^(->}[r]& \prod_{\mP\mid \mfp} \hat{L}_{\mP}\ar[d]^{\cong}\\
B\ot_A \hat{A}_{\mfp}\ar@{-}[d]\ar@^{(->}[r] & L\ot_K \hat{K}_{\mfp}\ar[d]^{\tr_{L/K}\ot_K \hat{K}_{\mfp}}\\
\hat{A}_{\mfp} \ar@^{(->}[r] & \hat{K}_{\mfp}
}
\]
The top-to-bottom map on the right is $\sum \tr_{\hat{L}_{\mP}/\hat{K}_{\mfp}}$. Then
\begin{align*}
\mathfrak \mfd_{B/A}^{-1} \ot_A \hat{A}_{\mfp}
&\cong \mathfrak D^{-1}_{B\ot_A \hat{A}_{\mfp}/\hat{A}_{\mfp}}\\
&\cong \mathfrak D^{-1}_{\prod_{\mP\mid \mfp}\hat{B}_{\mP}/\hat{A}_{\mfp}}\\
&\cong \prod_{\mP\mid \mfp} \mathfrak D^{-1}_{\hat{B}_{\mP}/\hat{A}_{\mfp}}\\
&\cong \prod_{\mP\mid \mfp} \mathfrak \mfd_{B/A}^{-1}\ot_B \hat{B}_{\mP}
\end{align*}
\end{enumerate}
\end{proof}
\section{Discriminant and ramification}
Recall $\ord_{\hat{\mP}} (\mathfrak D_{\hat B_{\mP}/\hat A_{\mfp}} = \ord_{\mP} (\mathfrak \mfd_{B/A})$.
Our goal is to show that $e_{\mP/\mfp}=1$ and $\kappa(\mP)/\kappa(\mfp)$ separable (i.e. $\mP$ is unramified over $K$, iff $\mP\nmid \mathfrak \mfd_{B/A}$.

In the CDVR case,
\[
\xymatrix{
\mP \ar@{^(->}[r] \ar@{-}[d] & B \ar@{-}[r]\ar@{-}[d] &L \ar@{-}[d]\\
\mfp\ar@{^(->}[r]& A\ar@{-}[r] &K
}
\]
we have $B/\mfp B=B/\mP^e$. 
\begin{lem}
\[
\tr_{L/K}(b)\bmod p = e\tr_{l/k} (\ol b).
\]
\end{lem}
\begin{proof}
For all $b\in B$, 
\[
0=\mP^e/\mP^e\subeq \mP^{e-1}/\mP^e\subeq \cdots \subeq \mP/\mP^e \subeq B/\mP^e.
\]
Each adjacent quotient is 1 dimensional over $l$ and hence $f$-dimensional over $k$.
Choose a basis $\{\ol w_i\}_{i=1}^n$ ($n=ef$) for $B/\mP^e$ as $k$-vector space, such that
\[
\spn_k(\{w_i\}_{i=(e-j)f+1}^{ef})=\mP^{e-j}/\mP^e.
\]
(The last $jf$ vectors span $\mP^{e-j}/\mP^e$.) Lift $\{\ol w_i\}$ to $w_i\in B$ such that $w_i\bmod{\mP^e}=\ol w_i$. The $w_i$ are a basis of $B$ over $A$. Now
\begin{align*}
\tr_{L/K}(b)&=\tr_K (m_b)\\
b w_i&=(b_{i,j}) (w_j)_{j=1}^n\\
\ol b\ol w_i &= (\ol b_{i,j})(\ol w_j).
\end{align*}
%On the LHS, w
We have
\[
\tr_{L/K}(b)=\sum_{i=1}^n b_{ii} \bmod{\mfp}.
\]
Now $(b_{i,j}\bmod{\mfp})_{fk+1\le i,j\le f(k+1)}$ represents the linear map (multiplication by $\ol b$)
\[
\fc{\mP^k/\mP^e}{\mP^{k+1}/\mP^{e}}\to \fc{\mP^k/\mP^e}{\mP^{k+1}/\mP^e}.
\]
The trace as a $k$-linear map is $\tr_{l/k}(\ol b)$. There are $e$ such $f\times f$ blocks.
\end{proof}
\begin{cor}
\[
\ord_{\mP}(\mathfrak \mfd_{B/A})\ge e-1.
\]
\end{cor}
\begin{proof}
It suffices to show
\[
\ord_{\mfp}(\mathfrak \mfd_{B/A})\ge (e-1)f.
\]
This is since $\mD_{B/A}=\mP^c$ implies $\mfd_{B/A} = \nm_{L/K}(\mP^c)=\mP^{cf}$.

Now
\[
\mfd_{B/A}=(\det \tr_{L/K}(w_iw_j))
\]
same as in the previous proof. Now $w_i\in \mP$ if $f+1\le i\le n=ef$, therefore $\ol w_i\in \mP/\mP^e$.
For all $j$, $\tr_{L/K}(w_iw_j)\in \mP\cap K=\mfp$, giving the result.
\end{proof}
Now consider the general case.
\begin{thm}
Suppose $A,B$ are Dedekind. Then
\begin{enumerate}
\item
$\ord_{\mP}(\mD_{B/A})\ge e_{\mP/\mfp}-1$.
\item
$\mP$ is unramified over $K$ iff $\mP\nmid \mD_{L/K}$.
\end{enumerate}
Serre does this by Eisenstein polys.
\end{thm}
\begin{proof}
\begin{enumerate}
\item
$\ord_{\mP}(\mD_{B/A})=\ord_{\hat{\mP}}(\mD_{\hat B_{\mP}/\hat A_{\mfp}})$. and $e_{\mP/\mfp}=e_{\hat{\mP}/\hat{\mfp}}$. Use the CDVR case.
\item
For ``$\Leftarrow$", note $\ord_{\mP}(\mD_{L/K})$ implies $0\ge e_{\mP/\mfp}-1$ i.e. $e_{\mP/\mfp}=1$.

For ``$\implies$", it suffices to prove $p\nmid \mfd_{B/A}$. Reduce to the CDVR case.
Now 
\[\det(\tr_{L/K}(w_iw_j))\bmod{\mfp}=e_{\mP/\mfp}\tr_{l/k}(\ol{w_iw_j})\ne 0\]
if $l/k$ is separable (Neukirch I.2).
\end{enumerate}•
\end{proof}
%if inseparable p can still divide \md_{B/A}.
\subsection{Types of ramification}
\begin{df}
$\mP$ is unramified if $e_{\mP/\mfp}$ and $l/k$ separable. For $\mP$ ramified,
\begin{enumerate}
\item
$\mP$ is \textbf{tamely ramified} if either $\chr k=0$ or $\chr k\nmid e_{\mpp}$.
\item
$\mP$ is \textbf{wildly ramified} otherwise. 
\end{enumerate}
\end{df}
\begin{thm}
$\mP$ is tamely ramified over $K$ iff
\[
\ord_{\mP}(\mathfrak \mfd_{L/K})=e_{\mpp}-1.
\]
\end{thm}
\begin{proof}
Reduce to the CDVR case.\\

\noindent\underline{Step 1:} We show that $\mP$ is tamely ramified iff $\tr_{L/K}(B)=A$. Observe that $\tr_{L/K}(B)$ is an ideal of $A$, so the latter is equivalent to $\tr_{L/K}(B)\pmod{\mfp}\ne 0$. But we know
\[
\tr_{L/K}(b)\bmod{\mfp} = e_{\mpp} \tr_{l/k} (\ol b),
\]
and $\tr_{l/k}(\ol b)\ne 0$ (not identically 0). Hence $e_{\mpp}\nequiv 0\pmod{\mfp}$ iff $\tr_{L/K}(b)\nequiv 0\pmod{\mfp}$.\\

\noindent\underline{Step 2:} $\tr_{L/K}(B)=A\iff \ord_{\mP}(\mD_{L/K})=e_{\mpp}-1$.

We've seen 
\[
\tr_{L/K}(\mb)\subeq \ma\iff \mb\subeq \ma\mD_{B/A}^{-1}.
\]
Plug in $\mb=B$ to get, as ideals of $B$,
\begin{align*}
A':=\tr(B)\subeq \ma&\iff B\subeq \ma\mD_{L/K}^{-1}\\
&\iff \mD_{L/K} \subeq \ma B
%&\iff \mD_{L/K}\cap K\subeq \ma B\cap K=\ma.
%uh -oh \mfp B=\mP^e.
\end{align*}
Write $A'=\mfp^{a}$. We have $\mfp^a\mid \mD$ iff $\mfp^a\mid A'$ for $a\in \Z$. (Power can be rational.)

For $a\in \Z$, $\ord_{\mfp}(A')\ge a$ iff $\ord_{\mfp}(\mD)\ge a$.

Thus we get
\[
\ord_{\mfp}(A')\le 
\underbrace{\ord_{\mfp}(\mD_{L/K})}_{\fc{\ord_{\mP}(\mD)}{e_{\mpp}}}
<\ord_{\mfp}(A')+1.
\]
Thus $\tr_{L/K}(B)=A$ iff $a=0$ iff $\ord_{\mP}(\mD)=e-1$.

Thus $\mP$ is tamely ramified iff $v(\mD)=e-1$.
\end{proof}
\subsection{Computation of different}
\begin{pr}\llabel{monogenous}
When $A$ and $B$ are CDVR's, $B$ is generated by one element over $A$ as an $A$-algebra:
\[
B=A[\be].
\]
(We say that $B$ is {\it monogenous} over $A$.)
%, i.e. there exists $\be\in B$ such that 
%\[
%B=A[\be]=\bigoplus_{i=0}^{n-1} A\cdot \be^i
%\]
%the last equality as $A$-module. 

Let $L:=\Frac(A)$, $K:=\Frac(B)$. When $L/K$ is totally ramified, then we can choose $\be$ to be any uniformizer $\pi_L$.
\end{pr}
\begin{proof}
Any element of $B$ can be written as $\sum_{k\ge 0} a_k\pi_k$ where $a_k$ are fixed representatives of $l=B/(\pi_L)$. But we can choose the $a_k$ to be representatives of $k=A/(\pi_K)$, since $k=l$.
\end{proof}
\begin{thm}
(Residue field extension separable.) 
$\mathfrak \mfd_{B/A}=(f'_{\be}(\be))$ where $f_{\be}(x)\in A[x]$ is the minimal polynomial of $\be$ over $K$.
\end{thm}
\begin{proof}
\begin{lem}
\[
\tr_{L/K}\pf{\be^k}{f'(\be)} =\begin{cases}
0,& 0\le i\le n-2\\
1,& i=n-1.
\end{cases}
\]
\end{lem}
\begin{proof}The eigenvalues of multiplication by $\beta$ are just the roots $\beta_{1},\ldots,\beta_{n}$
of the characteristic polynomial. Note that if $A$ is a linear operator
with eigenvalues $\lambda_{i}$ and $P$ is a polynomial then $P(A)$
has eigenvalues $P(\lambda_{i})$. Hence
\[
\text{tr}\left(\frac{\beta^{k}}{f'(\beta)}\right)=\sum_{i=1}^{n}\frac{\beta_{i}^{k}}{f'(\beta_{i})}\]

Let $D(x_{1},\ldots,x_{n})=\sum_{i<j}(x_{i}-x_{j})$. Noting $f'(\beta_{i})=\prod_{j\neq i}(\beta_{i}-\beta_{j})$,
the above equals
\[
\frac{1}{D(x_{1},\ldots,x_{n})}\underbrace{\sum_{i=1}^{n}\frac{x_{i}^{k}D(x_{1},\ldots,x_{n})}{\prod_{j\neq i}(x_{i}-x_{j})}}_{P(x_{1},\ldots,x_{n})}\]
evaluated at $(x_{1},\ldots,x_{n})=(\beta_{1},\ldots,\beta_{n})$.
Consider $P$. Note $P$ is zero whenever $x_{i}=x_{j}$ for some
$i\ne j$ (All except two terms are 0; those two cancel.). So $x_{i}-x_{j}\mid P$,
and $D\mid P$. However, $P$ has degree less than $\frac{(n-1)n}{2}$
when $k<n-1$, so must be 0. If $k=n-1$ then we know $P$ is a constant
multiple of $D$, look at the coefficient of any term to see that
in fact $P=D$.
\end{proof}
It suffices to prove
\[
(f_{\be}'(\be)^{-1})=B^*:=\set{b\in L}{\tr_{L/K}(bb')\in A\text{ for all }b'\in B}.
\]
The condition inside is equivalent to 
\[
\tr(b\be^j)\in A,\quad 0\le j\le n-1.
\]
(because $B=\bigoplus A\be^i$.)
But by the lemma,
\[
\tr\pa{
\sum_{i=0}^{n-1} a_i \fc{\be^{i+j}}{f'(\be)}}
=a_{n-1-j}+\cdots(>).
\]
``Triangular."
Therefore
\[
B^*=\bigoplus A\cdot \fc{\be^i}{f_{\be}'(\be)}=\prc{f'_{\be}(\be)}.
\]
\end{proof}
Good exercise: Compute $\mD_{\Q_p(\ze_{p^n})/\Q_p}$. This is tamely ramified only at $n=1$. Totally ramified tower. The first step is $(\Z/p)^{\times}$, tame, everything else is $p$, wild.

(Note $G(\Q(\ze_{p^n})/\Q)\cong (\Z/p^n\Z)^{\times}$ because $D_{p}\cong G(\Q_p(\ze_{p^n})/\Q_p)$.
\section{Ramification groups}
Local, CDVR setup.

\begin{df}
Let $i\ge -1$. The $i$th \textbf{ramification group} is
\begin{align*}
G_i&=\set{\si\in G}{b\in \sO_L,\,v_L(\si(b)-b)\ge i+1}\\
&=\set{\si\in G}{v_L(\si(\be)-\be)\ge i+1}.
\end{align*}
\end{df}
Observe $G_{-1}=G$, that $G_i\supeq G_j$ for $i\le j$ and $\bigcap_{i\ge -1}G_i=\{1\}$. Also note for all $i$, $G_i$ is a normal subgroup of $G$ because
\[
G_i=\ker(G\rightarrow \Aut(\sO_L/(\pi_L^{i+1})).
\]
In particular, $G_i/G_{i+1}$ is a group. Furthermore $G_i$ is defined even when $i$ is not an integer; we have $G_i=G_{\ce{i}}$. 

We will study $\{G_i\}_{i\ge -1}$. We want
\begin{enumerate}
\item
A formula for $v_L(\mD_{L/K})$. If at most tame, equals $e-1$, else greater.
\item
Look at quotients $G_i/G_{i+1}$. Abelian, cyclic, $p$-group, prime-to-$p$?
\end{enumerate}
\subsection{$\mD_{L/K}$ and $i_G$}
\begin{df}
Let $\si\in G(L/K)$. 
Define $i_G:G\to \N_0\cup \{\iy\}$ by
\[
i_G(\si)=\min\set{v_L(\si(\be)-\be)}{\be\in B}.
\]
\end{df}
Note that if $B=A[\be]$, then 
\[
i_G(\si)=v_L(\si(\be)-\be).
\]
Observe
\begin{itemize}
\item
$i_G(\si)=\iy$ iff $\si=1$.
\item
$G_i=\set{\si\in G}{i_G(\si)\ge i+1}$, so $\si\in G_i$ iff $i_G(\si)\ge i+1$, so doesn't depend on choice of generator.
\end{itemize}
Note 
\[
i_G(\tau\si\tau^{-1})=i_G(\si),\quad \si,\tau\in G.
\]
Because $G_i\unlhd G$. Note
\[
i_G(\si\tau)\ge \min(i_G(\si),i_G(\tau)).
\]
Because
\begin{align*}
i_G(\si\tau)&=v_L(\si\tau\be-\be)\\
&\ge \min(v_L(\si\tau(\be)-\tau(\be)),v_L(\tau(\be)-\be))\\
=\min(i_G(\si),i_G(\tau)).
\end{align*}
since $\sO_L=\sO_K[\be]=\sO_K[\tau\be]$.
\begin{pr}
\[
v(\mD_{L/K})=\sum_{\si\ne 1} i_G(\si)=\sum_{i\ge 0}(|G_i|-1).
\]
($\ma=(\pi_L^i)\implies v_L(a)=i$.)
\end{pr}
\begin{proof}
Let $f(x)$ be the minimal polynomial for $\be$. Letting $n=[L:K]$, 
\[
f(x)=\prod_{i=1}^n(X-\be_i).
\]
Now 
\begin{align*}
\mD_{L/K}&=(f'(\be))\\
&=\prod_{i>1}(\be-\be_i)\\
&=\prod_{\si\ne 1}(\be-\si(\be)).
\end{align*}
Take $v_L$ for (1).
\[
v_L(\mD_{L/K})=\sum_{\si\ne 1}\underbrace{v_L(\be-\si(\be))}_{i_G(\si)}.
\]
For (2), consider multiset
\[
\bigsqcup_{i\ge 0} (G_i\bs \{1\}).
\]
finite. Note $\si\in G$ appears in $G_0,G_1,\ldots, G_{i_G(\si)-1}$, there's $i_G(\si)$. Compute the size of the multiset in two different ways
\[
\sum_{i\ge 0} (|G_i|-1)=\sum_{\si\ne 1} i_G(\si).
\]
\end{proof}
Remark: $v_L(\mD_{L/K})=e-1$ iff $G_1=\{1\}$ (because $|G_0|=e$, iff $L/K$ is at most tame.

Let's understand $\mD_{L/K}$, $i_G$ under sub and quotient group. Consider $L/L^H/K$.

First, sub. 
\begin{pr}\llabel{ram-grp-sub}
\begin{align*}
i_H(\si)&=i_G(\si)\text{ for all $\si\in H$}\\
H_i&=H\cap G_i.
\end{align*}
\end{pr}
\begin{proof}
Same generator works for larger ring.
$\sO_L=\sO_K[\be]\implies \sO_L=\sO_{K'}[\be]$. Then true by def.
\end{proof}
\begin{cor}
\[
v_L(\mD_{L/K'})=\sum_{\si\ne 1,\si\in H} \underbrace{i_G(\si)}_{i_H(\si)}.
\]
\end{cor}
For quotient.
\begin{pr}\llabel{igh-ig}
For $H\unlhd G$, $\ol G\ne 1\in G/H$,
\[
i_{G/H}(\ol{\si})=\rc{e_{L/K'}}\sum_{\si\in G,\,\si\bmod H=\ol{\si}} i_G(\si).
\]
\end{pr}
\begin{cor}
\[
v_{K'/K}(\mD_{K'/K})=\rc{e_{L/K}'} \sum_{\si\nin H,\si\in G}i_G(\si).
\]
\end{cor}
Because by prev. $\sum_{\ol{\si}\ne 1} i_{G/H}(\ol{\si})$ equals RHS by prop.
\begin{proof}
Choose $\al\in \sO_{K'}$ and $\be\in \sO_L$ such that $\sO_{K'}=\sO_K[\al']$ and $\sO_L=\sO_K[\be]$. Then
\begin{align*}
e_{L/K'}i_{G/H}(\ol{\si})&=e_{L/K'}v_{K'}(\ol{\si}\al'-\al')\\
&=v_L(\ol{\si}\al'-\al').\\
\sum_{\si\in G}&=i_G(\si)\\
&=\sum_{\tau\in H} \underbrace{i_G(\si\tau)}_{v_L(\si\tau\be-\be)}\\
&=v_L\pa{\prod_{\tau\in H} (\si\tau (\be)-\be)}=before.
\end{align*}
fixing $\si$.

It suffices to prove $(\si\ol{\al'}-\al')=\prod_{\tau\in H}(\si\tau(\be)-\be)$.
Call LHS, RHS $\ma,\mb$.
\begin{enumerate}
\item
$\ma\mid \mb$: Consider
\[
g(X)=\prod_{\tau\in H}(X-\tau(\be))\in \sO_{K'}[x].
\]
minimal polynomial of $\be/K'$. 
\[
\si g(X)=\prod_{\tau\in H} (X-\si\tau(\be)).
\]
Observe $\si\al'-\al'$ divides coefficients of $\si g(X)-g(X)$. Because for all $a\in \sO_{K'}$, $a=a_0+a_1\al'$, $\si\al= a_0+a_1\si \al'+\cdots $. Note $\si\al'-\al'\mid \si\al'^i-\al'^i$. Note $g(\be)=0$. Take $x=\be$ to get
\[
\si\al'-\al'\mid \si g(\be)-\underbrace{g(\be)}_0.
\]
\item %Min poly div here
$\mb\mid \ma$. 
SWITCH $f$ and $g$ below.
Cook up a minimal polynomial to show divisibility. $\al'=\sO_K[\be]=\sO_L$. Write
\[
\al'=\sum_{i=0}^{n-1} a_i\be^i=:g(\be).
\]
$a_i\in \sO_K$. $g(X)\in \sO_K[X]$. Consider $g(X)-\al'\in \sO_{K'}[X]$. By construction has $\be$ as a root.

Hence plugging in $x=\be$,
\begin{align*}
f(X)\mid g(X)-\al\\
\si f(X)\mid \si(f(X))-\si(\al')\\
\si f(\be)\mid f(\be)-\si(\al')
\end{align*}
giving $\pm b\mid \pm a$.
\end{enumerate}
\end{proof}
{\color{red} End of unedited stuff.}
\subsection{Filtration of ramification groups}
\llabel{sec:ram-filt}
We know from (??) that
\[
G_{-1}/G_0\cong G/I_{L/K}\cong G(l/k).
\]
In particular, if $k$ is finite then $G_{-1}/G_0$ is finite cyclic and if $k=\ol k$ then $G_{-1}/G_0$ is trivial. 
%\Q[[x]] full Galois group in \Q...
From now on assume $i\ge 0$.

We aim to study the filtration 
\begin{equation}\llabel{ramification-filtration-eq}
G\supeq G_0\supeq G_1\supeq \cdots.
\end{equation}
To do this, we first study the filtration
\begin{equation}\llabel{units-filtration-eq}
L^{\times}\supeq U_L^0 \supeq U_L^1\supeq\cdots
\end{equation}
where 
\[U_L^i=
\begin{cases}
\sO_L^{\times},&i=0\\
1+\pi_L^i\sO_L,&i\ge 1.
\end{cases}
\]
The quotient groups in~(\ref{units-filtration-eq}) can be understood explicitly (Proposition~\ref{units-filtration}). We will relate the two filtrations by Proposition~\ref{ramification-to-unit-group}.\footnote{This will be important in local class field theory, which says there is a canonical isomorphism $K^{\times}/\nm_{L/K}(L^{\times})\cong G(L/K)$ if $L/K$ is finite abelian.} From this we get several important corollaries about the structure of the groups $G_s$. Understanding conjugates and commutators of elements in the $G_s$ gives us several more important properties.
%Would be true if use ``upper numbering."

%Here, relate $G_i/G_{i+1}$ to $U_L^i/U_L^{i+1}$. Just injection. Galois filtration hard to understand but $U_L$'s easy.
\begin{pr}\llabel{units-filtration}
Let $K$ be a complete field with discrete valuation (for instance, a local field), $k$ its residue field, and $\mm$ the associated maximal ideal. %, and $U_K^{(m)}:=1+\mm^m$. 
Then we have isomorphisms
\begin{align*}
U_K/U_K^{(1)}&\xra{\cong} k^{\times}&
U_K^{(m)}/U_K^{(m+1)}&\xra{\cong} k^+\\
u&\mapsto u\pmod{\mm}&
1+a\pi^m&\mapsto a\pmod{\mm}.
\end{align*}
\end{pr}
\begin{proof}
For the first just note that $1+\mm$ is the multiplicative unit of $A/\mm$. For the second, note $(1+a\pi^m)(1+b\pi^m)=1+(a+b)\pi^m+\cdots$.
%The second is composition $U_K^i/U_K^{i+1}\cong (\pi_K^i)/(\pi_K^{i+1})\cong \sO_K/(\pi_K)=k$, $(1+a)U_L^{i+1}$ to $a+(\pi_L^{i+1})$, $\pi_L^ia+(\pi_L^{i+1})$ to $a+(\pi_L)$.
\end{proof}
%%%%%lecture 11/15
To construct a map $G_i/G_{i+1}\to U_L^i/U_L^{i+1}$, we first need the following characterization of $G_i$. 

\begin{lem}\llabel{gi-quotient-crit}
Suppose $L/K$ is a finite Galois extension of local fields, $\pi$ is a uniformizer of $L$, and $G=G(L/K)$. 
For $i\in \N_0$ and $\si\in G_0$,
\begin{equation}\llabel{gi-quotient-crit-eq}
\si\in G_i\iff \fc{\si(\pi)}{\pi}\equiv 1\pmod{\pi_L^i}.
\end{equation}
\end{lem}
\begin{proof}
The RHS is equivalent to 
\begin{equation}\llabel{gi-quotient-crit1}
\si(\pi)-\pi\equiv 0\pmod{\pi_L^{i+1}}.
\end{equation}
We need to show this is equivalent to
\begin{equation}\llabel{gi-quotient-crit2}
\si(\be)-\be\equiv 0\pmod{\pi_L^{i+1}}\text{ for all }\be\in L.
\end{equation}
It is clear that~(\ref{gi-quotient-crit2}) implies~(\ref{gi-quotient-crit1}).

First suppose $L/K$ is totally ramified. Then $\sO_L=\sO_K[\pi]$ by Proposition~\ref{monogenous}, giving that~(\ref{gi-quotient-crit1}) implies~(\ref{gi-quotient-crit2}).

Now consider the general case. We know $L/L^{I_{L/K}}$ is totally ramified (Theorem~\ref{decomposition-and-inertia}), so the theorem holds for $L/L^{I_{L/K}}$.
Now, by Proposition~\ref{ram-grp-sub},
 $G_i(L/L^{I_{L/K}})=G_i\cap I_{L/K}=G_i$. Furthermore, since $\pi_L$ is the same for $L/K$ and $L/L^{I_{L/K}}$, the right hand-side of~(\ref{gi-quotient-crit-eq}) does not change whether we are talking about $L/K$ or $L/L^{I_{L/K}}$. Hence the theorem for $L/L^{I_{L/K}}$ implies the theorem for $L/K$.
\end{proof}
\begin{pr}\llabel{ramification-to-unit-group}
There is a well-defined injective group homomorphism
\begin{align*}
\te_i:G_i/G_{i+1}&\hra U_L^i/U_L^{i+1}\\
\si&\mapsto \fc{\si(\pi)}{\pi}
\end{align*}
that is independent of the choice of uniformizer $\pi$.
\end{pr}
\begin{proof}
Note that 
\begin{equation}\llabel{ram-unit1}
u\in \sO_L,\,\si\in G_i\implies\si(u)\equiv u\pmod{\pi^{i+1}}\implies \fc{\si(u)}{u}\in U_L^{i+1}.
\end{equation}

First we show $\te_i$ is a group homomorphism $G_i\to U_L^i/U_L^{i+1}$. We have
\[
\fc{\si\tau(\pi)}{\pi}
= \fc{\si(\pi)}{\pi}\cdot \fc{\tau(\pi)}{\pi}\cdot \fc{\si\pf{\tau(\pi)}{\pi}}{\fc{\tau(\pi)}{\pi}}.
\]
Since $\fc{\tau(\pi)}{\pi}\in \sO_L$ and $\tau\in G_i$,~(\ref{ram-unit1}) gives $\fc{\si\pf{\tau(\pi)}{\pi}}{\fc{\tau(\pi)}{\pi}}\in U_L^{i+1}$. 

Lemma~\ref{gi-quotient-crit} gives that the kernel is exactly $G_{i+1}$, so $\te_i$ induces an injective map $G_i/G_{i+1}\to U_L^i/U_L^{i+1}$.

Now suppose $\pi'$ is another uniformizer. Write $\pi'=u\pi$ with $u\in\sO_L^{\times}$. Then $\si\in G_i$ and~(\ref{gi-quotient-crit}) give
\[
\fc{\si(\pi')}{\pi'} = \fc{\si(\pi)}{\pi} \cdot \underbrace{\fc{\si(u)}{u}}_{\in U_L^{i+1}}.\qedhere
\]
\end{proof}
\begin{cor}
\begin{enumerate}
\item
$G_0/G_1$ is finite cyclic.
\item
If $\chr(l)=0$ then $G_1=\{1\}$; if $\chr(l)=p\ne 0$, then for each $i\ge 1$, \[G_i/G_{i+1}=(\Z/p\Z)^{n_i}\]
for some $n_i$.
%$, $i\ge 1$ is a product of $(\Z/p\Z)$'s.
\end{enumerate}
\end{cor}
\begin{proof}
\begin{enumerate}
\item
Proposition~\ref{ramification-to-unit-group} and~\ref{units-filtration} give $G_0/G_1\hra U_L/U_L^1\cong  l^{\times}$. But any finite subgroup of a finite field must be cyclic.
\item 
For $\chr(l)=0$, $l^{+}$ has no finite nontrivial subgroup. %$G_i/G_{i+1}=\{1\}$. 
For $\chr(l)=p$, we have $G_i/G_{i+1}\hra U_L^i/U_L^{i+1}\cong l^{+}$. Just note $l^+$ is a an abelian $p$-group.\qedhere
 %$\F_p$-vector space.
%l not nec finite
%any extension tamely ramified when char(l)=0, no wild ramification. Geometry on complex curves, never see wild ramification. Example $\C[[t]]$.
\end{enumerate}
\end{proof}
%%PUT STUFF ON TAME/WILD RAM. GRP.
\begin{cor}\llabel{cor:G(local)=solvable}
$G_0=I_{L/K}$ is solvable. If $G(l/k)=G_{-1}/G_0$ is solvable (in particular, if $k$ is finite) then $G$ is solvable. 
%Far from case in number fields! Galois theory easier.
\end{cor}
\begin{proof}
The series
\[
G_0\supeq G_1\supeq \cdots
\]
is a solvable series for $G$.
\end{proof}
\subsection{First ramification group}
Recall that we defined $G_0=I_{L/K}$ so that we can split $L/K$ into two parts: $L/L^{I_{L/K}}$ is totally ramified while $L^{I_{L/K}}/L$ is unramified. We can further split the extension $L/L^{I_{L/K}}$ into a wildly ramified and tamely ramified part.
\begin{df}
Define the \textbf{wild inertia group} and \textbf{tame inertia group} to be
\begin{align*}
%G_0&=I_{L/K}&\text{inertia group}\\
G_1&=I_{L/K}^{\text{wild}}\\%&\text{wild inertia group}\\
G_0/G_1&=I_{L/K}^{\text{tame}}.%&\text{tame inertia group.}
\end{align*}
\end{df}
\begin{thm}
The extension $L/L^{I_{L/K}^{\text{wild}}}$ is wildly ramified with Galois group $G_1=I_{L/K}^{\text{wild}}$ and the extension $L^{I_{L/K}^{\text{wild}}}/L^{I_{L/K}^{\text{tame}}}$ is tamely ramified with Galois group $G_1/G_0$.
\[
\xymatrix{
L \ls{d}\ar@{.}@/^1pc/[d]^{I_{L/K}^{\text{wild}}}
 \ar@{.}@/_1pc/[d]_{\text{wild ramification}}\\
L^{I_{L/K}^{\text{wild}}}\ls{d}
\ar@{.}@/^1pc/[d]^{I_{L/K}^{\text{tame}}}
\ar@{.}@/_1pc/[d]_{\text{tame ramification}}\\
L^{I_{L/K}} \ls{d}
\ar@{.}@/^1pc/[d]^{G/I_{L/K}=G(l/k)}
\ar@{.}@/_1pc/[d]_{\text{unramified}}\\
K
}
\]
Moreover, $G_1$ is the unique p-Sylow subgroup of $G_0$, and
%\end{thm}
%\begin{cor}
%Suppose $\chr(k)=p\ne 0$. Then the exact sequence
%\[
%\xymatrix{
%1\ar[r] & G_1\ar[r]\ar@{=}[d]& G_0\ar[r]\ar@{=}[d]& G_0/G_1\ar[r]\ar@{=}[d]& 1\\
%& I_{L/K}^{\text{wild}} & I_{L/K} & I_{L/K}^{\text{tame}}
%}
%\]
%splits. %if $\chr(k)=p\ne 0$. Hence 
%Hence
\[
G_0=G_1 \rtimes  G_0/G_1.
\]
%Note the first is a $p$-group and the latter is cyclic and prime-to-$p$.
%warning nor dir product
\end{thm}
\begin{proof}
Note $G_0/G_1\hra k^{\times}$ while $G_j/G_{j+1}\hra k^+$ for $j\ge 1$; we have $p\nmid |k^{\times}|$ while $|k|$ is a power of $p$; and $|G_1|=\prod_{1\le j\ll \iy} |G_j/G_{j+1}|$. Hence $G_1$ is a p-SSG of $G_0$; it is unique since it is normal and all $p$-SSGs are conjugate. Since the indices of the field extensions are the orders of the Galois groups, the result on tame and wild ramification follow.

Now we prove the semidirect product.
This follows directly from the Schur-Zassenhaus Lemma: If $H$ is a normal Hall subgroup of a finite group $G$, then $H$ has a complement, and hence $G=H\rtimes G/H$. (A Hall subgroup $H\subeq G$ is a group such that $\gcd(|H|,[G:H])=1$.)

The following is an alternate proof. We show the exact sequence
\[
\xymatrix{
1\ar[r] & G_1\ar[r]\ar@{=}[d]& G_0\ar[r]\ar@{=}[d]& G_0/G_1\ar[r]\ar@{=}[d]& 1\\
& I_{L/K}^{\text{wild}} & I_{L/K} & I_{L/K}^{\text{tame}}
}
\]
splits by showing there exists a right inverse $G_0/G_1\to G_0$ of the projection $G_0\to G_1$.\footnote{The image of $G_0/G_1$ is a {\it complement} $Q$ of $G_1$ in $G_0$; the elements of $Q$ act on $G_1$ by conjugation---this is what the semidirect product means.}
Since $G_0/G_1$ is cyclic of order $r:=|l^{\times}|$, it suffices to find a lift $\si\in G_0$ of the generator $\ol{\si}\in G_0/G_1$ with order $r$. Write $|G_0|=p^sr$. Let
\[
\si=\si'^{p^{\ph(r)t}}
\]
where $t$ is such that $\ph(r)t\ge s$. Note $r\nmid p$ implies $p^{\ph(r)t}\equiv 1\pmod{r}$. Since $\si'^r\in G_1$, this implies $\si$ is still a lift of $\ol{\si}$. Moreover $\ph(r)t\ge s$ gives that its order is $r$, so it is the desired lift.
\end{proof}
%The action of the semidirect product is given by the following.
\begin{pr}\llabel{conjugate-rg}
For $i\ge 1$, $\si\in G_0$, $\tau\in G_i/G_{i+1}$,
\[
\te_i(\si\tau\si^{-1})=\te_{0}(\si)^i\te_i(\tau).
\]
\end{pr}
(Here $\te_0(\si)^i$ is thought of as in $U_L/U_L^1\cong l^{\times}$, and $\te_i(\tau)\in U_L^i/U_L^{i+1}\cong l^+$.)
\begin{proof}
%Note $\te_0(\si)^i\in \sO_L^{\times}/U_L^1\cong l^{\times}$, acts on the one-dimensional vector space $l$, second factor in $l^{\times}$.
It is slightly more convenient to work additively rather than multiplicatively, so we consider
\begin{align*}
\te_i':G_i/G_{i+1}&\hra U_L^i/U_L^{i+1}\cong (\pi^i)/(\pi^{i+1})\\
\si&\mapsto \fc{\si(\pi)}{\pi} \mapsto 
\begin{cases}
\fc{\si(\pi)}{\pi},& i=0\\
\fc{\si(\pi)}{\pi}-1,& i\ge 1,
\end{cases}
\end{align*}
where $\pi$ is any uniformizer.

%Using this recasting, we need to show
%\[
%\fc{\si\tau\si^{-1}(\pi)}{\pi}-1\equiv \pf{\si(\pi)}{\pi}^i \pa{\fc{\tau(\pi')}{\pi'}-1}\pmod{\pi_L^{i+1}}.
%\]
%where $\pi'=\si^{-1}(\pi)$. $\tau\in G_i$. Write $\tau(\pi')=\pi'+a$ where $a\in (\pi^{i+1})$.
%
%The LHS equals 
%\[\fc{\si\tau(\pi')}{\pi}-1
%=\fc{\cancel{\si(\pi')}+\si(a)-\cancel{\pi}}{\pi}
%=\fc{\si(a)}{\pi}.
%\]
%The RHS equals
%\[
%\fc{\si(\pi)}{\pi}\pa{\fc{\pi'+a}{\pi'}-1}=\fc{\si(\pi)}{\pi}\fc{a}{\pi'}.
%\]
%Write $a=\pi^i\pi'a'$ where $a\in \sO_L$. The LHS minus RHS is
%\[
%\si(\pi)^j \si(a') - \si(\pi)^ia'
%= \si(\pi)^i \underbrace{(\si(a')-a')}_{\in (\pi)}\in (\pi)^{i+1}
%\] 
%because $\si\in G_0$.
%\end{proof}
%More group theory exercises!
%\begin{cor}
%\[
%\si\tau\si^{-1}\tau^{-1}\in G_{i+1}\iff \si^i\in G_1\text{ or }\tau\in G_{i+1}.
%\]
Define 
\[
\pi'=\si^{-1}(\pi)
\]
and let $a\in \sO_L^{\times}$ be such that
\[
\tau(\pi')=\pi'+a\pi'\pi^i.
\]
Note that
\[
\te_i'(\tau)=\fc{\tau(\pi')}{\pi'}=a\pi^i.
\]

Now we calculate, modulo $(\pi)^{i+1}$, that
%\begin{align*}
%\si^{-1}(\pi)&=\pi'\\
%\tau(\pi')&=\pi'+\pi'\pi
%\end{align*}
\begin{align*}
\te_i'(\si\tau\si^{-1})
&=\fc{\si\tau\si^{-1}(\pi)}{\pi}-1\\
&=\fc{\si\tau(\pi')}{\pi}-1\\
&=\fc{\si(\pi'+a\pi'\pi^i)}{\pi}-1\\
&=\fc{\pi+\si(a\pi'\pi^i)}{\pi}-1\\
&=\fc{a\si(\pi'\pi^i)}{\si(\pi')}&\text{since }\si(a)\equiv a\pmod{\pi^{i+1}}\\
&=\pf{\si(\pi)}{\pi}^ia\pi^i\\
&=\te_0'(\si)^i\te_i'(\tau).&\qedhere
\end{align*}
\end{proof}
\begin{pr}
If $\si\in G_i$ and $\tau\in G_j$, $i,j\ge 1$, then 
\[
\si\tau\si^{-1}\tau^{-1}\in G_{i+j+1}.
\]
\end{pr}
\begin{proof}
\end{proof}
\begin{cor}
For $i\ge 1$,
\[
\si\tau\si^{-1}\tau^{-1}\in G_{i+1}\iff \si^i\in G_1\text{ or }\tau\in G_{i+1}.
\]
\end{cor}
\begin{proof}
We have
\begin{align*}
\si\tau\si^{-1}\tau^{-1}\in G_{i+1}
&\iff  \si\tau\si^{-1} = \tau& \text{ in }G_i/G_{i+1}\\
&\iff  \te_i'(\si\tau\si^{-1}) = \te_i' (\tau) &\text{ in }(\pi^i)/(\pi^{i+1})\\
&\iff  \te_i'(\tau)%\underbrace{
(\te_0'(\si)^i-1)
%}_{\in l}
=0
&\text{by Proposition~\ref{conjugate-rg}}\\
&\iff \te_i'(\tau)=0\text{ or }\te_0'(\si^1)=1\\
&\iff \tau\in G_{i+1}\text{ or }\si^i\in G_1.
\end{align*}
%(pi^i)/... 1-dim l-vspace
\end{proof}
\begin{cor}\llabel{jump-multiple-go/g1}
Suppose $G$ is abelian and $|G_0/G_1|\nmid i$. Then $G_i=G_{i+1}$.
\end{cor}
%Where do jumps occurs? Strict inclusion in filtration. 
%For example, for $L/K=\Q_p(\ze_{p^m})/\Q_p$, $|G_0/G_1|=p-1$, $G_i\supset G_{i+1}$ only when $p-1\mid i$. Fact, happens when $i=p-1,p^2-1,\ldots, p^{m-1}-1$.
\begin{proof}
Write $G_0/G_1=\an{\ol{\si}}$ where $r=|G_0/G_1|$. Since $r\nmid i$, $\ol{\si}^i\ne 1$; for any lift $\si\in G_0$ of $\ol{\si}$, $\si^i\nin G_1$. Since $G$ is abelian, we get for all $\tau\in G_i$, $\si\tau\si^{-1}\tau^{-1}=1$. By the previous corollary, noting $\si^i\nin G_1$, we must have $\tau\in G_{i+1}$.
\end{proof}
\begin{df}
A \textbf{jump} for $L/K$ is an integer $i$ such that
\[
G_i\ne G_{i+1}.
\]
\end{df}
Corollary~\ref{jump-multiple-go/g1} tells us that jumps are divisible by $|G_0/G_1|$.
\section{Herbrand's Theorem}
\subsection{Functions $\ph$ and $\psi$}
Note that ramification groups behave nicely under taking subgroups (i.e. passing from $M/K$ to $M/L$), by Proposition~\ref{ram-grp-sub}.
However, the indices are screwed up when passing to quotient groups (i.e. passing from $M/K$ to $M/L$). We calculate exactly how the index changes (Herbrand's Theorem~\ref{herbrand}), and use it to define a different numbering scheme that is invariant under passing to quotient groups.

It is important to know how ramification groups behave under quotients because this gives a compatible system that allows us to look at larger and larger field extensions, i.e. pass to the inverse limit.
\begin{df}
%For $u\in \R_{\ge -1}$, define 
%\[
%G_u=\set{\si\in G(L/K)}{i_G(\si)\ge u+1}.
%\]
%Note $G_u=G_{\ce{u}}$.
Define $\ph_{L/K}:\R_{\ge 0}\to \R_{\ge 0}$ by
\[\ph_{L/K}(u)=\int_0^u \rc{[G_0:G_t]}dt\]
(recall $G_u=G_{\ce{u}}$) and extend $\ph_{L/K}$ to $\R_{\ge -1}\to \R_{\ge -1}$ by
\[\ph(u)=u, \quad -1\le u\le 0.\]
\end{df}
This is a piecewise linear increasing function with $\ph_{L/K}(-1)=-1$ and with derivative at least $\rc{|G_0|}$, so it is a bijection.
\begin{df}
Define $\psi_{L/K}:\R_{-1}\to \R_{-1}$ by $\psi_{L/K}=\ph_{L/K}^{-1}$. Define the \textbf{upper numbering} filtration by
\[
G^v :=G_{\psi_{L/K}(v)},\quad v\ge -1.
\]
\end{df}
%This is better for quotients.
%{\color{red} More unedited stuff.}
\subsection{Transitivity of $\ph$ and $\psi$}
%Setup $L/K'/K$. Everything finite Galois with separable residue field extension. We want to relate $\ph_{L/K}$ with $\ph_{L/K'}$ and $\ph_{K'/K}$and do the same with $\psi$.
The function $\ph_{L/K}$ gives the reindexing when we pass to the quotient Galois group.
\begin{thm}[Herbrand's Theorem]\llabel{herbrand-thm}
Let $L/K'/K$ be finite Galois extension with separable residue field extension.
For all $u\ge -1$, 
\[
G_uH/H=(G/H)_{\ph_{L/K'}(u)}.
\]
\end{thm}
Here, $G_u$ is the ramification group of $L/K$ and $(G/H)_{\ph_{L/K'}(u)}$ is the ramification group of $K'/K$. 

We will need several lemmas. First we relate the function $i_{G/H}(\ol{\si})$ and $i_G$ evaluated at the lifts of $\ol{\si}$ in $G$. 
\begin{lem}\llabel{igh-ig2}
For $\ol{\si}\in G/H$, $j(\ol{\si})=\max_{\si \in \ol{\si}H} i_G(\si)$,
\[
i_{G/H}(\ol{\si})-1=\ph_{L/K'} (j(\ol{\si})-1).
\]
\end{lem}
Thus applying $\ph_{L/K}$ has the effect of ``turning" $i_G$ into $i_{G/H}$. By writing out the criterion for $\si\in G_u$ or $(G/H)_u$ in terms of $i_G$ and $i_{G/H}$, respectively, we will get Herbrand's Theorem.
\begin{proof}
Pick $\si_0\in G$ mapping to $\ol{\si}$ such that $i_G(\si_0)=j(\ol{\si})$. Then by Proposition~\ref{igh-ig},
\begin{equation}\llabel{herbrand-pf1}
i_{G/H}(\ol{\si}) 
=\rc{e_{L/K'}}\sum_{\si \in \ol{\si}H} i_G(\si)
=\rc{e_{L/K'}}\sum_{\tau\in H} i_G(\si_0\tau).
\end{equation}
%We want to relate the right-hand side to $j(\ol{\si})$. Pick $\si_0\in G$ mapping to $\ol{\si}$ such that $i_G(\si_0)=j(\ol{\si})$. Then the right-hand side is
%\[
%\rc{e_{L/K'}}\sum_{\tau\in H} i_G(\si_0\tau).
%\]

We claim that
\[
i_G(\si_0\tau)=\min(i_G(\si_0),i_G(\tau))=\min(j(\ol{\si}), i_G(\tau))
\]
for all $\tau\in H$. Indeed, by the nonarchimedean inequality,
\[
i_G(\si_0\tau)=v_L(\si_0\tau(\be)-\be)\ge
\min(v_L(\si_0\tau(\be)-\tau(\be)),v_L(\tau(\be)-\be))= \min(i_G(\si_0),i_G(\tau)).
\]
Consider two cases.
\begin{enumerate}
\item
$i_G(\tau)=i_H(\tau)\ge i_G(\si_0)$. The above gives
\[
i_G(\si_0\tau)\ge \min(i_G(\si_0),i_G(\tau))\ge i_G(\si_0).
\]
Equality holds by the maximality assumption on $\si_0$.
\item
%If $\tau\nin H_{i_G(\si_0)-1}$ then 
$i_G(\tau)<i_G(\si_0)$. Then \[i_G(\si_0\tau)=\min(i_G(\si_0),i_G(\tau))=i_G(\tau).\]
\end{enumerate}
The RHS of~(\ref{herbrand-pf1}) then equals $\rc{e_{L/K'}}\sum_{\tau\in H} \min(i_G(\si_0),i_G(\tau))$; 
the result then follows from the next lemma.
\end{proof}
%We have $\tau\in H_{i_G(\si_0)-1}$ implies $i_G(\tau)=i_H(\tau)\ge i_G(\si_0)$. Hence by the nonarchimedean inequality,
%\[
%i_G(\si_0\tau)\ge \min(i_G(\si_0),i_G(\tau))\ge i_G(\si_0)
%\]
%since
%%%
%But maximality of $\si_0$ tells us $i_G(\si_0\tau)=i_{G}(\si_0)$.
\begin{lem}
\[
\ph_{L/K}(u)=\rc{e_{L/K}}\sum_{\si\in G} \min(j(\ol{\si}),u+1)-1.
\]
\end{lem}
\begin{proof}
Since both sides are piecewise linear functions, and both sides equal $u$ for $-1\le u\le 0$, it suffices to show their derivatives (slopes) are equal for $u>0$.

If $i-1<u<i$ where $i\in \N$, then the slope of the LHS is $\rc{[G:G_i]}$. For the RHS, since $i_G(\si)$ is an integer, each term is either $i_G(\si)$ or $u+1$; each term where $u+1$ is the minimum contributes to the slope. Hence the slope on the RHS is 
\[
\rc{e_{L/K}}\ab{
\set{\si\in G}{u+1<i_G(\si)}
}=\rc{e_{L/K}}\ab{
\set{\si\in G}{i_G(\si)\ge i+1}
}=\fc{|G_i|}{e_{L/K}}=\rc{[G_0:G_i]},
\]
as needed.
\end{proof}
%we get, by lemma applied to $L/K'$ with $H$,
%\[
%\ph_{L/K'}(i_G(\si)-1)+1.
%\]
%\end{proof}
\begin{proof}[Proof of Theorem~\ref{herbrand-thm}]
We have the following string of equivalences.
\begin{enumerate}
\item $\ol{\si}\in G_uH/H=G_u/G_u\cap H$
\item There is $\si\in G$ lifting $\ol{\si}$ so that $\si\in G_u$.
\item $j_G(\ol{\si})-1\ge u$.
\item $\ph_{L/K'}(j_G(\ol{\si})-1)\ge\ph_{L/K'}(u)$.
\item $i_{G/H}(\ol{\si})-1\ge\ph_{L/K'}(u)$.
\item $\ol{\si} \in (G/H)_{\ph_{L/K'}(u)}$.
\end{enumerate}
We have $(3)\iff(4)$ because $\ph_{L/K'}$ is monotonically increasing and $(4)\iff (5)$ by Lemma~\ref{igh-ig2}.
%Suppose $\ol{\si}$
%\[
%\ol{\si}\in G_uH/H=G_u/G_u\cap H.
%\]
%iff there is $\si\in G$ lifting $\ol{\si}$ such that $\si\in G_u$ (equivalently, $i_G(\si)\ge u$), iff $j_G(\ol{\si})-1\ge u$, iff (applying monotonically increasing $\ph_{L/K'}$), 
%$\ph_{L/K'}(j_G(\ol{\si})-1)\ge\ph_{L/K'}(u)$, iff (claim 1) $i_{G/H}(\ol{\si}-1) \ge \ph_{L/K'}(u)$, iff $\ol{\si} \in (G/H)_{\ph_{L/K'}(u)}$.
\end{proof}
Now we prove transitivity for $\ph$ and $\psi$. 
\begin{pr}\llabel{phi-psi-trans}
\begin{align*}
\ph_{L/K}&=\ph_{K'/K}\circ \ph_{L/K'}\\
\psi_{L/K}&=\psi_{L/K'}\circ \psi_{K'/K}.
\end{align*}
\end{pr}
\begin{proof}
It suffices to prove the first equation; the first implies the second since $\ph$ and $\psi$ are inverse. For $-1\le u\le 0$ both sides equal $u$. Thus it suffices to show the derivatives of both sides are equal for $u\ge 0$. For $u\nin \Z$, the derivative on the LHS is
\[
\ph'_{L/K}(u)=\rc{[G_0:G_u]}.
\]
By the chain rule, the slope on the RHS is
\begin{align*}
\ph'_{K'/K}(\ph_{L/K'}(u))\ph'_{L/K'}(u)&=\fc{|(G/H)_{\ph_{L/K'}(u)}|}{|(G/H)_0|}\fc{|H_u|}{|H_0|}\\
&=\fc{|G_uH/H||H_u|}{e_{K'/K}e_{L/K'}}&\text{by Herbrand's Theorem~\ref{herbrand}}\\
&=\fc{|G_u/H\cap G_u||H_u|}{e_{L/K}}\\
&=\fc{|G_u|}{|G_0|}
\end{align*}
using $H\cap G_u=H_u$ (Proposition~\ref{ram-grp-sub}) and multiplicativity of ramification index. The derivatives are equal, as needed.
%lower numbering ok wrt subgroups
\end{proof}
Finally, we prove the most important consequence of Herbrand's Theorem: namely, by using the upper numbering (i.e. numbering using the inverse of $\ph_{L/K}$), quotients of ramification groups are preserved.
\begin{pr}
For all $v\ge -1$,
\[
G^vH/H=(G/H)^v.
\]
\end{pr}
\begin{proof}
By Herbrand's Theorem~\ref{herbrand} and transitivity of $\psi$ (Proposition~\ref{phi-psi-trans}) ($\psi_{L/K}=\psi_{L/K'}\circ \psi_{K'/K}$), we get
\begin{align*}
G^vH/H&=G_{\psi_{L/K}(v)}H/H\\
&=(G/H)_{\ph_{L/K'}(\psi_{L/K}(v))}%&\text{Herbrand's Theorem}\\
&=(G/H)_{\psi_{K'/K}(v)}=(G/H)^v.
\qedhere
\end{align*}
\end{proof}
We can now define upper numbering for infinite algebraic extensions $L/K$.
\begin{df}
Define
\[
G(L/K)^v:=\varprojlim_{K'/K\text{ finite Galois}} G(K'/K)^v.
\]
\end{df}
\section{Hasse-Arf Theorem}
We have two different filtrations, the lower numbering filtration $\{G_u\}_{u\ge -1}$ and $\{G^v\}_{v\ge -1}$.
\begin{df}
A \textbf{jump} is $u$ such that $G_u\ne G_{u+\ep}$ or $v$ such that $G^v\ne G^{v+\ep}$.
\end{df}
They are the $x$ and $y$-coordinates of jump points, i.e. where the slope of $\ph$ changes.

Note a jump $u\in \Z$ since $G_u=G_{\ce{u}}$. Moreover, $u$ is a jump for the lower numbering iff $v=\ph_L/K(u)$ is a upper numbering, because $\ph,\psi$ are monotonically increasing.

\begin{thm}[Hasse-Arf Theorem]
If $G$ is finite abelian, then the jumps $v$ are integers.
\end{thm}
In the cyclotomic case, $G$ was abelian.
%deeper ramification, larger and larger denominators, less and less likely get an integer out of it.
\begin{rem}
There is a nonabelian example where $v\nin \Z$. (See HW.)
\end{rem}
We postpone the proof.
Applications.
\begin{enumerate}
\item
Used in local class field theory. 
\item
``Conductor of Galois representations" are in $\Z$, not just in $\Q$. Finite $L/K$, $G(L/K)\to \GL_n(\C)$. %conductor for elliptic curves
\end{enumerate}