\chapter{Introduction to Galois cohomology}
\llabel{galois-cohomology-ch}
\index{Galois cohomology}
We will apply group (co)homology as follows: Take a Galois extension $L/K$ and let $G:=G(L/K)$. Take as a $G$-module a multiplicative or additive subgroup $S$ of $L$. The special case that $G$ is cyclic will come up often, since if $L/K$ is an unramified extension of local fields, then $G$ is cyclic. Furthermore, the norm map $N_G$ has a natural interpretation:
\begin{enumerate}
\item
If $S\subeq L^{\times}$ then for $a\in S$,
\[
N_G(a)=\prod_{\si\in G}\si(a)=\nm_{L/K}(a).
\]
\item
If $S\subeq L^{+}$ then for $a\in S$,
\[
N_G(a)=\sum_{\si\in G}\si(a)=\tr_{L/K}(a).
\]
\end{enumerate}
In Section~\ref{kummer} we give an application to Kummer theory (characterizing certain abelian extensions $L$ of $K$ in terms of $L^{\times n}\cap K$). Kummer theory will allow us to  prove the linear independence of $n$th roots.

Finally, we give two interpretations of Galois cohomology groups.
\begin{enumerate}
\item
$H^1(G(L/K),\Aut(V))$ parameterizes algebraic structures defined over $K$ that become isomorphic in $L$ (Section~\ref{sec:nonabelian-galois-cohom}). This is called {\it descent}.
\item 
$H^2(G(L/K),L^{\times})$ parameterizes classes of $K$-algebras ``split" over $L$ (Section~\ref{brauer}), i.e. it is the {\it Brauer group}.
\end{enumerate}
\section{Basic results}\llabel{galois-cohomology}
We prove two fundamental theorems on the cohomology of $L^{\times}$ and $L^+$.
\index{Hilbert's Theorem 90}
\begin{thm}[Hilbert's Theorem 90]\llabel{h90} ($\dagger$)
Let $L/K$ be a Galois extension with Galois group $G$. Then
\[
H^1(G, L^{\times})=\{1\}.
\]
Moreover, if $G=\an{\si}$ is cyclic and $u\in L^{\times}$, then the following are equivalent.
\begin{enumerate}
\item
$\nm_{L/K} (u)=1$.
\item There exists $v\in L^{\times}$ such that $u=\si(v)v^{-1}$.
\end{enumerate}
\end{thm}
We will often abbreviate $H^1(G(L/K),L^{\times})$ as $H^1(L/K)$.
\begin{proof}
First suppose $G$ is finite. 
Let $c:G\to L^{\times}$ be a 1-cocycle; we have $c_{\si\tau}=\si(c_{\tau}) c_{\si}$. Consider the function
\[
b(e):=\sum_{\tau\in G} c_{\tau} \tau(e).
\]
By linear independence of the characters $\tau\in G$, $b$ is not identically zero; hence there exists $e\in L^{\times}$ so that $b(e)\ne 0$. Operating by $\si$ on both sides and using the cocycle condition gives
\beq{eq:h90pf}
\si(b(e))=\sum_{\tau\in G} \si(c_{\tau}) (\si\tau)(e)=\sum_{\tau\in G} c_{\si \tau} c_{\si^{-1}} (\si\tau)(e)=c_{\si}^{-1}b(e)
\eeq
and $c_{\si}=b(e)\si(b(e))^{-1}$, 
so $c$ is a coboundary.

The infinite case follows from the finite case and Theorem~\ref{group-hom-cohom}.\ref{profinite-lim2}.

For the second part, note that $H^1(G,L^{\times})=\ker(N)/\im(D)=0$ gives $\ker(N)=\im(D)$. Here $N$ is the norm map $\nm_{L/K}$ and $D$ is the map $\si-1$, i.e. the map $v\mapsto \frac{\si(v)}{v}$.
\end{proof}
Next we think of $L$ as an additive group.
\begin{thm}\llabel{h+0}
Let $L/K$ be a finite Galois extension. Then
\[
H^r(G,L^+)=0, \quad r>0.
\]
\end{thm}
\begin{proof}
From the normal basis theorem~\ref{galois}.\ref{nbt}, there exists $\al\in L$ such that $\set{\si\al}{\si\in G}$ is a basis for $L$ over $K$. We get that $K[G]\cong L$ as $G$-modules by the map
\[
\sum_{\si\in G} a_{\si} \si \mapsto \sum_{\si\in G} a_{\si} \si\al.
\]
Since $K[G]\cong \Ind_{\{1\}}^G(K)$,
\[
H^r(G,L^+)\cong H^r(\{1\},K)=0
\]
by Shapiro's Lemma~\ref{shapiro-lemma}.
\end{proof}

\index{Kummer theory}
\section{Kummer theory}\llabel{kummer}
We use Galois cohomology to prove the following.
\begin{thm}[Kummer theory]\llabel{kummer-theorem}
Suppose $K$ is a field containing a primitive $n$th root of 1. Then there is a bijection between
\begin{enumerate}
\item Finite abelian extensions of $K$ of exponent dividing $n$ (i.e. for any $\si$ in the Galois group $G(L/K)$, $\si^n=1$).
\item Subgroups of $K^{\times}$ containing $K^{\times n}$ as a subgroup of finite index (i.e. subgroups of $K^{\times}/K^{\times n}$).
\end{enumerate}
This correspondence is given by
\begin{align*}
L&\mapsto K^{\times }\cap L^{\times n}\\
K[B^{\frac{1}{n}}]&\mapsfrom B.\end{align*}
\end{thm}
Moreover,
\begin{equation}
\llabel{degree-order}
[L:K]=[K^{\times}\cap L^{\times n}:K^{\times}]
\end{equation}
(Note in the reverse map, which $n$th roots we take doesn't matter because $K$ contains $n$th roots of unity.)

In the course of proving this theorem, we will show the following useful proposition.
\begin{pr}\llabel{pr:kummer-char}
Let $K$ be a field containing a primitive $n$th root of 1 and $L/K$ an abelian extension with Galois group $G$. Then there is a natural isomorphism
\begin{align*}
K^{\times}\cap L^{\times n}/K^{\times n}&\cong H^1(G,\mu_n)=\Hom(G,\mu_n)\\
a&\mapsto \pa{\si \mapsto \fc{\si\pa{a^{\rc n}}}{a^{\rc n}}}.
\end{align*}
In particular, there is a natural isomorphism
\[
K^{\times}/K^{\times n} \cong H^1(G(K^s/K),\mu_n)=\Hom(G(K^s/K),\mu_n).
\]
\end{pr}
\begin{proof}
Let $G=G(L/K)$, and denote the forward map by $B(L)=K^{\times}\cap L^{\times n}$. The key step is showing that~\eqref{degree-order} holds; we do this by interpreting $K^{\times}\cap L^{\times n}$ as a 0th cohomology module.  
The inclusions $L\supeq K(B(L)^{\rc n})$ and $B(K(B^{\rc n}))\supeq B$ are easily seen to hold (Step 2), so~(\ref{degree-order}) will give that equality holds (Steps 3-4).\\

\noindent\underline{Step 1:} By Theorem~\ref{group-hom-cohom}.\ref{les-ext}, the short exact sequence of $G$-modules 
\[
1\to \mu_n \to L^{\times}\xrightarrow{x\mapsto x^n} L^{\times n}\to 1 
\]
induces the long exact sequence
\[
1\to H^0(G,\mu_n) \to H^0(G,L^{\times})\to H^0(G,L^{\times n})\to H^1(G, \mu_n)\to H^1(G,L^{\times})\to \cdots.
\]
We need not go further because Hilbert's Theorem 90 (Theorem~\ref{h90}) tells us
\[
H^1(G(L/K), L^{\times})=1.
\]
Next, note that $H^0(G, H)$ is simply the subgroup of $H$ fixed by $G$, and that the subfield of $L$ fixed by $G$ is $K$.  %(fixed field theorem~\ref{fields}.\ref{fixed-field}). 
As $\mu_n\sub K$, $G$ acts trivially on $\mu_n$ and $H^{1}(G,\mu_n)=\Hom(G,\mu_n)$ by Corollary~\ref{group-hom-cohom}.\ref{h1-is-hom}.  
The sequence becomes
\[
1\to \mu_n\to K^{\times}\xra{x\mapsto x^n} K^{\times}\cap L^{\times n} \to \Hom(G, \mu_n)\to 1,
\]
giving an isomorphism
\[
K^{\times}\cap L^{\times n}/K^{\times n}\cong \Hom(G, \mu_n).
\]
The map is $\partial^1(a)=\pa{\si\mapsto\frac{\si(a^{\rc n})}{a^{\rc n}}}$, as shown by tracing through the construction in Theorem~\ref{group-hom-cohom}.\ref{les}. This proves Proposition~\ref{pr:kummer-char}.
\[
\xymatrix{
&& K^{\times}\cap L^{\times n}\ar[d]\\
\mu_n\ar@{.>}[r]^i \ar@{.>}[d]^{d_1}&  {L^{\times}}\ar[r]^{x\mapsto x^n} \ar[d]^{d_1}
& L^{\times n}\\
\Der(G, \mu_n)\ar[r]^i&\Der(G,L^{\times})\\
}
\xymatrix{
&& a\ar[d]\\
& a^{\rc n} \ar[r]^{x\mapsto x^n} \ar[d]^{d_1}
& a\\
\pa{\si\mapsto \frac{\si(a^{\rc n})}{a^{\rc n}}}\ar[r]^i&\pa{\si\mapsto \frac{\si(a^{\rc n})}{a^{\rc n}}}\\
}
\]

We claim that $|\Hom(G,\mu_n)|=|G|$. Indeed, by the structure theorem for abelian groups, $G$ decomposes as $(\Z/n_1\Z)\times \cdots \times (\Z/n_m\Z)$ where $n_1,\ldots, n_m\mid n$. To choose a homomorphism for $G$ means choosing images for the generators of $\Z/n_1\Z,\ldots, \Z/n_m\Z$; there are $n_1,\ldots, n_m$ possibilities, respectively, for a total of $|G|$.

Then
\[
[L:K]=|G(L/K)|=[K^{\times}\cap L^{\times n}:K^{\times}].
\]
This shows~(\ref{degree-order}).

\noindent\underline{Step 2:} Next note the following two inclusions.
\begin{enumerate}
\item
$K[B(L)^{\rc n}]\subeq L$: Anything in $(K^{\times}\cap L^{\times n})^{\rc n}$ is in the form $(\be^n)^{\rc n}$ and hence in $L$.
\item
$B(K[B^{\rc n}])\supeq B$: Anything in $B$ is in the form $(b^{\rc n})^n$ and hence in $K^{\times}\cap K(B^{\rc n})^{\times n}$.
\end{enumerate}

\noindent\underline{Step 3:} We show that $K[B(L)^{\rc n}]= L$. By the inclusions in step 2,
\[
[L:K]\ge [K[B(L)^{\rc n}]:K]\stackrel{\eqref{degree-order}}{=}
[B(K[B(L)^{\rc n}]):K^{\times}]\ge [B(L):K^{\times}].
\]
But $[L:K]=[B(L):K^{\times}]$ by~(\ref{degree-order}), so equality holds everywhere. The first equality gives the conclusion.\\

\noindent\underline{Step 4:} We show that $B(K[B^{\rc n}])= B$. We apply step 1 with $L=K[B^{\rc n}]$ to get the isomorphism
\begin{align*}
B(L)=K^{\times}\cap L^{\times n}/K^{\times n}&\xra{\cong} \Hom(G,\mu_n)\\
a&\mapsto \pa{\si\mapsto \frac{\si(a^{\rc n})}{a^{\rc n}}}.
\end{align*}
Now $B\subeq B(L)$ gets mapped to a subgroup $H'\subeq \Hom(G,\mu_n)$, which can be identified with $\Hom(G/H,\mu_n)$\footnote{
The subgroups of $G$ are in bijective correspondence with the subgroups of $\Hom(G,\mu_n)$ via the map
\begin{align*}
H&\xra{\Phi} \set{h\in \Hom(G,\mu_n)}{H\subeq \ker h}\cong \Hom(G/H,\mu_n)\\
\bigcap_{h\in H'} \ker h&\xleftarrow{\Psi} H'
\end{align*}
Indeed, clearly $\Psi(\Phi(H))\supeq H$, and we have equality since for every $g\in G\bs H$ we can find $h\in \Hom(G,\mu_n)$ with kernel containing $H$, so that $h(g)\ne 1$. Since $\Hom(G,\mu_n)\cong G$, they have the same number of subgroups, and this is a bijection.
}.
But as the $a^{\rc n}$ generate $L$ over $K$ and the fixed field of $G$ is $K$, $\bigcap_{h\in H'} \ker h=1$. Thus $H=\{1\}$.
%The fixed field of $G$ is $K$ %(fixed field theorem~\ref{field}.\ref{fixed-field}) 
%so the only element of $B(L)$ fixed by all elements of $H$ is 1. Then $H=\Hom(G,\mu_n)$.
Hence $|B(L)|=|G|=|B|$, and $B=B(L)$.
\end{proof}
\index{roots, linear independence of}
\begin{cor}[$n$th roots are linearly independent]
Let $S$ be a set of nonzero integers so that $\frac{a}{b}$ is not a perfect $n$th power for any distinct $a,b\in S$. Then the elements
\[
\sqrt[n]{s},\quad s\in S
\]
are linearly independent over $\Q$.
\end{cor}
\begin{proof}
\noindent{\underline{Step 1:}}
It suffices to show that for distinct primes $p_1,\ldots, p_k$, we have
\begin{equation}\llabel{roots-right-degree}
[\Q(\sqrt[n]{p_1},\ldots, \sqrt[n]{p_k}):\Q]=n^k.
\end{equation}
Then a basis for this extension over $\Q$ is formed by taking products of basis elements for the $\Q(\sqrt[n]{p_j})$:
\begin{equation}\llabel{basis-sqrt}
\set{
\sqrt[n]{p_1^{a_1}\cdots p_k^{a_k}
}}{0\le a_j<n}.
\end{equation}
However, the radicands are exactly the representatives of elements in $\Q^{\times}/\Q^{\times n}$. The elements of $S$ are all represented by distinct elements of~(\ref{basis-sqrt}) modulo $\Q^{\times}$, so the theorem will follow. (To deal with $s\in S$ negative, note if $s$ is negative then $\sqrt[n]{s}$ is a not in $\R$.) 

We want to use Kummer theory to conclude~\eqref{roots-right-degree}. However, $\Q$ only has square roots of unity ($\pm1$), so we have to consider all other roots separately. We may as well assume $2\mid n$.\\

\noindent{\underline{Step 2:}} We first show
\begin{equation}\llabel{roots-indpt-index1}
[\Q(\sqrt{p_1},\ldots, \sqrt{p_k}):\Q]=2^k.
\end{equation}

Let $B$ be the subgroup of $\Q^{\times}$ generated by $p_1,\ldots, p_k$ and $\Q^{\times 2}$. %Note the elements of $S$ form distinct representatives in $\Q^{\times}/\Q^{\times 2}$. 
By Theorem~\ref{kummer-theorem}, 
\[
[\Q(B^{\rc 2}):\Q]=[B:\Q^{\times 2}]=2^k,
\]
as needed.\\

\noindent{\underline{Step 3:}} We now adjoin $n$th roots of unity such that we can apply Kummer theory for $n$th roots. Let $N$ be a positive integers such that $n\mid N$ and $\Q(\sqrt{p_1},\ldots, \sqrt{p_k})\subeq\Q(\ze_N)$ (every quadratic extension is contained in a cyclotomic extension; we can take $N=4p_1\cdots p_k n$).

However, if we look at $K:=\Q(\ze_N)$, what if elements that aren't $n$th powers in $\Q$ become $n$th powers? Fortunately, this doesn't happen for $n\ne 2$. We show that for even $n\ne 2$ and $m\in\Q$ not a perfect $\frac n2$th power,
\begin{equation}\llabel{root-not-in-cyclotomic}
\sqrt[n]{m}\nin \Q(\ze_N).
\end{equation}
By taking roots, we may assume that $m$ is not a perfect $d$th power for any $d\mid n$.

%Indeed, consider $\Q(\sqrt[n]{p})$; if it were contained in a cyclotomic extension then
%\[
%\Q(\sqrt[n]{p},\ze_n)\subeq \Q(\ze_M)
%\]
%for some $M$.
Note $L:=\Q(\sqrt[n]{m},\ze_n)$ is a Galois extension of $\Q$ since it is the splitting field of $X^n-m$. Note $X^n-m$ is irreducible over $\Q$ because the constant term of any proper factor must be in the form $m^{\frac jn}\nin \Q$ where $0<j<n$. Hence there exists $\tau\in G(L/\Q)$ sending $\sqrt[n]{m}$ to $\ze_n\sqrt[n]{m}$. 
Let $\si\in G(L/\Q)$ denote complex conjugation. Then
\begin{align*}
\si\tau(\sqrt[n]{m})&=\si(\ze_n\sqrt[n]{m})=\ze_n^{-1}\sqrt[n]{m}\\
\tau\si(\sqrt[n]{m})&=\tau(\sqrt[n]{m})=\ze_n\sqrt[n]{m}.
\end{align*}
Hence $G(L/\Q)$ is not abelian. Since all cyclotomic extensions are abelian, $L$ cannot be contained in an abelian extension, giving~(\ref{root-not-in-cyclotomic}).

Let $C$ be the subgroup of $\Q(\ze_N)^{\times}$ generated by $\sqrt{p_1},\ldots, \sqrt{p_k}$ and $\Q(\ze_N)^{\times \frac n2}$. We showed above that $\sqrt[n]{m}\nin\Q(\ze_N)^{\times \fc n2}$ for any  $m$ not a perfect $\frac n2$th power so $[C:\Q(\ze_N)^{\times \frac n2}]=\pf{n}{2}^k$. By Kummer Theory,
\[
[\Q(\ze_N,\sqrt[n]{p_1},\ldots, \sqrt[n]{p_k}):\Q(\ze_N)]=[\Q(C^{\frac n 2}):K]=[C:\Q(\ze_N)^{\times \frac n2}]=\pf{n}{2}^k.
\]
Since $\Q(\sqrt{p_1},\ldots, \sqrt{p_k})\subeq \Q(\ze_N)$ we get
\begin{equation}\llabel{roots-indpt-index2}
[\Q(\sqrt[n]{p_1},\ldots, \sqrt[n]{p_k}):\Q(\sqrt{p_1},\ldots, \sqrt{p_k})]=\pf n2^k.
\end{equation}

Combining~(\ref{roots-indpt-index1}) and~(\ref{roots-indpt-index2}) gives~(\ref{roots-right-degree}), as needed.
%Since representatives of $(B/K^{\times n})^{\rc n}$ generate the field extension, these representatives form a basis, i.e. are linearly independent.
\end{proof}
\section{Nonabelian Galois cohomology}\llabel{sec:nonabelian-galois-cohom}
\index{descent}
Because of the definition of $H^1$ in Section~\ref{group-hom-cohom}.\ref{nonabelian-cohom}, we find that we can often interpret $H^1(G(L/K),A)$ as parameterizing certain algebraic structures, specifically a set of them defined over $K$ that become isomorphic in $L$. (This is known as {\it descent} because it answers the question, how many ways can an algebraic structure (or in general, a variety) ``descend" from $L$ to $K$?)
In general,
\begin{multline}
H^1(G(L/K),\{\text{automorphisms preserving $V$ over $K$}\})\\
\cong \{\text{$K$-isomorphism classes that are $L$-congruent to $V$}\}
\llabel{eq:descent}
\end{multline}


In this section, we will see several examples where $A$ is an algebraic group. We could also take $A$ to be an abelian variety (see Silverman~\cite{Si86}, Theorem X.2.2, for instance).

In particular, we find in the next section that a special cohomology group classifies algebra structures over $K$: the Brauer group.

First, we need the following nonabelian generalization of Hilbert's Theorem 90~(\ref{h90}).
\begin{thm}[Generalization of Hilbert's Theorem 90]\llabel{thm:gen-h90}
For any finite Galois extension $L/K$, letting $G=G(L/K)$,
\[
H^1(G,\GL_n(L))=H^1(G,\SL_n(L))=1.
\]
\end{thm}
\begin{proof}
As in Theorem~\ref{h90}, given a 1-cocycle $c:G\to GL_n(L)$, consider the function
\begin{align*}
b:\GL_n(L)&\to \cal M_n(L)\\
b(A)&:=\sum_{\tau \in G}c_{\tau}\tau(A).
\end{align*}
Note that unlike in the proof of Theorem~\ref{h90}, we not only have to choose $A$  to be nonzero, but also invertible.

Also define $b$ on $L^n$ in the same way:
\begin{align*}
b:L^n&\to L^n\\
b(\mathbf x)&:=\sum_{\tau \in G}c_{\tau}\tau(\mathbf x).
\end{align*}
We show that $\set{b(\mathbf x)}{\mathbf x\in L^n}$ generate $L^n$ as a vector space over $L$.\footnote{Note $b$ is not a $L$-linear transformation; it is a $K$-linear transformation.} Suppose a linear functional $f:L^n\to L$ vanishes on all the $b(\mathbf x)$. Then for every $\al\in L$,
\[
0=f(b(\al \mathbf x))=\sum_{\tau\in G}f(c_{\tau}\tau(\al)\tau(\mathbf x))
=\sum_{\tau\in G} \tau(\al)f(c_{\tau}\tau(\mathbf x)).
\]
By linear independence of characters, we get that all the coefficients of the $\tau(\al)$ must be 0, i.e. $f(c_{\tau}\tau(\mathbf x))$ for all $c_{\tau}, \mathbf x$. But $c_{\tau}\in \GL_n(L)$ is invertible, so $f$ must vanish identically on $L^n$. We've shown that every linear functional vanishing on $\{b(\mathbf x)\}$ vanishes on $L^n$; therefore $\spn_L\{b(\mathbf x)\}=L^n$.

Thus we can choose $\mathbf x_1,\ldots, \mathbf x_n$ such that $\mathbf y_j=b(\mathbf x_j)$ form a basis for $L^n$ over $L$. Let $A$ be the matrix sending the canonical basis $\mathbf e_j$ to the $\mathbf x_j$. Then (note $\tau$ acts trivially on the $e_j$)
\[
b(A)\mathbf e_j=b(A\mathbf e_j)=\mathbf y_j
\]
so $b(A)$ is invertible.

The the rest of the proof of Theorem~\ref{h90} goes through: we have as in~(\ref{eq:h90pf}) that
\[
c_{\si}=b(A)\si(b(A))^{-1},
\]
i.e. $c$ is a coboundary. This shows $H^1(G,\GL_n(L))=1$.

For the second part, the exact sequence
\[
1\to \SL_n(L)\to GL_n(L)\xra{\det} L^{\times}\to 1
\]
gives the long exact sequence~\ref{group-hom-cohom}.\ref{thm:nonabelian-les}
\[
\xymatrix{
H^0(G,\GL_n(L))\ar[r]^{\det}\ar@{=}[d] & H^0(G,L^{\times}) \ar[r]\ar@{=}[d] & H^1(G,\SL_n(L))\ar[r] & H^1(G,\GL_n(L))\ar@{=}[d]\\
\GL_n(K) \ar@{->>}[r]^{\det}& K^{\times} & & 0.
}
\]
As the map on the left is surjective, we get $H^1(G,\SL_n(L))=1$. 
\end{proof}
We have now established~\eqref{eq:descent} when $V$ is a vector space: all vector spaces that become isomorphic in $L$ have the same dimension to begin with so are isomorphic in $K$, so the right-hand side of~\eqref{eq:descent} is $\{1\}$, and if $V=K^n$, $\GL_n(L)$ is the group of automorphisms preserving $V$ over $L$, and Theorem~\ref{thm:gen-h90} shows the right-hand side of~\eqref{eq:descent} is $\{1\}$. We now extend this to other algebraic structures.

To encode an algebraic structure, we consider vector spaces and tensors.
\index{tensor}
\begin{ex}\llabel{ex:tensors}$\,$\\
Let $V$ be a finite-dimensional vector space. The space $V^{\ot p}\ot V^{*\ot q}$ encodes$\ldots$

\noindent \begin{center}
\begin{tabular}{|c|c|c|}
\hline 
$p$ & $q$ & Structure\tabularnewline
\hline 
1 & 0 & vectors\tabularnewline
\hline 
0 & 1 & linear functionals\tabularnewline
\hline 
1 & 1 & linear operators\tabularnewline
\hline 
0 & 2 & bilinear forms\tabularnewline
\hline 
1 & 2 & algebra structures\tabularnewline
\hline 
\end{tabular} \end{center}
We focus on the case $p=1$, $q=2$. Given a tensor $\sum_i v_i\ot f_i\ot g_i\in V\ot V^{*\ot 2}$, define a (not necessarily commutative or associative) algebra structure on $V$ by
\[
v\cdot w = \sum_i f_i(v)g_i(w)v_i.
\]
Conversely, any algebra structure can be encoded in this way: Take a basis $\{v_i\}$ for $V$ and a dual basis $f_i$ for $V^*$, and encode the structure by $\sum_{i,j} (v_i\cdot v_j)\ot f_i\ot g_j$.
\end{ex}
\begin{df}
Let $V$ be a vector space over $K$ and $x\in V^{\ot p}\ot V^{*\ot q}$ be a tensor of type $(p,q)$. Two pairs $(V,x)$ and $(V',x')$ are isomorphic if there is a $K$-linear isomorphism 
\[f:V\to V'\]
such that $f(x)=x'$. Here, $f$ sends 
\beq{eq:extend-tensor}
x_1\ot\cdots \ot x_p\ot f_1\ot \cdots \ot f_q\mapsto f(x_1)\ot \cdots f(x_p)\ot (f_1\circ f^{-1})\ot \cdots \ot (f_q\circ f^{-1}).
\eeq
Given $(V,x)$ defined over $K$, we can consider it over $L$ by extending scalars; denote the resulting pair by $(V_L=V\ot_KL,x_L)$.

We say that $(V,x)$ and $(V',x')$ are $L$-isomorphic if $(V_L,x_L)$ and $(V'_L,x'_L)$ are isomorphic.
Let $E_{V,x}(L/K)$ denote the $L$-isomorphism classes of pairs that are $K$-equivalent to $(V,x)$. If $L/K$ is Galois, let $s\in G(L/K)$ act on $v\ot \al\in V\ot_K L=V_L$ by $s(v\ot_K \al):=v\ot_K s(\al)$ and let $s$ act on $A_L$ by conjugation:
\[
f^s:=s\circ f\circ s^{-1}.
\]
\end{df}
\begin{thm}[Descent for tensors]\llabel{thm:descent-tensors}
Let $L/K$ be a Galois extension, $G=G(L/K)$, and let $A_L$ be the group of $L$-automorphisms of $(V_L,x_L)$.
Define the map
\begin{align*}
\te:E_{V,x}(L/K)&\to H^1(G,A_L)\\
(V',x')&\mapsto (d:\si \mapsto f^{-1}\circ f^{\si }=f^{-1}\circ \si  \circ f \circ \si ^{-1})
\end{align*}
%where $f:V_L\to V_L'$ is any $L$-automorphism, 
where $f:(V_L,x_L)\to (V_L',x_L')$ is any $L$-automorphism. 
Then $\te$ is a bijection.
\end{thm}
\begin{proof}
We show the following.
\begin{enumerate}
\item $\te$ is well-defined. First, $\te(V',x')$ is a cocycle as
\[
d(\si t)=f^{-1}\si tft^{-1}\si^{-1}=(f^{-1}\si f\si^{-1})[\si (f^{-1}tft^{-1})\si^{-1}]=d(\si )\circ d(t)^{\si }.
\]
(See Definition~\ref{group-hom-cohom}.\ref{df:nonabelian-cocycles}.) 
Next, we show $\te(V',x')$ does not depend on the choice of $f$: Let $d_f(\si )=f^{-1}\si f\si^{-1}$ and $d_g(s)=g^{-1}\si g\si^{-1}$. Then
\[
d_g(\si )=g^{-1}\si g\si^{-1}= g^{-1}f(f^{-1}\si f\si^{-1})\si f^{-1}g\si^{-1}
=(fg^{-1})^{-1}d_f(\si ) (fg^{-1})^{\si }
\]
so $d_f$ and $d_g$ are cohomologous.
\item $\te$ is injective. Suppose $\te(V_1',x_1')=\te(V_2',x_2')$. We can choose the isomorphisms $f_1$ and $f_2$ such that $f_1^{-1}f_1^{\si }=f_2^{-1}f_2^{\si}$ for all $\si \in G(L/K)$. Then $(f_2f_1^{-1})^{\si }=f_2f_1^{-1}$ for all $\si \in G(L/K)$, i.e. $f_2f_1^{-1}$ is an isomorphism defined over $K$. %?
Thus $(V_1',x_1')$ and $(V_2',x_2')$ are $K$-isomorphic.
\item $\te$ is surjective. Let $c_{\si}$ be a 1-cocycle of $G$ with values in $A_L$. Since $A_L\subeq \GL(V_L)$, by Theorem~\ref{thm:gen-h90} there exists $f\in \GL(V_L)$ such that
\[
c_{\si}=f^{-1}\circ f^{\si}
\]
Let $f$ operate on $V^{\ot p}\ot V^{*\ot q}$ as in~(\ref{eq:extend-tensor}) and let $x'=f(x)$. As $x\in V_K^{\ot p}\ot V_K^{*\ot q}$ and $c_{\si}$ fixes $K$, we have 
\[
\si(x')=f^{\si}(\si(x))=f^{\si}(x)=f\circ c_{\si}(x)=f(x)=x'.
\]
Thus $x'$ is rational over $K$ (i.e. in $V_K^{\ot p}\ot V_K^{*\ot q}$), and $(V,x')$ maps to $c_{\si}$.
\end{enumerate}
\end{proof}
Note that since we always take an isomorphism $V\to V'$, we can really consider all the vector spaces to be the ``same," and just vary the tensors $x$. If we consider $V=V'$, then we abbreviate $f:(V_L,x_L)\to (V_L',x_L')$ by $f:x\to x'$.
\begin{ex}
We can use Galois cohomology to classify quadratic forms over a field $K$. Let $\Phi$ be a quadratic form (which corresponds to a bilinear form and can be represented by a tensor of type $(0,2)$), and $O_L(\Phi)$ be the orthogonal group of $\Phi$, i.e. linear transformations that preserve $\Phi$. Then $H^1(G(L/K),O_L(\Phi))$ classifies the quadratic forms over $K$ that are $L$-isomorphic to $\Phi$.
\end{ex}

\section{Brauer group}\llabel{brauer}
\index{Brauer group}
The Brauer group characterizes algebras over a field $K$. We already know a simple way of making algebras: just consider the algebra of $n\times n$ matrices, $\cal M_n(K)$. Thus, we will essentially ``mod out'' by these when constructing the group.

As we will see, there is an isomorphism to a second cohomology group. Thus, we can apply results about algebras over $K$ to Galois cohomology, or conversely, apply Galois cohomology to get information on algebras over $K$.

First, we need some results from noncommutative algebra. We refer the reader to Cohn~\cite{Co03}, Chapter 5, or Milne~\cite{Mi08}, Chapter IV.1--2, for the proofs. %a preliminary definition.

\subsection{Background from noncommutative algebra}
\index{central simple algebra}
%(Maybe include a subsection on some results from representation theory/noncommutative algebra, Schur, Double centralizer theorem, Wedderburn, Noether-Skolem)
\begin{df}
An \textbf{algebra} over a field $K$ is a ring $A$ with $K$ in its center\footnote{The center of a ring $R$ is the set of elements commuting with all elements of $R$.}. Its dimension is the dimension of $A$ as a $K$-vector space, denoted $[A:K]$. {\it In this chapter we assume all algebras to be finite-dimensional as $K$-vector spaces.}

An algebra over $K$ is
\begin{enumerate}
\item
\textbf{simple} if it has no proper two-sided ideals.
\item
\textbf{central} if its center in $K$.
\end{enumerate}
An algebra is a \textbf{division algebra} if every nonzero element has an inverse.
\end{df}
\begin{ex}
The algebra of $n\times n$ matrices $\cal M_n(K)$ is a central simple algebra over $K$.
\end{ex}
\index{representation}
\begin{df}
Let $A$ be an algebra over $K$. We use ``$A$-module" to mean any finitely generated left $A$-module $V$; the map $A\to \End(V)$ is called a \textbf{representation} of $A$. The module (or representation) is \textbf{faithful} if $av=0$ for all $v\in V$ implies $a=0$, i.e. $A\hra \End(V)$ is injective. A module is \textbf{simple} if it doesn't contain a proper $A$-submodule, and \textbf{indecomposable} if it is not the direct sum of two proper $A$-submodules. (Note that simple implies indecomposable, but not vice versa.) A module is \textbf{semisimple} if it is the direct sum of simple $A$-modules.\footnote{Equivalently, the radical of $A$ is trivial. If it is semisimple the factors in the decomposition are unique up to isomorphism (Jordan-H\"older).}

We say $A$ is semisimple if it is semisimple as a module.
\end{df}

We need some basic results from noncommutative algebra.
\index{centralizer}
\begin{df}
Let $B\subeq A$ be a subalgebra. Define the \textbf{centralizer} of $B$ to be the elements of $A$ commuting with $B$:
\[
C(B):=\set{a\in A}{ab=ba\text{ for all }b\in B}.
\]
\end{df}
\index{double centralizer theorem}
\begin{thm}[Double centralizer theorem]\llabel{thm:double-centralizer}
Let $A$ be a $K$-algebra, and $V$ a faithful semisimple $A$-module. Consider $A$ as a subalgebra of $\End_K(V)$. Then
\[
C(C(A))=A.
\]
\end{thm}
\begin{proof}
Milne~\cite{Mi08}, Theorem IV.1.3, or Etingof~\cite{Et10}, Theorem 4.54.
\end{proof}
\index{Wedderburn's structure theorem}
\begin{thm}[Wedderburn's structure theorem]\llabel{thm:wedderburn}
An algebra $A$ is semisimple iff it is isomorphic to the direct sum of matrix algebras over division algebras.

If $A$ is an algebra over an algebraically closed field $K$ and $K$, then any semisimple algebra over $K$ is isomorphic to a direct sum of matrix algebras over $K$.
\end{thm}
\begin{proof}
Milne~\cite{Mi08}, Theorem IV.1.15.

For the second part, we need to show the only division algebra over an algebraically closed field $K$ is $K$ itself. Suppose $D$ is a division algebra and $\al\in D$. As $[D:K]$ is finite-dimensional, $K(\al)$ is a finite extension of $K$. Hence $\al\in K$, giving $D=K$.
\end{proof}
\index{Noether-Skolem theorem}
\begin{thm}[Noether-Skolem theorem]\llabel{thm:noether-skolem}
Let $f,g:A\to B$ be homomorphisms, where $A$ is a simple $K$-algebra and $B$ is a central simple $K$-algebra. Then there exists $b\in B$ such that 
\[
f(a)=b\cdot g(a)\cdot b^{-1}
\]
for all $a\in A$, i.e. $f,g$ differ by an inner automorphism of $B$.

In particular, taking $A=B$ and $g=1$, all automorphisms of a central simple $K$-algebra are inner (come from conjugation). In particular, this is true for $\cal M_n(K)$. 
\end{thm}
\subsection{Central simple algebras and the Brauer group}
%\begin{pr}
%The center of a simple $K$-algebra is a field. Thus every simple $K$-algebra is central simple over a finite extension of $K$.
%\end{pr}
%\begin{proof}[Proof sketch]
%By Wedderburn's Theorem, we can write a simple $K$-algebra $A$ as $\cal M_n(D)\cong \cal M_n(K)\ot_K D$, where $D$ is a division algebra. Its center is $Z(\cal M_n(K)\ot_K D)=Z(\cal M_n(K))\ot Z(D)\cong K\ot_K Z(D)=Z(D)$.
%\end{proof}
We now define the Brauer group.
\begin{df}
Let $A$ and $B$ be simple algebras over $K$. We say $A$ and $B$ are similar and write $A\sim B$ if
\[
A\ot_K\cal M_m(K)\cong B\ot_K \cal M_n(K)
\]
for some $m,n$.

The \textbf{Brauer group} $\Br_K$ is the set of similarity classes of central simple algebras over $K$, with multiplication defined by
\[
[A][B]=[A\ot_K B].
\]

The Brauer group $\Br_{L/K}$ is the subgroup of classes of central simple algebras over $K$ that are \textbf{split} over $L$, i.e. such that $A\ot_K L$ is a matrix algebra.
\end{df}
\begin{proof}[Proof (sketch) that this is a group]
We need to check that$\ldots$
\begin{enumerate}
\item The tensor product of two central simple algebras is central simple. By Wedderburn's Theorem~\ref{thm:wedderburn} we can write the algebras as $A=M_m(D)$ and $B=M_{m'}(D')$, where $D,D'$ are division algebras. One can show $A\ot_K D'$ is simple; hence it equals $M_n(D'')$ for some $D''$. Then $A\ot_K B\cong M_{m'n}(D'')$ is simple. It is central because $C(A\ot_K B)=C(A)\ot_KC(B)=K$.
\item
``$\sim$" is an equivalence relation. If $A\sim B$ and $B\sim C$, then $A\ot_K \cal M_m(K)\cong B_K\ot_K \cal M_n(K)$, $B\ot_K\cal M_{n'}(K)\cong C\ot_K \cal M_p(K)$ for some $m,n,n',p$. Then
\[
A\ot_K \cal M_{mn'}(K)\cong A\ot_K \cal M_m(K)\ot_K \cal M_{n'}(K)\cong C\ot_K\cal M_n(K)\ot_K \cal M_p(K)\cong C\ot_K M_{np}(K).
\]
\item
``$\sim$" is preserved under the operation $\ot$. If $A_i\ot_K \cal M_{m_i}(K)\cong B_i\ot_K \cal M_{n_i}(K)$ for $i=1,2$, then $A_1\ot_K A_2\ot_K \cal M_{m_1m_2}(K)\cong B_1\ot_K B_2\ot_K \cal M_{n_1n_2}(K)$.
\item
$A$ has an inverse. Letting $A^{\text{opp}}$ be the opposite algebra, we find that
\[
A\ot_K A^{\text{opp}}\cong \cal M_n(K),\quad n=[A:K].
\]
\item
The operation is commutative and associative. This follows since tensor product is commutative and associative.
\end{enumerate}
\end{proof}
By Wedderburn's Structure Theorem~\ref{thm:wedderburn}, each (central) simple algebra is $M_n(D)\cong M_n(K)\ot_K D$ for some (central) division algebra $D$, so every similarity class is represented by a central division algebra. Thus to determine the Brauer group it suffices to classify central division algebras.
\begin{ex}
We have
\[
\Br_{\R}=\{\R,\Hq\}
\]
where $\Hq$ denotes the quaternions: the algebra with basis $1,\vi,\vj,\vk=\vi\vj$ and relations $\vi^2=1$, $\vj^2=1$, and $\vi\vj=-\vj\vi$.

Indeed, by Frobenius's Theorem, the only finite-dimensional (associative) division algebras over $\R$ are $\R$, $\C$, and $\Hq$, and only $\R$ and $\Hq$ have center equal to $\R$.
\end{ex}
%\begin{pr}$\,$
%\begin{enumerate}
%\item
%If $K$ is algebraically closed, then $\Br(K)=0$.
%\item
%$\Br(\R)=\{\R,\Hq\}$.
%\item 
%If $k$ is a finite field, then $\Br(k)=0$.
%\item 
%If $K$ is a nonarchimedean local field then $\Br(K)\cong \Q/\Z$.
%\end{enumerate}
%\end{pr}
%\begin{pr}[Extension of base field]
%Let $A$ be a central simple algebra over $K$, and let $L/K$ be a field extension. Then $A\ot_K L$ is a sentral simple algebra over $L$.
%\end{pr}
%\begin{cor}
%If $A$ is central simple over $K$ then $[A:K]$ is a square.
%\end{cor}
\begin{pr}
For any algebraically closed field $K$,
\[
\Br_{\ol K}=0.
\]
%and
%\[
%\Br_K=\bigcup_{L\text{ finite extension}} \Br_{L/K}.
%\]
\end{pr}
\begin{proof}
By Wedderburn's Theorem~\ref{thm:wedderburn}, all central simple algebras over $K$ are $\cal M_n(K)$ for some $n$.
\end{proof}
\subsection{Subfields and splitting of central simple algebras}
An important way of studying a central simple algebra is to look at its subfields.
\begin{thm}[Double centralizer theorem, generalization]\llabel{thm:dct-gen}
Let $A$ be a central simple $K$-algebra and $B$ be a simple $K$-subalgebra. Let $C=C(B)$. Then $C$ is simple, $C(C)=A$, and
\[
[B:K][C:K]=[A:K].
\]
\end{thm}
\begin{proof}
See Milne~\cite{Mi08}, Theorem IV.3.1.
\end{proof}
\begin{cor}
Let $A$ be central simple over $K$, and $L$ be a subfield with $K\subeq L\subeq A$. Then the following are equivalent.
\begin{enumerate}
\item
$L=C(L)$.
\item
$[A:K]=[L:K]^2$.
\item
$L$ is the maximal commutative $K$-subalgebra of $A$.
\end{enumerate}
\end{cor}
\begin{proof}
Milne~\cite{Mi08}, Corollary IV.3.4.
\end{proof}
The following describes the fields over which a central simple algebra splits.
\begin{cor}\llabel{cor:csa-split}
Let $A$ be central simple over $K$. A finite extension field $M$ splits $A$ iff there exists an algebra $B\sim A$ containing $M$ with $[B:K]=[L:K]^2$. In particular, any subfield $L$ of $A$ of degree $\sqrt{[A:K]}$ splits $A$.

If $D$ is a divison algebra of degree $n^2$ over $K$, and $L$ is a field of degree $n$ over $K$ (equivalently a maximal commutative subfield of $D$), then $L$ splits $D$, i.e. $D\cong \cal M_n(L)$.
\end{cor}
\begin{proof}
Milne~\cite{Mi08}, IV.3.6, and 3.7.
\end{proof}
%If $D$ is a central division algebra, and $L$ is a maximal subfield of $D$ containing $K$, then $[
\begin{thm}\llabel{thm:all-split}
Every central division algebra over $K$ is split over some finite Galois extension $L/K$. Therefore
\[
\Br_K=\Br_{\ol K/K}=\bigcup_{L/K\text{ finite Galois}} \Br_{L/K}.
\]
\end{thm}
\begin{proof}
When $K$ is perfect, this follows directly from Corollary~\ref{cor:csa-split}. The general case requires a separate argument; see Milne~\cite{Mi08}, IV.3.10.
\end{proof}
Similar to the commutative case, we can define a valuation on division algebras.
\begin{pr}\llabel{pr:div-alg-val}
Let $D$ be a division algebra of rank $n^2$ over a local field $K$. Then $D$ admits a discrete valuation extending the valuation on $K$, such that for any $a\in (0,1)$, $\ve{x}_D:=a^{v(x)}$ defines a norm on $D$. The set of integral elements $\set{x}{v(x)\ge 0}$ is a subring of $D$.
\end{pr}
\section{Brauer group and cohomology}
\subsection{The Brauer group is a second cohomology group}
\begin{df}
Let $\Br_{L/K,n}$ denote the subset of $\Br_{L/K}$ consisting of $[A]$ where $A\ot_K L\cong \cal M_n(L)$. Note that $\Br_{L/K}=\bigcup_{n\in \N}\Br_{L/K,n}$.
\end{df}
\begin{thm}[Cohomological interpretation of Brauer group]
\llabel{thm:brauer-cohom}
%Let
%$\cal A(L/K)$ be the classes of central simple algebras over $K$ containing $L$. Then there is a bijection 
%\[
%\ga:\Br(L/K)\xra{\cong} A(L/K)/\sim\xra{\cong} H^2(L/K).
%\]
%Let $\Br(n,L/K)$ denote the classes of $K$-algebras such that $A\ot_K L\cong \cal M_n(K)$.
There are canonical bijections
\[
\te_n:\Br_{L/K,n}\to H^1(G,\PGL_n(K))
\]
and canonical isomorphisms
\begin{align*}
\de:\Br_{L/K}&\to H^2(L/K)\\
\de:\Br_{K}&\to H^2(K)
\end{align*}
where $H^2(K):=H(\ol K/K)=\varinjlim_{L/K\text{ finite Galois}} H^2(L/K)$.
\end{thm}
\begin{proof}
We can represent elements of $\Br_{L/K,n}$ as algebras of dimension $n^2$ over $K$, that are $L$-isomorphic to the algebra $\cal M_n(L)$. By Example~\ref{ex:tensors}, we can encode the algebra $\cal M_n(L)$ by a tensor of type $(1,2)$. By Theorem~\ref{thm:descent-tensors},
\beq{eq:brauer-descent}
\Br_{L/K,n}\cong H^1(G,\Aut(\cal M_n(L))).
\eeq
By the Noether-Skolem Theorem~\ref{thm:noether-skolem}, every automorphism of $\cal M_n(L)$ is conjugation by an element of $\GL_n(K)$. Since the matrices that act trivially by conjugation are just the scalar matrices, we have the short exact sequence
\beq{eq:noether-skolem}
1\to L^{\times}\to \GL_n(L)\to \Aut(\cal M_n(L))\cong \PGL_n(L)\to 1.
\eeq
%We have
%\[
%\Aut(\cal M_n(L))\cong \GL_n(L)/L^{\times}=\PGL_n(L).
%\]
Along with~(\ref{eq:brauer-descent}) this proves the first part.

The long exact sequence \ref{group-hom-cohom}.\ref{thm:nonabelian-les} of~\eqref{eq:noether-skolem} gives 
\[
0=H^1(G,\GL_n(L))\to H^1(G,\PGL_n(L))\xra{\De_n} H^2(G,L^{\times}),
\]
where the LHS follows from Theorem~\ref{thm:gen-h90}. Let $\de_n=\De_n\circ \te_n$; then $\de_n$ is an injective map.

We show the following.
\begin{enumerate}
\item The $\de_n$ for different $n$ combine compatibly into an injective group homomorphism $\de:\Br(L/K)\to H^2(L/K)$: 
We need to show
\[
\de_{nn'}(A\ot A')=\de_n(A)\de_{n'}(A')
\]
for any $A\in \Br_{L/K,n}$ and $A'\in \Br_{L/K,n'}$.
%Let $e_{ij}$ be the standard basis for $\cal M_n(K)$ and $e_{ij}'$ be the standard basis for $\cal M_{n'}(K)$. Let $f_{ij}$ and $f_{ij}'$ be the dual bases. Then, using the construction in Example~\ref{ex:tensors}, $\cal M_n(K)$ and $\cal M_{n'}(K)$ are represented by $x=\sum_{1\le i,j,k\le n} f_{ij}\ot f_{jk}\ot e_{ik}$ and $x'=\sum_{1\le i,j,k\le n'}f'_{ij}\ot f_{jk}'\ot e_{ik}'$. If $A$ is represented by $a=\sum_{i,j,k,l,p,q}c_{ijklpq}f_{ij}\ot f_{kl}\ot e_{pq}$ and $A'$ is represented by $\sum_{i,j,k,l,p,q}c'_{ijklpq}f'_{ij}\ot f'_{kl}\ot e'_{pq}$, then $A\ot A'$ is represented by $a'=\sum_{i,j,\ldots, q'}c_{ijklpq} c'_{ijklpq}(f_{ij}\ot f'_{ij})\ot (f_{kl}\ot f'_{kl})\ot (e_{pq}\ot e'_{pq})$, i.e. the structure constants multiply in the obvious way. 

First, note that if $a,a'$ are tensors encoding the algebras $A,A'$ on $V\ot V^{*\ot 2}$ and $V'\ot {V'}^{*\ot 2}$, then $x\ot x'$ encodes the algebra $A\ot A'$ on $(V\ot V')\ot (V\ot V')^{*\ot 2}$. Let $x,x'$ encode $\cal M_n(K)$ and $\cal M_{n'}(K)$, so that $x\ot x'$ encodes $\cal M_{nn'}(K)$. 
If $f:x\to a$ and $f':x'\to a'$ are $L$-linear maps, then we have the $L$-linear map on $\cal M_{nn'}(L)$,
%(\cal M_n(L),x)\to (\cal M_n(L),a)$ and $f':(\cal M_{n'}(L),x')\to (\cal M_n(L),a')$ are $L$-linear maps, then we have the $L$-linear map
\[f\ot f':x\ot x'\to a\ot a'.\]
%(\cal M_{nn'}(L),x\ot x')\to (\cal M_{nn'}(L),
Now $\te_n,\te_{n'}$ map $A$ and $A'$ to $c_{\si}=f^{-1}\circ f^{\si}$ and $c'_{\si}={f'}^{-1}\circ {f'}^{\si}$. Suppose $c_{\si}$ and $c'_{\si}$ are represented by conjugation by $S_{\si}$ and $S'_{\si}$, respectively. Now $\te_{nn'}$ maps $A\ot A'$ onto $d_{\si}=(f\ot f')^{-1}\circ (f\ot f')^{\si}$, which corresponds to conjugation by $S_{\si}\ot S'_{\si}$. Then by the description of $\De$ in Theorem~\ref{group-hom-cohom}.\ref{df:nonabelian-cocycles}, we see that
\[
\de_{nn'}(A\ot A')=\bc{a_{\si,\tau}= i_{nn'}^{-1}[(S_{\si}\ot S'_{\si}) \si (S_{\tau}\ot S'_{\tau})(S_{\si\tau}\ot S'_{\si\tau})^{-1}]}=\de_n(A)\de_n(A')
\]
where $i_{nn'}$ is the inclusion map $L^{\times}\to \GL_{nn'}(L)$. Under the inverse of $i_{nn'}=i_n\ot i_{n'}$, tensor product becomes simply the product.
%conjugation action becomes just multiplication
\item
$\de$ is surjective. It suffices to show $\De_n$ is surjective, where $n=[L:K]$.\footnote{Incidentally, this shows that every equivalence class of algebras is represented by one of dimension at most $[L:K]^2$. This is consistent with results of the previous section.} Take an 2-cocycle $a_{\si,\tau}\in H^2(G,L^{\times})$. We need to show that
\[
a_{\si,\tau}=S_{\si} \si(S_{\tau})S_{\si\tau}^{-1}
\]
for some values of $S_{\si}\in \GL_n(L)$. We identify $L^n$ with the group algebra $L[G]$, and let $S_{\si}\in \GL(L[G])$ be the map sending $\tau$ to $a_{\si,\tau}\si\tau$ (it is invertible as $a_{\si,\tau}\in L^{\times}$). Then we calculate for every $u\in G\sub L[G]$,
\bal
[S_{\si}\si(S_{\tau})]u&=[a_{\si,\tau u} \si(a_{\tau,u})]\si\tau u\\
[a_{\si\tau} S_{\si\tau}]u&=[a_{\si,\tau}a_{\si\tau, u}]\si\tau u.
\end{align*}
The right-hand sides are equal since $a_{\si,\tau}$ is a cocycle. Hence
\[
a_{\si,\tau}=S_{\si}\si(S_{\tau})S_{\si\tau}^{-1}
\]
is in the image of $\De_n$.
\item $\de$ gives an isomorphism $\Br_K\cong H^2(K)$: This follows from Theorem~\ref{thm:all-split}, the following easy-to-check commutative diagram (which holds for any $K\subeq L\subeq M$),
\[
\xymatrix{
H^2(L/K)\ar@{^(->}[r]^{\Inf} \ar[d]_{\de} &H^2(M/K)\ar[d]_{\de} \\
\Br_{L/K} \ar@{^(->}[r] & \Br_{M/K},
}
\]
and taking the direct limit of the maps $\Br_{L/K}\to H^2(L/K)$.
\end{enumerate}
\end{proof}
\begin{rem}
Milne~\cite{Mi08} makes this correspondence more explicit. The relationship between the two approaches can be seen by choosing a basis for the tensor product $V\ot {V}^{*\ot 2}$; the coefficients are called the {\it structure constants} of the algebra. (We followed Serre; note that the isomorphism in Serre is the opposite of the isomorphism in Milne.)
\end{rem}
\subsection{Exact sequence of Brauer groups}
%
%\begin{thm}
%Let $K$ be a given field. 
%The following are equivalent.
%\begin{enumerate}
%\item
%For every finite separable $L/K$, $\Br(L)=0$. 
%%The Brauer group of every finite separable extension of $K$ is 0.
%\item 
%For any $L/K$ finite separable with $M/L$ finite Galois, $H(L/K)=0$.
%\item 
%For any $L/K$ finite separable with $M/L$ finite Galois, 
%$\nm_{L/K}$ is surjective.
%\end{enumerate}
%\end{thm}
%\begin{proof}
%Serre, Proposition X.7.11.
%\end{proof}
%\begin{thm}\llabel{brkur0}
%Let $K$ be a local field. Then $\Br(K\ur)=0$. 
%\end{thm}
%\begin{proof}[Proof 1]
%Use surjectivity of norm (Serre, V.4.7).
%\end{proof}
%\begin{proof}[Proof 2]
%Calculate the Brauer group algebraically (Serre, XII). We use two lemmas.
%\begin{lem}
%Suppose $D$ is a central division algebra of rank $n^2>1$ over $K$, and $k$ is perfect. Then there exists a  commutative subfield $L$ of $D$ properly containing $K$, unramified over $K$.
%\end{lem}
%\begin{lem}
%Keep the hypotheses. There is a subfield of $D$ of degree $n$ (so it is a maximal subfield) unramified over $K$.
%\end{lem}
%Suppose $D$ is a central division algebra over $K\ur$ of rank $n^2$. Then lemma 2 furnishes a subfield of $K\ur$ of degree $n$, unramified over $K\ur$. Hence $n=1$, and $D$ is trivial. Thus $\Br_{K\ur}=0$.
%\end{proof}
%Equivalences in Milne (p. 130).
%%\begin{pr}
%%Let $A$ be a central simple algebra over $k$, and $L$ any subfield of $A$ containing $k$, of degree $[A:k]^{\rc 2}$. Then $L$ splits $A$, i.e. $A\ot_k L\cong \Mat_n(L)$.
%%\end{pr}
%\begin{df}
%Define the \textbf{reduced norm} $\text{Nrd}:A\to K$ as follows: Let $A$ have dimension $n^2$ over $K$ and think of $a\in A$ as an element in $A\ot_k L\cong \Mat_n(L)$, and set $\text{Nrd}(a)=\det(a)$. (Why is this 
%
%Define $v'(x)=v(\text{Nrd}(x))$.
%\end{df}
%Note this is well defined since by Noether-Skolem, the only isomorphisms on $\Mat_n(L)$ are given by conjugation, and this doesn't change the determinant. Note this is called the {\it reduced} norm because $\text{Nrd}(x)=x^n$ for $x\in K$, instead of $x^{n^2}$.
The importance of the Brauer group in class field theory is given by the following proposition.
\begin{thm}\llabel{brauer2}
Let $M/L/K$ be Galois extensions. Then there is an exact sequence
\[
\xymatrix{
0\ar[r] & H^2(L/K)\ar[r]\ar@{=}[d] & H^2(M/K)\ar[r]\ar@{=}[d] & H^2(M/L)\ar@{=}[d]\\
& \Br_{L/K} & \Br_{M/K} & \Br_{M/L}.
}
\]
For any Galois extension $L/K$ there is an exact sequence
\[
\xymatrix{
0\ar[r] & H^2(L/K)\ar[r]\ar@{=}[d] & H^2(K)\ar[r]\ar@{=}[d] & H^2(L)\ar@{=}[d]\\
& \Br_{L/K} & \Br_{K} & \Br_{L}.
}
\]
%0\to H^2(L/K)\to \Br_K\to \Br_{L}.\]
\end{thm}
\begin{proof}
Since $H^1(L/K)=0$ by Hilbert's Theorem 90 (\ref{h90}), 
the inflation-restriction exact sequence~\ref{group-hom-cohom}.\ref{inflate-restrict} with $G=G(M/K)$ and $H=G(M/L)$ gives
\[
0\to H^2(L/K)\xra{\Inf} H^2(M/K)\xra{\Res} H^2(M/L). 
\]
The equality with the Brauer groups follows from Theorem~\ref{thm:brauer-cohom}.

Taking the direct limit over all finite Galois extensions $M/K$ gives the second result.
\end{proof}
%\subsection{Brauer group of a local field}
%\begin{thm}\llabel{brkur0}
%Let $K$ be a local field. Then $\Br(K\ur)=0$. 
%\end{thm}
%\begin{proof}[Proof 1]
%Use surjectivity of norm (Serre, V.4.7).
%\end{proof}
%\begin{proof}[Proof 2]
%Calculate the Brauer group algebraically (Serre, XII). We use two lemmas.
%\begin{lem}
%Suppose $D$ is a central division algebra of rank $n^2>1$ over $K$, and $k$ is perfect. Then there exists a  commutative subfield $L$ of $D$ properly containing $K$, unramified over $K$.
%\end{lem}
%\begin{lem}
%Keep the hypotheses. There is a subfield of $D$ of degree $n$ (so it is a maximal subfield) unramified over $K$.
%\end{lem}
%Suppose $D$ is a central division algebra over $K\ur$ of rank $n^2$. Then lemma 2 furnishes a subfield of $K\ur$ of degree $n$, unramified over $K\ur$. Hence $n=1$, and $D$ is trivial. Thus $\Br_{K\ur}=0$.
%\end{proof}
\section{Problems}
\begin{enumerate}
\item[2.1] (Artin-Schreier) Let $L/K$ be a Galois extension of degree $p$, with $K/\F_p$ a finite extension. Prove that $L=K(\al)$ for some $\al$ such that $\al^p-\al\in K$. (Hint: Consider a short exact sequence as in the proof of Kummer theory. However, use the map $x\mapsto x^p-x$ instead of $x\mapsto x^p$, and consider additive instead of multiplicative groups.)
\end{enumerate}
%\begin{comment}
%\section{Lubin-Tate Theory}
%\subsection{Formal groups}
%\begin{df}
%A \textbf{formal group} $\mathscr F$ over the ring $R$ is a power series $F(X,Y)\in R[[X,Y]]$ satisfying
%\begin{enumerate}
%\item
%$F(X,Y)=X+Y+G(X,Y)$ where all terms of $G$ have degree at least 2.
%\item (Associativity)
%$F(F(X,Y),Z)=F(X,F(Y,Z))$.
%\item (Commutativity)
%$F(X,Y)=F(Y,X)$.
%\item (Inverse) There is a unique power series $i(T)\in R[[T]]$ such that $F(T,i(T))=0$.
%\item (Identity) $F(X,0)=F(0,X)=X$.
%\end{enumerate}
%(Note items 1-2 imply 4-5.)
%\end{df}
%\begin{df}
%A homomorphism from $(\mathscr F,F)$ to $(\mathscr G,G)$ is a power series $f(T)\in R[[T]]$ with
%\[
%f(F(X,Y))=G(f(X),f(Y)).
%\]
%$\mathscr F$ and $\mathscr G$ are isomorphic over $R$ if there are homomorphisms $f:\mathscr F\to \mathscr G$ and $g:\mathscr G\to \mathscr F$ such that $f(g(T))=g(f(T))=T$.
%\end{df}
%\begin{ex}
%The formal additive and multiplicative groups are given by
%\begin{align*}
%F(X,Y)&=X+Y\\
%F(X,Y)&=X+Y+XY=(1+X)(1+Y)-1.
%\end{align*}
%\end{ex}
%\end{comment}