\def\filepath{templates}
%\def\filepath{C:/Users/holden-lee/Dropbox/Math/templates}
%\def\filepath{G:/transfer/Math/templates}

\documentclass[12pt]{book}
\usepackage{etex}
%this prevents the "No room for a new \dimen" error that comes from loading too many packages (tikz+xy in particular)

%load geometry first (sets up page)
\usepackage[top=1.2in, bottom=1.2in, left=1in, right=1in]{geometry}

%main packages
\usepackage{amscd}
\usepackage{amsmath}
\usepackage{amssymb}
\usepackage{amsthm}
\usepackage{array}
\usepackage{bbm}
%\usepackage{asymptote}
\usepackage{cancel}
\usepackage{chemarrow}
\usepackage{cmap}
\usepackage{courier}
\usepackage[usenames,dvipsnames]{color}%%%
%\usepackage{color}
%\usepackage{ctable} %You must load ctable after tikz.
\usepackage{enumerate}
\usepackage{enumitem}%resume lists
\usepackage{fancyhdr}
\usepackage{listings}
\lstset{
	basicstyle=\small\ttfamily,
	keywordstyle=\color{blue},
	language=python,
	xleftmargin=16pt,
}
\usepackage{makeidx}
%\usepackage{marvosym}%doesn't work
\usepackage{mathdots}%iddots: dots going northeast
\usepackage{mathtools}
\usepackage{mathrsfs}
%\usepackage{hyperref}
%\usepackage{sidecap}
\usepackage{stackrel}
\usepackage{stmaryrd}%\mapsfrom
\usepackage{tabularx}
\usepackage{tikz}
\usepackage{ctable}
\usepackage{titlesec}
\usepackage{titletoc}
\usepackage{url}
\usepackage{verbatim}
\usepackage{wasysym}
\usepackage{wrapfig}
\usepackage{yhmath}
%\usepackage{yhmath}%arcs
\usepackage[all,cmtip]{xy}%Commutative diagrams
%\usepackage[usenames,dvipsnames]{xcolor}%tikz loads xcolor


\usepackage[usenames,dvipsnames]{color} % Required for specifying custom colors and referring to colors by name
\usepackage[pdftex]{hyperref} % For hyperlinks in the PDF
\hypersetup{
  colorlinks=true,
  linkcolor=MyBlue, 
  citecolor=MyRed,
  urlcolor= MyBlue
}

\definecolor{MyRed}{rgb}{0.99, 0.0, 0.0} 
\definecolor{MyGreen}{rgb}{0.0,0.4,0.0} 
\definecolor{MyBlue}{rgb}{0.0, 0.0, 0.6}

%load hyperref last
%\usepackage{hyperref}
%\usepackage{listings}
%\lstset{
%	basicstyle=\small\ttfamily,
%	keywordstyle=\color{blue},
%	language=python,
%	xleftmargin=16pt,
%}
%this causes an error. No idea why.

\usetikzlibrary{calc,trees,positioning,arrows,chains,shapes.geometric,%
    decorations.pathreplacing,decorations.pathmorphing,shapes,%
    matrix,shapes.symbols,shadows,fadings}

%\input xy
%\xyoption{all}

%http://www.simonsilk.com/content/simonsilk/2011-jun/latex-list-notations-nomenclature
\usepackage[refpage]{nomencl}
\renewcommand{\nomname}{List of Notations}
\renewcommand*{\pagedeclaration}[1]{\unskip\dotfill\hyperpage{#1}}
\makenomenclature
%The first line invokes the nomenclature package, and the option refpage means that the list will include, for each symbol in the list,  the page number on which you added it with the \nomenclature command. Leave it out to remove page numbers. The second line is the title at the top of the list of notations. The third line changes the page numbers in the list so they are right-justified with a line of dots connecting them back to the description of the symbol. By default, they follow the description after a comma and the word "page." The last line tells Latex you're using nomenclature so it will generate and look for the associated intermediate files during successive runs.

\makeindex

\setcounter{tocdepth}{3}
\setcounter{secnumdepth}{3}
%\pagenumbering{arabic}

%http://tex.stackexchange.com/questions/142982/how-to-get-the-current-chapter-name-section-name-subsection-name-etc?lq=1
\usepackage{etoolbox}
% Patch the sectioning commands to provide a hook to be used later
\preto{\chapter}{\def\leveltitle{\chaptertitle}}
\preto{\section}{\def\leveltitle{\sectiontitle}}
\preto{\subsection}{\def\leveltitle{\subsectiontitle}}
\preto{\subsubsection}{\def\leveltitle{\subsubsectiontitle}}

\makeatletter
% \@sect is called with normal sectioning commands
% Argument #8 to \@sect is the title
% Thus \section{Title} will do \gdef\sectiontitle{Title}
\pretocmd{\@sect}
  {\expandafter\gdef\leveltitle{#8}}
  {}{}
% \@ssect is called with *-sectioning commands
% Argument #5 to \@ssect is the title
\pretocmd{\@ssect}
  {\expandafter\gdef\leveltitle{#5}}
  {}{}
% \@chapter is called by \chapter (without *)
% Argument #2 to \@chapter is the title
\pretocmd{\@chapter}
  {\expandafter\gdef\leveltitle{#2}}
  {}{}
% \@schapter is called with \chapter*
% Argument #1 to \@schapter is the title
\pretocmd{\@schapter}
  {\expandafter\gdef\leveltitle{#1}}
  {}{}
\makeatother
%%%%%%%%%%%%%%%%%%%%%%%%%%%%%%%%%%%%%%%%%%%%%%%%%
%%%Theorem styles
\newtheoremstyle{norm}
{12pt}
{12pt}
{}
{}
{\bf}
{:}
{.5em}
{}

\newtheorem{thm}{Theorem}[section]
\newtheorem*{thm*}{Theorem}
\newtheorem{clm}[thm]{Claim}
\newtheorem*{clm*}{Claim}
\newtheorem{conj}[thm]{Conjecture}
\newtheorem*{conj*}{Conjecture}
%\newtheorem{cons}{Construction}
\newtheorem{cor}[thm]{Corollary}
\newtheorem{lem}[thm]{Lemma}
\newtheorem*{lem*}{Lemma}

\theoremstyle{norm}
\newtheorem{prb}[thm]{Problem}%[section]
\newtheorem*{prb*}{Problem}

\newtheorem{alg}[thm]{Algorithm}
\newtheorem{ax}[thm]{Axiom}
\newtheorem*{ax*}{Axiom}
\newtheorem{df}[thm]{Definition}
\newtheorem*{df*}{Definition}
\newtheorem{ex}[thm]{Example}
\newtheorem*{ex*}{Example}
\newtheorem{exr}[thm]{Exercise}
\newtheorem{expl}[thm]{Exploration}%prb
\newtheorem{fct}[thm]{Fact}
\newtheorem{mdl}[thm]{Model}
\newtheorem{pos}[thm]{Postulate}
\newtheorem*{pos*}{Postulate}
%\newtheorem{sprb}{Problem}%numbering for solutions
\newtheorem{pr}[thm]{Proposition}
\newtheorem*{pr*}{Proposition}
\newtheorem{qu}[thm]{Question}
\newtheorem*{qu*}{Question}
\newtheorem{rem}[thm]{Remark}
\newtheorem*{rem*}{Remark}

%%%%%%%%%%%%%%%%%%%%%%%%%%%%%%%%%%%%%%%%%%%%%%%%%

%%%%%%%%%%%%%%%%%%%%%%%%%%%%%%%%%%%%%%%%%%%%%%%%%
%%DEFINING BOX COMMANDS
\newcommand{\prbox}[1]{{
\noindent
\centering
\begin{tikzpicture}
\node [prbox] (box){
\begin{minipage}[l]{6in}
#1
\end{minipage}
};
\end{tikzpicture}\\%
}
}
\newcommand{\prbbox}[1]{\prbox{\begin{prb}#1\end{prb}}}
\newcommand{\expbox}[1]{\prbox{\begin{expl}#1\end{expl}}}
\newcommand{\sprbbox}[1]{\prbox{\begin{sprb}#1\end{sprb}}}

\newcommand{\thbox}[1]{{
\noindent
\centering
\begin{tikzpicture}
\node [thbox] (box){
\begin{minipage}[l]{6in}
#1
\end{minipage}
};
\end{tikzpicture}\\%
}}
%6.68

\newcommand{\thmbox}[1]{\thbox{\begin{thm}#1\end{thm}}}
\newcommand{\dfbox}[1]{\thbox{\begin{df*}#1\end{df*}}}

\newcommand{\grbox}[1]{{
\noindent
\centering
\begin{tikzpicture}
\node [cpbox] (box){
\begin{minipage}[l]{6in}
#1
\end{minipage}
};
\end{tikzpicture}\\%
}}
\newcommand{\cpbox}[1]{{
\noindent
\centering
\begin{tikzpicture}
\node [cpbox] (box){
\begin{minipage}[c]{.2in}
%\hspace{-.2in}
%\dbend
\includegraphics[scale=0.35]{\filepath/key}
\end{minipage}
\begin{minipage}[t]{6in}%6.25
#1
\end{minipage}
};
\end{tikzpicture}\\%
}}


\newcommand{\wrbox}[1]{{
\noindent
\centering
\begin{tikzpicture}
\node [wrbox] (box){
\begin{minipage}[c]{.2in}
%\hspace{-.2in}
%\dbend
{\Huge \bf !}
\end{minipage}
\begin{minipage}[t]{6in}%6.25
#1
\end{minipage}
};
\end{tikzpicture}\\%
}}
\newcommand{\hintbox}[1]{{
\noindent
\centering
\begin{tikzpicture}
\node [hnbox] (box){
\begin{minipage}[l]{6in}
\textbf{Hint:} {#1}
\end{minipage}
};
\end{tikzpicture}\\%
}
}
%white box
\newcommand{\whbox}[1]{{
\noindent
\centering
\begin{tikzpicture}
\node [hnbox] (box){
\begin{minipage}[l]{6in}
{#1}
\end{minipage}
};
\end{tikzpicture}\\%
}
}
%6.68
%END DEFINING BOX COMMANDS
%%%%%%%%%%%%%%%%%%%%%%%%%%%%%%%%%%%%%%%%%%%%%%%%%

%Box commands
%\thbox{...} makes a theorem box
%\thmbox{...} makes a theorem box labeled "Theorem"
%\prbox{...} makes a problem box
%\prbbox{...} makes a problem box labeled "Problem"
%\cpbox{...} makes a concept box labeled "Concept"
%\wrbox{...} makes a warning box.
%\dfbox{...} makes a definition box (same as problem box) labeled "> Definition"
%\sprbbox{...} makes a problem box labeled "Problem" (use this when you're making a copy of a previous problem box to put the solution afterwards; it has a numbering system separate from problems).

 
% for testing purposes only
\usepackage[english]{babel} 
\usepackage{blindtext} 
%%%%%%%%
%begin doc
%%%%%%%%
%%%%%%%%%%%%%%%%%%%%%%%%%%%%%%%%%%%%%%%%%%%%%%%%%
%DEFINING THE BOXES.
% Problem box (simple box, blue)
\tikzstyle{prbox} = [draw=black, fill=blue!20, very thick,
    rectangle, inner sep=10pt, inner ysep=10pt]
% Theorem box (double border, gray)
\tikzstyle{thbox} = [draw=black,double, fill=blue!10, very thick,
    rectangle, inner sep=10pt, inner ysep=10pt]
%\tikzstyle{fancytitle} =[fill=red, text=white]
% Concept box (shadowed, light green)
\tikzstyle{cpbox} = [drop shadow={
    shadow scale=1}, draw=black, fill=green!10, very thick,
    rectangle, inner sep=10pt, inner ysep=10pt]
\tikzstyle{wrbox} = [drop shadow={
    shadow scale=1}, draw=black, fill=yellow!10, very thick,
    rectangle, inner sep=10pt, inner ysep=10pt]
\tikzstyle{hnbox} = [draw=black, fill=white, very thick,
    rectangle, inner sep=10pt, inner ysep=10pt]
%END DEFINING BOXES
%%%%%%%%%%%%%%%%%%%%%%%%%%%%%%%%%%%%%%%%%%%%%%%%%

%LETTERS

%UPPERCASE
\newcommand{\sA}[0]{\mathscr{A}}
\newcommand{\cA}[0]{\mathcal{A}}
\newcommand{\A}[0]{\mathbb{A}}
\newcommand{\cB}[0]{\mathcal{B}}
\newcommand{\C}[0]{\mathbb{C}}
\newcommand{\sC}[0]{\mathcal{C}}%deprecated
\newcommand{\cC}[0]{\mathcal{C}}
\newcommand{\sD}[0]{\mathscr{D}}
\newcommand{\mD}[0]{\mathfrak D}
\newcommand{\cE}[0]{\mathscr{E}}%deprecated
\newcommand{\sE}[0]{\mathscr{E}}
\newcommand{\E}[0]{\mathbb{E}}
\newcommand{\EE}[0]{\mathop{\mathbb E}}
%\scalebox{1.25}{$\mathbb E$}}
\newcommand{\F}[0]{\mathbb{F}}
\newcommand{\cF}[0]{\mathscr{F}}%deprecated.
\newcommand{\sF}[0]{\mathscr{F}}
\newcommand{\G}[0]{\mathbb{G}}
\newcommand{\cG}[0]{\mathscr{G}}%deprecated.
\newcommand{\sG}[0]{\mathscr{G}}
\newcommand{\cH}[0]{\mathscr H}%deprecated
\newcommand{\sH}[0]{\mathscr H}
\newcommand{\Hq}[0]{\mathbb{H}}
\newcommand{\bfI}[0]{\mathbf{I}}
\newcommand{\I}[0]{\mathbb{I}}
\newcommand{\cI}[0]{\mathscr{I}}%deprecated
\newcommand{\sI}[0]{\mathscr{I}}
\newcommand{\cJ}[0]{\mathscr{J}}
\newcommand{\cL}[0]{\mathscr{L}}
\newcommand{\N}[0]{\mathbb{N}}
\newcommand{\fN}[0]{\mathfrak{N}}
\newcommand{\cP}[0]{\mathcal{P}}
\newcommand{\Pj}[0]{\mathbb{P}}
\newcommand{\mP}[0]{\mathfrak{P}}
\newcommand{\mQ}[0]{\mathfrak{Q}}
%!
\newcommand{\cO}[0]{\mathcal{O}}
\newcommand{\sO}[0]{\mathscr{O}}
\newcommand{\Q}[0]{\mathbb{Q}}
\newcommand{\R}[0]{\mathbb{R}}
\newcommand{\bS}[0]{\mathbb{S}}
\newcommand{\T}[0]{\mathbb{T}}
\newcommand{\X}[0]{\mathfrak{X}}
\newcommand{\Z}[0]{\mathbb{Z}}
\newcommand{\one}[0]{\mathbbm{1}}
%lowercase
\newcommand{\mba}[0]{\mathbf{a}}%idele a
\newcommand{\ma}[0]{\mathfrak{a}}%ideal a
\newcommand{\mb}[0]{\mathfrak{b}}
\newcommand{\mc}[0]{\mathfrak{c}}
\newcommand{\mfd}[0]{\mathfrak d}
\newcommand{\mf}[0]{\mathfrak{f}}
\newcommand{\fg}[0]{\mathfrak{g}}
\newcommand{\vi}[0]{\mathbf{i}}%vector i
\newcommand{\vj}[0]{\mathbf{j}}
\newcommand{\vk}[0]{\mathbf{k}}
\newcommand{\mm}[0]{\mathfrak{m}}%ideal m
\newcommand{\mfp}[0]{\mathfrak{p}}
\newcommand{\mq}[0]{\mathfrak{q}}
\newcommand{\mr}[0]{\mathfrak{r}}
\newcommand{\mv}[0]{\mathbf{v}}
\newcommand{\mx}[0]{\mathbf{x}}%idele x
\newcommand{\my}[0]{\mathbf{y}}
%More sequences of letters
\newcommand{\et}[0]{_{\text{\'et}}}
\newcommand{\Gm}[0]{\mathbb{G}_m}
\newcommand{\Fp}[0]{\mathbb{F}_p}
\newcommand{\fq}[0]{\mathbb{F}_q}
\newcommand{\fpb}[0]{\ol{\mathbb{F}_p}}
\newcommand{\fpt}[0]{\mathbb{F}_p^{\times}}
\newcommand{\fqt}[0]{\mathbb{F}_q^{\times}}
\newcommand{\Kt}[0]{K^{\times}}
\newcommand{\RP}[0]{\mathbb{R}P}
\newcommand{\qb}[0]{\ol{\mathbb{Q}}}
\newcommand{\qp}[0]{\mathbb{Q}_p}
\newcommand{\qpb}[0]{\ol{\mathbb{Q}_p}}
\newcommand{\ql}[0]{\mathbb Q_{\ell}}
\newcommand{\sll}[0]{\mathfrak{sl}}
\newcommand{\vl}[0]{V_{\ell}}
\newcommand{\zl}[0]{\mathbb Z_{\ell}}
\newcommand{\zp}[0]{\mathbb Z_{p}}
\newcommand{\zpz}[0]{\mathbb Z/p\mathbb Z}
%Shortcuts for greek letters
\newcommand{\al}[0]{\alpha}
\newcommand{\be}[0]{\beta}
\newcommand{\ga}[0]{\gamma}
\newcommand{\Ga}[0]{\Gamma}
\newcommand{\de}[0]{\delta}
\newcommand{\De}[0]{\Delta}
\newcommand{\ep}[0]{\varepsilon}
\newcommand{\eph}[0]{\frac{\varepsilon}{2}}
\newcommand{\ept}[0]{\frac{\varepsilon}{3}}
\newcommand{\ka}[0]{\kappa}
\newcommand{\la}[0]{\lambda}
\newcommand{\La}[0]{\Lambda}
\newcommand{\ph}[0]{\varphi}
\newcommand{\Ph}[0]{\Phi}
\newcommand{\rh}[0]{\rho}
\newcommand{\te}[0]{\theta}
\newcommand{\Te}[0]{\Theta}
\newcommand{\vte}[0]{\vartheta}
\newcommand{\om}[0]{\omega}
\newcommand{\Om}[0]{\Omega}
\newcommand{\si}[0]{\sigma}
\newcommand{\Si}[0]{\Sigma}
\newcommand{\ze}[0]{\zeta}

%SYMBOLS

%Shortcuts for symbols
\newcommand{\nin}[0]{\not\in}
\newcommand{\opl}[0]{\oplus}
\newcommand{\bigopl}[0]{\bigoplus}
\newcommand{\ot}[0]{\otimes}
\newcommand{\bigot}[0]{\bigotimes}
\newcommand{\sub}[0]{\subset}
\newcommand{\nsub}[0]{\not\subset}
\newcommand{\subeq}[0]{\subseteq}
\newcommand{\supeq}[0]{\supseteq}
\newcommand{\nsubeq}[0]{\not\subseteq}
\newcommand{\nsupeq}[0]{\not\supseteq}
\newcommand{\nequiv}[0]{\not\equiv}
\newcommand{\bs}[0]{\backslash}
\newcommand{\iy}[0]{\infty}
\newcommand{\no}[0]{\trianglelefteq}%normal subgroup
\newcommand{\qeq}[0]{\stackrel?=}
\newcommand{\subsim}[0]{\stackrel{\sub}{\sim}}
\newcommand{\supsim}[0]{\stackrel{\supset}{\sim}}
\newcommand{\na}[0]{\nabla}
\newcommand{\mcv}[0]{*_-}
\newcommand{\dlim}[0]{\varinjlim}
\newcommand{\ilim}[0]{\varprojlim}
\newcommand{\gle}[0]{\begin{array}{c}
\ge\\
\le
\end{array}}
\newcommand{\lge}[0]{\begin{array}{c}
\le\\
\ge
\end{array}}
%\stackrel{-}{*}}%minus convoluion.

%COMMON TIME-SAVERS

%Fractions
\newcommand{\rc}[1]{\frac{1}{#1}}
\newcommand{\prc}[1]{\pa{\rc{#1}}}
\newcommand{\ff}[2]{\left\lfloor\frac{#1}{#2}\right\rfloor}
\newcommand{\cf}[2]{\left\lceil\frac{#1}{#2}\rceil\rfloor}
\newcommand{\fc}[2]{\frac{#1}{#2}}
\newcommand{\sfc}[2]{\sqrt{\frac{#1}{#2}}}
\newcommand{\pf}[2]{\pa{\frac{#1}{#2}}}%Shortcut for fraction with parentheses
\DeclareRobustCommand{\stl}{\genfrac\{\}{0pt}{}}

\newcommand{\pd}[2]{\frac{\partial #1}{\partial #2}}%Partial derivatives
\newcommand{\dd}[2]{\frac{d #1}{d #2}}
\newcommand{\pdd}[1]{\frac{\partial}{\partial #1}}
\newcommand{\ddd}[1]{\frac{d}{d #1}}
\newcommand{\af}[2]{\ab{\fc{#1}{#2}}}
\newcommand{\ddt}[2]{\frac{d^2 #1}{d {#2}^2}}
\newcommand{\pdt}[2]{\frac{\partial^2 #1}{\partial {#2}^2}}
\newcommand{\pdxy}[3]{\frac{\partial^2 #1}{\partial {#2}\partial {#3}}}
\newcommand{\pl}[0]{\partial}
\newcommand{\nb}[0]{\nabla}
\newcommand{\onb}[0]{\ol{\nabla}}
\newcommand{\dy}{\,dy}
\newcommand{\dx}{\,dx}

%Arrows
\newcommand{\lar}[0]{\leftarrow}
\newcommand{\ra}[0]{\rightarrow}
\newcommand{\dra}[0]{\dashrightarrow}
\newcommand{\lra}[0]{\leftrightarrow}
\newcommand{\rra}[0]{\twoheadrightarrow}
\newcommand{\hra}[0]{\hookrightarrow}
\newcommand{\tra}[0]{\twoheadrightarrow}
\newcommand{\send}[0]{\mapsto}
\newcommand{\xra}[1]{\xrightarrow{#1}}
\newcommand{\xla}[1]{\xleftarrow{#1}}
\newcommand{\xhra}[1]{\xhookrightarrow{#1}}
\newcommand{\xlra}[1]{\xleftrightarrow{#1}}
\newcommand{\xrc}[0]{\xrightarrow{\cong}}
\renewcommand{\cir}[0]{\circlearrowright}
\newcommand{\cil}[0]{\circlearrowleft}

%Brackets
\newcommand{\ab}[1]{\left| {#1} \right|}
\newcommand{\an}[1]{\left\langle {#1}\right\rangle}
\newcommand{\ba}[1]{\left[ {#1} \right]}
\newcommand{\bc}[1]{\left\{ {#1} \right\}}
\newcommand{\bra}[1]{\langle{#1}|}
\newcommand{\braa}[1]{\langle\langle{#1}|}
\newcommand{\ce}[1]{\left\lceil {#1}\right\rceil}
\newcommand{\fl}[1]{\left\lfloor {#1}\right\rfloor}
\newcommand{\ro}[1]{\left\lfloor {#1}\right\rceil}
\newcommand{\ket}[1]{|{#1}\rangle}
\newcommand{\kett}[1]{|{#1}\rangle\rangle}
\newcommand{\bk}[2]{\langle{#1}|{#2}\rangle}
\newcommand{\braak}[2]{\langle\langle{#1}|{#2}\rangle\rangle}
\newcommand{\kbb}[1]{|{#1}\rangle\langle{#1}|}
\newcommand{\kb}[2]{|{#1}\rangle\langle{#2}|}
\newcommand{\kettb}[2]{|{#1}\rangle\rangle\langle\langle{#2}|}
\newcommand{\pa}[1]{\left( {#1} \right)}
\newcommand{\pat}[1]{\left( \text{#1} \right)}
\newcommand{\ve}[1]{\left\Vert {#1}\right\Vert}
\newcommand{\nv}[1]{\frac{#1}{\left\Vert {#1}\right\Vert}}
\newcommand{\nvl}[2]{\frac{#1}{\left\Vert {#1}\right\Vert_{#2}}}
\newcommand{\rb}[1]{\left.{#1}\right|}
\newcommand{\nl}[1]{\left\Vert #1 \right\Vert_{L^1}}
\newcommand{\ad}[0]{|\cdot|}
\newcommand{\ved}[0]{\ve{\cdot}}
\newcommand{\set}[2]{\left\{{#1}:{#2}\right\}}
\newcommand{\sett}[2]{\left\{\left.{#1}\right|{#2}\right\}}

%under and overlines, etc.
\newcommand{\ch}[1]{\check{#1}}
\newcommand{\dch}[1]{\check{\check{#1}}}
\newcommand{\olra}[1]{\overleftrightarrow{#1}}
\newcommand{\ol}[1]{\overline{#1}}
\newcommand{\ul}[1]{\underline{#1}}
\newcommand{\ub}[2]{\underbrace{#1}_{#2}}
\newcommand{\ora}[1]{\overrightarrow{#1}}
\newcommand{\ura}[1]{\underrightarrow{#1}}
\newcommand{\wt}[1]{\widetilde{#1}}
\newcommand{\wh}[1]{\widehat{#1}}

%other
\newcommand{\fp}[1]{^{\underline{#1}}}%Falling power
\newcommand{\rp}[1]{^{\overline{#1}}}
\newcommand{\pp}[1]{^{(#1)}}
\newcommand{\sh}[0]{^{\sharp}}
\newcommand{\ri}[0]{\ddagger}%replace this with rosati involution
\newcommand{\bit}[0]{\{0,1\}}

%TEXT
\newcommand{\btih}[1]{\text{ by the induction hypothesis{#1}}}
\newcommand{\bwoc}[0]{by way of contradiction}
\newcommand{\by}[1]{\text{by~(\ref{#1})}}
\newcommand{\idk}[0]{{\color{red}I don't know.} }
\newcommand{\ore}[0]{\text{ or }}
\newcommand{\wog}[0]{ without loss of generality }
\newcommand{\Wog}[0]{ Without loss of generality }
\newcommand{\step}[1]{\noindent{\underline{Step {#1}:}}}
\newcommand{\prt}[1]{\noindent{\underline{Part {#1}:}}}
\newcommand{\tfae}[0]{ the following are equivalent}
\newcommand{\fabfm}[0]{for all but finitely many }

%FUNCTIONS
%Functions, etc.
\newcommand{\Abs}{(\text{Ab}/S)}
\newcommand{\abk}{(\text{Ab}/k)}
\newcommand{\abc}{(\text{Ab}/\C)}
\newcommand{\Alb}{\operatorname{Alb}}
\newcommand{\Ann}{\operatorname{Ann}}
\newcommand{\Area}{\operatorname{Area}}
\newcommand{\amax}{\operatorname{argmax}}
\newcommand{\amin}{\operatorname{argmin}}
\newcommand{\Ass}{\operatorname{Ass}}
\newcommand{\Art}{\operatorname{Art}}
\newcommand{\Aut}{\operatorname{Aut}}
\newcommand{\avg}{\operatorname{avg}}
\newcommand{\bias}[0]{\operatorname{bias}}
\newcommand{\Binom}{\operatorname{Binom}}
\newcommand{\Bl}[0]{\operatorname{Bl}}
\newcommand{\Br}{\operatorname{Br}}
\newcommand{\chr}{\operatorname{char}}
\newcommand{\cis}{\operatorname{cis}}
\newcommand{\cl}{\operatorname{Cl}}
%\newcommand{\Cl}{C}%changed notation%this is confusing
\newcommand{\Cl}{\operatorname{Cl}}
\newcommand{\Coh}{\operatorname{Coh}}
\newcommand{\Coinf}[0]{\operatorname{Coinf}}
\newcommand{\Coind}[0]{\operatorname{Coind}}
\newcommand{\coker}{\operatorname{coker}}
\newcommand{\colim}{\operatorname{colim}}
\newcommand{\Comp}{\operatorname{Comp}}
\newcommand{\conv}{\operatorname{conv}}
\newcommand{\oconv}{\ol{\conv}}
\newcommand{\Cor}{\operatorname{Cor}}
\newcommand{\Cov}{\operatorname{Cov}}
\newcommand{\cro}{\operatorname{cr}}
\newcommand{\degs}[0]{\deg_{\text{s}}}%separable degree
\newcommand{\Dec}{\operatorname{Dec}}
\newcommand{\depth}{\operatorname{depth}}
\newcommand{\Der}{\operatorname{Der}}
\newcommand{\diag}{\operatorname{diag}}
\newcommand{\diam}{\operatorname{diam}}
\renewcommand{\div}{\operatorname{div}}
\newcommand{\Dir}{\operatorname{Dir}}
\newcommand{\Div}{\operatorname{Div}}
\newcommand{\Disc}{\operatorname{Disc}}%discrepancy
\newcommand{\disc}{\operatorname{disc}}%discriminant
\newcommand{\dom}{\operatorname{dom}}
\newcommand{\Ell}{\mathcal{E}{\rm{ll}}}
\newcommand{\Enc}{\operatorname{Enc}}
\newcommand{\End}{\operatorname{End}}
\newcommand{\Ent}{\operatorname{Ent}}
\newcommand{\Ext}{\operatorname{Ext}}
\newcommand{\Exp}{\operatorname{Exp}}
\newcommand{\err}{\operatorname{err}}
\newcommand{\ess}{\operatorname{ess}}
\newcommand{\ext}{\operatorname{ext}}
\newcommand{\pfcg}{(p\text{-FCGp}/k)}
\newcommand{\fcg}{(\text{FCGp}/k)}
\newcommand{\Frac}{\operatorname{Frac}}
\newcommand{\Frob}{\operatorname{Frob}}
\newcommand{\Fun}{\operatorname{Fun}}
\newcommand{\FS}{\operatorname{FS}}
\newcommand{\Gal}{\operatorname{Gal}}
\newcommand{\grad}{\operatorname{grad}}
\newcommand{\GL}{\operatorname{GL}}
\newcommand{\gps}{(\text{Gp}/S)}
\newcommand{\Hess}{\operatorname{Hess}}
\newcommand{\Het}{H_{\text{\'et}}}
\newcommand{\Hom}{\operatorname{Hom}}
\newcommand{\Homeo}{\operatorname{Homeo}}
\newcommand{\chom}[0]{\mathscr{H}om}
\newcommand{\Homc}{\operatorname{Hom}_{\text{cont}}}
\newcommand{\id}{\mathrm{id}}
\newcommand{\Id}{\operatorname{Id}}
\newcommand{\im}[0]{\text{im}}
\newcommand{\imp}[0]{\im^{\text{pre}}}
\newcommand{\Ind}[0]{\operatorname{Ind}}
\newcommand{\Inf}[0]{\operatorname{Inf}}
\newcommand{\inv}[0]{\operatorname{inv}}
\newcommand{\Isoc}[0]{\operatorname{Isoc}}
\newcommand{\Isog}[0]{\operatorname{Isog}}
\newcommand{\Isom}[0]{\operatorname{Isom}}
\newcommand{\Jac}[0]{\operatorname{Jac}}
\newcommand{\li}[0]{\operatorname{li}}
\newcommand{\Li}[0]{\operatorname{li}}
\newcommand{\Lie}[0]{\operatorname{Lie}}
\newcommand{\Line}[0]{\operatorname{Line}}
\newcommand{\Lip}[0]{\operatorname{Lip}}
\newcommand{\lcm}[0]{\operatorname{lcm}}
\newcommand{\Mat}{\operatorname{Mat}}
\newcommand{\Mod}{\operatorname{Mod}}
\newcommand{\Dmod}{\Mod_{W[F,V]}^{\text{fl}}}
\newcommand{\nil}[0]{\operatorname{nil}}
\newcommand{\nm}{\operatorname{Nm}}
\newcommand{\NS}{\operatorname{NS}}
\newcommand{\ord}{\operatorname{ord}}
\newcommand{\PDer}{\operatorname{PDer}}
\newcommand{\PGL}{\operatorname{PGL}}
\newcommand{\Pic}{\operatorname{Pic}}
\newcommand{\Pico}{\operatorname{Pic}^{\circ}}
\newcommand{\Pois}{\operatorname{Pois}}
\newcommand{\poly}{\operatorname{poly}}
\newcommand{\polylog}{\operatorname{polylog}}
\newcommand{\PSL}{\operatorname{PSL}}
\newcommand{\Per}{\operatorname{Per}}
\newcommand{\perm}{\operatorname{perm}}
\newcommand{\PrePer}{\operatorname{PrePer}}
\newcommand{\Prob}{\operatorname{Prob}}
\newcommand{\Proj}{\operatorname{Proj}}
\newcommand{\Prj}[0]{\textbf{Proj}}
\newcommand{\QCoh}{\operatorname{QCoh}}
\newcommand{\Rad}{\operatorname{Rad}}
\newcommand{\range}{\operatorname{range}}
\newcommand{\rank}{\operatorname{rank}}
\newcommand{\red}{\operatorname{red}}
\newcommand{\Reg}{\operatorname{Reg}}
\newcommand{\Res}{\operatorname{Res}}
\newcommand{\Ric}{\operatorname{Ric}} %Ricci curvature
\newcommand{\rot}{\operatorname{rot}}
\newcommand{\Scal}{\operatorname{Scal}} %Ricci curvature
\newcommand{\schs}{(\text{Sch}/S)}
\newcommand{\schk}{(\text{Sch}/k)}
\newcommand{\se}{\text{se}}
\newcommand{\srank}{\text{srank}}
\newcommand{\hse}{\wh{\text{se}}}
\newcommand{\Sel}{\operatorname{Sel}}
\newcommand{\sens}{\operatorname{sens}}
\newcommand{\Set}{(\text{Set})}
\newcommand{\sgn}{\operatorname{sign}}
\newcommand{\sign}{\operatorname{sign}}
\newcommand{\SL}{\operatorname{SL}}
\newcommand{\SO}{\operatorname{SO}}
\newcommand{\Spec}{\operatorname{Spec}}
\newcommand{\Specf}[2]{\Spec\pa{\frac{k[{#1}]}{#2}}}
\newcommand{\rspec}{\ul{\operatorname{Spec}}}
\newcommand{\Spl}{\operatorname{Spl}}
\newcommand{\spp}{\operatorname{sp}}
\newcommand{\spn}{\operatorname{span}}
\newcommand{\Stab}{\operatorname{Stab}}
\newcommand{\SU}{\operatorname{SU}}
\newcommand{\Supp}{\operatorname{Supp}}
\newcommand{\ssupp}{\operatorname{sing}\operatorname{supp}}
\newcommand{\Sym}{\operatorname{Sym}}
\newcommand{\Th}{\operatorname{Th}}
\newcommand{\Tor}{\operatorname{Tor}}
\newcommand{\tor}{\operatorname{tor}}
\newcommand{\Tr}[0]{\operatorname{Tr}}
\newcommand{\tr}[0]{\operatorname{tr}}
\newcommand{\val}[0]{\text{val}}
\newcommand{\Var}[0]{\operatorname{Var}}
\newcommand{\var}[0]{\text{var}}
\newcommand{\vcong}{\operatorname{vcong}}
\newcommand{\Vol}[0]{\text{Vol}}
\newcommand{\vol}[0]{\text{vol}}
\providecommand{\cal}[1]{\mathcal{#1}}
\renewcommand{\cal}[1]{\mathcal{#1}}
\providecommand{\bb}[1]{\mathbb{#1}}
\renewcommand{\bb}[1]{\mathbb{#1}}

%Text super/subscripts
\newcommand{\abe}[0]{^{\text{ab}}}
\newcommand{\gal}[0]{^{\text{gal}}}%galois closure
\newcommand{\op}{^{\text{op}}}
\newcommand{\pre}[0]{^{\text{pre}}}
\newcommand{\rd}[0]{_{\text{red}}}
\newcommand{\sep}[0]{^{\text{sep}}}
\newcommand{\tp}{^{\text{top}}}
\newcommand{\tors}{_{\text{tors}}}
\newcommand{\ur}[0]{^{\text{ur}}}
\newcommand{\urt}[0]{^{\text{ur}\times}}

%COMMUTATIVE DIAGRAMS

%Commutative diagram shortcuts
\newcommand{\commsq}[8]{\xymatrix{#1\ar[r]^-{#6}\ar[d]_{#5} &#2\ar[d]^{#7} \\ #3 \ar[r]^-{#8} & #4}}
%Makes a diagram like this
%1->2
%|    |
%3->4
%Arguments 5, 6, 7, 8 on arrows
%  6
%5  7
%  8
\newcommand{\pull}[9]{
#1\ar@/_/[ddr]_{#2} \ar@{.>}[rd]^{#3} \ar@/^/[rrd]^{#4} & &\\
& #5\ar[r]^{#6}\ar[d]^{#8} &#7\ar[d]^{#9} \\}
\newcommand{\back}[3]{& #1 \ar[r]^{#2} & #3}
%Syntax:\pull 123456789 \back ABC
%1=upper left-hand corner
%2,3,4=arrows from upper LH corner, going down, diagonal, right
%5,6,7=top row (6 on arrow)
%8,9=middle rows (on arrows)
%A,B,C=bottom row
%composition
\newcommand{\cmp}[9]{
\xymatrix{
#1 \ar[r]^{#4}{#5} \ar@/_2pc/[rr]^{#8}_{#9} & #2 \ar[r]^{#6}_{#7} & #3
}
}
\newcommand{\ctr}[9]{
\xymatrix{
#1 \ar[rr]^{#4}_{#5}\ar[rd]^{#6}_{#7} && #2\ar[ld]_{#8}^{#9}\\
& #3 &
}
}
\newcommand{\ctrr}[9]{
\xymatrix{
#1 \ar[rr]^{#4}_{#5} && #2\\
& #3 \ar[lu]_{#6}^{#7}\ar[ru]_{#8}^{#9}&
}
}

\newcommand{\ctri}[9]{
\xymatrix{
#1 \ar[rr]^{#4}_{#5}\ar[rd]^{#6}_{#7} && #2\\
& #3 \ar[ru]^{#8}_{#9}&
}
}
\newcommand{\rcommsq}[8]{\xymatrix{#1 &#2\ar[l]_-{#6} \\ #3 \ar[u]^{#5} & #4 \ar[u]_{#7}\ar[l]_-{#8}}}


%Arrow shortcuts
\newcommand{\ha}[1]{\ar@{^(->}[#1]}
\newcommand{\ls}[1]{\ar@{-}[#1]}
\newcommand{\sj}[1]{\ar@{->>}[#1]}
\newcommand{\aq}[1]{\ar@{=}[#1]}
\newcommand{\acir}[1]{\ar@{}[#1]|-{\textstyle{\circlearrowright}}}
\newcommand{\acil}[1]{\ar@{}[#1]|-{\textstyle{\circlearrowleft}}}
\newcommand{\ard}[1]{\ar@{.>}[#1]}
\newcommand{\mt}[1]{\ar@{|->}[#1]}
\newcommand{\inm}[1]{\ar@{}[#1]|-{\in}}
\newcommand{\inr}{\ar@{}[d]|-{\rotatebox[origin=c]{-90}{$\in$}}}
\newcommand{\inl}{\ar@{}[u]|-{\rotatebox[origin=c]{90}{$\in$}}}

%Other
%\newcommand{\set}[2]{\left.\left\{{#1}\right|{#2}\right\}}
%\newcommand{\sett}[2]{\left\{{#1}\left|{#2}\right\}\right.}

%SUMS, ETC.
\newcommand{\sumr}[2]{\sum_{\scriptsize \begin{array}{c}{#1}\\{#2}\end{array}}}%sum with 2 rows
\newcommand{\prr}[2]{\prod_{\scriptsize \begin{array}{c}{#1}\\{#2}\end{array}}}%product with 2 rows
\newcommand{\maxr}[2]{\max_{\scriptsize \begin{array}{c}{#1}\\{#2}\end{array}}}%product with 2 rows
\newcommand{\minr}[2]{\min_{\scriptsize \begin{array}{c}{#1}\\{#2}\end{array}}}%product with 2 rows
\newcommand{\trow}[2]{\scriptsize \begin{array}{c}{#1}\\{#2}\end{array}}
\newcommand{\seqo}[3]{\{#3\}_{#1=1}^{#2}}
\newcommand{\seqos}[3]{\{#3_#1\}_{#1=1}^{#2}}
\newcommand{\seqz}[3]{\{#3\}_{#1=0}^{#2}}
\newcommand{\seqzs}[3]{\{#3_#1\}_{#1=0}^{#2}}
\newcommand{\su}[0]{\sum_{n=0}^{\iy}}
\newcommand{\suo}[0]{\sum_{n=1}^{\iy}}
\newcommand{\sumo}[2]{\sum_{#1=1}^{#2}}
\newcommand{\sumz}[2]{\sum_{#1=0}^{#2}}
\newcommand{\prodo}[2]{\prod_{#1=1}^{#2}}
\newcommand{\prodz}[2]{\prod_{#1=0}^{#2}}
\newcommand{\oplo}[2]{\bigopl_{#1=1}^{#2}}
\newcommand{\cupo}[2]{\bigcup_{#1=1}^{#2}}
\newcommand{\capo}[2]{\bigcap_{#1=1}^{#2}}
\newcommand{\iiy}[0]{\int_0^{\infty}}
\newcommand{\iny}[0]{\int_{-\infty}^{\infty}}
\newcommand{\iiiy}[0]{\int_{-\infty}^{\infty}}%DEPRECATED
\newcommand{\sui}[0]{\sum_{i=1}^{n}}
\newcommand{\suiy}[0]{\sum_{i=1}^{\iy}}
\newcommand{\cui}[0]{\bigcup_{i=1}^n}
\newcommand{\suj}[0]{\sum_{j=1}^{n}}
\newcommand{\suij}[0]{\sum_{i,j}}
\newcommand{\limn}[0]{\lim_{n\to \infty}}

%MATRICES

%Matrices
\newcommand{\coltwo}[2]{
\begin{pmatrix}
{#1}\\
{#2}
\end{pmatrix}}
\newcommand{\colthree}[3]{
\begin{pmatrix}
{#1}\\
{#2}\\
{#3}
\end{pmatrix}
}
\newcommand{\colfour}[4]{
\begin{pmatrix}
{#1}\\
{#2}\\
{#3}\\
{#4}
\end{pmatrix}
}
\newcommand{\cth}[3]{
\begin{pmatrix}
{#1}\\
{#2}\\
{#3}
\end{pmatrix}
}
\newcommand{\detm}[4]{
\ab{
\begin{matrix}
{#1}&{#2}\\
{#3}&{#4}
\end{matrix}
}}
\newcommand{\sdetm}[4]{
\ab{
\begin{smallmatrix}
{#1}&{#2}\\
{#3}&{#4}
\end{smallmatrix}
}}
\newcommand{\matt}[4]{
\begin{pmatrix}
{#1}&{#2}\\
{#3}&{#4}
\end{pmatrix}
}
\newcommand{\smatt}[4]{
\left(\begin{smallmatrix} 
{#1}&{#2}\\
{#3}&{#4}
\end{smallmatrix}\right)
}
\newcommand{\mattn}[9]{
\begin{pmatrix}
{#1}&{#2}&{#3}\\
{#4}&{#5}&{#6}\\
{#7}&{#8}&{#9}
\end{pmatrix}
}

\newcommand{\bt}[2]{
\left\{\begin{matrix}
\text{#1}\\
\text{#2}
\end{matrix}
\right\}}
\newcommand{\bth}[3]{
\left\{\begin{matrix}
\text{#1}\\
\text{#2}\\
\text{#3}
\end{matrix}
\right\}}

\newcommand{\tcase}[4]{
\begin{cases}
#1 & #2\\
#3 & #4
\end{cases}}
\newcommand{\txcase}[4]{
\begin{cases}
#1 & \text{#2}\\
#3 & \text{#4}
\end{cases}}

\newcommand{\ifoth}[3]{
\begin{cases}
#1 & \text{if }#2\\
#3 & \text{otherwise}
\end{cases}}

%EQUATIONS
\newcommand{\beq}[1]{\begin{equation}\llabel{#1}}
\newcommand{\eeq}[0]{\end{equation}}
\newcommand{\bal}[0]{\begin{align*}}
\newcommand{\eal}[0]{\end{align*}}%this doesn't work; i don't know why
\newcommand{\ban}[0]{\begin{align}}
\newcommand{\ean}[0]{\end{align}}
\newcommand{\ig}[2]{\begin{center}\includegraphics[scale=#2]{#1}\end{center}}
\newcommand{\ign}[2]{\includegraphics[scale=#2]{#1}}

%COURSE-SPECIFIC

%measure theory, analysis
\newcommand{\am}[0]{(a.e., $\mu$)}
\newcommand{\ftla}[1]{\int e^{i\lambda\cdot x}{#1}\,d\lambda}
\newcommand{\lime}[0]{\lim_{\varepsilon\to 0}}

%number theory
\newcommand{\modt}[1]{\,(\text{mod}^{\times}\, {#1})}
\newcommand{\tl}[0]{T_{\ell}}
 %multiplicative subgroup
\newcommand{\zmod}[1]{\Z/{#1}\Z}
\newcommand{\md}[1]{\,(\text{mod }#1)}
\newcommand{\ks}[1]{K(\sqrt{#1})}
\newcommand{\ksq}[1]{K(\sqrt{#1})/K}
\newcommand{\nksq}[1]{\nm_{K(\sqrt{#1})/K}(K(\sqrt{#1})^{\times})}
\newcommand{\mpp}[0]{\mP/\mfp}

%manifolds
\newcommand{\cvd}[0]{\frac{D}{\partial t}}
\newcommand{\cvs}[0]{\frac{D}{\partial s}}
\newcommand{\np}[1]{\nabla_{\pd{}{#1}}}

%algebraic geometry
\newcommand{\av}[0]{A^{\vee}}


%arithmetic combinatorics
\newcommand{\Deq}[1]{\Delta^{\times}_{#1}}

%modular forms
\newcommand{\GLAi}[0]{\text{GL}_2(\mathbb A)_{\infty}}

%noise sensitivity part iii essay
\newcommand{\Dict}[0]{\operatorname{\mathsf{Dict}}}
\newcommand{\Maj}[0]{\operatorname{\mathsf{Maj}}}
\newcommand{\MAJ}[0]{\operatorname{\mathsf{Maj}}}
\newcommand{\Parity}[0]{\operatorname{\mathsf{Parity}}}
\newcommand{\Tribes}[0]{\operatorname{\mathsf{Tribes}}}
\newcommand{\Ns}[0]{\operatorname{NS}}
\newcommand{\DT}[0]{\operatorname{\mathsf{DT}}}
\newcommand{\DNF}[0]{\operatorname{\mathsf{DNF}}}
\newcommand{\CNF}[0]{\operatorname{\mathsf{CNF}}}
\newcommand{\Circuit}[0]{\operatorname{\mathsf{Circuit}}}
\newcommand{\Ac}[0]{\mathsf{AC}}
\newcommand{\EB}[0]{\text{EB}}
%comp complexity
%complexity theory
\newcommand{\ACC}[0]{\operatorname{\mathsf{ACC}}}
\newcommand{\BPP}[0]{\mathsf{BPP}}
\newcommand{\BPTIME}[0]{\operatorname{\mathsf{BPTIME}}}
\newcommand{\CLIQUE}[0]{\operatorname{\mathsf{CLIQUE}}}
\newcommand{\EXP}[0]{\mathsf{EXP}}
\newcommand{\NEXP}[0]{\mathsf{NEXP}}
\newcommand{\NC}[0]{\operatorname{\mathsf{NC}}}
\newcommand{\NP}[0]{\operatorname{\mathsf{NP}}}
\newcommand{\NTIME}[0]{\operatorname{\mathsf{NTIME}}}
\newcommand{\msP}[0]{\operatorname{\mathsf{P}}}
\newcommand{\PCP}[0]{\operatorname{\mathsf{PCP}}}
\newcommand{\SAT}[0]{\mathsf{SAT}}
\newcommand{\SUBEXP}[0]{\mathsf{SUBEXP}}
\newcommand{\TIME}[0]{\operatorname{\mathsf{TIME}}}
\newcommand{\zz}{\textcircled{Z}}
%statistics
\newcommand{\hten}[0]{\widehat{\theta_n}}

%EDITING HELP
%blu: key idea or concept. red: unresolved issue/question. 
\newcommand{\blu}[1]{{\color{blue}#1}}
\newcommand{\grn}[1]{{\color{green}#1}}
\newcommand{\redd}[1]{{\color{red}#1}}
\newcommand{\pur}[1]{{\color{purple}#1}}
\newcommand{\oge}[1]{{\color{orange}#1}}
\newcommand{\concept}[1]{#1}
\newcommand{\fixme}[1]{{\color{red}#1}}
\newcommand{\cary}[1]{{\color{purple}#1}}%deprecated
%short for commentary
\newcommand{\llabel}[1]{\label{#1}\text{\fixme{\tiny#1}}}
\newcommand{\nref}[1]{\ref{#1}}
%%%%%%%%%%%
%Use the above when working on the document, so labels will be displayed by their theorems, and you can write reminders to yourself. Switch to the below commands when publishing.
%%%%%%%%%%%
%\newcommand{\fixme}[1]{}
%\newcommand{\concept}[1]{{\color{blue}#1}}
%\newcommand{\llabel}[1]{\label{#1}}
%%%%%%%%%%%

\newcommand{\itag}[1]{\index{#1}[\##1]}
\newcommand{\ilbl}[1]{[#1]\index{#1}}
\newcommand{\iadd}[1]{\index{#1} #1}
\newcommand{\arxiv}[1]{\url{http://www.arxiv.org/abs/#1}}

%markup
\newcommand{\ivocab}[1]{\index{#1}\textbf{#1}}
\newcommand{\vocab}[1]{\textbf{#1}} %also index automatically.
\newcommand{\summq}[1]{\textbf{Summary Question: }#1}
\newcommand{\keypt}[1]{{\it #1}}
\newcommand{\subprob}[1]{\noindent\textbf{#1}\\}
\newcommand{\qatable}[1]{\begin{center}
    \begin{longtable}{ | p{5.5cm} | p{10cm} |}
#1
\end{longtable}
\end{center}}

%Page breaks in equations
\allowdisplaybreaks[2]

%SHA
\DeclareFontFamily{U}{wncy}{}
    \DeclareFontShape{U}{wncy}{m}{n}{<->wncyr10}{}
    \DeclareSymbolFont{mcy}{U}{wncy}{m}{n}
    \DeclareMathSymbol{\Sh}{\mathord}{mcy}{"58} 
\renewcommand{\thesection}{\arabic{section}}
%% this file is for things that cannot go in the individual chapters, but only
%% in the main document

\newcommand{\rref}[1]{\cref{#1}}
\newcommand{\txtn}[1]{\textnormal{#1}}

\renewcommand{\thethm}{\arabic{chapter}.\arabic{section}.\arabic{thm}}

\def\name{Elliptic Curves}

\def\textline#1{%
  \hbox to \hsize{%
    \vbox{\centering #1}}}%

\def\maketitle{%
  \null
  \pagestyle{empty}%
  \vbox to .9\vsize{%
    \vss
    \vbox to 1\vsize{%
      \vfill
      \textline{{\LARGE \name}
      }
      \vfill
    }%
    \vss
  }
}
\makeatother

\begin{document}
\maketitle

%\input{beginning/intro.tex}
\tableofcontents
\frontmatter

\mainmatter 
\chapter*{Introduction}

\section*{Resources}

These are some notes (in progress) on elliptic curves that I've combined from various sources, including the following classes:
\begin{enumerate}
\item
Andrew Sutherland's course at MIT (18.783) from spring 2012~\url{http://co.mit.edu/18.783} (parts of these notes are from notes scribed by students in the class and edited by Sutherland),
\item
a reading course with Sug Woo Shin at MIT on class field theory and complex multiplication, and
\item
Tom Fisher's course at the University of Cambridge from autumn 2013 (lecture notes at~\url{https://dl.dropboxusercontent.com/u/27883775/math\%20notes/part_iii_elliptic.pdf}; alternate version at~\url{http://www.pancratz.org/notes/Elliptic.pdf}).
\end{enumerate}
I will draw heavily on the following books:
\begin{enumerate}
\item
Silverman, The arithmetic of elliptic cuves~\ref{Si86}.
\item 
Silverman, Advanced topics in the arithmetic of elliptic curves, by Silverman~\ref{Si94}.
\item
Washington, Elliptic curves and cryptography~\ref{Wa08}.
\item
Cox, Primes of the form $x^2+ny^2$~\ref{Co89}.
\end{enumerate}
I would also like to put in material such as from the following:
\begin{enumerate}
\item
Silverman and Hindry, Diophantine Geometry: An Introduction~\ref{HS00}.
\item
Koblitz, Elliptic Curves and Modular Forms~\ref{Ko84}.
\item
Lang, Elliptic functions~\ref{La87}.
\item 
Diamond and Shurman, A First Course in Modular Forms~\ref{DS05}.
\end{enumerate}

\section*{Using these notes}
I eventually want these to be a complete set of notes, but for the time being, see it more as a ``reading guide" or ``road map," to be used as a supplement to textbooks or course material.

When you see things like ``ADD a discussion on ...", take this as a sign that you should be able to discuss the topic in your studies on elliptic curves.

I would like the prerequisites for these notes to be minimal. However, I will assume familiarity with basic things such as exact sequences and equivalence of categories that make many theorems much more compact to state. Currently, I will have to refer a reader elsewhere for the basics of algebraic geometry.

\section*{Philosophy}
(I.e., what I'd like to do differently from texts already out there.)
In these notes I would like to focus on...
\begin{enumerate}
\item
intuition.
How to talk about the material in a non-rigorous way? A strong way of building intuition is connecting to previous topics, even if the analogies are imperfect. I also hope to add in discussion of big-picture questions such as ``why does geometry matter for arithmetic questions?" and ``what the heck do elliptic curves have to do with modular forms?" As a result, take everything outside of formal statements of theorems and proofs with a grain of salt.
\item motivations, and connections.  How would someone come up with the statements or proofs? How are different subtopics related to one another? 
\item
summaries and ``index-carding." What is the most compact way you can remember the topics? What is the big-picture?
\item
road maps. For topics where I do not have notes for yet, what resources are out there? What is the big picture?
\item
problem-solving based learning. Have problems before theorems to get the reader thinking, and possibly derive some of the ideas of the theorems on his/her own.
\item
fun problems. Stray away from the core material occasionally.
\item
algorithms. Elliptic curves is a very computational subject, so it is helpful to learn how to program algorithms involving EC. I'll try to include SAGE code with comments. (For more on SAGE, see~\ref{http://www.sagemath.org/}.)
\end{enumerate}
\section*{Collaboration}
Send me any comments or corrections at holdenlee@alum.mit.edu.
In particular, let me know if you'd like to collaborate on these notes.

%Add in link to philosophy of math notes on tiddlywiki

\pagestyle{fancy}
\chapter{Conics}

Before we study elliptic curves, we gain some geometric experience by studying some more basic curves: conics, which are defined by quadratic equations. We'll see that we can understand all conics, and that they are basically the same geometrically.

First, we'll consider a common equation: the Pythagorean equations.

%\section{Conics}
%We give 3 levels of discussion on Pythagorean triples. Level 1 is just finding some of them and looking for patterns. Level 2 is parameterizing them using number theory. Level 3 is algebraic geometry-based.
\section{Pythagorean triples}

\dfbox{
A \textbf{Pythagorean triple} is a triple of integers $(a,b,c)$ that are the side lengths of a right triangle. Here $a$ and $b$ are lengths of the two legs and $c$ is the length of the hypotenuse.\\

A \textbf{primitive Pythagorean triple} is a Pythagorean triple $(a,b,c)$ where the greatest common divisor of $a$, $b$, and $c$ is 1.
}

%It is nice when Pythagorean triples pop up in geometry problems, because this means we don't have to worry about getting square roots from the Pythagorean formula.

%\subsection{Level 1}
%You may already be familiar with some small Pythagorean triples.
%\[
%(3,\,4,\,5)\quad (5,\,12,\,13).
%\]
%\prbbox{
%Can you produce an infinite number of Pythagorean triples, just from knowing one of these triples?
%}
%
%Yes, if we multiply one of the triples by any integer, we still get a Pythagorean triple.
%\begin{gather*}
%(6,\,8,\,10)\\
%(9,\,12,\,15)\\
%(12,\,16,\,20)\\
%\vdots
%\end{gather*}
%
%
%\prbbox{
%Note how in these two triples, $b$ is 1 less than $c$. Can you find another Pythagorean triple where this is case? Try to find triples in this form where the smallest length is 7 and 9.
%\begin{align*}
%(7,\,?,\,?)\\
%(9,\,?,\,?).
%\end{align*}
%
%Can you give a method to find all Pythagorean triples $(a,b,c)$ with $c=b+1$?
%}
%\vskip0.15in
%We look for triples such that $c=b+1$. We have
%\[
%a^2=c^2-b^2=(b+1)^2-b^2=2b+1
%\]
%so we set
%\[
%b=\frac{a^2-1}{2}.
%\]
%In order for $b$ to be an integer, $a$ must be odd. Then we get the triple 
%\[
%(a,\,b,\,c)=\pa{a,\frac{a^2-1}{2},\frac{a^2+1}{2}}.
%\]
%For example, with $a=7$ and $a=9$ we get
%\[
%(7,\,\frac{7^2-1}{2},\,\frac{7^2+1}{2})=(7,\,24,\,25)\quad (9,\,\frac{9^2-1}{2},\,\frac{9^2+1}{2})=(9,\,40,\,41).
%\]
%
%\prbbox{Are all primitive Pythagorean triples of this form? If not, can you find one that isn't?}
%
%Not all primitive Pythagorean.

\subsection{Number theoretic solution}

%(Taken from AwesomeMath notes. Need to rewrite.)
%
%A triple $(x,y,z)$ of integers is called Pythagorean if
%\begin{equation}
%x^2+y^2=z^2. \label{eq2}
%\end{equation}

\begin{thm}\llabel{thm:pythag}
Any Pythagorean has the form
$$a=(m^2-n^2)k , \ b=2mnk, \ c=(m^2+n^2)k$$
$$a=2mnk, \ b=(m^2-n^2)k,\ c=(m^2+n^2)k, $$
where
\begin{enumerate}
\item $\gcd(m,n)=1, \ \gcd(x,y)=k.$
\item $m,n$ are of different parity.
\item $m>n>0$, $k>0$.
\end{enumerate}
\end{thm}

\begin{proof}
Idea: Rewrite as $a^2=c^2-b^2=(c-b)(c+b)$. Assume $a,b,c$ have no common factor, so $c-b,c+b$ have no common factor except possibly 2; they must each be a square or 2 times a square.
\end{proof}

%\begin{proof}
%Let $\gcd(x,y)=k$. Then $x=ka$, $y=kb$, $\gcd(a,b)=1$. Then
%$k^2(a^2+b^2)=z^2$. We get $k\mid z$ and set $z=kc$. We obtain
%$$a^2+b^2=c^2.$$
%
%Suppose that $a$ is an odd number. Then $b$ is even since otherwise
%$c^2=a^2+b^2 \equiv 2 \pmod{4}$, a contradiction.
%
%Thus $c$ is odd. We have
%$$b^2=(c-a)(c+a), $$
%which is equivalent to
%$$\left( \frac{b}{2}\right)^2=\frac{c-a}{2} \frac{c+a}{2} .$$
%Note that $\gcd\left( \cfrac{c-a}{2},\cfrac{c+a}{2}\right) =1$.
%Otherwise there exists prime $p$ such that $p\mid \cfrac{c-a}{2}$,
%$p\mid \cfrac{c+a}{2}$. We get $p\mid \cfrac{c-a}{2}\pm
%\cfrac{c+a}{2}=c,a$ which implies $p\mid b$, a contradiction. Hence
%$$\frac{c-a}{2}=n^2, \ \frac{c+a}{2}=m^2,\ \frac{b}{2}=mn $$
%and we obtain
%$$c=m^2+n^2, \ a=m^2-n^2, \ b=2mn. $$
%\end{proof}

\subsection{Geometric solution}
We now explore a more geometric way of finding all Pythagorean triples.

\prbbox{
\begin{enumerate}
\item
We can reduce the problem of finding all primitive Pythagorean triples to finding all right triangles whose hypotenuse is 1 and whose legs are rational numbers. Why?
%rational points on the circle $x^2+y^2=1$. Why?

%Place a vertex of the right triangle at the origin. What can you say about the other vertex? Can you restate the problem?
\item
We want to find all rational points on the circle $x^2+y^2=1$. Let's consider a vertical line $\ell$ going through the origin. Let $A$ be the point $(-1,0)$, and $B$ be any other rational point on $x^2+y^2=1$. What can you say about the intersection of $\ol{AB}$ with $\ell$?
\item 
Now suppose we have a rational point on $\ell$, $(0,z)$. Let $B$ be the second intersection of the line going through $A$ and $(0,z)$ with the circle. Find the coordinates of $B$. What can you say about $B$?
\item
You have now found all rational points on the circle $x^2+y^2=1$. Why? Now use this to find all Pythagorean triples.
\end{enumerate}
}

\begin{enumerate}
\item
Given a Pythagorean triple, we can find a right triangle with rational legs and hypotenuse 1 by dividing all lengths by the hypotenuse:
\[
(a,\,b,\,c)\mapsto \pa{\frac ac,\,\frac bc,\,1}
\]

If we're given a right triangle with rational legs and hypotenuse 1, we can get a primitive Pythagorean triple by multiplying through by the least common denominator. This is the unique primitive triple that's a multiple of the original lengths.

These two operations are inverse to each other.

If we place the right triangle with hypotenuse 1 at the origin, its other vertex is on the circle $x^2+y^2=1$. Indeed, this is just the Pythagorean formula.

So we've reduced the problem of finding all Pythagorean triples to \textbf{finding all rational points on $x^2+y^2=1$}. (We just need the points in the first quadrant.)
\item
The line going through 2 points with rational coordinates will be of the form $y=mx+b$, with $m$ and $b$ both rational. Indeed, the line going through $(-1,0)$ and $(r,s)$ is 
\[
y=\frac{s}{r+1}(x+1)=\frac{s}{r+1}x+\frac{s}{r+1}.
\]
This means its intersection with $\ell$ is also rational: $(0,b)=\pa{0,\frac{s}{r+1}}$.
\item
Now we're going the other way, drawing the line through a point on $\ell$ and looking at its intersection with the circle. The line going through $(-1,0)$ and $(1,z)$ has equation
\[
y=z(x+1).
\]
We substitute this into the equation for the circle $x^2+y^2=1$ and get
\begin{align*}
x^2+[z(x+1)]^2&=1\\
x^2+z^2(x+1)^2-1&=0\\
(1+z^2)x^2+2z^2x+(z^2-1)&=0.
\end{align*}
This looks like a rather nasty quadratic. But before we pull out the quadratic formula to solve for $x$, note that this equation represents the intersection points of the line with the circle, and we already know one intersection point -- it is $(-1,0)$, when $x=-1$. The sum of the roots is $-\frac{2z^1}{1+z^2}$ so the other solution is
\[
-\frac{2z^1}{1+z^2}-(-1)=\frac{1-z^2}{1+z^2}.
\]
Then
\[
y=z(x+1)=z\pa{\frac{1-z^2}{1+z^2}+1}=\frac{2z}{1+z^2}.
\]
The second intersection is
\[
\pa{\frac{1-z^2}{1+z^2},\frac{2z}{1+z^2}}.
\]
In particular, since $z$ is rational, it is rational!
\item
We've established a 1-to-1 correspondence between rational points on $\ell$ and rational points on the circle (excluding the point $(-1,0)$). 

Now we simply have to go from rational points on the circle back to Pythagorean triples, as we said in step 1. Write $z=\frac mn$ is lowest terms. We have a right triangle with legs
\[
\pa{\frac{1-\pf mn^2}{1+z\pf mn^2},\frac{\fc mn}{1+\pf mn^2},1}=\pa{\fc{n^2-m^2}{n^2+m^2},\fc{2mn}{m^2+n^2},1}.
\]
Multiplying through by the denominator $m^2+n^2$, we get 
\[
\pa{\frac{1-\pf mn^2}{1+z\pf mn^2},\frac{\fc mn}{1+\pf mn^2},1}=\pa{n^2-m^2,2mn,n^2+m^2}.
\]
Note this is a primitive Pythagorean triple because the greatest common divisor divides $(n^2+m^2)-(n^2-m^2)=2m^2$ and $2mn$. \fixme{Deal with parity.}
\end{enumerate}

%%%%%%%%%%%

\section{General conics}

For general conics, we can do something similar.

\section{Group law}
\fixme{Introduce a group law for conics. For example, this shows how we can take solutions to Pell equations and produce more solutions. This is good for motivating the group law on elliptic curves later.}

See \url{http://www.quora.com/Elliptic-Curves/Why-is-there-a-group-law-on-an-elliptic-curve}.

\prbbox{
Suppose you are given two Pythagorean triples $(a_1,b_1,c_1)$ and $(a_2,b_2,c_2)$. Produce (in a nontrivial way) a Pythagorean triple $(a,b,c)$ with $c=c_1c_2$?}

\section{Hasse-Minkowski}

(It's good to know about how the local-to-global principle works before seeing how it fails in the case of elliptic curves!)

\section{Summary}

You should now be able to find all solutions to any conic over $\Q$ (or prove that it has no solutions).

%%%%%%%%%%%%%%%%%%%%%%%%%%%%%
%%%%%%%%%%%%%%%%%%%%%%%%%%%%%
\chapter{Introduction to algebraic geometry}
We introduce some algebraic geometry that we'll need.

We'll cover the following.
\begin{enumerate}
\item
Varieties (affine and projective), morphisms, and rational maps: Define the basic objects we study in algebraic geometry and maps between them.
\item
Curves: Understand the equivalence of categories between curves and certain field extensions. Talk about degree and ramification of maps between curves.
\item
Divisors
\item Differentials
\item Genus, and the Riemann-Roch Theorem
\end{enumerate}

We assume the reader can do the following. See Silverman~\cite{Si86}[Chapter I-II].
\begin{itemize}
\item
Define affine variety; understand the relationship between ideals of a polynomial ring and varieties.
\item Define projective variety, and how the above relationship is modified in this case. Why do we study projective rather than affine varieties?
\item Understand the local ring at a point.
\item Define dimension and smoothness. (What are the two definitions of smoothness, and when are they equivalent?)
\item Understand ``field of definition" and Galois action.
\item Define morphism and rational map.
\item Optional: understand all the above in terms of schemes.
\end{itemize}

\section{Varieties}
\subsection{Affine varieties}
\subsection{Projective varieties}
\subsection{Morphisms and rational maps}
%\begin{df}\llabel{rem:cam2-9}
%A rational map $C\dra \Pj^n$ is given by
%\[
%P\mapsto (f_0(P):f_1(P):\cdots :f_n(P))
%\]
%where $f_0,f_1,\ldots, f_n\in K(C)$ are not all zero.
%\end{df}
\section{Curves}
\begin{df}
A \textbf{curve} is a projective variety of dimension 1.
\end{df}

\subsection{Curves correspond to field extensions}
Our main result is the following.

\thmbox{\llabel{thm:curves-fields}
%\begin{thm}
There is an contravariant equivalence of categories between the following.
\begin{enumerate}
\item
Objects: Smooth curves defined over $K$

Maps: Non-constant rational maps defined over $K$
\item
Extensions $L/K$ of transendence degree 1 and $L\cal \ol K=K$.

Maps: field injections fixing $K$.
\end{enumerate}
%The equivalence is given by the following.
%\bal
%\pat{Smooth curves}&\to \pat{Field extensions with $\tr\deg_K L=1$}\\
%C/K&\mapsto K(C)\\
%\pat{Non-constant rational maps}&\to \pat{Field injections}\\
%(\phi:C_1\to C_2)&\mapsto (\phi^*:K(C_2)\to K(C_1)).
%\end{align*}
The equivalence is given by sending $C/K$ to $K(C)$ and $\phi:C_1\to C_2$ to $\phi^*:K(C_2)\hra K(C_1)$ with $\phi^*f=f\circ \phi$.
\[
\xymatrix{
C_1/K\ar[r]^{\phi}\ard{d} & C_2/K \ard{d}\\
K(C_1)& \ar[l]^{\phi^*} K(C_2)\\
f\circ \phi&\mt{l} f.
}
\]
}
%\end{thm}

Why do we consider functions and divisors on curves? There are two good motivations, depending on your background:
\begin{enumerate}
\item
Algebraic number theory: Let $K$ be a number field. We know the following.
\begin{enumerate}
\item 
Primes: $K$ has a set of primes. Call it $\Spec \sO_K=\{\mfp\text{ prime in }K\}$.
\item 
Unique factorization: Each fractional ideal $\ma$ in $K$ has a unique factorization. In other words, we can think of the elements of $K$ as functions from $\Spec \sO_K$ to $\Z$ that are zero almost everywhere; the function gives the orders with respect to various primes.
\begin{enumerate}
\item
Discrete valuation: When we localize at $\mfp$, we get a local field $K_{\mfp}$, which has a \textbf{discrete valuation} $\ord_{\mfp}$.
\end{enumerate}
\item
Class group: $K$ has finite \textbf{class group} $\Cl_K$. In other words, the factorizations of elements of $K^{\times}$ is cofinite in the group of all possible factorizations (of ideals),
\beq{eq:Cl-K}
1\to\sO_K^{\times}\to  K^{\times}\to \Id_K\to \Cl_K\to 0.
\eeq
\item 
Field extensions: Three kinds of behavior can result. Namely, a prime can split, remain inert, and ramify. We can define \textbf{ramification} indices, and find they multiply when we have field extensions $K/L$ and $M/K$.
\end{enumerate}
\item
Riemann manifolds: \fixme{Add motivations in}
\end{enumerate}
(We often say algebraic number theory is ``algebraic geometry in dimension 0." For more information, look up Arakelov geometry.) 
Because each curve has an associated field extension, it makes sense to consider analogues of the above concepts. A curve in %$\A^2$ (abusing our definition of ``curve" just for intuition's sake) is associated to some field $\fc{K(x,y)}{\an{f(x,y)}}$, and the elements of this ring are {\it actually} functions.
has some associated function field $\ol K(C)$, and the elements here are {\it actually} functions; we can define discrete valuations when we localize at a point. (Note we have to be careful with the analogy because we're dealing with {\it projective} curves; geometry is really necessary here.)

(Todo: make more precise. See chapter 3 of Ravi Vakil's Algebraic geometry~\cite{Vakil}.)

Here's a partial dictionary.
\bal
\{\text{prime ideals of }K\}&\lra\{\text{points of }C\}\\
\Id_K&\lra \Div(C)\\
\Cl_K&\lra \Pic(C)\\
e_{L/K}&\lra e_{\phi}(C)
\end{align*}

\begin{pr}\llabel{pr:dvp}
Let $C$ be a curve and $P\in C$ a smooth point. Then $\ol K[C]_P$ is a discrete valuation ring.

Thus for each $P$, we have a discrete valuation $\ord_P:K(C)^*\rra \Z$:
\begin{enumerate}
\item
$\ord_P(f_1f_2)=\ord_P(f_1)\ord_P(f_2)$.
\item
$\ord_P(f_1+f_2)\ge \min (\ord_P(f_1),\ord_P(f_2))$.
\end{enumerate}
\end{pr}

\begin{df}
$t\in K(C)$ is a \textbf{uniformizer} at $P$ if $\ord_P(t)=1$.
\end{df}
Because $\ol K[C]_P$ is a DVR, once we've found a uniformizer at $P$, we can then write functions as power series in the uniformizer.

\subsection{Divisors and the Picard group}
\begin{df}
A \textbf{divisor} is a formal sum of points on $C$, \[D=\sum_{P\in C}n_PP\] with $n_P\in \Z$ and $n_P=0$ for all but finitely many $P$. 
\begin{enumerate}
\item
Define the \textbf{degree}
\[
\deg D=\sum_{P\in C} n_P.
\]
\item
$D$ is \textbf{effective} (written $D\ge 0$) if $n_P\ge 0$ for all $P$. 
\item If $f\in K(C)^{\times}$ then define
\[
\div(f):=\sum_{P\in C} \ord_P(f)P.
\]
\end{enumerate}
\end{df}
You can think of divisors as functions from the points of $C$ to $\Z$, just like fractional ideals in $K$ were functions from the primes of $K$ to $\Z$. With this analogy, effective divisors correspond to proper ideals (as opposed to fractional ideals), and the map $\div$ corresponds to the map $K^{\times}\to \Id_K$.

One important fact is that $\div(f)$ always has degree 0. (We don't have this behavior for $K$; this nice fact comes from the fact that we're working with projective varieties. Rational functions have the same number of zeros as poles, so have degree 0; this count only works if we think about the point at infinity.)

We will define $\Div(C)$ similar to $\Cl_K$.
\begin{df}\llabel{df:picard}
Divisors $D_1,D_2\in \Div(C)$ are \textbf{linearly equivalent} (written $D_1\sim D_2$) if there exists $f\in \ol K(C)^{\times}$ with $\div(f)=D_1-D_2$. Write
\[
[D]=\set{D'\in \Div(C)}{D'\sim D}.
\]
Define the \textbf{Picard group}
\begin{align*}
\Pic(C)&=\fc{\Div(C)}{\sim}\\
\Pic^0(C)&=\fc{\Div^0(C)}{\sim}.
\end{align*}
where $\Div^0(C)$ is the group of divisors on $E$ of degree 0. %The size of each fiber is the same. 0 and \iy
\end{df}

We summarize the main result similar to~\eqref{eq:Cl-K}.

\thbox{
\begin{pr}\llabel{pr:div-es}
There is an exact sequence 
\[
1\to \ol K^{\times} \to \ol K(C)^{\times} \xra{\div} \Div^0(C)\to \Pic^0(C)\to 1.
\]
\end{pr}}

\begin{proof}
We need to check that
\begin{enumerate}
\item
$\deg(\div(f))=0$. See the proof after Proposition~\ref{pr:deg-basics}.
\item
If $\div(f)=0$, then $f\in \ol K^{\times}$. 
\fixme{Add.}
%Suppose $f=\fc{g}{h}$. $g$ and $h$ have the same zeros, i.e., $Z(g)=Z(h)$, iff $\sqrt{\an{g}}=\sqrt{\an{h}}$, i.e, $\fc{g^m}{h^n}$ is in $\sO_{\ol K(C)}^{\times}=\ol K^{\times}$, i.e, $\fc{g^m}{h^n}\in K^{\times}$. A zero of $g$ has the same multiplicity as a zero of $h$, 
%But the multiplicities of the zeros in $f$ are not 0 unless $g,h$.
\end{enumerate}
\end{proof}
Projectivity is essential for both these statements.

\subsection{Maps are like field extensions}
Using the fact that a morphism of curves corresponds to a field extension (see Theorem~\ref{thm:curves-fields}), we can take notions that apply to field extensinos (degree, separability, ramification) and apply them to morphisms.
\begin{df}\llabel{df:degree-morphism}
Let $\phi:C_1\to C_2$ be a morphism of smooth projective curves. Recall that we defined (Theorem~\ref{thm:curves-fields})
\begin{align*}
\phi^*: K(C_2)&\to K(C_1)\\
f&\mapsto f\circ \phi
\end{align*}
(This is a ring homomorphism and hence an embedding of fields.)
\begin{enumerate}
\item
Define the \textbf{degree} to be 
\[\deg \phi:=[K(C_1):\phi^*K(C_2)].\]
\item
Define the \textbf{separable/inseparable degree} to be the separable/inseparable degree of $K(C_1)/\phi^*K(C_2)$.
$\phi$ is \textbf{separable} if $K(C_1)/\phi^*K(C_2)$ is separable.
\end{enumerate}
\end{df}
Note that separability is automatic if $\chr(K)=0$.

Note that $\phi$ is an isomorphism iff $\deg\phi=1$ (Proposition~\ref{pr:deg1-iso}). 

\subsubsection{Degree and ramification}
There is another way of thinking of degree (cf. the Riemann manifold viewpoint): Fix a point on $C_2$; the number of points on $C_1$ mapping to it, counted with appropriate multiplicity (see below), is always constant and equal to the degree.
\begin{df}
Supoose $P\in C_1,Q\in C_2,\phi(P)=Q$. Let $t\in K(C_2)$ be a uniformizer at $Q$. Define
\[
e_{\phi}(P)=\ord_P(\phi^* t).
\]
\end{df}
This is always at least 1, and independent of the choice of $t$. \fixme{why?}

\thbox{
\begin{thm}\llabel{thm:cam2-8}
Let 
$\phi:C_1\to C_2$ be a nonconstant morphism of smooth projective curves (over algebraically closed $K$). Then 
\beq{eq:sum-ram}
\sum_{P\in \phi^{-1}(Q)} e_{\phi}(P)=\deg(\phi)
\eeq
for all $Q\in C_2$. Moreover if $\phi$ is separable then $e_{\phi}(P)=1$ for all but finitely many $P$.
In particular,
\begin{enumerate}
\item
$\phi$ is surjective %(we require $K=\ol K$)
\item
$|\phi^{-1}(Q)|\le \deg \phi$, and if $\phi$ is separable, then we have equality \fabfm $Q\in C_2$.
\end{enumerate}
\end{thm}}

Compare this to the following theorem from algebraic number theory: Given an extension of number fields $L/K$ and a prime $\mfp$ in $K$, we have 
\[
\sum_{\mP\mid \mfp}e_{L/K}(\mP)f_{L/K}(\mP)=[L:K].
\]
The absence of $f$ is because we are working with smooth curves. \fixme{Comment on why.} Furthermore, only finitely many primes ramify in $L$, and if $L/K$ is Galois, we have the nice fact that $e_{L/K}(\mP)$ are equal for all $\mP\mid \mfp$, and this reduces to 
\[
e_{L/K}f_{L/K}g_{L/K}=[L:K].
\]
Later we will see that this formula~\eqref{eq:sum-ram} becomes similarly nice for elliptic curves.
\subsubsection{Basic facts on degree}

\fixme{Define lower-star}
\begin{pr}\llabel{pr:deg-basics}
Let $\phi:C_1\to C_2$ be a non-constant map of smooth curves. Then for all $D_i\in \Div(C_i)$, $f_i\in \ol K(C_i)^{\times}$
\begin{enumerate}
\item $\deg(\phi^* D_2)=\deg(\phi)\deg(D_2)$.
\item $\phi^*(\div f_2)=\div(\phi^* f_2)$.
\item $\deg(\phi_* D)=\deg D$.
\item $\phi_*(\div f_1)=\div(\phi_* f_1)$.
\item $\phi_*\circ \phi^*$ is multiplication by $\deg \phi$ on $\Div(C_2)$.
\item For $\psi:C_2\to C_3$,
$(\psi\circ \phi)^*=\phi^*\circ \psi^*$ and $(\psi\circ \phi)_*=\psi_*\circ \phi_*$.
\end{enumerate}
\end{pr}
\begin{proof}
\cite{Si86}[II.3.6]
\end{proof}

A very useful relation is
\[
\div(f)=f^*((0)-(\iy)).
\]
This holds because noting $t\mapsto t$ is a uniformizer at 0 on $\Pj^1$ and $t\mapsto \rc t$ is a uniformizer at $\iy$ for $\Pj^1$,
\begin{align}
f^*((0))&=\sum_{Q\in f^{-1}(0)}\ord_Q(f^*t)=\sum_{Q\in f^{-1}(0)}\ord_Q(f)=\sum_{P,\ord_P(f)>0} \ord_P(f)P\\
f^*((\iy))&=\sum_{Q\in f^{-1}(\iy)}\ord_Q\pa{f^*\rc t}=\sum_{Q\in f^{-1}(\iy)}\ord_Q\prc{f}=-\sum_{P,\ord_P(f)<0} \ord_P(f)P\\
\div(f)%&=\sum_P \ord_P(f)P\\
%&=\sum_{Q\in f^{-1}(0)}\ord_Q(f) - \sum{Q\in f^{-1}(\iy)} \ord_Q\prc{f}\\
%&=\sum_{Q\in f^{-1}(0)}\ord_Q(f^*t) - \sum{Q\in f^{-1}(\iy)} \ord_Q(f^*\prc t)\\
&=f^*((0)-(\iy)).\llabel{eq:f*0-iy}.
%e_f(0)&=\ord_P(f^*t)
\end{align}
\begin{proof}[Proof of Proposition~\ref{pr:div-es}]
We have 
\[
\deg(\div(f))=\deg(f^*((0)-(\iy)))=\deg(f)\ub{\deg((0)-(\iy))}0=0.
\]
\end{proof}
We summarize:

\begin{center}
\begin{tabular}{c|c}
Number fields & Elliptic curves\tabularnewline
\hline
$\sum_{\mP\mid\mfp}e_{L/K}(\mP)f_{L/K}(\mP)=[L:K].$ & $\sum_{P\in\phi^{-1}(Q)}e_{\phi}(P)=\deg(\phi)$\tabularnewline
FABFM $Q$, $|\phi^{-1}(Q)|=\deg_{s}\phi$ & Finitely many primes ramify.\tabularnewline
$e_{\psi\circ\phi}(P)=e_{\phi}(P)e_{\psi}(\phi P)$ & $e_{M/L}(\mQ/\mP)e_{L/K}(\mP/\mfp)=e_{M/K}(\mQ/\mfp)$\tabularnewline
\end{tabular}
\end{center}

\subsubsection{Separability}

Just like we can break up a field extension into a purely inseparable and a separable part, we can do the same for maps between smooth curves.
\begin{pr}
Let $\psi:C_1\to C_2$ be a rational map of smooth curves. Then we can factor $\psi=\la\circ \phi_q$, where 
\begin{enumerate}
\item
$\phi_q$ is the Frobenius map, which is purely inseparable, with $q=\deg_i(\psi)$.
\item
$\la$ is purely separable.
\end{enumerate}
\[
\xymatrix{
C_2&\\
& C_1^{(q)}\ar[lu]_{\la\text{ separable}}\\
C_1\ar[ru]^{\phi_q\text{ inseparable}}&
}
\]
\end{pr}
\begin{proof}

\end{proof}

\subsection{Rational maps are morphisms}
In algebraic geometry there are two kinds of maps: morphisms and rational maps. For curves these are actually the same. (Again, projectivity is essential. If a rational map tries to blow up, that's fine, because a point at infinity exists!)

\thmbox{\llabel{thm:rat-morphism}
Let $C_1$ be a smooth curve and $V\subeq \Pj^N$ be a projective variety, and
\[\phi:C_1\dra V\subeq \Pj^N\] 
be a rational map. Then $\phi$ is a morphism.
}

\begin{proof}
Write the map as $f=[f_0:\cdots :f_n]$. Given a point $P$, we may have trouble with $f(P)$ if $f_0(P)=\cdots =f_n(P)=0$. 
{\it Because $C$ is smooth at $P$}, we have a discrete valuation at $P$ (Proposition~\ref{pr:dvp}).
Let $v=\min_i\ord_P(f_i)$, let $t$ be a uniformizer for $P$. Then we can define
\[
f(P)=\ba{
\fc{f_0}{t^v}(P):\cdots :\fc{f_n}{t^v}(P)
}
\]
because all of the $\fc{f_i}{t^v}$ have positive valuation at $P$, and at least one of them have valuation 0 so is nonzero.
{\it Because $V$ is projective}, this point is in $V$.
\end{proof}
\begin{pr}\llabel{pr:deg1-iso}
A rational map of degree 1 between smooth curves $C_1,C_2$ is an isomorphism.
\end{pr}
\begin{proof}
\fixme{proof}
\end{proof}

\section{Differentials}

\fixme{Add motivations.} 
Why would we consider differentials in algebra? See \url{http://math.stackexchange.com/questions/307439/appearance-of-formal-derivative-in-algebra}.
From Silverman~\cite{Si86}[II.4], differentials...
\begin{enumerate}
\item
perform the traditional calculus role of linearization.
\item
give a useful criterion for determining when an algebraic map is separable (cf. a field extension is separable iff the minimal polynomial of each element has nonzero derivative).
\end{enumerate}


Let $C$ be a smooth projective curve over $K=\ol K$. The space of differentials $\Om_C$ is the $K(C)$-vector space generated by $df$ for $f\in K(C)$ subject to relations
\begin{enumerate}
\item
$d(f+g)=df+dg$ for all $f,g\in K(C)$.
\item
$d(fg)=fdg+gdf$  for all $f,g\in K(C)$.
\item
$da=0$ for all $a\in K$.
\end{enumerate} 
\begin{pr}
If $C$ be a curve,
$\Om_C$ is a 1-dimensional $K(C)$-vector space. 

Hence, if $\om\in \Om_C\bs\{0\}$, $P\in C$, and $t\in K(C)$ is a uniformizer at $P$, then $\om =fdt$ for some $f\in K(C)$.
\end{pr}
\begin{df}
Keep the notation above. 
Define \[\ord_P(\om):=\ord_P(f).\] 
Note this is independent of the choice of $t$.
%doens't depend on which pick, doesn't change order of vanishing.

Moreover $\ord_P(\om)=0$ for all but finitely many $P\in C$. We define $\div(\om)=\sum_{P\in C}\ord_P(\om)P$. 
\end{df}

\section{Riemann-Roch Theorem}
See Silverman~\cite{Si86}[II.5].
\begin{df}
The \textbf{Riemann-Roch space} of $D\in \div(C)$ is
\[
\cL(D)=\set{f\in K(C)^*}{\div(f)+D\ge 0}\cup \{0\}\]
i.e., the $K$-vector space of rational functions on $C$ with poles no worse than specified by $D$. Denote its dimension by
\[
\ell(D)=\dim_{\ol K} \cL(D).
\]
\end{df}
We have the following basic facts.
\begin{pr}[Silverman~\cite{Si86}[II.5.2]
\llabel{pr:L(D)-basic}
Let $D\in \Div(C)$.
\begin{enumerate}
\item (We don't need to worry about negative divisors) If $\deg D<0$, then 
\[
\cL(D)=\{0\}\text{ and }\ell(D)=0.
\]
\item (Finite-dimensionality) $\cL(D)$ is a finite-dimensional $\ol{K}$ vector space.
\item If $D'\sim D$, then 
\[
\cL(D)\cong \cL(D') \text{ and }\ell(D)=\ell(D').
\]

\end{enumerate}
\end{pr}

%\fixme{Define genus. The below is for genus 1 only.}

The Riemann-Roch theorem tells us the dimension of these spaces based on an invariant called the genus.

\begin{thm}[Riemann-Roch]\llabel{thm:rr}
Let $C$ be a smooth curve and $K_C=\div(\om)$ a canonical divisor on $C$. Then there is an integer $g\ge 0$, called the \textbf{genus} of $C$, such that for every divisor $D\in \Div(C)$,
\[
\ell(D)-\ell(K_C-D)=\deg D-g+1.
\]
\end{thm}
The genus is the same as the topological genus if the curve is considered over $\C$. \fixme{Find a precise statement of this.}

The left hand side is a difference of $\ell$'s. To get $\ell(D')$ for some $D'$, we have to make the other term 0. We can do this by noting that $\ell(D)=0$ for $D<0$ and $D=0$.
\begin{cor}[Computation of $\ell(D)$]
As above, let $C$ be a smooth curve and $K_C=\div(\om)$ a canonical divisor on $C$.
We have the following.
\begin{enumerate}
\item
$\ell(K_C)=g$.
\item
$\deg K_C=2g-2$.
\item
If $\deg D>2g-2$, then 
\[
\ell(D)=\deg D-g+1.
\]
\end{enumerate}
\end{cor}
\begin{proof}
\begin{enumerate}
\item
Use the Riemann-Roch Theorem~\ref{thm:rr} with $D=0$ and note $\cL(0)=\ol K$.
\item
Use (1) and Riemann-Roch with $D=K_C$. 
\item
From (2) we get $\deg(K_C-D)<0$, so by Proposition~\ref{pr:L(D)-basic}, $\ell(D)-0=\deg D-g+1$. 
\end{enumerate}
\end{proof}

\begin{cor}[Riemann-Roch for elliptic curves]\llabel{thm:rr-ec}
If the genus is 1 (i.e. $C$ is an elliptic curve), then
\[
\dim \cL(D)=\begin{cases}
\deg(D),&\text{if }\deg D>0\\
0\text{ or }1,&\text{ if }\deg D=0\\
0,&\text{ if }\deg D<0.
\end{cases}
\]
\end{cor}
\begin{proof}
Put in $g=1$.
\end{proof}

%\fixme{Motivate this with st}

%\subsection{Rationality}
%Assume $K=\ol K$ and $\chr(K)\ne 2$. 
%\begin{df}\llabel{df:rat}
%A (irreducible) plane affine algebraic curve $C=\{f(x,y)=0\}\sub \A^2$ is \textbf{rational} if it has a rational parametrization, i.e. there exist $\phi(t)$ and $\psi(t)\in K(t)$ such that 
%\begin{enumerate}
%\item
%The rational map
%\begin{align*}
%\A^1&\dra\A^2\\
%t&\mapsto (\phi(t),\psi(t))
%\end{align*}
%is injective on $\A^1\bs S$ where $S$ is a finite set.
%\item
%$f(\phi(t),\psi(t))=0$. (The image lies on the curve.)
%\end{enumerate}
%\end{df}
%\begin{ex}
%%\begin{enumerate}
%%\item
%Any non-singular plane conic is rational. To see this, pick a point on the curve.
%
%For example, for $x^2+y^2=1$, pick the point $(-1,0)$. Consider the line with slope $t$, $y=t(x+1)$. 
%We use the difference of two squares to bring out the $x-1$:
%\begin{align*}
%y&=t(x+1)\\
%\implies x^2+t^2(x+1)^2&=1\\
%\implies (x+1)(x-1+t^2(x+1))&=0\\
%\implies x&=-1 \text{ or } x=\fc{1-t^2}{1+t^2}.
%\end{align*}
%The rational parametrization is
%\[
%(x,y)=\pa{\fc{1-t^2}{1+t^2},\fc{2t}{1+t^2}}.
%\]
%(Note this may be familiar from doing integrals; it helps to rationally parametrize a circle. This is exactly what we did in Section () when we found all Pythagorean triples.)
%%\item
%%Any singular plane cubic is rational. We'll show an example, but the method works more generally.
%%
%%Consider $y^2=x^3$. Take the singular point and a line $y=tx$ through the singular point. We expect the line to meet the curve in 3 points, but the singular point counts double, so it only meets the curve in one other point. The point of intersection is $(x,y)=(t^2,t^3)$; this is the rational parametrization.
%%
%%For $y^2=x^2(x+1)$, setting $y=tx$ similarly gives a rational parametrization.
%%\item
%%Corollary~\ref{cor:ec-no-pts-func-field} shows elliptic curves are not rational.
%%\end{enumerate}
%\end{ex}
%\subsection{Algebraic geometry of curves}
%
%
%Let $C$ be a smooth projective curve.
%\begin{df}
%A \textbf{divisor} is a formal sum of points on $C$, say $D=\sum_{P\in C}n_PP$ with $n_P\in \Z$ and $n_P=0$ for all but finitely many $P$. Define
%\[
%\deg D=\sum_{P\in C} n_P.
%\]
%$D$ is \textbf{effective} (written $D\ge 0$) if $n_P\ge 0$ for all $P$. If $f\in K(C)^*$ then 
%\[
%\div(f)=\sum_{P\in C} \ord_P(f)P.
%\]
%The \textbf{Riemann-Roch space} of $D\in \div(C)$ is
%\[
%\cL(D)=\set{f\in K(C)^*}{\div(f)=D\ge 0}\cup \{0\}\]
%i.e., the $K$-vector space of rational functions on $C$ with poles no worse than specified by $D$.
%\end{df}
%
%The Riemann-Roch theorem tells us the dimension of these spaces based on the genus (in this case, 1).
%\begin{thm}[Riemann-Roch]\llabel{thm:rr-ec}
%We have
%\[
%\dim \cL(D)=\begin{cases}
%\deg(D),&\text{if }\deg D>0\\
%0\text{ or }1,&\text{ if }\deg D=0\\
%0,&\text{ if }\deg D<0.
%\end{cases}
%\]
%\end{thm}
%\begin{ex}
%$C$ is an elliptic curve with Weierstrass equation $y^2=f(x)$, with point at $\iy$, $P$. Then example~\ref{ex:order-xy-ec} gives
%\[\cL(3P)=\an{1,x,y}\]
%since $1,x,y$ are linearly independent elements of $\cL(3P)$ and Riemann-Roch says $\dim \cL(3P)=3$.
%\end{ex}


\chapter{Introduction and geometry}

What are elliptic curves and why are they important? See Andrew Sutherland's slides~\url{http://math.mit.edu/classes/18.783/Lecture1.pdf} for an introduction.

In this chapter we'll present an elementary, computational approach side by side with a more theoretical, geometric approach. (The reader may choose to focus on one or the other.)

\section{Definition and motivations}

An elliptic curve is essentially the ``next simplest curve" besides a conic. We give two definitions of an elliptic curve which make this precise.
\begin{df}[Elementary definition]\llabel{df:ec1}
%\begin{enumerate}
%\item
An \textbf{elliptic cuve} $E$ over a field $K$ is the projective closure of a plane affine curve \[y^2=f(x)\] with a specified rational point (typically the point at infinity). 
Here, $f\in K[x]$ is a monic cubic polynomial with distinct roots in $\ol K$.\footnote{When $\chr(K)=2$, we have to allow equations of the form $y^2+a_1xy+a_3y=f(x)$. More on this when we talk about the Weierstrass form.}
%\item For any field extension $L/K$, let
%\[
%E(L)=\set{(x,y)\in L}{y^2=f(x)}\cup\{O\}.
%\]
%\end{enumerate}
\end{df}

\begin{df}[Algebraic geometry definition]\llabel{df:ec2}
An \textbf{elliptic curve} $E$ over a field $K$ is a smooth projective curve of genus 1 with a distinguished point in $K$.
\end{df}

Why is this interesting to study? We give an example of a problem where elliptic curves naturally arise.

%\subsection{Problems involving elliptic curves}

\subsection{The congruent number problem}
As elliptic curves are the ``next simplest curves" apart from conics, and there is a lot more freedom for elliptic curves, it is natural that a lot of Diophantine equations reduce to problems about elliptic curves. This is a large part of the motivation for studying them.

One famous still unsolved Diophantine equation is the congruent number problem, which we'll now consider. See Keith Conrad's article at \url{http://www.thehcmr.org/issue2_2/congruent_number.pdf} for an in-depth discussion, and~\cite{Ko84} for the approach to the problem using modular forms.

%\section{Fermat's method of descent}
Consider a right triangle $\triangle$ with legs $a$, $b$, and hypotenuse $c$. It satisfies the following.
\begin{align*}
a^2+b^2&=c^2\\
\Area(\triangle)&=\rc2ab.
\end{align*}
\begin{df}
We say
\begin{enumerate}
\item
$\triangle$ is \textbf{rational} if $a,b,c\in \Q$. 
\item
$\triangle$ is \textbf{primitive} if $a,b,c\in \Z$ and $\gcd(a,b,c)=1$. (This is the same as saying that $a,b,c$ are pairwise coprime.)
\end{enumerate}
\end{df}
Recall the following parametrization.
\begin{lem}[Rephrasing Theorem~\ref{thm:pythag}]\llabel{lem:primitive-triangle}
Every primitive triangle is of the form $(u^2-v^2,2uv,u^2+v^2)$ for $u,v\in \Z$ where $u>v>0$.
\end{lem}
\begin{df}
$D\in \Q^+$ is \textbf{congruent} if it is the area of some rational right-angled triangle.
%equivalent to AS of three terms.
\end{df}
Note that it suffices to consider squarefree integers $D\in \N$ since they form a coset of $\Q^{\times 2}$ in $\Q$.

For example, 5 and 6 are congruent.

\thbox{\begin{lem}\llabel{lem:cong-iff-ell}
$D\in \Q^+$ is congruent iff $Dy^2=x^3-x$ for some $x,y\in \Q$ with $y\ne 0$, iff $y^2=x^3-D^2x$ for some $y\ne 0$.
\end{lem}}

Thus the problem of whether or not a number is congruent is equivalent to whether or not there exists a 

\begin{proof}
We first show the first two conditions are equivalent. 
Suppose $D$ is congruent. 
Then there is $w$ such that $Dw^2$ is the area of a primitive triangle. 
By lemma~\ref{lem:primitive-triangle}, we can write the sides in the form $u^2-v^2$, $2uv$, and $u^2+v^2$ for $u,v,w\in \Q$. Then
\begin{align*}
Dw^2&=\rc2(u^2-v^2)2uv\\
D\pf{w}{v^2}^2&=\pf uv\pa{\pf uv^2-1}.
\end{align*}
Set $x=\fc uv$ and $y=\fc{w}{v^2}$. Conversely, given $x,y$, let $u,v$ be the numerator and denominator of $x$, and let $w=yv^2$.

To go between $Dy^2=x^3-x$ and $y^2=x^3-D^2x$, make the substitution $y\mapsfrom \fc{y}{D^2}$ and $x\mapsfrom \fc{x}{D}$.
\end{proof}
Fermat showed $D=1$ is not congruent.
\begin{thm}\llabel{thm:1-not-cong}
There are no solutions to
\begin{equation}\llabel{eq:no-soln-1-cong}
w^2=uv(u-v)(u+v)
\end{equation}
for $u,v,w\in \Z$. Hence 1 is not a congruent number.
\end{thm}
\begin{proof}
An elementary approach is to use infinite descent. See the notes at~\url{https://dl.dropboxusercontent.com/u/27883775/math\%20notes/part_iii_elliptic.pdf}.
%Without loss of generality $w>0$. We can replace $(u,v)$ by $(-u,-v)$, $(-v,u)$, or $(v,-u)$, so  we can assume $u,v>0$. WLOG $\gcd(u,v)=1$; then $\gcd(u,u\pm v)=1$ as well. If a prime $p\mid u-v$ and $p\mid u+v$ then $p\mid 2u$ and $p\mid 2v$. Thus the only possible common factor among $u,v,u+v,u-v$ is 2, and this happens when $u\equiv v\pmod 2$.
%
%If $u\equiv v\pmod 2$, replace $(u,v,w)$ with $\pa{\pf{u+v}{2},\pf{u-v}{2},\fc{w}{2}}$. Now our new $u,v$ are coprime, or else $u,v$ are divisible by 2, in which case we divide it out.
%
%Thus we can assume $u\nequiv v\pmod 2$, so $u,v,u\pm v$ are coprime integers.
%
%By unique factorization of $\Z$,
%\begin{align*}
%u&=a^2\\
%v&=b^2\\
%u+v&=c^2\\
%u-v&=d^2
%\end{align*}
%for some $a,b,c,d\in \Z$. Since $u\nequiv v\pmod 2$, $c\equiv d\equiv 1\pmod 2$. Now
%\[
%\pf{c+d}{2}^2+\pf{c-d}{2}^2=\fc{c^2+d^2}2=a^2
%\]
%so $\pa{\fc{c+d}2,\fc{c-d}2,a}$ is a primitive triangle. %check
%The area is $\rc2\cdot \fc{c^2-d^2}4=\fc{2v}{8}=\fc{v}{4}=\pf b2^2$.
%
%Let $w_1=\fc b2$. By Lemma~\ref{lem:primitive-triangle} again, $w_1^2=u_1v_1(u_1^2-v_1^2)$ for some $u_1,v_1\in \Z$. This is another solution to~\eqref{eq:no-soln-1-cong}. We have $4w_1^2=b^2=v$. By the original equation, $v\mid w^2$ so $v\le w^2$, giving
%\[
%4w_1^2\le w^2\implies w_1\le \fc{w}{2}<w.
%\]
%We are done by Fermat's method of infinite descent.
\end{proof}

\section{The equation of an elliptic curve}
\subsection{Every elliptic curve can be put in Weierstrass form}
We would like to have some ``standard form" for an elliptic curve, that is
\begin{enumerate}
\item
a form involving as few terms as possible, such that every elliptic curve is isomorphic to an elliptic curve in that form, and
\item
an easy way to tell if two elliptic curves in that form are isomorphic.
\end{enumerate}
We'll investigate two natural forms for an elliptic curve: the Weierstrass and Legendre forms.
\begin{df}
A \textbf{(long) Weierstrass equation} is an eqution in the following form
\begin{align*}
\text{affine coordinates}&& y^2+a_1xy+a_3y&=x^3+a_2x^2+a_4x+a_6\\
\text{projective coordinates}&& Y^2Z+a_1XYZ+a_3YZ^2&=X^3+a_2X^2Z+a_4XZ^2+a_6Z^3
\end{align*}
A \textbf{(short) Weierstrass equation} is an equation in the following form
\begin{align*}
\text{affine coordinates}&& y^2=x^3+Ax+B\\
\text{projective coordinates}&& Y^2Z=X^3+AXZ^2+BZ^3.
\end{align*}
\end{df}
A note on the numbering in the coefficients: they are the weights for the coefficients that make the equation $y^2+a_1xy+a_3y=x^3+a_2x^2+a_4x+a_6$ homogeneous if $y$ has weight 3 and $x$ has weight 2.

Our main theorem in this section is the following. (The proof may be safely skipped for an elementary course, by using Definition~\ref{df:ec1} for an elliptic curve.)
\begin{thm}\llabel{thm:cam3-1}
Every elliptic curve $E$ is isomorphic over $K$ to a curve in Weierstrass form, via an isomorphism mapping $O_E\mapsto (0:1:0)$.
\end{thm}
%Our previous lemma~\ref{lem:cam2-7} was a special case: $E$ was a smooth plane cubic and $O_E$ a point of inflection. This is more general because it's an arbitrary curve of genus 1, and the point may not be a point of inflection.
\begin{fct}
Let $D$ be a divisor on $E$, i.e., a formal sum of $\ol K$-points on $E$. If $D$ is defined over $K$ (i.e., $D$ is fixed by the action of $G(\ol K/K)$), then $\cL(D)$ has a basis consisting of rational functions defined over $K$, not just in $\ol K(E)$. \fixme{Proof?}
\end{fct}
\begin{proof}[Proof of Theorem~\ref{thm:cam3-1}]
\ul{Step 1:}
The idea is that if we pick some functions $x,y$ to be our coordinates, by Riemann-Roch, we will get a linear dependence relations between terms $x^iy^j$ before too long.
\begin{enumerate}
\item
$\cL(2O_E)$ is 2-dimensional by Riemann-Roch~\ref{thm:rr-ec}. Pick a basis $1,x$.
\item $\cL(3O_E)$ is 3-dimensional by Riemann-Roch. Extend to a basis $1,x,y$.
\item
Look at the elements $1,x,y,x^2,xy,x^3,y^2$: these are all in $\cL(6,O_E)$. But the dimension of the space is 6 and there are 7 elements, so there is a linear dependence relation.

Leaving out either $x^3$ or $y^2$ gives a basis for $\cL(6O_E)$ since each term has a different order pole at $O_E$. Thus the coefficients of $x^3$ and $y^2$ are nonzero. Rescaling $x$ and $y$ we get
\[
y^2+a_1xy+a_3y=x^3+a_2x^2+a_4x+a_6
\] 
for some $a_i\in K$.
%(If $E$ is defined over $K$, then $a_i\in K$.)
\end{enumerate}
We obtain a rational map
\begin{align*}
\phi:E&\to E'\subeq \Pj^2\\
P&\mapsto [x(P):y(P):1].
\end{align*}
A rational map from a smooth curve to a smooth curve is always a morphism (Theorem~\ref{thm:rat-morphism}).

\ul{Step 2:} We show $\phi$ is an isomorphism by showing its degree is 1. By Theorem~\ref{thm:cam2-8} \fixme{(how?)} on the point $\iy\in \Pj^1$ with inverse $O_E$ under $x,y$, we have

\begin{align*}
[K(E):K(x)]&=\deg(E\xra{x}\Pj^1)=\ord_{O_E}\prc x=2\\
[K(E):K(y)]&=\deg(E\xra{y}\Pj^1)=\ord_{O_E}\prc y=3.
\end{align*}
We write down the fields involved.
\[
\xymatrix{
& K(E)\ls{rdd}^3\ls{d} &\\
& K(x,y) \ls{rd} &\\
K(x)\ls{uur}^2\ls{ur} && K(y)
}
\]
The tower law says that $[K(E):K(x,y)]$ divides 2 and 3 so $K(E)=K(x,y)$ thus $\phi^*K(E')=K(E)$ and $\deg (\phi)=1$. Thus $\phi$ is birational.

We know $E$ is projective, and we need $E'$ to be smooth. If $E'$ is singular, we can find a rational parametrization, so $E$ and $E'$ are rational. This is a contradiction because $E$ has genus 1.

Thus $E'$ is smooth and $\phi$ is an isomorphism.

We check the image of $O_E$. Since $x$ has a pole of order 2 at $O_E$ and $y$ has a pole of order 3,
\begin{align*}
\phi:E&\to E'\\
P&\mapsto \ba{\fc xy(P):1:\rc y(P)}\\
O_E&\mapsto [0:1:0].
\end{align*}
\end{proof}

\subsection{Transforming an elliptic curve}
We can use the proof of Theorem~\ref{thm:cam3-1} to find when 2 curves in Weierstrass form are isomorphic.
\begin{pr}\llabel{pr:cam3-2}
Let $K$ be algebraically closed.
Let $E,E'$ be elliptic curves over $K$ in Weierstrass form. Then $E\cong E'$ over $K$ iff the equations are related by substitutions 
\begin{align*}
x&=u^2x'+r\\
y&=u^3y'+u^2sx'+t
\end{align*}
for some $r,s,t,u\in K$ with $u\ne 0$.
\end{pr}
%specified pt on E to E'
\begin{proof}
Because $x,y$ and $x',y'$ are Weierstrass coordinates, we must have $x,x'\in \cL(2O_E)$ and $y,y'\in \cL(3O_E)\bs \cL(2O_E)$. %Look at the previous proof: how much freedom did we have when picking $x$ and $y$? 

Again by Riemann-Roch, we have
\[
\an{1,x}=\cL(2O_E)=\an{1,x'}.
\]
From this we see $x=\la x'+r$ for some $\la,r\in K$. Similarly,
\[
\an{1,x,y}=\cL(3O_E)=\an{1,x',y'}
\]
and hence $y=\mu y' +\si x' + t$ for some $\mu,\si,t\in K$ and $\mu\ne 0$. Looking at coefficients of $y^2$ and $x^3$ we need $\la^3=\mu^2$, so $\la=u^2$ and $\mu=u^3$ for some $u\in K$. Put $s=\fc{\si}{u^2}$.
\end{proof}
\fixme{A warning when $K$ is not algebraically closed: we can have curves isomorphic over $\ol K$ but not $K$. Quadratic, cubic, sextic twists.}

A Weierstrass equation defines an elliptic curve iff it defines a smooth curve, iff $\De(a_1,\ldots, a_6)\ne 0$ where $\De\in \Z[a_1,\ldots, a_6]$ is a certain polynomial (see the formula sheet!).

If $\chr(K)\ne 2,3$, we can reduce to the case $y^2=x^3+ax+b$, with discriminant
\[
\De=-16(4a^3+27b^2).
\]
(this is 16 times the usual formula for the discriminant of a polynomial).

\begin{cor}\llabel{cor:cam3-3}
Assume $\chr(K)\ne 2,3$. The elliptic curve 
\begin{align*}
E:y^2&=x^3+ax+b\\
E':y^2&=x^3+a'x+b'
\end{align*}
\end{cor}
are isomorphic over $K$ iff $a'=u^4a$ and $b'=u^6b$ for some $u\in K^*$.
\begin{proof}
$E$ and $E'$ are related by a substition as in Proposition~\ref{pr:cam3-2} with $r=s=t=0$. 
\end{proof}

%\subsubsection{Change of coordinates}


\subsection{Legendre form}
Another useful for an elliptic curve is the \textbf{Legendre form}.
\begin{lem}\llabel{lem:cam2-7}
Let $C\sub \Pj^2$ be a smooth plane cubic, and $P\in C$ a point of inflection. Then we can change coordinates such that
\[
C:Y^2Z=X(X-Z)(X-\la Z),\qquad \la\ne 0,1,\qquad P=(0:1:0).
\]
\end{lem}
\begin{proof}
We change coordinates %use a 3x3 invertible matrix
such that $P=(0:1:0)$ and $T_pC=\{Z=0\}$. Then
\[
C:\{F(X,Y,Z)=0\}\sub \Pj^2.
\]
A point of inflection means the line meets the curve with multiplicity 3, so we get a triple root
\[
F(t,1,0)=ct^3.
\]
Thus there are no terms $X^2Y,XY^2, Y^3$. Thus
\[
F\in \an{Y^2Z,XYZ,YZ^2,X^3,X^2Z,XZ^2, Z^3},
\]
with the coefficient of $Y^2Z$ nonzero (otherwise $P\in C$ would be singular \fixme{why?}), otherwise everything is divisible by $Z$, and $C$ contains $\{Z=0\}$ (curves are irreducible).
%why is reducibility preserved

We can rescale $X,Y,Z,F$, so WLOG
\[
C:Y^2Z+a_1XYZ+a_3YZ^2=X^3+a_2X^2Z+a_4XZ^2+a_6Z^3.\qquad \text{Weierstrass form}
\]
Substituting $Y\lar Y-\rc2 a_1X-\rc 2 a_3Z$ (completing the square) we may assume $a_1=a_3=0$ (we assume $\chr(K)\ne 2$), giving
\[
C: Y^2Z=Z^3f\pf{X}{Z}.
\]
$C$ is smooth, so $f$ has distinct roots, without loss of generality, 0, 1, $\la$. Thus
\[
C: Y^2Z=X(X-Z)(Z-\la Z).\qquad \text{Legendre form}
\]
\end{proof}



\subsection{Invariants of an elliptic curve}

\fixme{Motivate the $j$-invariant. Quotient of 2 things with weight 12.}

\begin{df}\llabel{df:j-inv}
The \textbf{$j$-invariant} of $E$ is
\[
j(E)=\fc{1728(4a^3)}{4a^3+27b^3}.
\]
\fixme{Also define the discriminant and invariant differential.}
\end{df}
\begin{cor}
$E\cong E'$ implies $j(E)=j(E')$ and the converse holds over $K=\ol K$. 
\end{cor}
\begin{proof}
We have $E\cong E'$ iff $a'=u^4a$, $b'=u^6b$ for some $u\in K^*$. This implies $(a^3:b^2)=((a')^3:(b')^2)$, which is true iff $j(E)=j(E')$. The converse holds if the field is algebraically closed (we need to extract roots).
\end{proof}


\subsection{Singular cubics}
[Weierstrass form, etc.]



\section{Group law}
There is a natural group law on elliptic curve. There are several ways to think about it.
\begin{enumerate}
\item
The key point here is that every line intersects an elliptic curve in 3 points, counted with multiplicity.
Given two points $P$ and $Q$ on an elliptic curve $E$, let the third point of intersection be $-(P+Q)$ (and its reflection across the $x$-axis be $P+Q$). We can obtain explicit expressions for $P+Q$.
\item 
Every algebraic curve has a group associated with it, the group of divisors. It turns out that the divisor class group of an elliptic curve can be put in direct correspondence to the points on the elliptic curve.
\item
For an elliptic curve over $\C$, we can define the group law by finding some analytic map $\C\to E(\C)$. Then addition on $\C$ pushed forward gives a group law on $E$. This is just like the fact that the addition formulas for $\sin,\cos$ give a group law on the circle. We'll see this in the chapter on elliptic curves over $\C$.
\end{enumerate}

\subsection{Elementary approach, I}

The following is an adaptation of the proof in \cite[p.\ 28]{Cassels}. The proof is nice, but fails to capture all cases, so we will give a more correct, but messier, proof later.

\begin{thm*}
Let $P$, $Q$, and $R$ be three points on an elliptic curve $E(K)$ for some field $K$ that we may assume is algebraically closed.
Assume that $P$, $Q$, $R$, and the zero point $O$ are all in \emph{general position} (this means that in the diagram below there are no relationships among the points other than those that necessarily exist by construction). Then
\[
(P+Q)+R=P+(Q+R).
\]
\end{thm*}
\begin{proof}
The line $\ell_0$ through $P$ and $Q$ meets the curve $E$ at a third point, $-(P+Q)$, and the line $m_2$ through $O$ and $-(P+Q)$ meets $E$ at $P+Q$.
Similarly, the line $m_0$ through $P$ and $R$ meets $E$ at $-(P+R)$, and the line $\ell_2$ through $O$ and $-(P+R)$ meets $E$ at $P+R$.
Let $S$ be the third point where the line $\ell_1$ through $Q+P$ and $R$ meets $E$, and let $T$ be the third point where the line $m_1$ through $Q$ and $P+R$ meets $E$.  See the diagram below.

\begin{center}
\begin{tikzpicture}[scale=1.5]
\footnotesize
\foreach \x in {0,1,2} \draw[black] (\x,-0.5) -- (\x,2.5) node[above] {{$m_\x$}};
\foreach \y in {0,1,2} \draw[black] (-1,\y) -- (3,\y) node[right] {{$\ell_\y$}};
\foreach \x in {0,1,2} \foreach \y in {0,2} \draw[fill=blue!50] (\x,\y) circle (0.05);
\foreach \x in {0,2} \draw[fill=blue!50] (\x,1) circle (0.05);
\draw[fill=blue!50] (1.2,1) circle (0.05) node at (1.2,0.83) {{$S$}};
\draw[fill=blue!50] (1,1.2) circle (0.05) node[left] {{$T$}};
\node at (-0.15,-0.15) {{$P$}};
\node at (0.85,-0.15) {{$Q$}};
\node at (2.5,-0.15) {{$-(P+Q)$}};
\node at (-0.15,0.85) {{$R$}};
\node at (2.4,0.85) {{$P+Q$}};
\node at (-0.5,1.85) {{$-(P+R)$}};
\node at (1.35,1.85) {{$P+R$}};
\node at (2.15,2.15) {{$O$}};
\end{tikzpicture}
\end{center}

We have $S=-(Q+P)+R$ and $T=-(Q+(P+R))$.  It suffices to show $S=T$.
Suppose~not.
Let $g(x,y,z)$ be the cubic polynomial formed by the product of the lines $\ell_0,\ell_1,\ell_2$ in homogeneous coordinates,
and similarly let $h(x,y,z)=m_0m_1m_2$.
We may assume $g(T)\ne 0$ and $h(S)\ne 0$, since the points are in general position and $S\ne T$.
Thus $g$ and $h$ are linearly independent elements of the $k$-vector space $V$ of homogeneous cubic polynomials in $k[x,y,z]$.
The space $V$ has dimension 10, thus the subspace of homogeneous cubic polynomials that vanish at the eight points $O$, $P$, $Q$, $R$,
$\pm(Q+P)$, and $\pm(P+R)$ has dimension 2 and is spanned by $g$ and $h$.
The homogeneous polynomial $f(x,y,z)=x^3+Axz^2+Bz^3-zy^2$ that defines $E$ is a nonzero element of this subspace, so we may write $f=ag+bh$ as a linear combination of $g$ and $h$.
But $f(S)=f(T)=0$, since $S$ and $T$ are both points on $E$, which implies that $a$ and $b$ are both zero.
This contradicts the linear independence of $g$ and $h$, since $f$ is not the zero polynomial.
\end{proof}
\subsection{Elementary approach, II: The group law in algebraic terms}
Let $P=(x_1,y_1,z_1)$ and $Q=(x_2,y_2,z_2)$ be two points on $E$.
We will compute the sum $P+Q=R=(x_3,y_3,z_3)$ by expressing the coordinates of $R$ as rational functions of the coordinates of $P$ and $Q$.
If either $P$ or $Q$ is the point at infinity, then $R$ is simply the other point, so we assume that $P$ and $Q$ are affine points with $z_1=z_2=1$.
There are two cases:
\begin{description}
\item[Case 1.] $x_1\neq x_2$. The line $\overline{PQ}$ has slope
  $m=(y_2-y_1)/(x_2-x_1)$, which yields the equation $y-y_1=m(x-x_1)$.
  The point $-R=(x_3,-y_3,1)$ is on this line, thus $-y_3=m(x_3-x_1)+y_1$.
  Substituting for $y_3$ in the Weierstrass equation for $E$ yields
  $$(m(x_3-x_1)+y_1)^2=x_3^3+Ax_3+B.$$
  Simplifying, we obtain $0=x_3^3-m^2x_3^2+\cdots$, where the ellipsis hides lower order terms.
  The values $x_1$ and $x_2$ satisfy the same cubic equation, and the quadratic coefficient $-m^2$
  must be the sum of the roots.  Thus $x_3=m^2-x_1-x_2$.  To sum up, we have
\begin{align*}
m &= \fc{y_2-y_1}{x_2-x_1},\\
x_3 &= m^2-x_1-x_2,\\
y_3&=m(x_1-x_3)-y_1.
\end{align*}
  Thus to compute $P+Q=R$, we need one inversion and three multiplications (one of which is a squaring).
  We'll denote this cost 3M+I.

\item[Case 2.] $x_1=x_2$. If $y_1\neq y_2$, then they must be opposite
  points and $R=0$. Otherwise $P=Q$, and we compute the slope of the
  tangent line by implicitly differentiating the Weierstrass equation for $E$.
  This yields $2y\dy=3x^2\dx+A\dx$, so 
\[m=\frac{dy}{dx}=\frac{3x_1^2+A}{2y_1}.\]
  The formulas for $x_3$ and $y_3$ are then the same as
  the previous case.  Note that we require an extra multiplication here,
  so computing $R=2P$ has a cost of 4M+I.
\end{description}

With these equations in hand, we can now prove associativity as a formal
identity, treating $x_1,y_1,z_1,x_2,y_2,z_2,x_3,y_3,z_3,A,B$ as indeterminants
subject to the three relations implied by the fact that $P$, $Q$, and
$R$ all lie on the curve $E$.  See the Sage worksheet
\begin{center}
\url{https://hensel.mit.edu:8002/home/pub/1/}
\end{center}
for details, which includes checking all the special cases.

The equations above can be converted to projective coordinates by replacing $x_1,y_1,x_2$, and $y_2$ with
$x_1/z_1$, $y_1/z_1$, $x_2/z_2$, and $y_2/z_2$ respectively, and then writing the resulting expressions for $x_3/z_3$ and $y_3/z_3$ with a common denominator.
This has the advantage of avoiding inversions, which are more costly than multiplications (in a finite field of cryptographic size inversions may be 50 or even 100 times more expensive).
This increases the number of multiplications to 12M in case 1
(addition), and 14M in case 2 (doubling).

\subsubsection{Edwards curves*}
There are many alternative representations of elliptic curves that have been proposed.
We give just one example here, Edwards curves \cite{bl,edwards}, which have two significant advantages over Weierstrass equations.
Let $d$ be a non-square element of a field $k$ (assumed to have characteristic not equal to 2, as usual).
Then the equation
\begin{equation*}
x^2+y^2 = 1+dx^2y^2
\end{equation*}
defines an elliptic curve with distinguished point $(0,1)$.\footnote{Technical point: there are two points at infinity, both of which are singular,  violating our requirement that an elliptic curve be smooth.  However, this plane curve can be desingularized by embedding it in $\mathbb{P}^3(k)$.  The points at infinity are then no longer rational, and do not play a role in the group operation on $E(k)$.}
The group operation is given by
\begin{equation*}
(x_3,y_3)=\left(\frac{x_1y_2+y_2x_1}{1+dx_1x_2y_1y_2},\frac{y_1y_2-x_1x_2}{1-dx_1x_2y_1y_2}\right).
\end{equation*}
As written, this involves five multiplications and two inversions (ignoring the multiplication by $d$, which we can choose to be small),
which is greater than the cost of the group operation in Weierstrass form.
However, in projective coordinates we have
\begin{equation*}
\frac{x_3}{z_3}=\frac{z_1z_2(x_1y_2+x_2y_1)}{z_1^2+z_2^2+dx_1x_2y_1y_2},\qquad\frac{y_3}{z_3}=\frac{z_1z_2(y_1y_2-x_1x_2)}{z_1^2+z_2^2+dx_1x_2y_1y_2}\,.
\end{equation*}
There are a bunch of common subexpressions here, and in order to compute $z_3$, we need a common denominator.
Let $r=z_1z_2$, let $s=x_1y_2+x_2y_1$, let $t=dx_1y_2x_2y_1$, and let $u=y_1y_2-x_1x_2$.
We then have
\begin{equation*}
x_3=rs(r^2-t),\qquad y_3=ru(r^2+t),\qquad z_3=(r^2+t)(r^2-t).
\end{equation*}
This yields a cost of 12M, and, if you are clever, you can reduce it to 11M.

The remarkable thing about these formulas is that they handle every case; there
are not separate formulas for addition and doubling, and adding opposite points or the identity element works the same as the general case.
Such formulas are called \emph{complete}, and they have two distinct advantages.
First, they can be implemented very efficiently because there is no branching.
Second, they protect against what is known as a \emph{side-channel} attack.  If an adversary can distinguish whether you are doubling or adding points, e.g. by monitoring the CPU and noticing the difference in the time required by each operation, they can break a cryptosystem that performs scalar multiplication by an integer that is meant to be secret.

Having said that, if you know you are going to be doubling and are not concerned about a side-channel attack, there are several optimizations that can be made (these include replacing $1+dx^2y^2$ with $x^2+y^2$).
This reduces the cost of doubling a point on an Edwards curves to 7M, which is a huge improvement over the 14M cost of doubling a point in Weierstrass coordinates.

The explicit formulas database at \url{http://hyperelliptic.org/EFD/} contains optimized formulas for Edwards curves and various generalizations, as well as many other forms of elliptic curves.  Operation counts and verification scripts are provided with each set of formulas.

We should note that, unlike Weierstrass equations, not every elliptic curve can be put into Edwards form.
In particular, an Edwards curve always has a rational point of order 4, the point $(1,0)$, but this is not true of many elliptic curves.

\subsection{Group law via divisors}

Recall the definition of the Picard group~\ref{df:picard}. 
We define 
\begin{align*}
\phi:E&\to \Pic^0(E)\\
P&\mapsto [P-O_E].
\end{align*}
For clarity, we temporarily write the group law on $E$ with $\opl$.
\begin{pr}\llabel{pr:cam4-2}
\begin{enumerate}
\item We have $\phi(O_E)=0$ and 
$\phi(P \opl Q)=\phi(P)+\phi(Q)$.
\item
$\phi$ is a bijection.
\end{enumerate}
\end{pr}
%Associativity comes for free after we make this correspondence.
This shows that addition on the elliptic curve is the pullback under $\phi$ of addition on $\Pic^0(E)$. Since $\Pic^0(E)$ is an abelian group, we get $E$ is an abelian group under addition. (In particular, $\opl$ is associative.)

\begin{proof}
\begin{enumerate}
\item Let $\ell=0$ be the line through $P,S,Q$ and $m=0$ be the line through the point $O_E,S,R=P\opl Q$. 
We have
\begin{align*}
\div(\ell/m)&=(P)+(S)+(Q)-(O_E)-(S)-(R)\\
&=(P)+(Q)-(P\opl Q)-(O_E)\\
\implies (P)+(Q)&\sim(P\opl Q)+(O_E)\\
\implies (P\opl Q)-(O_E)&\sim (P)-(O_E)+(Q)-(O_E)\\
\implies \phi(P\opl Q)&=\phi(P)+\phi(Q).
\end{align*}
\item \ul{Injectivity}: Suppose $\phi(P)=\phi(Q)$ and $P\ne Q$. Then there exists $f\in \ol K(E)^*$ such that $\div(f)=P-Q$. Then there is a rational map $f:E\to \Pj^1$ which is automatically a morphism. What is its degree? Only 1 point, $P$, maps to 0 with ramification index 1, so the degree is 1:\[\deg(f)=\sum_{P\in f^{-1}(0)}e_{f}(P)=\sum_{P\in f^{-1}(0)}\ord_P(f)=\sum_{P,\ord_P(f)>0}\ord_P(f)=1.\] A morphism of degree 1 is an isomorphism so $E\cong \Pj^1$ (Proposition~\ref{pr:deg1-iso}), contradiction.

\ul{Surjectivity}: Let $D\in \Div^0(E)$. Then $D+(O_E)$ has degree 1. Riemann-Roch~\ref{eq:rr-ec} tells us that $\dim(\cL(D+(O_E)))=1$ so there exists $f\in \ol K(E)^{\times}$ such that 
\[
\div(f)+D+(O_E)\ge 0.
\]
where the LHS has degree 1. Thus
\[
\div(f)+D+(O_E)=(P)
\]
for some $P\in E$. 
We see $D\sim (P)-(O_E)$; taking the divisor class,
\[
[D]=\phi(P).
\]
%We conclude that $\phi$ identifies $(E,\opl)$ with $\Pic^0(E),+)$, so $\opl$ is associative.
\end{enumerate}
\end{proof}

%\begin{proof}[Proof of group law]
%\begin{enumerate}
%\item
%Suppose $A+B+C=O$ on the elliptic curve, i.e., $A,B,C$ lie on the same line (with multiplicity). 
%Let $f=0$ be the projective equation of the line in $\Pj^2$. (For example, the line $y-x+1=0$ would become $\fc{y}{z}-\fc{x}{z}+1=0$.) %Suppose it intersects $E$ in $A,B,C$ (with multiplicity). 
%Then
%\[
%\div(f)=(A)+(B)+(C)-3(O).
%\]
%First, letting $B=-A$, $C=O$ gives
%\[
%\phi(A)+\phi(-A)=[(A)+(-A)-2(O)]=0
%\]
%so $\phi(-A)=-\phi(A)$. 
%We have
%\[
%\phi(A)+\phi(B)+\phi(C)=
%[(A)+(B)+(C)-3(O)]=0
%\]
%for all $A+B+C=0$. Writing $C=-A-B$, we get
%\[
%\phi(A+B)=-\phi(-A-B)=\phi(A)+\phi(B),
%\]
%as needed.
%
%\end{enumerate}•
%\end{proof}
\subsection{Elliptic curves are group varieties}
\begin{thm}
Elliptic curves are group varieties, i.e.,
\begin{align*}
[-1]:E&\to E; & P&\mapsto \ominus P\\
\opl: E\times E&\to E;& (P,Q)&\mapsto P\opl Q. 
\end{align*}
\end{thm}
Being a group variety is more than just being a group and being a variety. The group laws are actually morphisms.
\begin{proof}
This requires no further calculation, but there is some subtlety.
\begin{enumerate}
\item The above formulas says $[-1]:E\to E$ is a {\it rational} map. Thus $[-1]$ is a morphism, since $E$ is a smooth projective curve.
\item The above formula say $\opl:E\times E\to E$ is a rational map regular on $U=\set{(P,Q)\in E\times E}{P,Q,P\opl Q,P\ominus Q\ne O_E}$.
The result we quoted above only works on a smooth projective curve, not a surface, so we need another trick here.

For $P\in E$, let 
\begin{align*}
\tau_P:E&\to E\\
X&\mapsto X\opl P
\end{align*}
be translation by $P$. We have $\tau_P$ is a rational map, therefore a morphism. We factor $\opl:E \times E \to E$ as
\[
E\times E\xra{\tau_{\ominus A}\times \tau_{\ominus B}} E\times E \xra{\opl} E\xra{\tau_{A\opl B}} E.
\]
Thus $\opl$ is regular on $(\tau_A\times \tau_B)(U)$ for all $A,B\in E$. Clearly they agree on overlaps.

Here's an informal map: we want to avoid the diagonals; $U$ is anything not on those lines. %is this clear?

Thus $\opl$ is regular on $E\times E$ and $\opl$ is a morphism.
\end{enumerate}
\end{proof}

\subsection{The group $E(K)$ for different $K$}
What is the group $E(K)$? We will prove the following later on.
\begin{enumerate}
\item
For $K=\C$, $E(\C)\cong \C/\La\cong \R/\Z\cong \R/\Z$ for $\La$ a lattice. (It's a torus.)
\item
$K=\R$: 
\[E(\R)\cong 
\begin{cases}
\Z/2\Z\times \R/\Z\text{ if }\De>0\\
\R/\Z\text{ if }\De<0.
\end{cases}
\]
\item For $K=\F_q$,
\[
|E(\F_q)-(q+1)|\le2\sqrt q.
\]
This is Hasse's Theorem.
\item When $[K:\Q_p]<\iy$, $E(K)$ is a contains a subgroup of finite index isomorphic to $(\sO_K,+)$. 
\item When $[K:\Q]<\iy$, $E(K)$ is a finitely generated abelian group (Mordell-Weil Theorem).
\end{enumerate}
%Note that the isomorphisms on (i), (ii), and (iv), respect the relevant topologies.


%%%%%%%%%%
\section{Isogenies}
We give two approaches to isogenies. 
\begin{itemize}
\item
We'll give a hands-on approach that shows us how to compute with isogenies (by writing out the rational functions), and lets us understand the degree of an isogeny.
\item
We'll also see what we can do with a more theoretical approach that avoids calculations.
\end{itemize}

\begin{df} \llabel{df:isogeny}
Let $E_1$ and $E_2$ be elliptic curves defined over $k$.  An \textbf{isogeny} is a morphism $\alpha\colon E_1\to E_2$ that preserves the distinguished point (i.e. $\alpha(0) = 0$). 
\begin{itemize}
\item
We denote the set of isogenies by $\Hom(E_1,E_2)$. It is an abelian group, where addition is defined by addition in $E_2(\ol{k})$:
$$(\alpha + \beta)(P)=\alpha(P)+\beta(P).$$
\end{itemize}

When $E_1 = E_2$, we say that $\alpha$ is an \textbf{endomorphism}
\begin{itemize}
\item
We write $\Hom(E,E)= \End(E)$. This is a ring where multiplication is given by composition.
\end{itemize}•
 and we write $\Hom(E,E)= \End(E)$.
\end{df}

We have a group structure on an elliptic curve, so it's natural to restrict to isogenies that are group homomorphisms. In fact, we don't need to, because all isogenies are group homomorphisms!

\begin{thm}[Silverman~\cite{Si86}, Theorem III.4.8]\label{thm:grp-hom}
Let $E_1$ and $E_2$ be elliptic curves defined over $K$.
A regular rational map $\alpha\colon E_1\to E_2$ is an isogeny if and only if $\alpha\colon E_1(\overline{K}) \rightarrow E_2(\overline{K})$ is a group homomorphism. 
\end{thm}
We will prove this fact in Section~\ref{sec:isog-hom} using some algebraic geometry. Alternatively, we can use Theorem~\ref{thm:grp-hom} as our definition, since for all the isogenies we will be interested in it is easy (and useful) to show that they are group homomorphisms.
%\begin{df*}
%Let $E_1$ and $E_2$ be elliptic curves defined over $k$.  An \textbf{isogeny} is a morphism $\alpha\colon E_1\to E_2$ {\it that is a group homomorphism}.
%\end{df*}
%At first, this condition seems much stronger than the definition. Rather than proving this equivalence, however, we will take the condition in Theorem~\ref{thm:grp-hom} as our definition, since for all the isogenies we will be interested in it is easy (and useful) to show that they are group homomorphisms.



\subsection{Isogenies are group homomorphisms}\label{sec:isog-hom}

\begin{proof}[Proof of Theorem~\ref{thm:grp-hom}]

\end{proof}

\subsection{Isogenies: explicit approach}\label{sec:hom}
%We start by defining a few concepts that can be found in Section 2.9 of Washington~\cite{Wa08} which discusses endomorphisms of elliptic curves.  We first consider the more general notion of a homomorphism of elliptic curves, and then focus on endomorphisms, homomorphisms from an elliptic curve to itself.

%The standard definition of a homomorphism of elliptic curves is the following.

%\begin{df} Let $E_1$ and $E_2$ be elliptic curves defined over $k$.  An \emph{isogeny} is a regular rational map $\alpha\colon E_1\to E_2$ that preserves the distinguished point (i.e. $\alpha(0) = 0$).
%\end{df}

%A rational map is a map in which each coordinate is specified by a rational function.
%The regularity condition ensures that $\alpha$ can be evaluated at every point on $E_1$.
%This topic is treated in full generality in Section I.3 of Silverman \cite{Si86}, but, following Washington, we will shortly give a more narrow definition that addresses this issue explicitly for elliptic curves in Weierstrass form.

%The term isogeny means nothing more or less than a homomorphism of elliptic curves, but it helps to remind us that it is more than just a morphism of algebraic curves, it is also a group homomorphism. We have the following equivalence.


When working with isogenies it is often more convenient to work with affine coordinates and we will do so for the next two lectures.
But it is important to remember that whenever refer to the point $(x,y)$ in affine space, we are actually referring to the point $(x:y:1)$ of projective space, and $0$ refers to $(0:1:0)$, since, as usual, we assume elliptic curves are specified in Weierstrass form $y^2 = x^3+Ax+B$.
We begin by showing that without loss of generality we can assume that isogenies are specified in a standard form.

\begin{lem}\label{lem:hom-ec}
Suppose $E_1,E_2$ are elliptic curves in Weierstrass form.

Any isogeny $\al:E_1\to E_2$ can be written as
\[
\al(x,y)=\ba{\frac{u(x)}{v(x)}, \frac{s(x)}{t(x)}y}
\]
%
%Suppose $\alpha : E_1 \rightarrow E_2$ is a homomorphism of elliptic curves specified by
%\begin{align*}
%\alpha(x,y) &= 
%	\begin{cases}
%	(R_1(x,y),R_2(x,y)) & \text{when } R_1,R_2 \text{ are both defined,}\\
%	0 & \text{otherwise,}
%	\end{cases}
%\end{align*}
%where $R_1,R_2\in  \overline{K}(x,y)$ are rational functions in $x$ and $y$. Then 
%we can write $R_1$ and $R_2$ in the form 
%\bal
%R_1 &= \frac{u(x)}{v(x)}\\
%R_2 &= \frac{s(x)}{t(x)}y,
%\end{align*}
where $u,v,s,t \in \overline{k}[x]$ are polynomials in $x$.
\end{lem}

\begin{proof}
Write $\al(x,y)=(R_1(x,y),R_2(x,y))$, where $R_1,R_2\in  \overline{K}(x,y)$ are rational functions in $x$ and $y$.

Let's begin with $R_1$.  We can always write $R_1(x,y) = \frac{p_1(x)+p_2(x)y}{p_3(x)+p_4(x)y}$ because for any power of $y$ greater than one, we can substitute $y^2 = x^3+Ax+B$.  Multiply the top and the bottom by $p_3(x)-p_4(x)y$ to get \[R_1(x,y) = \frac{q_1(x)+q_2(x)y}{q_3(x)}.\]
Recall that a point $(x,y)$ on an elliptic curve in Weierstrass form has inverse $(x,-y)$, so the $x$-coordinate does not change under inversion. Since $\alpha$ is a group homomorphism, we must have $R_1(x,-y) = R_1(x,y)$.  Therefore, $q_2(x) = 0$.

The argument for $R_2$ is similar:  We have $R_2(x,y) = -R_2(x,-y)$, so for $R_2(x,y) = \frac{r_1(x)+r_2(x)y}{r_3(x)}$, we must have $r_1(x) = 0$.
\end{proof}

We may assume that the polynomials $u$ and $v$ of Lemma \ref{lem:hom-ec} are relatively prime, equivalently, that they have no common root in $\overline{k}$, and we write $u\perp v$ to denote this constraint.  Similarly we assume $s\perp t$.
We now give a more precise definition of an isogeny that is particularly convenient to work with.

\begin{df}[cf. Definition~\ref{df:isogeny}]\llabel{df:isogeny2}
Let $E_1$ and $E_2$ be elliptic curves over a field $k$ with characteristic not $2$.  Let $u,v,s,t \in \overline{k}[x]$ and let $\alpha$ be a map $E_1 \rightarrow E_2$ be given by
\begin{align*}
\alpha(x,y) &= 
	\begin{cases}
	\left(\frac{u(x)}{v(x)},\frac{s(x)}{t(x)}y\right) & \text{if } v(x)t(x) \neq 0,\\
	0, & \text{otherwise, }
	\end{cases}
\end{align*}
such that $\alpha\colon E_1(\overline{k}) \rightarrow E_2(\overline{k})$ is a group homomorphism.  Then $\alpha$ is an \emph{isogeny} from $E_1$ to~ $E_2$.
\end{df}

With~$\alpha$ in this form, we make the following definitions:

\begin{df}\label{df:degree-elem}
The \emph{degree} of a nonzero isogeny $\alpha$ is $\deg \alpha= \max \{\deg u, \deg v\}$.
By convention, the zero isogeny has degree 0.
\end{df}

\begin{df}\label{df:separable-elem}
A nonzero isogeny $\alpha$ is \emph{separable} if $\left(\dfrac{u}{v}\right)'\ne 0$ (as functions) and is \emph{inseparable} otherwise.
The zero isogeny is separable.
\end{df}

We can check that this agrees with the definitions of the degree and separability of a rational map.
\begin{pr}
Definitions~\ref{df:degree-elem} and~\ref{df:separable-elem} agree with Definition~\ref{df:degree-morphism}.
\end{pr}
\begin{proof}
\fixme{Add me.}
\end{proof}

%\begin{rem}
%This terminology is related to the use of the term ``separable" to mean a polynomial with no repeated roots (see the proof of Theorem \ref{countker}), and also to the notion of a separable field extension (see \cite[\S III.4]{Si86}).
%\end{rem}

\subsection{Examples of isogenies}

Our first example of an isogeny is a simple endomorphism, the multiplication by $2$ map, which doubles points on an elliptic curve.
This is obviously a group homomorphism, and we can easily show that it is defined by rational maps.

\begin{ex}[Doubling]\label{ex:mult-2}
Let $\alpha(P) = 2P$ on the elliptic curve $y^2 = f(x) = x^3+Ax+B$.
Recall that the formula for doubling a point is \[\alpha(x,y) = (m^2-2x, \ m(x-(m^2-2x))-y),\qquad \text{where }m = \frac{3x^2+A}{2y}.\]
We compute
\begin{align*}
\frac{u(x)}{v(x)} &= \frac{(3x^2+A)^2}{4y^2} - 2x\\
&= \frac{(3x^2+A^2)-8xf(x)}{4f(x)},\\
\frac{s(x)}{t(x)} &= \frac{3x^2+A}{2y}\Bigg( 3x-\frac{(3x^2+A)^2}{4y^2}\Bigg)-y\\
&= \frac{(3x^2+A)(12xy^2-(3x^2+A)^2)-8y^4}{8y^3}\\
&= \frac{(3x^2+A)(12xf(x)-(3x^2+A)^2)-8f(x)^2}{8f(x)^2}y\\
&= \frac{x^6+5Ax^4+20Bx^3-5A^2x^2-4ABx-A^3-8B^2}{8f(x)^2}y.
\end{align*}
\end{ex}

Even in this simple example, we see that it is already non-trivial to write down $u$,~$v$,~$s$, and $t$ for the multiplication by $2$ map. In the next section we will introduce \emph{division polynomials} to tackle the general multiplication by $m$ case.

Our second example is the Frobenius endomorphism.
\begin{ex}[Frobenius endomorphism]\label{ex:frob-end}
Let $E/\Fp$ be an elliptic curve and let $\pi\colon E \rightarrow E$ be the map
\begin{center}$\pi (x,y) = (x^p,y^p) = (x^p, f(x)^{\frac{p-1}{2}}y)$.
\end{center}
In this case it is easy to see that $\pi(x,y)$ is specified by rational functions, in fact polynomials: $u(x) = x^p$, $v(x) = 1$, $s(x) =  f(x)^{\frac{p-1}{2}}$, and $t(x) = 1$.

We now show that $\pi$ is a group endomorphism of $E(\overline{k})$.\footnote{Note that $\pi$ is \emph{not} the multiplication by $p$ map, which is another endomorphism that we will see soon.}
We first recall several facts about the Frobenius map over a finite field, which we also denote by $\pi$, given by $\pi\colon \overline{\F}_p \rightarrow \overline{\F}_p$, $\pi(x) = x^p$. The map $\pi$ is a field automorphism of $\Fp$, as we may check by noting that
\begin{enumerate}
\item $0^p=0$ and $1^p=1$.
\item$(ab)^p = a^pb^p$, $(a^{-1})^p = (a^p)^{-1}$ for all $a\in \overline{\F}_p$.
\item$(a+b)^p = \sum {p \choose i} a^ib^{p-i} = a^p+b^p$ for all $a,b\in \overline{\F}_p$.
\item$(-a)^p = -a^p$  for all $a\in \overline{\F}_p$.
\end{enumerate}
(Note: these properties also hold hold for the map $\pi(x)=x^q$ over $\mathbb{F}_q$, where $q = p^n$.)

This implies that for any $g \in \Fp[x_1,\ldots,x_k]$, and hence any rational function $g \in \Fp(x_1,\ldots,x_k)$, we have
\[
g(x_1,\ldots,x_k)^p = g(x_1^p,\ldots,x_k^p).
\]
Applying this to the expressions for adding, doubling, and negating points, and noting that $\pi(0)=0$, we see that the Frobenius map is an endomorphism on elliptic curves.
\end{ex}

Note that if $(x,y)$ is a point in $E(\Fp)$, then $\pi(x,y)=(x^p,y^p)=(x,y)$, so the Frobenius endomorphism acts trivially on $E(\Fp)$.
However it is important remember that when we are talking about endomorphisms on elliptic curves, we should be thinking about the group of points over $\overline{k} = \overline{\F}_p$.  The Frobenius endomorphism does not act trivially on $E(\ol{\Fp})$.
In fact, $\Fp$ is precisely the subset of $\ol{\Fp}$ fixed by $\pi$, and it follows that $E(\Fp)$ is precisely the subgroup of $E(\ol{\Fp})$ fixed by $\pi$.

%Before moving on, we recall a few standard facts about finite fields.
%\begin{enumerate}
%\item Every finite field has order $q=p^n$ for some prime $p$ and integer $n>0$.
%\item All finite fields of the same order are isomorphic (but see below).
%\item $\F_q^*$ is cyclic of order $q-1$.
%\item $\overline{\mathbb F}_{p} = \bigcup_{n>0} \mathbb{F}_{p^n}$.
%\item $\mathbb{F}_{p^n} \subset \mathbb{F}_{p^m}$ if and only if $n \mid m$.
%\end{enumerate}
%
%Although all finite fields of the same order are isomorphic, finite fields of non-prime order may be represented in many different ways.  One common way to represent such a finite field is to choose an irreducible polynomial $f$ of degree~$n$ in $\Fp[x]$. Then  $\mathbb{F}_{p^n} \simeq \Fp[x]/(f)$.\footnote{An alternative approach is to represent elements of $\F_q^*$ as powers of a generator.  This makes multiplication easy, but addition becomes much more difficult.}
%In problem 5 of the first problem set, for example, we represented $\mathbb{F}_{p^2}=\Fp(f)$ using $f = x^2-x-1$, which was a particularly convenient choice.  But we could also have used $f=x^2-5$.
%As discussed in Lecture 4, the representation $\F_q\simeq \Fp[x]/(f)$ allows us to efficiently implement finite field operations using polynomial arithmetic, which can in turn be performed using integer arithmetic.

%\subsection{Multiplication-by-$m$ maps: $[m]$}\label{sec:mult-m}
We now discuss endomorphisms that multiply points on an elliptic curve by an integer~$m$, which we denote $[m]$.
These are clearly group homomorphisms from $E(\ol{k})$ to $E(\ol{k})$, but we need to express them as rational maps.
To represent these maps generically, we will define what are known as ``division polynomials.''  We've already seen these polynomials in the case $m=2$ (Example~\ref{ex:mult-2}).  To compute polynomials for the general case, rather than using affine or standard projective coordinates, it is more convenient to use weighted projective coordinates, also known as \emph{Jacobian coordinates} (which we may then transform back to our standard affine format).

In weighted project coordinates our standard Weierstrass curve equation becomes
\[y^2 = x^3 + Axz^4 + Bz^6,\]
where we think of $x$ as having weight 2 and $y$ having weight 3; this makes the equation homogeneous of degree 6.
In weighted projected coordinates, our equivalence relation on triples changes: now $(x,y,z)\sim (\lambda^2x,\lambda^3y,\lambda z)$ for any scalar $\lambda\in\ol{k}^*$.
The triple $(x:y:z)$ corresponds to the affine point $\left(\frac{x}{z^2},\frac{y}{z^3}\right)$.
%
%As an aside, we can now explain the numbering of the coefficients in the general Weierstrass equation,
%\[y^2 +a_1xy + a_3y = x^3 + a_2x^2 + a_4x +a_6\]
%The subscript $i$ of $a_i$ is the power of $z$ that makes the equation homogeneous in weighted projective coordinates.

In order to use these weighted projective coordinates, we need to write down the group law for them.
We can do this using the corresponding affine points.
For example, to double the point $(x_1:y_1:z_1)$ we compute
\begin{align*}
m & = \frac{3(x_1/z_1^2)+A}{2(y_1/z_1^3)} = \frac{3x_1^2+Az_1^4}{2y_1z_1}\\
\frac{x_3}{z_3^2} & = m^2 - 2\frac{x_1}{z_1^2} = \frac{(3x_1^2+Az_1^4)^2-8x_1y_1^2}{z_3^2} \hspace{5mm}\text{where } z_3 = 2y_1z_1\\
\frac{y_3}{z_3^3} & = m\left(\frac{x_1}{z_1^2} - \frac{x_3}{z_3^2}\right)-\frac{y_1}{z_1^3} = \frac{(3x_1^2+Az_1^4)(4x_1y_1^2-x_3)-8y_1^4}{z_3^3}.
\end{align*}
The addition law may be computed similarly.

If we start with a generic point $P=(x:y:1)$, and apply the group law to compute $2P$, $3P$, $4P$, \ldots we obtain generic formulas for the multiplication-by-$m$ maps as triples of polynomials $(\phi_m : \omega_m : \psi_m)$ in $\Z[x,y,A,B]$.  Here we treat $A$ and $B$ as variables in order to get generic formulas, but in practical applications these will be instantiated with the coefficients of a particular curve equation.
To put these formulas in standard form we use the curve equation $y^2=x^3+Ax+B$ to reduce powers of~$y$, and then put the maps in affine form $(\phi_m/\psi_m^2, \omega_m/\psi_m^3)$, eliminating any common factors from the numerator and denominator of each coordinate. See the Sage worksheet for details:
\begin{center}
\url{https://hensel.mit.edu:8002/home/pub/4/}
\end{center}

In principal this approach can be used to generate rational maps for $[m]$ for any $m$.
However, it turns out that the computation can be simplified dramatically by focusing just on the polynomial $\psi_m$ for the $z$-coordinate, which satisfies a set of recurrences that allow us to compute $\psi_m$ much more efficiently, and can also be used to define the polynomials $\phi_m$ and~$\omega_m$.
Thus the polynomials $\psi_m$ are traditionally known as ``the" division polynomials, although we may use term more generically to refer to any of the polynomials associated with the multiplication-by-$m$ maps.

Note that the points where $\psi_m$ vanishes are precisely the non-trivial points in the kernel of the endomorphism $[m]$, correspond to the $m$-torsion subgroup of $E(\ol{k})$.
We will see in later lectures that any finite subgroup of $E(\ol{k})$ uniquely determines an isogeny (in this case, an endomorphism), which explains why $\psi_m$ effectively determines $[m]$.

\subsection{Division Polynomials}

Let $\psi_0=0$, and let $\psi_1,\psi_2,\psi_3,\psi_4$ be as computed in Sage:
\begin{align*}
\psi_1&=1\\
\psi_2&=2y\\
\psi_3&=3x^4+6x^2A-A^2+12xB\\
\psi_4&=4x^6y+20x^4yA-20x^2yA^2+80x^3yB-4yA^3 -16xyAB-32yB^2
\end{align*}
To compute $\psi_m$ for $m>4$, we may apply the following recurrences:
\begin{align*}
\psi_{2m+1} & = \psi_{m+2}\psi_{m}^3-\psi_{m-1}\psi_{m+1}^3, & m \geq 2\\
\psi_{2m} & = \frac{1}{2y}\psi_{m}(\psi_{m+2}\psi_{m-1}^2-\psi_{m-2}\psi_{m+1}^2), & m \geq 3
\end{align*}
It is not difficult to show that $\psi_{m}(\psi_{m+2}\psi_{m-1}^2-\psi_{m-2}\psi_{m+1}^2)$ is always divisible by $2y$, so that $\psi_{2m}$ is in fact a polynomial.

We next define the polynomials $\phi_m$ and $\omega_m$ for the $x$ and $y$ coordinates in terms of $\psi_m$.
\begin{align*}
\phi_{m} & := x\psi_{m}^2-\psi_{m+1}\psi_{m-1} & m \geq 1\\
\omega_{m} & := \frac{1}{4y}(\psi_{m+2}\psi_{m-1}^2-\psi_{m-2}\psi_{m+1}^2) & m \geq 1,\,\psi_{-1} = -1
\end{align*}
We now record some key properties of these polynomials.

\begin{lem}
 Let $f(x) = x^3+Ax+B$. Then
\begin{align*}
\psi_{n} \bmod (y^2-f(x)) \text{ lies in} & \begin{cases} \Z[x,A,B] & n \text{ odd}\\
2y\Z[x,A,B] & n \text{ even,}\end{cases}\\
\phi_{n} \bmod (y^2-f(x)) \text{ lies in } & \hspace{4mm} \Z[x,A,B] \hspace{5mm} \text{ for all } n,\\
\omega_{n} \bmod (y^2-f(x)) \text{ lies in} & \begin{cases} \Z[x,A,B] & n \text{ even}\\
y\Z[x,A,B] & n \text{ odd.}\end{cases}
\end{align*}
\end{lem}
\begin{proof} See Lemmas 3.3 and 3.4 in Washington \cite{Wa08}.
\end{proof}

\begin{thm}
Let $P = (x,y)$ be a point on an elliptic curve $E : y^2 +x^3+Ax+B$ over a a field of characteristic different from 2.  Then
\[nP = \left( \frac{\phi_n(x)}{\psi_n^2(x)},\frac{\omega_n(x,y)}{\psi_n^3(x,y)} \right) \hspace{8mm} \text{for all } n >0.\]
\end{thm}

\begin{proof}
The standard proof uses complex analysis and the Weierstrass $\wp$-function (as in Chapter 9 of Washington \cite{Wa08}).
However, it can be given a purely computational proof using the group law, as we did earlier in Sage.
See Exercise 3.7 in Silverman \cite{Si86}.
\end{proof}

\begin{thm}
The polynomials $\phi_n$ and $\psi_n$ are in the form
\begin{align*}
\phi_n(x) & = x^{n^2}+ (\text{lower degree terms}),\\
\psi_n(x) & = \begin{cases} nx^{\frac{n^2-1}{2}} + (\text{lower degree terms}),  & n\text{ odd}\\
y(nx^{\frac{n^2-4}{2}} + (\text{lower degree terms})), & n\text{ even.}\\
\end{cases}
\end{align*}
\end{thm}

\begin{proof}
We'll just do the case where $n = 2m+1$ and $m$ is odd, we'll leave the rest as an exercise.
In this case the leading term of $\psi_n = \psi_{2m+1}$ is
\begin{align*}
& (m+2)x^{\frac{(m+2)^2-1}{2}}m^3x^{3\cdot \frac{m^2-1}{2}}-(m-1)x^{\frac{(m-1)^2-4}{2}}(m+1)^3x^{3\cdot \frac{(m+1)^2-4}{2}}y^2\\
&= -(m^4+2m^3)x^{\frac{4m^2+4m}{2}}+(m^4+2m^3+2m+1)x^{\frac{4m^2+4m}{2}}\\
&= (2m+1)x^{\frac{(2m+1)^2-1}{2}}\\
&= nx^{\frac{n^2-1}{2}}.\qedhere
\end{align*}
\end{proof}
%\subsection{Isogenies: explicit approach}
%
%In this lecture $p$ is the characteristic of the field $k$ over which all elliptic curves under consideration are defined (possibly $p=0$).
%We assume throughout $p\ne 2,3$ and that our elliptic curves are in short Weierstrass form $y^2=x^3+Ax+B$.
%The main theorems (everything denoted \textbf{Theorem} in these notes) also hold when $p$ is 2 or 3, even though we do not prove this.
%In this lecture we lay the groundwork for the proof of Hasse's theorem.
%
%Some comments on solutions to the first problem set are recorded in Appendix \ref{app:pset-notes}.

%\subsection{Isogenies and Endomorphisms}
%We start with some more background on isogenies and endomorphisms.
%Recall our definition of an isogeny.
%
%\begin{df} An \emph{isogeny} between elliptic curves $E_1/k$ and $E_2/k$ is a rational map that is a group homomorphism, $\alpha\colon E_1(\ol{k})\rightarrow E_2(\ol{k})$
%\end{df}

%Isogenies are the morphisms in the category of elliptic curves.
%The set $\Hom(E_1,E_2)$ of isogenies from $E_1$ to $E_2$ is an abelian group, where addition is defined by addition in $E_2(\ol{k})$:
%$$(\alpha + \beta)(P)=\alpha(P)+\beta(P).$$
%
%Recall that an \emph{endomorphism} is an isogeny from an elliptic curve $E$ to itself, an element of $\Hom(E,E)$. We can compose endomorphisms, which gives us a way to multiply elements in $\Hom(E,E)$, yielding a (not necessarily commutative) ring $\End(E)$.

%In the last lecture we showed that we may assume without loss of generality that every nonzero isogeny $\alpha$ can be written in the form\footnote{Here we need the equation for $E_1$ to be of the form $y^2=f(x)$, which requires $p\ne 2$.}
%
%\begin{equation}\label{alphadef}
% \alpha(x,y)=\begin{cases} \left(\dfrac{u(x)}{v(x)},\dfrac{s(x)}{t(x)}y\right)&\mbox{if } v(x)t(x) \ne  0, \\
%0 & \mbox{otherwise,} \end{cases}  
%\end{equation}
%where $u,v,s,t\in \ol{k}[x]$ satisfy $u\perp v$ and $s\perp t$ (the notation $u\perp v$ indicates that $u$ and $v$ have no common roots in $\ol{k}$). And of course we also have $\alpha(0)=0$.
%Note that $u\perp v$ and $s\perp t$ implies that none of these polynomials is the zero polynomial.


\subsubsection{The degree and separability of the multiplication-by-\texorpdfstring{$n$}{} map}
In several of our later proofs we use the fact that multiplication by $n$ is a separable isogeny whenever $p\nmid n$, and has degree $n^2$.
To prove this, we begin with some easy lemmas.

\begin{lem}\label{quoderiv0lma}
If $u\perp v$, then $\left(\dfrac{u}{v}\right)'=0$ if and only if $u'=v'=0$.
\end{lem}

\begin{proof}
The proof is a simple computation:

$$\left(\dfrac{u}{v}\right)'=\dfrac{u'v-uv'}{v^2}$$

If $u'=v'=0$, then clearly $\left(\dfrac{u}{v}\right)'=0$. Conversely, if $\left(\dfrac{u}{v}\right)'=0$, then we have $u'v=v'u$. Consider the roots of the left hand side, with multiplicity. Since $u\perp v$, every root of $v$ must be a root of $v'$. Thus $v'$ has at least $\deg v$ roots.
But this is impossible unless $v'=0$, since otherwise $\deg v' < \deg v$.
The same argument shows $u'=0$.
\end{proof}

In characteristic zero the only polynomials with zero derivatives are the constant functions.
But $u$ and $v$ cannot both be constants: if they were then $\alpha$ would have trivial kernel (since $v\ne 0$) and finite image (at most two points have $x$-coordinate $u/v$), which is impossible, since its domain $E_1(\ol{k})$ is infinite.
Thus in characteristic zero every isogeny is separable.  But in positive characteristic things get more interesting.

\begin{lem}\label{derivequals0lma}
In any field $k$, a polynomial $f\in k[x]$ has $f'=0$ if and only if $f(x)=g(x^p)$ for some $g\in k[x]$, where $p=\chr k$.
\end{lem}

\begin{proof}
Let $f(x)=\sum_i a_i x^i$. Then $f'(x)=\sum ia_ix^{i-1}=0$ if and only if $ia_i=0$ for all~$i$. This holds if and only if $p|i$ for every $i$ with $a_i\ne0$, equivalently $f(x)=g(x^p)$, where $g(x)=\sum_j a_{pj}x^j$.
\end{proof}

\begin{ex}
For $k=\Fp$, the Frobenius endomorphism, $\pi(x,y)=(x^p,y^p)$ has degree $p$ and is inseparable, since $(x^p)'=px^{p-1}=0$ in characteristic $p$. 
\end{ex}

\begin{thm}\label{nmapdeg}
The multiplication-by-$n$ map $[n]=\left(\dfrac{\phi_n(x)}{\psi_n^2(x)},\dfrac{\omega_n(x,y)}{\psi_n^3(x,y)}\right)$ has degree $n^2$, and is separable if $p$ does not divide $n$.
\end{thm}

\begin{proof}
We use a shortcut here that simplifies in the proof in Washington \cite[Corollary 3.7]{Wa08}.

If $\phi_n\perp \psi_n^2$ then the proof follows immediately from Theorem 5.9 given in the last lecture (and proved in Problem Set 2), since the leading term of $\phi_n$ is $x^{n^2}$ and $\deg \psi_n^2 < n^2$ (note that $\phi_n'$ has leading coefficient $n^2$, which is zero only if $p$ divides $n$).

So suppose that $\phi_n$ and $\psi_n$ have a common root $x_0\in\ol{k}$.
Let $y_0\in \ol{k}$ satisfy $y_0^2=x_0^3+Ax_0+B$, where $y^2=x^3+Ax+B$ is the Weierstrass equation for $E$.
Notice that such a $y_0$ exists because $\ol{k}$ is algebraically closed.

We now consider the point $P=(x_0,y_0)$ (which is not~0, because it is an affine point).
We may assume $n>1$, since $\psi_1^2=1$ has no roots.  From the definition of $\phi_n$ we have
\[
\phi_n(x_0)=x_0\psi_n(x_0)^2-\psi_{n+1}(x_0)\psi_{n-1}(x_0).
\] 
Applying $\phi_n(x_0)=\psi_n(x_0)=0$ yields
\[
0=\psi_{n+1}(x_0)\psi_{n-1}(x_0).
\]
Thus $x_0$ is a root of either $\psi_{n+1}$ or $\psi_{n-1}$.
Note that $nP=0$ if and only if $x_0$ is a root of~$\psi_n$, by the definition of $[n]$, see \eqref{alphadef}.
Thus either $(n-1)P=0$ or $(n+1)P=0$, but in either case we may add or subtract $nP=0$ to obtain $P=0$, which is a contradiction.
\end{proof}
%
%\subsection{The structure of the subgroup \texorpdfstring{$E[n]$}{}}
%\begin{df}
%The $n$-\emph{torsion subgroup} $E[n]$ of $E(\ol{k})$ is the kernel of $[n]$.
%For any field $K$ containing $k$, we define $E(K)[n]=\{P\in E(K):nP=0\}$. 
%\end{df}
%\begin{rem}
%The \emph{rational} torsion subgroup $E(k)[n]$ is not the same thing as $E[n]$; it is the intersection $E[n]\cap E(k)$, which is typically a proper subgroup of $E[n]$ (here the word ``rational" means ``with coordinates in the field of definition").
%\end{rem}
%
%We now prove a technical lemma that will simplify the proofs that follow by allowing us to focus on just the $x$-coordinate of an isogeny.
%
%\begin{lem}\label{xissuff}
%Let $\left(\frac{u}{v} ,\frac{s}{t} y \right)$ be a nonzero isogeny $E_1\to E_2$ with $u\perp v$ and $s\perp t$, where $E_1$ is defined by $y^2=x^3+A_1x+B_1$ and $E_2$ is defined by $y^2=x^3+A_2x+B_2$, over a field $k$.
%\renewcommand{\labelenumi}{(\roman{enumi})}
%\begin{enumerate}
%\item $v(x_0)=0$ if and only if $t(x_0)=0$, for every $x_0\in\ol{k}$.
%\item Suppose $k$ has characteristic $p>0$.  If $\frac{u}{v}=r_1(x^p)$, then $\frac s t y\equiv_{E_1} r_2(x^p)y^p$,
%where $r_1, r_2\in\ol{k}(x)$ and $\equiv_{E_1}$ indicates equivalence modulo $y^2-x^3-A_1x-B_1$.
%\end{enumerate}
%\end{lem}
%
%\begin{proof}
%By substituting $\left(\frac{u}{v}, \frac{s}{t} y \right)$ for $(x,y)$ in the equation for $E_2$ we obtain
%$$\left(\frac{s}{t}y\right)^2 = \left(\frac{u}{v}\right)^3+A_2\frac{u}{v}+B_2.$$
%Using the equation for~$E_1$ to eliminate $y^2$ yields
%$$\frac{s^2(x^3+A_1x+B_1)}{t^2}= \frac{u^3+A_2uv^2+B_2v^3}{v^3}.$$
%Setting $w=(u^3+A_2uv^2+B_2v^3)$ and clearing denominators then gives
%\begin{equation}\label{eq:vstw}
%v^3s^2(x^3+A_1x+B_1)=t^2w.
%\end{equation}
%We first prove (i). Note that $u\perp v$ implies $v\perp w$, since any common root of $v$ and $w$ must be a root of $u$.
%Since $v$ divides the LHS of \eqref{eq:vstw}, and $v \perp w$, every root of $v$ must be a root of~$t$.
%This prove the forward implication in (i).  For the converse, suppose $t(x_0)=0$.
%Then $x_0$ is a double root of the RHS of \eqref{eq:vstw}, and must therefore be a double root of the LHS.
%Since $E_1$ nonsingular (by definition), $x^3+A_1x+B_1$ has no multiple roots.  So $x_0$ must be a root of $v^3s^2$, and $s\perp t$ implies that $x_0$ is a root of $v$, completing the proof of (i).
%
%We now prove (ii). 
%Suppose $\frac u v =r_1(x^p)$.
%Then by Lemma \ref{quoderiv0lma} we must have $u'=v'=0$. We then have
%\[
%w'=3u'u^2+A_2u'v^2+2A_2v'vu+3B_2v'v^2 = 0.
%\]
%This implies that $\left(\frac{w}{v^3}\right)'=\left(\frac{s^2f}{t^2}\right)'=0,$
%where $f(x)=x^3+A_1x+B_1$. We may write $f=f_1f_2$ with $f_1\vert t$, $f_2\perp t$, and $f_1\perp f_2$, for some $f_1,f_2\in\ol{k}[x]$.
%Let $t_1=\frac{t}{f_1}$, so $t^2=t_1^2f_1^2$. Then
%
%$$0=\left(\frac{s^2f_1f_2}{t_1^2f_1^2}\right)'=\left(\frac{s^2f_2}{t_1^2f_1}\right)'.$$
%
%Since the numerator and denominator of the fraction on the RHS are relatively prime, we have $(s^2f_2)'=0$ and $(t_1^2f_1)'=0$, by Lemma \ref{quoderiv0lma}. Then by Lemma \ref{derivequals0lma}, there exist polynomials $g,h\in \ol{k}[x]$ such that $s^2f_2=g(x^p)$ and $t_1^2f_1=h(x^p)$. Every root of these polynomials occurs with multiplicity that is divisible by $p$. Thus we may write $s^2f_2=s_2^2f_2^p$, and $t_1^2f_1=t_2^2f_1^p$ for some $s_2,t_2\in\ol{k}[x]$ (here we use the fact that $p-1$ is even).
%Since $s_2^2f_2^p=g(x^p)$ and $t_2^2f_1^p=h(x^p)$, the polynomials $s_2$ and $t_2$ are both functions of $x^p$ (to see this, differentiate and apply Lemma \ref{quoderiv0lma}).
%Thus $s_2=g_2(x^p)$ and $t_2^2=h_2(x^p)$ for some $g_2$ and $h_2$ in $\ol{k}[x]$.
%Using the equation $y^2=f(x)$ for $E_1$, we then obtain
%\[
%\left(\frac{s}{t}y\right)^2=
%\frac{s^2f}{t^2}=
%\frac{s^2f_2f_1^p}{t_1^2f_1f_1^p}=
%\frac{s_2^2f_2^pf_1^p}{t_2^2f_1^{2p}}=
%\frac{g_2(x^p)^2f(x)^p}{h_2(x^p)^2f_1(x)^{2p}}=
%\frac{g_2(x^p)^2y^{2p}}{h_2(x^p)^2f_1(x^p)^2}=
%\left(\frac{g_2(x^p)y^p}{h_2(x^p)f_1(x^p)}\right)^2,
%\]
%which implies that $\frac{s}{t}y=r_2(x^p)y^p$ with $r_2=\frac{g_2}{h_2}{f_1}$, as desired.
%\end{proof}
%
%\begin{cor}\label{factoralpha}
%Let $\alpha$ be a nonzero inseparable isogeny over a field of characteristic $p>0$. Then $\alpha=\alpha_1\circ \pi$ for some $\alpha_1:E_1\rightarrow E_2$, where $\pi(x,y)=(x^p,y^p)$.
%\end{cor}
%
%\begin{proof}
%Let $\alpha=(\frac u v,\frac s t y)$ with $u\perp v$ and $s\perp t$.
%Since $\alpha$ is inseparable, we have $\left(\frac u v \right)'=0$, and since $u\perp v$ we have $u'=v'=0$, by Lemma \ref{quoderiv0lma}. Then by Lemma \ref{derivequals0lma}, $u=f(x^p)$ and $v=g(x^p)$ for some $f,g\ \in \ol{k}[x]$.
%Thus $\frac u v=r_1(x^p)$ for some $r_1\in\ol{k}(x)$, and by Lemma \ref{xissuff} we have $\frac s t y =r_2(x^p)y^p$ for some $r_2\in\ol{k}(x)$.
%Thus $\alpha=\alpha_1\circ \pi$, where $\alpha_1=(r_1(x),r_2(x)y)$. 
%\end{proof}
%
%\begin{rem}
%By repeatedly applying the Corollary \ref{factoralpha}, we may write any isogeny $\alpha$ as $\alpha=\alpha_{\rm sep}\circ \pi^n$, where $\alpha_{\rm sep}$ is separable and $n$ is a nonnegative integer.
%The degree of $\alpha_{\rm sep}$ is called the \emph{separable degree} of $\alpha$, and the integer $p^n$ is the \emph{inseparable degree} of $\alpha$.
%\end{rem}
%
%\begin{lem}
%Let $\alpha,\beta\colon E_1\rightarrow E_2$ be isogenies.
%\renewcommand{\labelenumi}{(\roman{enumi})}
%\begin{enumerate}
%\item If $\alpha$ and $\beta$ are inseparable, then $\alpha+\beta$ is either inseparable or $0$.
%\item If $\alpha$ is inseparable, and $\beta\ne 0$ is separable, then $\alpha+\beta$ is separable.
%\end{enumerate}
%\end{lem}
%
%\begin{proof}
%(i)
%If $\alpha$ and $\beta$ are inseparable, then by Corollary \ref{factoralpha} we may write $\alpha=\alpha_1\circ \pi$ and $\beta=\beta_1\circ \pi$.
%We then have
%$$\alpha+\beta=\alpha_1\circ\pi+\beta_1\circ\pi=(\alpha_1+\beta_1)\circ\pi,$$
%which is either inseparable or the zero isogeny.
%
%(ii) If $\alpha+\beta$ is inseparable, then so is $-(\alpha+\beta)$, and (i) implies that $\alpha+(-(\alpha+\beta))=\beta$ is either inseparable or 0.
%Taking the contrapositive yields (ii).
%\end{proof}
%
%\begin{thm}\label{thm:surjective}
%Let $\alpha\colon E_1\rightarrow E_2$ be a nonzero isogeny. Then $\alpha$ is surjective (onto $E_2(\ol{k})$).
%\end{thm}
%
%\begin{proof}
%Let $\alpha=(\frac u v , \frac s t y)$, with $u\perp v$.
%Let $(a,b)$ be any nonzero point in $E_2(\ol{k})$, and let $f=u-av$. We break the proof into two cases:
% \medskip
%
%\noindent
%\textbf{Case 1}: $f$ has a root $x_0\in \ol{k}$.\\
%Pick $y_0\in\ol{k}$ so that $(x_0,y_0)\in E_1(\ol{k})$ (this is possible since $\ol{k}$ is algebraically closed).
%We have $f(x0)=u(x_0)-av(x_0)=0$ with $v(x_0)\ne 0$, since $u\perp v$ implies $f\perp v$.
%Thus $a=u(x_0)/v(x_0)$, so $\alpha(x_0,y_0)=(a,b')\in E_2(\ol{k})$ for some $b'$.
%Let $E_2$ be given by $y^2=x^3+A_2x+B_2$. Since both $(a,b)$ and $(a,b')$ are on the curve $E_2$, we have 
%
%$$b'^2=a^3+A_2a+B_2=b^2.$$
%
%Thus $b'=\pm b$, and either $\alpha(x_0,y_0)=(a,b)$ or $\alpha(x_0,-y_0)=(a,b)$.
%It follows that $\alpha$ is surjective (note $\alpha(0)=0$).
%
%\textbf{Case 2}: $f$ has no roots in $\ol{k}$\\
%Since every non-constant polynomial over an algebraically closed field has a root, this means that $f$ is a constant.
%But, as noted earlier, $u$ and $v$ cannot both be constant.
%It follows that $a$ is uniquely determined as the ratio of the leading coefficients of $u$ and $v$.
%
%Since $a$ is unique, there can be at most 2 points, $(a,b)$ and $(a,\pm b)$, that do not lie in the image of $\alpha$ (all other points must fall into Case 1).
%Choose $(a',b')=\alpha(P_1)$ so that $(a,b)+(a',b')\ne (a,\pm b)$ (this is possible since $E_1(\ol{k})$ and $E_2(\ol{k})$ are infinite).
%Then $(a,b)+(a',b')$ lies in the image of $\alpha$ and is equal to $\alpha(P_2)$ for some $P_2\in E_1(\ol{k})$.  But then
%\[
%\alpha(P_1-P_2)=(a,-b)\qquad\text{and}\qquad\alpha(P_2-P_1)=(a,b),
%\]
%so $(a,b)$ and $(a,-b)$ are  in the image of $\alpha$.  Thus $\alpha$ is surjective.
%\end{proof}
%
%\begin{thm}\label{countker}
%Let $\alpha$ be a nonzero isogeny from $E_1$ to $E_2$. 
%\renewcommand{\labelenumi}{(\roman{enumi})}
%\begin{enumerate}
%\item If $\alpha$ is separable, then $\deg \alpha=\#\ker \alpha$
%\item If $\alpha$ is inseparable, then $\deg \alpha>\#\ker \alpha$
%\end{enumerate}
%\end{thm}
%
%\begin{proof}
%Let $(a,b)$ be a point in the image of $\alpha$ with $a,b\ne0$ (this is possible since $\alpha$ is nonzero).
%Let $\alpha$ be in our standard form $(\frac u v, \frac s t y)$.
%Consider the set
%$$S(a,b)=\{(x,y)\in E_1: \alpha(x,y)=(a,b)\}$$
%of points in the pre-image of $(a,b)$.
%Since $\alpha$ is a group homomorphism, $\#S(a,b)=\#\ker\alpha$. 
%
%If $(x_0,y_0)\in S(a,b)$ then
%$$\frac{u(x_0)}{v(x_0)}=a,\qquad \frac{s(x_0)}{t(x_0)}y_1=b.$$
%We must have $t(x_0)\ne 0$, since $\alpha$ is defined at $(x_0,y_0)$, and $b\ne 0$ implies $s(x_0)\ne 0$.
%It follows that $y_0=\frac{t(x_0)}{s(x_0)}b$ is uniquely determined by $x_0$.
%Thus to compute $\#S(a,b)$ it suffices to count the number of distinct possibilities for $x_0$.
%
%As in the proof of Theorem~\ref{thm:surjective}, we let $f=u-av$ so that $\alpha(x_0,y_0)=(a,b)$ if and only if $f(x_0)=0$.
%We may assume $\deg f = \deg\alpha$, since there is at most one choice of $a$ for which $\deg f < \deg\alpha$ (this can occur only if $a$ is the ratio of the leading coefficients of $u$ and $v$), and there are infinitely many points $(a,b)$ to choose from (since $\alpha$ is surjective and $E_2(\ol{k})$ is infinite).
%The cardinality of $S(a,b)$ is then equal to the number of \emph{distinct} roots of $f$; of course $f$ has exactly $\deg f = \deg\alpha$ roots, when counted with multiplicity.
%
%A root $x_0$ of $f$ is a multiple root of $f$ if and only if $f(x_0)=f'(x_0)=0$.
%Equivalently, $x_0$ is a multiple root if and only if $av(x_0)=u(x_0)$ and $av'(x_0)=u'(x_0)$. If we multiply opposing sides of these equations and cancel the $a$ we get
%\begin{equation}\label{eq1}
%u'(x_0)v(x_0)=v'(x_0)u(x_0).
%\end{equation}
%
%If $\alpha$ is separable, then $u'v-v'u$ is not the zero polynomial and has only finitely many roots.
%We may assume that $(a,b)$ was chosen so that \eqref{eq1} is not satisfied for any $(x_0,y_0)$ in $S(a,b)$.
%Then every root of $f$ is distinct and we have $\#S(a,b)=\deg f=\deg \alpha$ as desired.
%
%If $\alpha$ is inseparable, then $u^\prime v-v^\prime u=0$ and every root of $f$ is a multiple root no matter which point $(a,b)$ we  choose.
%Thus the number of distinct roots of $f$ is strictly less than its degree and we have $\#S(a,b) < \deg f = \deg\alpha$.
%\end{proof}
%
%\begin{rem}
%We won't need this, but in the inseparable case, $\deg \alpha=p^n\#\ker \alpha$, where $p^n$ is the inseparable degree of $\alpha$.
%\end{rem}
%
%We can now determine the structure of the abelian group $E[n]$.
%
%\begin{thm}
%Let $E$ be an elliptic curve defined over a field of characteristic $p$.
%Then $E[n]\cong \Z/n\Z \oplus \Z/n\Z$ if $p\nmid n$, and if $p>0$ then either $E[p]\cong \Z/p\Z$ or $E[p]\cong \{0\}$.
%\end{thm}
%
%\begin{proof}
%Assume $p\nmid n$ and let $\ell$ be a prime dividing $n$.
%The subgroup $E[\ell]\subseteq E[n]$ is a finite abelian of order $\deg[\ell]=\ell^2$\ (by Theorems~\ref{countker} and~\ref{nmapdeg}), in which every nonzero element has order $\ell$.  It follows that $E[\ell]\cong \Z/\ell\Z\otimes\Z/\ell Z$, and the $\ell$-rank of $E[n]$ must be 2 (the $\ell$-Sylow subgroup of $E[n]$ may be larger than $E[\ell]$, but its rank cannot be larger).
%If $\ell^e$ is the largest power of $\ell$ dividing $n$, then the $\ell$-Sylow subgroup of $E[n]$ is $E[\ell^e]$, which must be isomorphic to $\Z/\ell^e\Z\oplus\Z/\ell^e\Z$, since it has order $\ell^{2e}$, rank 2, and no elements of order greater than $\ell^{e}$.
%It follows that $E[n]\cong \Z/n\Z \oplus \Z/n\Z$.
%
%If $p>0$, then $[p]$ is inseparable, and by Theorem~\ref{countker}, its kernel $E[p]$ has order strictly less than $\deg [p]=p^2$.
%Since $E[p]$ is a $p$-group of order less than $p^2$, it must be isomorphic to either $\Z/p\Z$ or $\{0\}$.
%\end{proof}

%%%%%%%%%%%%%%
\section{Appendix: calculations}
\subsection{The group law in SAGE}
The following code computes the group law. % as rational functions
\begin{lstlisting}
RR.<Px,Py,Qx,Qy,Rx,Ry,A,B> = PolynomialRing(QQ,8)
# represent projective points on E uniquely, as either affine points (x,y,1) or the point O=(0,1,0) at infinity
P=(Px,Py,1); Q=(Qx,Qy,1); R=(Rx,Ry,1); O=(0,1,0);
I=RR.ideal(Py^2-Px^3-A*Px-B, Qy^2-Qx^3-A*Qx-B, Ry^2-Rx^3-A*Rx-B)
SS=RR.quotient(I)

def add(P,Q):
    """ general addition algorithm for an elliptic curve in short Weierstrass form"""
    if P == O: return Q
    if Q == O: return P
    x1=P[0]; y1=P[1]; x2=Q[0]; y2=Q[1];
    if x1 != x2:
        m = (y2-y1)/(x2-x1)         # usual case: P and Q are distinct and not opposites
    else:
        if y1 == -y2: return O      # P = -Q (includes case where P=Q is 2-torsion)
        m = (3*x1^2+A) / (2*y1)     # P = Q so we are doubling
    x3 = m^2-x1-x2
    y3 = m*(x1-x3)-y1
    return (x3,y3,1)

def negate(P):
    if P == O: return O
    return (P[0],-P[1],1)
    
def reduced_fractions_equal(p,q):
    return SS(p.numerator()*q.denominator()-p.denominator()*q.numerator()) == 0
    
def on_curve(P):
    return reduced_fractions_equal(P[1]^2*P[2],P[0]^3+A*P[0]*P[2]^2+B*P[2]^3)
    
def equal(P,Q):
    return reduced_fractions_equal(P[0],Q[0]) and reduced_fractions_equal(P[1],Q[1]) 
\end{lstlisting}
As a sanity check, let's first verify that the output of add(P,Q) is always on the curve, and check the identity, inverses, and commutativity.
\begin{lstlisting}
print on_curve(O) and on_curve(negate(P)) and on_curve(add(P,Q)) and on_curve(add(P,P)) and on_curve(add(P,negate(P)))
print add(P,O) == P and add(O,P) == P
print add(P,negate(P)) == O
print add(P,Q) == add(Q,P) 
\end{lstlisting}
%Good, now for associativity in the general case...
%\begin{lstlisting}
%add(add(P,Q),R) == add(P,add(Q,R)) 
%\end{lstlisting}


\chapter{Elliptic curves over finite fields}

Now that we know how to do arithmetic in finite fields, we can talk about elliptic curves over finite fields.  We know that an elliptic curve over a finite field $E(\Fp)$ forms a finite abelian group.  The two main questions we will answer are the following:
\begin{enumerate}
\item
How big is $E( \Fp)$?
\item
What is the structure of $E( \Fp)$?
\end{enumerate}
Hasse's Theorem answers the first question by saying that the size of $E(\Fp)$ is remarkably close to the ``expected value" $p+1$.

For the second question, we know that any finite abelian group can be written as a direct sum of cyclic groups.  Can any direct sum of cyclic groups show up as a possible $E(\Fp)$, or are there constraints? 

%We will answer these two questions in this lecture and the next. 
%In this lecture, we will first discuss homomorphism of elliptic curves (Section~\ref{sec:hom}). Then we look at two important examples that will help us understand $E(\F_q)$: the Frobenius endomorphism (Example~\ref{ex:frob-end}) and multiplication by $m$ (Section~\ref{sec:mult-m}).
%
%\section{Hasse's Theorem}
%
%Blahblah
%
%
%The two possibilities for $E[p]$ admitted by the theorem lead to the following definitions.
%We do not need this terminology today, but it will be important in the weeks that follow.
%\begin{df}
%Let $E$ be an elliptic curve defined over a field of characteristic $p>0$.
%If $E[p]\cong \Z/p\Z$ then $E$ is said to be \emph{ordinary}, and if $E[p]\cong \{0\}$, we say that $E$ is \emph{supersingular}.
%\end{df}
%
%\begin{rem}
%The term ``supersingular" is unrelated to the term ``singular" (recall that an elliptic curve is nonsingular by definition: it must have a well-defined tangent at every point).
%Supersingular refers to the fact that such elliptic curves are rare and interesting.
%\end{rem}
%
%\begin{cor}
%Let $E$ be defined over a field $k$. Every finite subgroup of $E(\ol{k})$ is a product of two (possibly trivial) cyclic groups.
%\end{cor}
%
%\begin{proof}
%Let $p$ be the characteristic of $k$, and let $T$ be a subgroup of $E(K)$ with finite order~$n$.
%If $p\nmid n$, then $T\subseteq E[n]\cong \Z/n\Z\otimes\Z/n\Z$ can clearly be written as a product of two cyclic groups.
%Otherwise we may write $T\cong G\otimes H$ where $H$ is the $p$-Sylow subgroup of $T$, and we have $G\subset E[m]\cong \Z/m\Z\otimes\Z/m\Z$, where $m=|G|$ is prime to $p$ and $H$ has $p$-rank at most 1.  It follows that $T$ can always be written as a product of two cyclic groups (at most one of which has order divisible by $p$).
%\end{proof}
%
%\begin{rem}
%If we let $k$ be a finite field $\F_q$, then the group $E(\F_q)$ is a finite torsion subgroup of $E(\ol{\F_q})$, and therefore the product of two (possibly trivial) cyclic groups.
%\end{rem}
%
%\subsection{Preliminaries for Hasse's Theorem}
%
%For any endomorphism $\alpha\colon E\rightarrow E$, let $\alpha_n$ denote the restriction of $\alpha$ to $E[n]$.
%Then $\alpha_n\colon E[n]\rightarrow E[n]$ is a group endomorphism.
%For $p\nmid n$, we have $E[n]\cong \Z/n\Z\oplus\Z/n\Z$, and after choosing a basis for $\Z/n\Z\oplus\Z/n\Z$ we may identify $\alpha_n$ with the $2\times 2$ matrix over $\Z/n\Z$ determined by the action of $\alpha_n$ on this basis (which is $\Z/n\Z$-linear, since $\alpha_n$ is an endomorphism of $E[n]$).
%Note that the matrix $\alpha_n$ is only determined up to conjugacy, but its determinant and trace do not depend on the choice of basis.
%The following theorem allows us to compute the determinant of $\alpha_n$.
%
%\begin{thm}
%$\det \alpha_n \equiv \deg \alpha \bmod n$.
%\end{thm}
%
%We will postpone the proof until later in the course, after we introduce the Weil pairing, which will make the proof much easier.\\
%
%\begin{rem}
%The restriction of $[m]$ to $E[n]$ is just the scalar matrix
%\[
%[m]_n = \begin{pmatrix}m_n & 0\\0 & m_ n\end{pmatrix},
%\]
%where $m_n \equiv m\bmod n$.
%\end{rem}
%
%\subsection{Hasse's Theorem}
%We now state Hasse's theorem.
%
%\begin{thm}\textbf{\emph{(Hasse's Theorem)} }
%Let $E/\F_q$ be an elliptic curve. Then:
%$$\#E(\F_q)=q+1-t,$$
%where $|t|\le 2\sqrt{q}$.
%\end{thm}
%
%This theorem will be proved in the next lecture, but we note here how remarkable it~is.
%For an elliptic curve $E/\F_q$ in Weierstrass from $y^2=x^3+Ax+B$, the cardinality of $E(\F_q)$ is determined by how often $f(x)=x^3+Ax+B$ is a quadratic residue in $\F_q$, as $x$ varies.
%On average, we expect this to occur about half the time, but \emph{a priori} there is no obvious reason why it couldn't happen for \emph{every} $x$, in which case we would have $\#E(\F_q)=2q$, or \emph{none} of them, in which case we would have $\#E(\F_q)=1$.
%Hasse's theorem tells us that the maximum variation from the expected value is bounded by $O(\sqrt{q})$, rather than $O(q)$.
%
%We are now ready to prove Hasse's theorem, which bounds the number of rational points on an elliptic curve $E/\F_q$.
%The proof relies on three key facts that we saw in Lecture 6:
%\begin{enumerate}
%\item[(1)] The degree of a nonzero separable isogeny is equal to the size of its kernel.
%\item[(2)] For any integers $r$ and $s$ with $s$ prime to $q$, the endomorphism $r\pi+s$ is separable.
%\item[(3)] For any endomorphism $\alpha$ and any integer $n$ prime to $q$ we have $\deg\alpha\equiv_n\det \alpha_n$.
%\end{enumerate}
%Recall that $\alpha_n$ is a $2\times 2$ matrix with entries in $\Z/n\Z$ that gives the action of $\alpha$ on the $n$-torsion subgroup $E[n]$ with respect to some chosen basis; its determinant does not depend on the choice of basis.
%The first two facts were proven in class.  We will prove the third later in the course when we cover the Weil pairing, at which point it will be an easy corollary  (see Proposition 3.15 in Washington \cite{Washington}).
%
%\begin{thm}[Hasse] Let $E/\F_q$ be an elliptic curve.  The cardinality of $E(\F_q)$ is given by
%\[
%\#E(\F_q)=q+1-t,
%\]
%where the integer $t$ satisfies $|t|\le 2\sqrt{q}$
%\end{thm}
%\begin{proof}
%Let $\pi(x,y)=(x^q,y^q)$ be the Frobenius endomorphism.
%The finite field $\F_q$ is precisely the subset of $\ol{\F_q}$ fixed by the Frobenius automorphism $x\to x^q$.
%It follows that $E(\F_q)$ is precisely the subset of $E(\ol{\F_q})$ fixed by $\pi$.  Therefore $E(\Fp)=\ker (\pi-1)$.
%The endomorphism $\pi-1$ is separable, since $-1$ is not divisible by the characteristic $p$ of $\F_q$, by (2) .
%It then follows from (1) that
%\[
%\#E(\F_q) = \#\ker(\pi-1) = \deg (\pi-1).
%\]
%Let $t$ be the integer
%\[
%t=q+1-\#(\F_q) = q+1-\deg(\pi-1).
%\]
%Let $r$, $s$, and $n$ be integers with $s$ and $n$ prime to $q$, and let $\pi_n=\begin{bmatrix}a&b\\c&d\end{bmatrix}$.  By (3) we have
%\begin{align*}
%\deg (r\pi-s) &\equiv_n \det \left(r\begin{bmatrix}a&b\\c&d\end{bmatrix} - s\begin{bmatrix}1&0\\0&1\end{bmatrix}\right)\\
%&\equiv_n \det\begin{bmatrix}ra-s&rb\\rc&rd-s\end{bmatrix}\\
%&\equiv_n (ra-s)(rd-s)-r^2bc\\
%&\equiv_n r^2(ad-bc)-rs(a+d)+s^2\\
%&\equiv_n r^2\det \pi_n - rs\tr \pi_n + s^2
%\end{align*}
%But note that $\det \pi_n\equiv_n q$, and $\tr\pi_n\equiv t$ since
%\begin{align*}
%t &= q+1-\det(\pi-1)\\
%&\equiv_n \det\pi_n+1-\det(\pi_n-[1]_n)\\
%&\equiv_n ad-bc+1-\bigl((a-1)(d-1)-bc)\bigr)\\
%&\equiv_n a+d = \tr \pi_n
%\end{align*}
%Thus we have $\deg(r\pi-s)\equiv_n r^q-rst+s^2$.
%This holds for arbitrarily large $n$ and $\deg(r\pi-s)$ is finite, so it must actually be an equality over $\Z$, and we have
%\[
%\deg(r\pi-s) = r^q-rst+s^2
%\]
%for any integers $r$ and $s$ with $s$ prime to $q$.  Dividing by $s^2$ and noting that $\deg(r-\pi s)\ge 0$ yields the inequality
%\[
%q\left(\frac{r}{s}\right)^2 - \left(\frac{r}{s}\right)t+1\ge 0.
%\]
%This holds for a set of rational numbers $\frac{r}{s}$ that are dense in $\R$, so we must have $qx^2-tx+1\ge 0$ for all real numbers $x$.
%It follows that the discriminant $t^2-4q$ cannot be positive, and this yields the desired bound $|t|\le 2\sqrt{q}$.
%\end{proof}
%
%The bound in Hasse's theorem is the best possible.
%Later in the course we will see how to explicitly construct elliptic curves $E/\F_q$ with cardinalities matching every integer value in the interval $[q+1-2\sqrt{q},q+1+2\sqrt{q}]$ when $q$ is prime, and all but at most two integers when~$q$ is not prime.
%
%
%\begin{cor}
%Let $E/\F_q$ be an elliptic curve, with $t=q+1-\#E(\F_q)$.
%The Frobenius endomorphism $\pi$ satisfies the identity $\pi^2-t\pi+q=0$.
%\end{cor}
%\begin{proof}
%Let $n$ be any integer prime to $q$ and let $\pi_n=\begin{bmatrix}a&b\\c&d\end{bmatrix} $.
%The characteristic polynomial of $\pi_n$ is
%\begin{align*}
%\det(\pi_n-\lambda I) &= \det \begin{bmatrix}a-\lambda&b\\c&d-\lambda\end{bmatrix}\\
%&= \lambda^2-(a+d)\lambda+ad-bc\\
%&=\lambda^2-(\tr \pi_n)\lambda + \det\pi_n\\
%&\equiv_n \lambda^2-t\lambda+q.
%\end{align*}
%Thus $\pi_n-t\pi_n+qI \equiv_n 0$, which implies that $E[n]$ lies in the kernel of of the endomorphism $\pi^2-t\pi+q$.
%This holds for infinitely many $n$, hence the kernel is infinite and $\pi^2-t\pi+q$ must then be the zero endomorphism.
%\end{proof}
%
%The polynomial $\lambda^2-t\lambda+q$ is called the \emph{characteristic polynomial of Frobenius}.
%%%%%%%%%%%%
\section{Hasse's Theorem}
Let $E$ be the curve $y^2+a_1xy+a_3y=x^3+a_2x^2+a_4x+a_6$. A quadratic function $ay^2+by+c$ in a finite field $\F_q$ attains about half of the values, so given a value of $x$, there is about a $\rc2$ chance that the equation is solvable in $y$. When it is solvable, there will usually be 2 solutions. Thus we expect the number of solutions to be close to $q$. Including the point at infinite, the expected number becomes $q+1$. Hasse's Theorem tells us that the number of solutions is not too far from $q+1$.
\begin{thm}
Let $E/\F_q$ be an elliptic curve. Then
\[
||E(\F_q)|-(q+1)|\le 2\sqrt q.
\]
\end{thm}
\begin{proof}
The idea is to count the number of points of $E(\ol{\F_q})$ in $E(\F_q)$ by viewing $E(\F_q)$ as the kernel of $1-\phi_q$, where $\phi_q$ is the Frobenius map. The kernel equals the degree of the map. We know the degree of $1$ and $\phi_q$; to get the degree of $1-\phi_q$ we use the fact that $\deg$ is a quadratic form, and a version of the Cauchy-Schwarz inequality.

Let $\phi_q:E\to E$ be the $q$th power Frobenius morphism. 
Since $\F_q$ consists of exactly the solutions to $x^q=x$, we have
\[
\ol{\F_q}^{\phi_q}=\F_q,
\]
i.e. the fixed field of $\ol{\F_q}$ under $\phi_q$ is $\F_q$. Hence $P\in E(\F_q)$ iff $\phi_q(P)=P$, iff $P\in \ker(1-\phi_q)$. We have
\[
|E(K)|=|\ker(1-\phi_q)|=\deg(1-\phi).
\] 
The latter from CITE. SEPARABLE. Now $\deg$ is a positive definite quadratic form. We use the following.
\begin{lem}[Cauchy-Schwarz inequality for groups]
Let $A$ be an abelian group and $d:A\to \Z$ be a positive definite quadratic form. Then for all $\psi,\phi\in A$,
\[
|d(\psi-\phi)-d(\phi)-d(\psi)|\le 2\sqrt{d(\phi)d(\psi)}
\]
\end{lem}
(If $d(\phi)=\an{\phi,\phi}$, then the LHS is $2\an{\phi,\phi}$. We don't divide by 2, just so we can stick to something $\Z$-valued.)
\begin{proof}
The proof is similar to the proof for the ordinary Cauchy-Schwarz inequality. Since $d$ is a quadratic form,
\[
B(\psi,\phi)=d(\psi-\phi)-d(\psi)-d(\phi)
\]
is bilinear. Since $d$ is positive definite,
\[
0\le d(m\psi-n\phi)=m^2d(\psi)+mnB(\psi,\phi)+n^2d(\phi).
%-\rc2L(m\psi-n\psi,m\psi-n\psi)=
%(d(m\psi)+d(n\phi)+L(\psi,\phi))^2
\]
The RHS is quadratic in $m$ and $n$, hence obtains minimum at $\fc mn=-\fc{B(\psi,\phi)}{2d(\psi)}$. So take $m=-B(\psi,\phi)$ and $n=2d(\psi)$. We get 
\[
0\le d(\psi)[4d(\psi)d(\phi)-L(\psi,\phi)^2].
\]
When $\psi\ne 0$, we get $4d(\psi)d(\phi)\ge L(\psi,\phi)^2$, from which the desired inequality follows from taking square roots.
For $\psi=0$ the result is obvious.
\end{proof}
Since $\deg \phi_q=q$, the Cauchy-Schwarz inequality gives
\[
||E(\F_q)|-1-q|=| \deg(1-\phi)-\deg(1)-\deg(\phi)|\le 2\sqrt q
\]
as needed.
\end{proof}
Application to character sums (Silverman, p. 132).
%Given a value for $x$, there are 

%%%%%%%%%%%%

\section{Counting points on elliptic curves over finite fields}

We now consider the problem of actually computing $\#E(\F_q)$ for an elliptic curve $E/\F_q$ given by a Weierstrass equation $y^2=x^3+Ax+B$.
The most na\"ive approach one could take would be to simply evaluate this equation for every pair $(x,y)\in\F_q^2$, count the number of solutions, and then remember to add 1 for the point at infinity.  This take $O(q^2\textsf{M}(\log q))$ time.
Note that the input to this problem is simply the pair of coefficients $A,B\in\F_q$, which each have $O(n)$ bits, where $n=\log q$.
Thus in terms of the size of the input, this algorithm takes time $O(\exp (2n) \textsf{M}(n))$, exponential in $n$.
But we can certainly do better.

Recall that for an odd prime $p$ the Legendre symbol $\pf{a}{p}$ satisfies
\[
\left(\frac{a}{p}\right) =
\begin{cases}
1\qquad&\text{if $x^2=a$ has two solutions mod $p$},\\
0\qquad&\text{if $x^2=a$ has one solution mod $p$},\\
-1\qquad&\text{if $x^2=a$ has no solutions mod $p$}.\\
\end{cases}
\]
We extend the Legendre symbol to finite fields $\F_q$ of odd characteristic by defining
\[
\left(\frac{a}{\F_q}\right) =
\begin{cases}
1\qquad&\text{if $x^2=a$ has two solutions in $\F_q$},\\
0\qquad&\text{if $x^2=a$ has one solution in $\F_q$},\\
-1\qquad&\text{if $x^2=a$ has no solutions in $\F_q$}.\\
\end{cases}
\]
Note that in every case, $1+\pf{a}{\F_q}$ counts the solutions to $x^2=a$ in $\F_q$.  It follows that
\begin{align}\notag
\#E(\F_q) &= 1 + \sum_{x\in\F_q} \left(1 + \left(\frac{x^3+Ax+B}{\F_q}\right)\right)\\
&= q+1+\sum_{x\in\F_q} \left(\frac{x^3+Ax+B}{\F_q}\right).\label{eq:sum}
\end{align}
We note that Hasse's Theorem is equivalent to the statement that the sum in \eqref{eq:sum} has absolute value bounded by $2\sqrt{q}$.
This is remarkable, given that one might na\"ively suppose that the sum could potentially have absolute value as large as $q$.

We can apply \eqref{eq:sum} to compute $\#E(\F_q)$ in $O(\exp(n)\textsf{M}(n))$ time by computing a table of quadratic residues in $\F_q$ and then evaluating $x^3+Ax+B$ for all $x\in\F_q$.  Alternatively, we can compute the Legendre symbol using Euler's criterion $\pf{a}{p}=a^{(p-1)/2}$, which generalizes to any finite field $\F_q$.
This uses much less space, since we don't need to store a table of quadratic residues, but it increases the running time slightly, to $O(\exp(n)\textsf{M}(n)n)$.

So far we have not yet taken advantage of Hasse's theorem, which tells us that the integer $\#E(\F_q)$ which roughly the same size as $q$ actually lies in an interval of width $4\sqrt{q}$.
This suggests that we ought to be able to compute it more efficiently by exploiting this fact.
To do so we first consider the problem of computing the order of a point $P\in E(\F_q)$.

\subsection{Computing the order of a point}\label{sec:pointorder}

The least positive integer $m$ for which $mP=0$ is the \emph{order} of $P$, which we denote $|P|$.
We know that $|P|$ must divide the group order $\#E(\F_q)$, thus the \emph{Hasse interval}
\[
\mathcal{H}(q)=\bigl[q+1-2\sqrt{q},\ q+1+2\sqrt{q}\bigr],
\]
contains at least one multiple $M$ of $|P|$, namely, $\#E(\F_q)$.
To find such a multiple, we set $M_0=\lceil q+1-2\sqrt{q}\rceil$, compute $M_0P$, and then generate the sequence of points
\[
M_0P, (M_0+1)P, (M_0+2)P, \ldots, MP=0,
\]
using repeated addition by $P$. 
We then compute the prime factorization $M=p_1^{e_1}\cdots p_w^{e_w}$, which is easy compared to the time required to find $M$), and compute $m=|P|$ as follows:
\begin{enumerate}
\item Set $m=M$.
\item For each prime $p_i|M$, while $p_i|m$ and $(m/p_i)P=0$ replace $m$ by $m/p_i$.
\end{enumerate}
When this procedure is complete we know that $mP=0$ and $(m/p)\ne 0$ for every prime $p$ dividing $m$, which implies that $m=|P|$.
You will analyze the efficiency of this algorithm and develop several improvements to it in Problem Set 2.

The time to compute $|P|$ is dominated by the time for find the initial multiple $M$, which involves $O(\sqrt{q})$ operations in $E(\Fp)$, yielding a bit complexity of $O(\sqrt{q}\ \textsf{M}(\log q))$ or $O(\exp (n/2)\textsf{M}(n)$.
We will shortly see how to improve this to $O(\exp(n/4)\textsf{M}(n))$, but first we consider how we may use our algorithm for computing $|P|$ to compute $\#E(\Fp)$.
If we are lucky (and when $q$ is large we usually will be), the multiple $M$ of $|P$ that we find will actually be the \emph{unique} multiple of $|P|$ in $\mathcal{H}(q)$.
If this happens, then we must have $M=\#E(\F_q)$.  Otherwise, we might try our luck with a different point $P$.  If we can find any combination of points such that the least common multiple of their orders has a unique multiple in $\mathcal{H}(q)$, then we can determine the group order.

\subsection{The group exponent}
\begin{df}
For a finite group $G$, the \emph{exponent} of $G$, denoted $\lambda(G)$, is defined by
\[
\lambda(G) = \lcm \{|\alpha|:\alpha\in G\}.
\]
\end{df}
Note that $\lambda(G)$ is a divisor of $|G|$ and is divisible by the order of every element of $G$.
Thus $\lambda(G)$ is the maximal possible order of an element of $G$, and when $G$ is abelian this maximum is achieved: their necessarily exists an element with order $\lambda(G)$.
To see this, note that the structure theorem for finite abelian groups allows us to write 
\[
G\simeq \Z/n_1\Z \times \Z/n_2\Z\times\cdots\times\Z/n_r\Z
\]
with $n_i|n_{i+1}$ for $1\le i < r$.
Thus $\lambda(G)=n_r$, and any generator for $\Z/n_r\Z$ has order $\lambda(G)$.

It is clear that if we compute the least common multiple of a sufficiently large subset of a finite abelian group $G$ we will obtain $\lambda(G)$.
If we pick points at random, how many points do we expect to need in order to obtain $\lambda(G)$?
The answer is surprisingly small: just two random points are usually enough.

\begin{thm}\label{thm:exponent}
Let $G$ be a finite abelian group with exponent $\lambda(G)$.  Let $\alpha$ and $\beta$ be uniformly distributed random elements of $G$.
Then
\[
\Pr[\lcm(|\alpha|,|\beta|) = \lambda(G)] > \frac{6}{\pi^2}.
\]
\end{thm}
\begin{proof}
We first reduce to the case that $G$ is cyclic.
As noted above, $G$ is isomorphic to a direct product of cyclic groups $C_1\times C_2\times \cdots\times C_r$, where $C_r$ has order $\lambda(G)$.
Let $\alpha_r$ and $\beta_r$ be the projection of $\alpha$ and $\beta$ to $C_r$.
Then $\lcm(|\alpha_r|,|\beta_r|)=\lambda(G)$ implies $\lcm(|\alpha|,|\beta|)=\lambda(G)$, and therefore
\[
\Pr[\lcm(|\alpha|,|\beta|) = \lambda(G)] \ge \Pr[\lcm(|\alpha_r|,|\beta_r|) = \lambda(G)] .
\]

So we now assume that $G$ is cyclic with generator $\gamma$.  Let $p_1^{e_1}\cdots p_k^{e_k}$ be the prime factorization of $\lambda(G)$.
Let $\alpha=a\gamma$, with $0\le a <  \lambda(G)$.  Unless $a$ is a multiple of $p_i$, which occurs with probability $1/p_i$,
the order of $\alpha$ will be divisible by $p_i^{e_i}$, and similarly for $\beta$.  These two probabilities are independent, thus with probability $1-1/p_i^2$ at least one of $\alpha$ and $\beta$ has order divisible by $p_i^{e_i}$.   Call this event $E_i$.
The events $E_1,\ldots, E_k$ are independent, since we may write $G$ as a direct product of cyclic groups of order $p_1^{e_1},\ldots p_k^{e_k}$, and the projections of $\alpha$ and $\beta$ in each of these cyclic groups are uniformly and independently distributed.  Thus
\[
\Pr[\lcm(|\alpha|,|\beta|)=\lambda(G)] = \prod_{p|\lambda(G)}( 1-p^{-2}) >\prod_p (1-p^{-2})= \left(\sum_{n=1}^\infty\frac{1}{n^2}\right)^{-1} = \frac{1}{\zeta(2)} = \frac{6}{\pi^2},
\]
where $\zeta(s)=\sum n^{-s}$ is the Riemann zeta function.
\end{proof}

The theorem implies that if we generate random points $P\in E(\F_q)$ and accumulate the least common multiple $N$ of their orders, we should expect to obtain $\lambda(E(\F_q))$ within $O(1)$ iterations.
Regardless of when we obtain $\lambda(E(\F_q))$, at every stage we know that $N$ divides $\#E(\F_q)$, and if we ever find that $N$ has a unique multiple in the Hasse interval, then we know that this multiple is the group order.
Unfortunately this might not ever happen, it could be that $\lambda(E(\F_q))$ is smaller than $4\sqrt{q}$ and actually has more than one multiple in the Hasse interval.
To deal with this problem we need to consider the \emph{quadratic twist} of $E$.

\subsection{The quadratic twist of an elliptic curve}
Suppose $d$ is an element of $\F_q$ that is \emph{not} a quadratic residue, so that $\pf{d}{\Fp}=-1$.
If we consider the elliptic curve $\tilde{E}$ defined by $y^2=d(x^3+Ax+B)$, then
\begin{align*}
\#\tilde{E}(\F_q) &= q+1+\sum_{x\in\F_q}\left(\frac{d(x^3+Ax+B)}{\F_q}\right)\\
&= q+ 1 + \sum_{x\in\F_q}\left(\frac{d}{\F_q}\right)\left(\frac{(x^3+Ax+B)}{\F_q}\right) \\
&= q+1 - \sum_{x\in\F_q}\left(\frac{x^3+Ax+B}{\F_q}\right).
\end{align*}
Thus if $\#E(\F_q)=q+1-t$, then $\#\tilde{E}(\F_q)=q+1+t$.
The curve $\tilde{E}$ is called the \emph{quadratic twist} of $E$ (by $d$).
We can put the curve equation for $\tilde{E}$ in standard Weierstrass form by substituting $x/d$ for $x$ and $y/d$ for $y$ and then clearing denominators,  yielding
\[
y^2=x^3+d^2Ax+d^3B.
\]
If we instead choose $d$ to be a (nonzero) quadratic residue, say $d=a^2$, then $\tilde{E}$ is isomorphic to $E$ over $\F_q$ (substitute $a^2x$ for $x$ and $a^3y$ for $y$ and divide both sides by~$a^6$).
Moreover, it does not matter which non-residue $d$ we choose:
if $d$ and $d'$ are any two non-residues in~$\F_q$, then the corresponding curves $\tilde{E}$ and$\tilde{E'}$ are isomorphic over $\F_q$ (use $a^2=d/d'$ to obtain the isomorphism).

Note that the curves $E$ and $\tilde{E}$  are isomorphic over the quadratic extension $\F_q[x]/(x^2-d)\simeq \F_{q^2}$.
In general, curves defined over a field $k$ that are isomorphic over $\ol{k}$ are called \emph{twists}, and if they are isomorphic over a quadratic extension of $k$ they are called \emph{quadratic twists}, as above.
This technically includes the case where the curves are already isomorphic over $k$, but when we refer to ``the" quadratic twist of an elliptic curve $E$ we always mean a curve $\tilde{E}$ constructed as above using a non-residue $d\in k$ so that $E$ and $\tilde{E}$ is not isomorphic.

Our interest in the quadratic twist of $E$ lies in the fact that
\[
\#E(\F_q) + \#\tilde{E}(\F_q) = 2q+2.
\]
Thus if we can compute either $\#E(\F_q)$ or $\#\tilde{E}(\F_q)$ than we can easily determine both values.


\subsection{Mestre's Theorem}

It is not necessarily the case that the group exponent of $\lambda(E(\Fp))$ has a unique multiple in the Hasse interval.
But if we also consider the quadratic twist $\tilde{E}(\Fp)$, then a theorem of Mestre (published by Schoof in \cite{Schoof}) ensures that for all sufficiently large $p$, either $\lambda(E(\Fp))$ or $\lambda(\tilde{E}(\Fp))$ has a unique multiple in the Hasse interval $\mathcal{H}(p)=[(\sqrt{p}-1)^2,(\sqrt{p}+1)^2]$.

\begin{thm}[Mestre]\label{thm:mestre}
Let $p>229$ be prime, and let $E/\Fp$ be an elliptic curve with quadratic twist $\tilde{E}/\Fp$.
Then either $\lambda(E(\Fp))$ or $\lambda(\tilde{E}(\Fp)$  has a unique multiple in $\mathcal{H}(p)$.
\end{thm}
\begin{proof}
Let $E(\Fp)\simeq \Z/n\Z\times\Z/N\Z$ and  $\tilde E(\Fp)\simeq \Z/m\Z\times\Z/M\Z$, where $n|N$ and $m|M$.
We have $E[n]=E(\Fp)[n]$, so the Frobenius endomorphism $\pi$ fixes $E[n]$ and the matrix $\pi_n$ is the identity.
Thus $p=\deg\pi\equiv_n\det\pi_n = 1$, thus $n$ divides $p-1$.  By the same argument, so does $m$.

Let $t=p+1-nN$ be the trace of Frobenius of $E$.  Then
\begin{align*}
4p-t^2 &= 4p-(p+1-nN)^2\\
               &\equiv_{n^2} 4p-(p+1)^2 = 4p-p^2-2p-1 = -(p-1)^2\\
               &\equiv_{n^2} 0
\end{align*}
Thus $n^2$ divides $4p-t^2$, and so does $m^2$, by the same argument.

Since $n$ divides $nN$ and $p-1$, we have $t=p-1+2-nN\equiv_n 2$, and similarly $t\equiv_m -2$.
Thus $t=an+2$ and $t=bm-2$, for some integers $a$ and $b$, and subtracting these equations yields $an-bm=4$, which implies $\gcd(m,n)\le 4$.
Therefore $\gcd(m^2,n^2)\le 16$, and since $m^2$ and $n^2$ both divide $4p-t^2$, we have
\begin{equation}\label{eq:m1}
\frac{m^2n^2}{16} \le 4p-t^2 \le 4p
\end{equation}
Now suppose for the sake of contradiction that $N=\lambda(E(\Fp))$ and $M=\lambda(\tilde{E}(\Fp))$ both have more than one multiple in $\mathcal{H}(p)$.
Then $M$ and $N$ are both at most $\sqrt{4p}$, so $MN\le 4p$.
Since $mM$ and $nN$ lie in $\mathcal{H}(p)$, they are both greater than $(\sqrt{p}-1)^2$, hence $mnMN \ge (\sqrt{p}-1)^4$.
It follows that $mn\ge (\sqrt{p}-1)^4/(4p)$.  Dividing by 4 and squaring both sides yields
\begin{equation}\label{eq:m2}
\frac{m^2n^2}{16}\ge\frac{(\sqrt{p}-1)^8}{256p^2}.
\end{equation}
 Combining \eqref{eq:m1} and \eqref{eq:m2} yields
\begin{equation}
1024p^3\ge (\sqrt{p}-1)^8.
\end{equation}
This implies that if neither $M$ nor $N$ have a unique multiple in $\mathcal{H}(p)$, then $p\le 1284$.
An exhaustive computer search for $p\le 1284$ then finds that in fact we must have $p\le 229$.
\end{proof}



\subsection{Computing the group order with Mestre's Theorem}
We now give a complete algorithm to compute $\#E(\Fp)$ using Mestre's theorem, assuming that $p$ is a prime greater than 229 (if $p$ is smaller than this we can easily just count points as before).\footnote{There is a generalization of Mestre's theorem that applies to arbitrary finite fields $\F_q$, and handles all $q>49$, see \cite{CS10}.
With this the algorithm we give here can be modified to handle arbitrary finite fields.}
As usual, $\mathcal{H}(p)=[p+1-2\sqrt{p},\ p+1+2\sqrt{p}]$ denotes the Hasse interval.

\renewcommand{\labelenumii}{\alph{enumii}.}
\begin{enumerate}
\item Compute a quadratic twist $\tilde{E}$ of $E$ using a randomly chosen non-residue $d\in\F_q$.
\item Let $E_0=E$ and $E_1=\tilde{E}$, set $N_0=N_1=1$ and $i=0$.
\item While neither $N_0$ nor $N_1$ has a unique multiple in $\mathcal{H}(p)$:
\begin{enumerate}
\item Generate a random point $P\in E_i(\Fp)$.
\item Find an integer $M\in\mathcal{H}{p}$ such that $MP=0$.
\item Use $M$ to compute $|P|$ as in \S\ref{sec:pointorder}.
\item Set $N_i=\lcm(N_i,|P|)$ and set $i=1-i$.
\end{enumerate}
\item If $N_0$ has a unique multiple $M$ in $\mathcal{H}(p)$ then return $M$, otherwise return $2p+2-M$, where $M$ is the unique multiple of $N_1$ in $\mathcal{H}(p)$.
\end{enumerate}

It is clear that the output of the algorithm is correct, and it follows from Theorems~\ref{thm:exponent} and \ref{thm:mestre} that the expected number of iterations of step 3 is $O(1)$.  Thus we have a Las Vegas algorithm to compute $\#E(\Fp)$.  Its running time is dominated by the time to find $M$ in step~3b.
If we simply compute $aP$ for every integer $a\in\mathcal{H}(p)$, we obtain an expected running time of $O(\sqrt{p}\ \textsf{M}(\log p)$, or $O(\exp(n/2)\textsf{M}(n)$,
but this can be improved to $O(\exp(n/4)\textsf{M}(n))$ if we speed up step 3b using the baby-steps giant-steps method discussed below.

\subsection{The baby-steps giant-steps method}
Let $a=\lceil p+1-2\sqrt{p}\rceil$ and let $b=\lceil p+1+2\sqrt{p}\rceil$, so $[a,b)$ contains every integer in the Hasse interval $\mathcal{H}(p)$.
In its simplest form, the baby-steps giant-steps method proceeds as follows:
\begin{enumerate}
\item Pick integers $r$ and $s$ such that $rs \ge b-a$.
\item Compute the set $S_{\rm baby}=\{0,P,2P,\ldots,(r-1)P\}$ of \emph{baby steps}.
\item Compute the set $S_{\rm giant}=\{aP,(a+r)P, (a+2r)P,\ldots,(a+(s-1)r)P\}$ of \emph{giant steps}.
\item For each giant step $P_{\rm giant}=(a+ir)P\in S_{\rm giant}$, check whether its negation is equal to a baby step $P_{\rm baby}=jP\in S_{\rm baby}$, and if so output $M=a+ir+j$.
\end{enumerate}

Note that $-P_{\rm giant}=P_{\rm baby}$ implies $P_{\rm giant}+P_{\rm baby}=(a+ir)P+jP=MP=0$, thus $M$ is a multiple of~$|P|$.
Since \emph{every} integer in $\mathcal{H}(p)$ can be written in the form $a+i+jr$ with $0\le i < r$ and $0\le j < s$, the algorithm is guaranteed to find such an $M$.

To implement this algorithm efficiently, we typically store the baby steps $S_{\rm baby}$ in a lookup table (e.g. a binary tree or a hash table) and as each giant step $P_{rm giant}$ is computed, we lookup $-P_{\rm giant}$ in this table.  Alternatively, one may compute the sets $S_{\rm baby}$ and $S_{\rm giant}$ in their entirety, sort both sets, and then efficiently search for a match.
In both cases, we assume that the points in $S_{\rm baby}$ and $S_{\rm giant}$ are uniquely represented, which may require converting them to affine form.
In the next lecture we will see how \emph{batching} can be used to do this efficiently.
Assuming this is done, if we choose $r\approx s \approx 2p^{1/4}$, then the running time of the algorithm above is $O(\exp(n/4)\textsf{M}(n))$.

Note that its space complexity is $O(\exp(n/4)n)$, which is actually the limiting factor in practically implementations, thus one may choose to make a time-space trade-off by picking a smaller value for $r$ and a larger values of $s$.
We will discuss this and other optimizations to the baby-steps giant-steps method in the next lecture.

\chapter{Cryptography and other applications}
\chapter{Modular forms}
(Placeholder)
\chapter{Elliptic cuves over $\C$}
\chapter{Formal groups}

Why are we interested in elliptic curves over different fields? We are interested in elliptic curves over $\Q$ because much of number theory is concerned with solving equations over $\Q$; we are interested in elliptic curves over finite fields $k$ because of its applications to cryptography, factoring, and primality proving.

On the other hand, even if we just wanted information about an elliptic curve $E$ over $\Q$, it helps to investigate it over other fields. $\Q$ is a difficult field to work with because it is a {\it global} field. So we make two reductions.
\begin{enumerate}
\item Consider $E$ over the local fields $\Q_p$ (and $\C$!). Then combine this information to get information on $E(\Q)$.
\item Consider $E$ over finite fields $\F_p=\Q_p/p\Q_p$. Use this to get information about $E$ over local fields $\Q_p$.
\end{enumerate}
We look at item 2 more closely. Suppose $K$ is a local field; let $k$ be its residue field. 
Note that when we reduce an elliptic curve modulo $p$, some points will get sent to 0; this is the kernel of reduction $E_1(K)$. Thus we get a exact sequence
\[
\xymatrix{
0\ar[r]& E_1(K)\ar[r]&  E_0(K)\ar[r]&  \wt{E}_{\text{ns}}(k)\ar[r]&  0\\%criteria and consequences for good/bad reduction.
%nonsingular points form a group.
& \wh{E}(\mm)\ar@{=}[u] &  & &}
\]
(Here we have the technicality that when we reduce to $k$, some points may be singular, so $E_0(K)$ is the set of nonsingular points.)
How can we study $E_1(K)$? We know these are points whose coordinates are in the maximal ideal $\mm$ associated to $K$ (so get sent to 0 upon reduction). 
%To go from elliptic curves over finite fields to elliptic curves over local fields, we use the following exact sequences.
To get these points we take a uniformizer $z\in \mm$, and write the coordinates of a point in $E_0(K)$ as power series in $z$. Thus we investigate the elliptic curve over a {\it ring of formal power series}, and the group law becomes a group law for power series, called a {\it formal group law} (see Section BLAH). We then get to $E(K)$ from the exact sequence $0\to E_0(K)\to E(K)\to E(K)/E_0(K)\to 0$; fortunately, $E(K)/E_0(K)$ is finite.

To get from $E$ over local fields to $E$ over global fields, we would like to use the Hasse principle as in quadratic forms. However, this fails. There is, however, a way of measuring the failure of the Hasse principle using the Selmer and Shafarevich-Tate groups (see Silverman, X).

\wrbox{
$E_1(K)$ is the set of points which ``in the maximal ideal," i.e.,  are 0 modulo $k$. But this is with respect to the equation where the point at infinity has been transformed to the origin, so in the original coordinates, they are points with powers of $\pi$ in the {\it denominator}, not the numerator. 
}

\section{Formal groups}
%\section{Formal groups}
A formal group is basically a group law defined on power series. It can be thought of as a ``group law without a group"; for applications we will this group law operate on a maximal ideal in a complete ring. (This way, the power series for the group law will converge.)
\begin{df}
A (1-dimensional commutative) \textbf{formal group} $\mathscr F$ over the ring $R$ is a power series $F(X,Y)\in R[[X,Y]]$ satisfying the following three conditions.
\begin{enumerate}
\item
$F(X,Y)=X+Y\pmod{(X,Y)^2}$. %+G(X,Y)$ where all terms of $G$ have degree at least 2.
\item (Associativity)
$F(F(X,Y),Z)=F(X,F(Y,Z))$.
\item (Commutativity)
$F(X,Y)=F(Y,X)$.
\end{enumerate}
We also write $X+_F Y$ for $F(X,Y)$.
\end{df}
\begin{pr} Keep the above notation.
A formal group has the following additional properties:
\begin{enumerate}
\item[4.] (Inverse) There is a unique power series $i(T)\in R[[T]]$ such that $F(T,i(T))=0$.
\item[5.] (Identity) $F(X,0)=F(0,X)=X$.
\end{enumerate}
\end{pr}
Formal groups originally appeared geometrically: Suppose we are interested in an infinitesimal neighborhood of a point on a variety. Consider the ring of rational functions defined in this infinitesimal neighborhood. Taking the completion of this ring, we get a ring of power series; the variable is an uniformizer.\footnote{For the scheme-theoretically inclined, imagine a group law on $\A^1_{\Spec A}=\Spec A[T]$; we have to give a multiplication map
\begin{align*}
\A^1_{\Spec A}&\to \A^1_{\Spec A} \\
\Spec A[X]\ot_A A[Y] & \to \Spec A[T]\\
A[X,Y] & \mapsfrom A[T].
\end{align*}
This is equivalent to giving the image of $T$ in $A[X,Y]$. Looking at it locally (say at 0) means localizing at $(x)$, giving a power series. 
%affine formal group scheme, degree 1 approximation
}

\subsection{Formal groups and Lie algebras}

\fixme{(I don't understand this really well)} One can say that formal groups are like an intermediate between {\it Lie algebras }and {\it Lie groups}\footnote{A Lie group is basically a manifold with a continuous group structure. A Lie algebra is an algebra with a {\it bracket} operation that ``measures" noncommutativity.} (in a way we won't make precise), going from 
\begin{enumerate}
\item
the complete local ring (power series ring) $R$ at a point, to
\item
a subset of points on the curve.
\end{enumerate}
An elliptic curve is a Lie group, and the tangent space at 0 forms a 1-dimensional (trivial) Lie algebra.
\begin{gather*}
\text{Usual paradigm: }\pat{Lie algebra} \dashrightarrow \pat{Lie group}\\
\xymatrix{
\fixme{\text{unsatisfactory}}\ar[d] &&\\
\pat{Lie algebra of $E$}
\ar@{<->}[rr] \ar@{<->}[rd]&&E\\
\text{so consider...}& {\color{blue}\pat{formal group of $E$}} \ar@{<->}[ru]^{\color{blue}\text{exp}}&
}
\end{gather*}

The analogy is the following.
\begin{enumerate}
\item
The Lie algebra is the tangent space at a point {\it with a Lie bracket $[,]$}, and that under the {\it exponential map}, a vector in the tangent space gets sent to a point on the Lie group (manifold) in the the direction and magnitude of that vector. The Lie bracket tells us about how the group law on the Lie group works (and in particular measures the failure of commutativity).
\item
Here, the formal group law will take the place of the Lie bracket, ``transfering" the group law from the curve to the power series ring. We will see there similarly exists an exponential map that identifies $R$ with the formal group operation, to $R$ with normal addition of power series. (Since the group law is commutative, we don't really want to consider a Lie algebra, which would be trivial; we say that the formal group is an intermediary, in that it gives more information.) There is a caveat that since we plan to work over local rings (rather than over $\C$ as with classical Lie groups), we can map to points on the curve in the maximal ideal, as mentioned in the introduction.
\end{enumerate}
\fixme{Can we make this functorial?}
\subsection{Basic examples}

The formal group law will be easier to study than addition on the curve, because we are just working in a power series ring, and we understand the algebra of a power series ring very well (it's almost like just working with polynomials!). Thus formal groups are useful in simplifying problems in algebraic number theory and geometry.
\begin{ex}
The formal additive $\hat{\mathbb G}_a$ and multiplicative groups $\hat{\mathbb G}_m$ are given by
%XY, bring identity element to 0.
%completion of G_a,G_m.
\begin{align*}
F(X,Y)&=X+Y\\
F(X,Y)&=X+Y+XY=(1+X)(1+Y)-1.
\end{align*}
To motivate this, consider the group varieties $\G_a(K)=K^+$ and $\G_m(K)=K^{\times}$. For $\hat{\mathbb G}_a$, we consider the local ring at 0, and for $\hat{\mathbb G}_m$ we consider the local ring at 1; the law is just ``multiplication around 1." Let $s=x-1$ be a uniformizer. Thinking of the expression $x-1$ as something where we substitute an actual number for $x$, if $s=x-1$ and $t=y-1$, then the product is $xy$, and $xy-1=(s+1)(t+1)-1$, hence the formula.
\end{ex}

\begin{df}
A homomorphism from $(\mathscr F,F)$ to $(\mathscr G,G)$ is a power series $f(T)\in R[[T]]$ with
\[
f(F(X,Y))=G(f(X),f(Y)).
\]
$\mathscr F$ and $\mathscr G$ are isomorphic over $R$ if there are homomorphisms $f:\mathscr F\to \mathscr G$ and $g:\mathscr G\to \mathscr F$ such that $f(g(T))=g(f(T))=T$.
\end{df}
Note we must have $f(X)\in \an{X}$.
%%A[[T]]->A[[X]]
%%maximal ideal: in schemes (T)->(X).
%For short we write $f\circ F=G\circ f$ or $f(X+_F Y)=f(x)+_G f(Y)$.
%\begin{df}[Base change]
%Let $B$ be an $A$-algebra, via $\ph:A\to B$. Let $(\mathscr F,F)$ be a formal group over $A$. Define
%\[
%F\ot_A B:=\ph(F)\in B[[X,Y]].
%\]
%\end{df}
%We care about Lubin-Tate formal groups for local class field theory. Suppose $K/\Q_p$ is finite; write $k=\F_q$; let $L/K$ be a complete unramified extension (for example, finite unramified extension $K_n=K(\mu_{q^n-1})$ of $\hat K^{\text{un}}$). Let $\sO_L$ be the base ring. Consider $(\pi, f)$ where $\pi$ is a uniformizer of $L$ and $f(X)\in \sO_L[[X]]$ such that 
%\begin{enumerate}
%\item
%$f(X)\equiv \pi X \pmod{(X^2)}$.
%\item
%$f(X)\equiv X^q\pmod{\pi\sO_L[[X]]}$.
%\end{enumerate}
%The fundamental example is the following.
%\begin{ex}
%%get cyclotomic extensions out of $\Q_p$
%$K=\Q_p$, $\pi=p=q$. Take
%\[
%f(X)=(X+1)^p-1=p(\cdots )+X^p.
%\]
%\end{ex}
%\begin{df}
%A Lubin-Tate group is a formal group over $\sO_L$ of height 1 with $\sO_K$-action.
%(We say $f$ has height $h$ if $f(X)\equiv X^{q^h}\pmod{\pi}$.)
%%general case - nonabelian local class field theory.
%
%Let $\pi,\pi'$ be uniformizers of $L$. Define
%\[
%\Theta_{\pi,\pi'} :=\set{\te\in \sO_L}{\te\pi = \te^{\ph} \pi'}
%\]
%where $\ph:L\to L$ is the arithmetic Frobenius ($x\mapsto x^q$ on residue field).
%%\ph(\te)(\pi')
%%f is like a group homomorphism
%%gadget relate one choice of uniformizer to another.
%\end{df}
%
%
\section{Formal groups over DVR's}
We use the formal group to study $p$-torsion. 
\begin{pr}[Silverman IV.4.3]
Let $\mathscr F,\mathscr G$ be formal groups with normalized invariant differentials $\om_{\mathscr F},\om_{\mathscr G}$. Then
\[
\om_{\mathscr G}(f)=f'(0) \om_{\mathscr F}
\]
\end{pr}

The key proposition we will repeatedly use is the following, which tells us what the formal series for multiplication by $p$ looks like.
\begin{pr}\llabel{pr:formal-group-p}
Let $\mathscr F/R$ be a formal group and $p$ a prime. Then there exist power series $f,g\in tR[[t]]$ such that
\[
[p]T=pf(T)+g(T^p).
\]
\end{pr}
\begin{proof}

\end{proof}

The ``$T^p$" term above suggests some kind of ``ramification" happening when we work over local fields of characteristic $p$. (Example: $\Q_p(\ze_p)/\Q_p$ is ramified, and $\ze_p$ satisfies $X^p-1=0$.) 
We now use formal groups to investigate $p$-torsion points. 
\begin{pr}
Let $R$ be complete with maximal ideal $\mm$, and let
$\mathscr F/R$ be a formal group. Suppose $x\in \mathscr F(\mm)$ is a $p^n$-torsion point: $[p^n]x=0$. Then
\[
v(x)\le \fc{v(p)}{p^n-p^{n-1}}.
\]
\end{pr}
\begin{ex}
This already gives us something nonobvious when applied to $\G_m(\Q_p)$. Namely, let $x=\ze_{p^n}-1$ (recall that $s\in \Q_p^{\times}$ will be represented by $s-1$ in the formal group). The above says
\[
v(\ze_{p^n}-1)\le \rc{p^n-p^{n-1}}
\]
and therefore equality must occur (\fixme{why can't it be 0?}). Since $\ze_{p^n}$ satisfies $X^{p^n-p^{n-1}}+\cdots +X^{p^{n-1}}+1=0$, we get that $\Q_p[\ze_{p^n}-1]/\Q_p$ is totally ramified of degree $p^n-p^{n-1}$, and that $\ze_{p^n}-1$ is a uniformizer in the extension.
\end{ex}
\begin{proof}

\end{proof}

Now we consider formal groups in characteristic $p$.

\prbbox{
We showed that $\exp_{\cF}:\hat{\G}^a\to \mathscr F$ over $R$ when $R$ has characteristic 0. What goes wrong if the residue field of $R$  has positive characteristic? Can you salvage it with a weaker result?
}
\vskip0.15in
The problem is that the power series for $\exp_{\cF}$ may not converge! 
But when a power series doesn't converge, {\it look at a smaller neighborhood.}
When do they converge? We find a criterion on the coefficients using some basic calcuations with valuations. 
\footnote{(Note to self: look at subgroup-trivial-cohom in CFT, same stategy of looking at subset?)}
\begin{pr}
\begin{enumerate}
\item
The power series
\[
f(T)=\suo \fc{a_n}{n}T^n
\]
converges for $v_p(T)>0$.
\item
The power series 
\[g(T)=\suo \fc{b_n}{n!}T^n.\]
converges for $v_p(T)\ge \color{blue}\fc{v(p)}{p-1}$ with $v(g(x))=v(x)$.
\end{enumerate}
\end{pr}
\begin{proof}
The valuation of a factorial satisfies $v_p(n!)<(n-1){\color{blue}\fc{v(p)}{p-1}}$.

Just do it!
\end{proof}
Thus we get a isomorphism on an open subset of $R$.
\begin{thm}[Silverman 6.4]
The formal logarithm induces an isomorphism
\[\log_{\mathscr F} \mathscr F(\mm^r)\to \hat{\G}^a(\mm^r)
\]
{\it when $r>\fc{v(p)}{p-1}$}.
\end{thm}
Note that we only have trouble defining the reverse map (exp) in the backwards direction, so we can define $\log_{\mathscr F}:\mathscr F(\mm)\to K^+$, but this will not be a homomorphism of formal groups, and only be a homomorphism of groups.

This theorem will tell us that when $K$ is a local field, there is a subgroup of $E(K)$ isomorphic to $K^+$.
\section{Formal groups in characteristic $p$}

In characteristic $p$, the first term in~\ref{pr:formal-group-p} disappears. Then we get $f(T)=g(T^p)$. This reminds us of separability, and we know separability is an issue for isogenies between elliptic curves in characteristic $p$. (Moreover, this separability can be measured by looking at the powers in the polynomials defining the isogenies.) 
Is there a way we can naturally connect 
\begin{enumerate}
\item
separability degree of isogenies between elliptic curves with
\item
 homomorphisms of formal groups?
\end{enumerate}
Yes.
\begin{df}
Let $f:\mathscr F\to \mathscr G$ be a homomorphism of formal groups over $R$ of characteristic $p$. Define the \textbf{height} $\text{ht}(f)$ to be the largest $h$ such that there exists a power series $g$ with
\[
f(T)=g(T^{p^h}).
\]
\end{df}

\begin{thm}\llabel{thm:height-degree}
Let $K$ be a field of characteristic $p>0$, $E_1,E_2/K$ be elliptic curves, and $\phi:E_1\to E_2$ be an isogeny. Let $\mathscr F:\hat{E_1}\to \hat{E_2}$ be the associated homomorphism of formal groups\fixme{ (say how this comes about)}. Then
\[
p^{\text{ht}(f)}=\deg_i(\phi).
\]

For any elliptic curve,
\[
\text{ht}(\hat E)=1 \text{ or }2.
\]
\end{thm}
We first prove some basic facts about heights using the invariant differential. 
\begin{pr}\llabel{pr:fg-height}
Let $f:\cF\to \cG$ be a homomorphism of formal groups over $R$.
\begin{enumerate}
\item
If $h=\text{ht}(f)$ and $f(T)=g(T^{p^h})$, then $g'(0)\ne0$. In particular, if $f'(0)=0$, then $\text{ht}(f)\ge 1$, i.e., $f(T)=g(T^p)$ for some $g\in R[[T]]$.
\item
(Height is additive) Given $g:\cG\to \mathscr H$, we have
\[
\text{ht}(g\circ f)=\text{ht}(f)+\text{ht}(g).
\]
\end{enumerate}
\end{pr}
Compare the additivity of height with the multiplicativity of degree.
\begin{proof}
\begin{enumerate}
\item
We first show the ``in particular" part. Let $\om_{\mathscr G}=P(T)\,dT.$
We write $\om_{\mathscr G}(f(T))$ in two ways: 
\begin{align*}
\om_{\mathscr G}(f(T))&=f'(0)\om_{\mathscr F}(T)=0\\
\om_{\mathscr G}(f(T))&=P(f(T))f'(T)\,dT.
\end{align*}
Since $P(f(T))\in R[[T]]$, we get $f'(T)=0$ in $R$. Since we are in characteristic $p$, this implies $f(T)=g(T^p)$ for some $g\in R[[T]]$.

We already have the result for the case $h=0$. How do we reduce the general case to this? By using the $p^h$th power Frobenius. Let $\mathscr F^{(q)}$ denote the formal group with law $\sum a_{ij}^pX^iY^j$. 
Since taking the $q$th power is a homomorphism in characteristic $p$, the homomorphism $f$ transfers to a formal group law $g:\cF^{(q)}\to \cG^{(q)}$.
\[
\xymatrix{
(\cF,F)\ar[r]^f\ar@{.>}[d]_{\hat{\,}q}&(\cG,G)\ar@{.>}[d]^{\hat{\,}q}\\
(\cF,F^{(q)})\ar[r]^g & (\cG^{(q)},G).
}
\]
\begin{align*}
g(F^{(q)}(X^q,Y^q))&= f(F(X,Y))\\
&=G(f(X),f(Y))\\
%&=G(f(X^q),f(Y^q))\\
& = G(g(X^q),g(Y^q)),
\end{align*}
i.e., $g$ is a homomorphism of formal groups $\cF\to \cG$.

Now, the fact that $\cF$ has height $h$ means that $g$ has height 0. We've shown this means $g=0$.
\item
This follows immediately from the characterization in part 1, after writing $f=f_1(T^{p^{\text{ht}(f)}})$ and $g=g_1(T^{p^{\text{ht}(g)}})$.
\end{enumerate}
\end{proof}

\begin{proof}[Proof of Theorem~\ref{thm:height-degree}]
Every isogeny can be written as a composition of a separable and purely inseparable (namely, the Frobenius) isogeny. Since degree is multiplicative and height is additive (Proposition~\ref{pr:fg-height}), it suffices to show the theorem for separable and purely inseparable isogenies.
\begin{enumerate}
\item For the Frobenius isogeny, $\deg_i(\phi_q)=q$, and the associated homomorphism of formal groups is $T^q$, so the theorem checks.
\item For a separable isogeny, $\deg_i(\phi)=1$. We will use the differential as a way to go between the degree of the isogeny and the height of the associated homomorphism of formal groups. Separability implies $\phi^*\om\ne 0$. In the realm of formal groups this says $\om_{\cG}\circ f\ne0$, or $f'(0)\om_{\cF}\ne 0$. (Check that $\om_{\cG}$ corresponds to $\om$.) Thus $f'(0)\ne 0$, and Proposition~\ref{pr:fg-height} tells us the height is 0.
% and the associated homomorphism of formal groups satisfies $
\end{enumerate}
For the second part, note that $\deg_i([p])=p$ or $p^2$, so $\text{ht}([p])=1$ or 2.
\end{proof}
%Compare this with the baby example of field extensions of a finite field $\F_p$. We have the map $

cf. ANT inseparability?
\chapter{Elliptic curves over local fields}
\fixme{
Summary.
\begin{enumerate}
\item
\item
\item
\item Define unramified for $G(\ol K/K)$-sets by saying that the action is entirely captured by $G(K\ur/K)$, i.e, $I_v$ acts trivially. $E[m], T_{\ell}(E)$ are unramified for $m,\ell\perp \chr(k)$ because of injectivity $E[m]\hra \wt E(k')$, which means it suffices to look at the action on $l/k$, i.e. $G(K\ur /K)\cong G(k^s/k)$.
\item Define good, multiplicative (split/nonsplit), additive reduction. 5.4
\begin{enumerate}
\item
Cannot improve from unramified extension (any change of coordinates over an unramified extension can be done over the base field, up to a unit, since the $u'$ doesn't have fractional valuation).
\item Good/multiplicative reduction preserved: we'll still have $v(\De), v(c_4)=/\ne0$ in extension.
\item Can improve from additive reduction: intuition---$v(c_4)>0$---by extending valuation can make CoC so that $v(c_4)=0$. Use Legendre form.
\item Good reduction iff $j$-invariant integral: if good reduction, then $\De\in (R')^{\times}$, need to show in $R$. But $j=\fc{c_4'}{\De'}$, and by good reduction $\De'$ is in $R'^{\times}$. If $j$-invariant integral---note that $j$-invariant is most closely linked to the Legendre form, so we look at that---$\la$ must be integral, and not ``almost" a repeated root, so $\De$ (think of it as the discriminant of the polynomial!) is good.  
\end{enumerate}
\item With NOS, we have a strong link between good reduction and non-ramification, in the sense that a weak result about non-ramification of torsion points (just $E[m]$ unramified at $v$ for infinitely many integers $\perp \chr(k)$) implies good reduction.

Pf. Choose $m$ super-large and relatively prime. The key is that $m$ is more than $E/E_0(K^{\text{nr}})$---because then $E_1$ can't capture everything and this forces the existence of $(\Z/\ell\Z)^2$ in the reduction part, which cannot happen for mult/add reduction.

7.2 Isogenous $\implies$ both have good reduction, or bad reduction. Look at $m$-torsion!

7.3 Pot good reduction iff $I_v$ acts on $T_{\ell}(E)$ through finite quotient for some/all primes $\ell\ne \chr(k)$. Forward direction easy---just extend field until good reduction. Backwards direction: look at fixed field, can find finite subfield giving good reduction (use giving compositum).
\end{enumerate}
}

\section{Introduction}
Let $K$ be a field. 
\begin{df}\llabel{df:cam8-1}
$v:\Kt\to \Z$ is a \textbf{discrete valuation} if
\begin{enumerate}
\item
$v(xy)=v(x)+x(y)$.
\item
$v(x+y)\ge \min(v(x),v(y))$.
\end{enumerate}
($v(0)$ is either 0 or undefined.)
\end{df}
\begin{ex}
\begin{enumerate}
\item
$K=\Q$ prime. $v=v_p$ where $v_p(p^r\fc{a}{b})=r$ with $a,b\in \Z$ coprime to $p$.
\item
$[K:\Q]<\iy$, ring of integers $\sO_K$ where $\mfp\subeq \sO_K$ is a prime ideal.
\end{enumerate}
We have $v=v_{\mfp}$ where $v_{\mfp}(x)$ is the power of $\mfp$ in the factorization of $x\sO_K$ into prime ideals. 
\end{ex}
\begin{rem}
Let $x,y\in K$. The definition gives 
\[
\begin{cases}
v(x+y)&\ge \min(v(x),v(y))\\
v(x)&\ge \min(v(x+y),v(y)=v(-y)).
\end{cases}
\]
So if $v(x)<x(y)$ then $v(x+y)=v(x)$. Hence if $v(x)\ne v(y)$, then $v(x+y)=\min(v(x),v(y))$. 
\end{rem}
\begin{lem}\llabel{lem:cam8-2}
Let $E/K$ be an elliptic curve, with Weierstrass equation 
\[
y^2+a_1xy+a_3y=x^3+a_2x^2+a_4x+a_6.
\]
Assume $v(a_i)\ge 0$ for all $i$. Let $O\ne P=(x,y)\in E(K)$. Then either
\begin{enumerate}
\item
$v(x),v(y)\ge 0$, or
\item
$v(x)=-2r, v(y)=-3r$ for some $r\ge 1$.
\end{enumerate}
\end{lem}
\begin{proof}
\ul{Case $v(x)\ge 0$:} Suppose $v(y)<0$. Then $v(y^2+a_1xy+a_3y)=2v(y)<0$ and $v(RHS)\ge 0$, contradiction. Thus $v(y)\ge 0$.\\

\ul{Case $v(x)<0$:} We have $v(LHS)\ge \min(2v(y),v(x)+v(y),v(y))$, $v(RHS)=3v(x)$, and therefore (checking a few cases) $v(y)<v(x)<0$. Then $v(LHS)=2v(y)$. Thus $v(x)=-2r$, $v(y)=-3r$ for some $r\ge 1$.
\end{proof}
Notation: Denote the valuation ring, unit group, maximal ideal, and residue field by 
\begin{align*}
\sO_K&=\set{x\in K^{\times}}{v(x)\ge 0}\cup \{0\}\\
\sO_K^{\times}&=\set{x\in K^{\times}}{v(x)=0}\cup \{0\}\\
\pi\sO_K&=\set{x\in K^{\times}}{v(x)\ge 1}&\text{picking }\pi\in K,v(\pi)=1.\\
%setdry
k&=\sO_K/\pi\sO_K.
\end{align*}
\begin{df}
A Weierstrass equation for $E$ with coefficients $a_1,\ldots, a_6\in K$ is
\begin{enumerate}
\item
\textbf{integral} if $a_1,\ldots,a_6\in \sO_K$.
\item
\textbf{minimal} if $v(\De)$ is minimal among all integral models for $E$ (among all the Weierstrass equations you can write down).
\end{enumerate}
\end{df}
%amongst all the Weierstrass equations you can write down, the discriminant is as small as possible.
\begin{rem}
\begin{enumerate}
\item
Putting $x=u^2x'$ and $y=u^3y'$ gives $a_i=u^ia_i'$, therefore integral models exist.
\item
$a_1,\ldots, a_6\in \sO_K$ gives $\De\in \sO_K$ and $v(\De)\in \N_0$.
\end{enumerate}
\end{rem}
We'd like to apply formal groups to elliptic curves over local fields.

Fixing any $0<c<1$ we define a metric on $K$ by
\[
\begin{cases}
c^{v(x-y)}&\text{if }x\ne y\\
0&\text{ if}x=y.
\end{cases}
\]
Definition~\ref{df:cam8-1} gives 
\[
d(x,z)\le \max(d(x,y),d(y,z)),
\]
the ultrametric inequality (much stronger than the triangle inequality). 
Let $\hat K$ be the completion of $K$ with respect to $d$. By continuity, $+,\times$ extend to $\hat K$, so $\hat K$ is a field. $v$ extends to $\hat K$ and is a discrete valuation. Note that the residue field does not change.

In examples 1 and 2, we write $\hat K=\Q_p$ and $K_{\mfp}$. For the rest of this section, $K$ is a field complete with respect to a discrete valuation $v:\Kt\to \Z$. $\sO_K,\pi,k$ are defined as before.

We further assume $\chr(K)=0$ and $\chr(k)=p>0$. (For example, $K=\Q_p$, $\sO_K=\Z_p$, $\pi\sO_K=p\Z_p$, $k=\F_p$.) $K$ is complete so $\sO_K$ is complete with respect to $\pi^r\sO_K$ for any $r\ge 1$. In \S7 we put $t=-\fc xy$ we put $t=-\fc xy$, $w=-\rc y$. Then 
\begin{align*}
\hat E(\pi^r\sO_K)&=
\set{(x,y)\in E(K)}{-\fc xy,-\rc y \in \pi^r\sO_K}\cup \{0\}\\
&=\set{(x,y)\in E(K)}{v\pf{x}{y},v\pf{1}{y} \ge r}\cup \{0\}\\
&\stackrel{\text{Lem.~\ref{lem:cam8-2}}}= 
\set{(x,y)\in E(K)}{v(x)\le -2r,v(y)\le -3r}\cup \{0\}.
\end{align*}
By Lemma~\ref{lem:cam7-2}, this is a subgroup of $E(K)$, say $E_r(K)$.

Thu. 7/11

We get a filtration
\[
\cdots \subeq E_3(K)\subeq E_2(K)\subeq E_1(K)=\hat E(\pi \sO_K).
\]
More generally, if $\cF$ is a formal group over $\sO_K$, 
\[
\cdots\subeq  \cF(\pi^3\sO_K)\subeq \cF(\pi^2\sO_K)\subeq \cF(\pi\sO_K).
\]
\begin{pr}\llabel{pr:cam8-3}
Let $\cF$ be a formal group over $\sO_K$. Let $e=v(p)$. If $r>\fc{e}{p-1}$, then
\begin{align*}
\log:\cF(\pi^r\sO_K)& \to \wh{\G}_a(\pi^r \sO_K)\\
\exp:\wh{\G}_a(\pi^r\sO_K)& \to\cF(\pi^r \sO_K)
\end{align*}
are inverse homomorphisms.
\end{pr}
\begin{proof}
%We need to show the power series converge; then the fact they are homomorphism and inverses will follow from what we've already shown.

Let $x\in \pi^r\sO_K$. We must show the power series $\exp$ and $\log$ of Theorem~\ref{thm:cam7-3} converge. Then the fact they are homomorphism and inverses will follow from what we've already shown. Recall 
\[
\exp(T)=T+\fc{b_2}{2!}T^2+\fc{b_3}{3!}T^3+\cdots
\]
for $b_2,b_3,\ldots, \in \sO_K$. We have
\bal
v(n!)&=ev_p(n!)\\
&=e\pa{\sum_{j=1}^{m} \ff{n}{p^j}},\qquad p^m\le n<p^{m+1}\\
&\le e\sum_{j=1}^m \fc{n}{p^j}\\
&=en \fc{\rc p-\rc{p^{m+1}}}{1-\rc p}\\
&=en\rc{p-1} (1-p^{-m})\le \fc{e}{p-1}(n-1).
\end{align*}
Therefore 
\begin{align*}
v\pa{\fc{b_n}{n!}x^n}&\ge nr-\fc{e}{p-1}(n-1)\\
&=(n-1)\ub{\pa{r-\fc{e}{p-1}}}{>0}+r.
\end{align*}
This is $\ge r$ and goes to $\iy$ as $n\to \iy$. 
Because we have the ultrametric law, convergence of terms in a sum implies convergence of the sum. Thus $\exp(x)$ converges and belongs to $\pi^r\sO_K$. 

For $\log$ the denominators are smaller, so a fortiori the series for $\log$ converges.\footnote{In complex analysis we're used to $\exp$ having better convergence, but here it's the other way around.}
\end{proof}
So for $r$ sufficiently large,
\[
\cF(\pi^r\sO_K)\cong \wh{\G}_a(\pi^r\sO_K)\cong (\pi^r\sO_K,+)\cong (\sO_K,+).
\]
\begin{lem}[Filtration of formal groups]\llabel{lem:cam8-4}
If $r\ge 1$ then $\fc{\cF(\pi^r\sO_K)}{\cF(\pi^{r+1}\sO_K)}\cong (k,+)$.
\end{lem}
\begin{proof}
note $F(X,Y)=X+Y+XY(\cdots)$. If $x,y\in \sO_K$ then $F(\pi^rx,\pi^ry)\equiv \pi^r(x+y)\pmod{\pi^{r+1}}$. We get a  group homomorphism
\bal
\cF(\pi^r\sO_K)&\to k\\
\pi^rx &\mapsto x\pmod \pi
\end{align*}
surjective (by definition of $k$ as a residue field) with kernel $\cF(\pi^{r+1}\sO_K)$. Hence
\[
\fc{\cF(\pi^r\sO_K)}{\cF(\pi^{r+1}\sO_K)}\xrc (k,+).
\]
\end{proof}
\begin{cor}
If $|k|<\iy$ then $\cF(\pi\sO_K)$ has a subgroup of finite index isomorphic to $(\sO_K,+)$. 
\end{cor}
We use the following notation for reduction modulo $\pi$:
\bal
\sO_K&\to \frac{\sO_K}{\pi \sO_K}=k\\
x&\mapsto \wt x.
\end{align*}

Let $E/K$ be an elliptic curve. 
\begin{pr}
The reductions modulo $\pi$ of two minimal Weierstrass equations for $E$ define isomorphic curves over $k$. 
\end{pr}
\begin{proof}
Say the Weierstrass equations are related by $[u;r,s,t]$, $u\in \Kt$ and $r,s,t\in K$. Then $\De_1=u^{12}\De_2$. If both equations are minimal, then $v(\De_1)=v(\De_2)$, then $v(u)=0$, i.e., $u\in \sO_K^{\times}$.

In the case where $\chr(k)\ne 2,3$, we can work with shorter Weierstrass forms, and we only need to worry about $u$.

In the general case, the transformation formulae for $a_i$'s and $b_i$'s (for example $u^2b_2'=b_2+12r$) gives $r,s,t\in \sO_K$. The Weierstrass equations for the reduction $\pi$ are now related by $[\wt u\ne 0;\wt r,\wt s,\wt t]$. 
\end{proof}
\begin{df}\llabel{df:ec-reduction}
The \textbf{reduction} $\wt E/k$ of $E/K$ is defined by the reduction of the reduction of a minimal Weierstrass equation modulo $pi$. We say $E$ has \textbf{good reduction} if $\wt E$ is non-singular (i.e., $\wt E$ is an elliptic curve). Otherwise, it has \textbf{bad reduction}.
\end{df}
For an integral Weierstrass equation, 
\begin{itemize}
\item
$v(\De)=0$ implies good reduction,
\item
$0<v(\De)<12$ implies bad reduction (we can only change $\De$ by powers of 12 so we can't get $v(\De)=0$; the discriminant vanishes when you reduce modulo $\pi$.), and 
\item
when $v(\De)\ge12$, the Weierstrass equation might not be minimal. \fixme{good or bad reduction?}
\end{itemize}

There is a well-defined map 
\begin{align*}
\Pj^2(K)&\to \Pj^2(k)\\
(x:y:z)&\mapsto (\wt x:\wt y:\wt z)
\end{align*}
where we choose the representative with $\min(v(x),v(y),v(z))=0$. 

We restrict this map to get 
\begin{align*}
E(K)&\to \wt E(k)\\
P&\mapsto \wt P.
\end{align*}
Lemma~\ref{lem:cam8-2} gives that $E_1(K)=\set{P\in E(K)}{\wt P=O}$.

Why should $\wt E(k)$ be a group? In the case of bad reduction we don't have a group law... but we can get one by deleting a singular point. Let 
\[
\wt E_{ns} =\begin{cases}
 \wt E &\text{ if $E$ has good reduction,}\\
\wt E\bs \{\text{singular point}\}&\text{ if $E$ has bad reduction.} 
\end{cases}
\]

The chord and tangent process defines a group law on $\wt E_{ns}$: If the line passes through the singular point, it passes through with multiplicity 2. Hence the third point of intersection will not be the singular point.

In the case of bad reduction, then over $\ol k$, $\wt E\cong \G_a$ or $\G_m$.

For simplicity, assume $\chr(k)\ne 2$. Note $\wt E:y^2-f(x)$ has $\deg(f)=3$. There are two possibilities.
\begin{enumerate}
\item
The singular point is a double root, for example $y^2=x^2(x+1)$. This is a curve with a node, and we get multiplicative reduction.
\item
The singular point is a triple root, for example, $y^2=x^3$. This is a curve with a cusp, and we get additive reduction.
\end{enumerate}

{\color{blue}Lecture 9/11}

(Picture)

We check that we have a group law on a singular curve with the singular point removed. We'll just do the special case of $y^2=x^3$.
We have 
\bal
\G_a&\xrc \wh E_{ns}\\
t&\mapsto (t^{-2},t^{-3})\\
O&\mapsto O \text{ (point at $\iy$)}\\
\fc xy& \mapsfrom (x,y).
\end{align*}
We check this is a group homomorphism. Let $P_1,P_2,P_3$ be on the line $ax+by=1$. Let $P_i=(x_i,y_i)$. Then  $x_i^3=y_i^2=y_i^2(ax_i+by_i)$, so $t_i=\fc{x_i}{y_i}$ is a root of $x^3-aX-b$, and $t_1+t_2+t_3$. 

This is the additive case; there is a similar calculation for the multiplicative case. The one thing to notice is that rational parametrizations aren't unique, because we can compose by a Mobius transformation. We want to make sure that the point at $\iy$ on $\Pj^1$ goes to the singular line. In the multiplicative case, 2 points on the projective line map to the singular point.
We get a rational map between the curve and $\Pj^1$ with $0,\iy$ deleted, the multiplicative group. We want 1 to be mapped to the identity; this helps us pick the right Mobius map.

\begin{df}
Define
\[
E_0(K)=\set{P\in E(K)}{\wt P\in \wt E_{ns}(k)}.
\]
\end{df}
\begin{pr}\llabel{pr:cam8-6}
$E_0(K)\subeq E(K)$ is a subgroup, and reduction modulo $\pi$ is a surjective group homomorphism, $E_0(K)\rra \wt E_{ns}(k)$. 
%good reduction: no singular point to delete.
\end{pr}
%chord and tangent process
\begin{proof}
(Group homomorphism) A line in $\Pj^2$ defined over $K$ is given by $\ell:aX+bY+cZ=0$ for some $a,b,c\in K$ not all 0. We may assume $\min(v(a),v(b),v(c))=0$. Reduction modulo $\pi$ gives a line $\wt{\ell}:\wt aX+\wt bY+\wt cZ$ where $\wt a,\wt b,\wt c$ re not all 0. 
Thinking of lines as points in dual projective space, our method of reducing lines is the same as our method of reducing points.

If $P_1,P_2,P_3\in E(k)$ with $P_1+P_2+P_3=O$, then they lie on a line $\ell$. Then $\wt P_1,\wt P_2,\wt P_3\in \wt E(k)$ lie on the line $\wt{\ell}$.
%original 3 points distinct
%intersect at line, look for roots of cubic polynomial

If $\wt P_1,\wt P_2\in \wt E_{ns}(k)$, then $\wt P_3\in \wt E_{ns}(k)$. So of $P_1,P_2\in E_0(K)$ then $P_3\in E_0(K)$ and $\wt P_1+\wt P_2+\wt P_3=0$.

(Surjective) Let $f(x,y)=y^2+a_1xy+a_3y-x^3-a_2x^2-a_4x-a_6$. Let $\wt P\in \wt E_{ns}(k)\bs \{0\}$, say $\wt P=(\wt x_0,\wt y_0)$ for some $x_0,y_0\in \sO_K$. Since $\wt P$ is nonsingular, either $\pd fx(x_0,y_0)\nequiv 0\pmod{\pi}$ or $\pd fy(x_0,y_0)\nequiv 0\pmod{\pi}$. If $\pd fx(x_0,y_0)\nequiv 0 \pmod{\pi}$ then let $g(t)=f(t,y_0)\in \sO_K[t]$. We will use Hensel's lemma~\ref{lem:hensel-ring} to turn our approximate root into an exact root.

Then
\[
\begin{cases}
g(x_0)\equiv 0\pmod{\pi}\\
g'(x_0)\in \sO_K^{\times}.
\end{cases}
\]
Hensel's lemma gives $b\in \sO_K$ such that $g(b)=0$, $b\equiv x_0\pmod{\pi}$. Then $P=(b,y_0)\in E(K)$ reduces to $\wt P=(\wt x_0,\wt y_0)$. (In fact, $P\in E_0(K)$.) The case where $\pd fy(x_0,y_0)\nequiv 0\pmod{\pi}$ works in the same way.
\end{proof}
This is a general argument to lift points on curves by Hensel's lemma;  the form of $f$ didn't really matter.

We have for $r>\fc{e}{p-1}$,
\[
\xymatrix{
\hat E(\pi^r\sO_K)\aq{d} &&&\hat E(\pi\sO_K)\aq{d}&&\\
E_r(K)\aq{d} \ha{r}_{(k,+)} & \cdots \ha{r}_{(k,+)} & E_2(K)\ha{r}_{(k,+)} & E_1(K)\ha{r}_{\wt E_{ns}(k)} & E_0(K) \ha{r} & E(K)\\
(\sO_K,+) &&&&&
}
\]
where each quotient before $E_1(K)$ is isomorphic to $(k,+)$ and $\fc{E_0(K)}{E_1(K)}\cong \wt E_{ns}(k)$. 
%keep passing to groups of finite index. 
Our goal is to show $E(K)$ contains a subgroup of finite index isomorphic to $(\sO_K,+)$; it remains to  study the top layer, how $E_0(K)$ sits inside $E(K)$. 
If good reduction we're already done; we just need to consider bad reduction. %Neron model, we will do a compactness model to show the index is finite

\begin{lem}
If $|k|<\iy$, then $\Pj^n(K)$ is compact with respect to the $\pi$-adic topology. 
\end{lem}
\begin{proof}
We have
\[
\fc{\pi^r \sO_K}{\pi^{r+1} \sO_K} \cong \fc{\sO_K}{\pi\sO_K}\cong k
\]
by $\pi^r x\bmod{\pi^{r+1}}\mapsto x\bmod \pi$. Because $k$ is finite, $\fc{\sO_K}{\pi^r\sO_K}$ is finite for all $r$. 

Let $(x_n)$ be a squence in $\sO_K$. $(x_n)$ has a subsequence $(x_n^{(1)})$ that is constant modulo $\pi$, and inductively $(x_n^{(i)})$ has a subsequence $(x_n^{(i+1)})$ that is constant modulo $\pi^{i+1}$. Then $(x_n^{(n)})$ is a Cauchy sequence, and hence converges. Thus $\sO_K$ is sequentially compact and hence compact. (In general, profinite groups are compact.)

$\Pj^n(K)$ is the union of compact sets $\set{[a_0:\cdots a_{i-1}:1:a_{i+1}:\cdots a_n]}{a_i\in \sO_K}$ and hence compact.
\end{proof}
\begin{lem}
Suppose $E_0(K)\subeq E(K)$ has finite index.
\end{lem}
\begin{proof}
$E(K)\subeq \Pj^2(K)$ is a closed subset and hence a compact topological group. If $\wt E$ has a singular point $(\wt x_0,\wt y_0)$, $E(K)\bs E_0(K)=\set{(x,y)\in E(K)}{v(x-x_0)\ge 1, v(y-y_0)\ge 1}$ is a closed subset (as $v$ is continuous). Thus $E_0(K)\subeq E(K)$ is an open subgroup. 
The cosets of $E_0(K)$ in $E(K)$ form an open cover of $E(K)$. 
%subgroup open iff closed of finite index?
Then $E(K)$ is compact, and $[E(K):E_0(K)]<\iy$. The index
\[
c_K(E)=[E(K):E_0(K)]<\iy
\]
is called the \textbf{Tamagawa number}.
\end{proof}
\begin{rem}
If $E$ has good reduction, then $c_K(E)=1$, but the converse is false. If you do more geometry, using Neron models, then you get a better understanding of $c_K$.

If you work in a more abstract setting, reduction consists of several components. When we look at the Weierstrass equation, we see only one component; the other ones get contracted to a singular point and we don't see them. ($c_K$ is the number of components.) One can prove various nice facts about $c_K$: either $c_K(E)=v(\De)$ or $c_K(E)\le 4$. We've insisted on the minimal Weierstrass equation, but only needed it in two places: the reduction is well-defined, and in the above fact on $c_K$.

%Note that we get additive or multiplicative reduction when we assume completeness
Split multiplicative reduction is where get multiplicative reduction without having to make a field extension.
\end{rem}
\begin{thm}\llabel{thm:cam8-9}
Let $K$ be a field complete with respect to a discrete valuation, $\chr(K)=0$ with finite residue field. Then $E(K)$ contains a subgroup of finite index isomorphic to $K^+$. 
In particular, $E(K)\tors$ is finite.
%whenever have abelian group, can look at torsion subgroup.
\end{thm}
\begin{rem}
The fields in Theorem~\ref{thm:cam8-9} are exactly the finite extensions of $\Q_p$. 
%The hypotheses for this theorem $K$ finite extension for $\Q_p$. 
\end{rem}
Next time we'll prove a result useful when we look at elliptic curves over global fields, about unramified extensions.

{\color{blue}{Lecture 12-11}}
We recall some facts on local fields. Let $K$ be a finite extension of $\Q_p$. Let $K$ be a finite extension of $\Q_p$. Note that $K$ is complete wrt $v_K$. Let $L/K$ be a finite extension. Then we have a commutative diagram
\[
\xymatrix{
\Kt\sj{r}^{v_K} \ha{d} & \Z\ar[d]^{\times e}\\
L^{\times} \sj{r}^{v_L} & \Z.
}
\]
where $[L:K]=ef, f=[k':k]$, and $k,k'$ are the residue fields of $K$ and $L$ (and have characteristic $p$). If $L/K$ is Galois then there is a natural group homomorphism 
\[G(L/K)\to G(k'/k)\] (since the Galois action preserves the valuation). This map is surjective with kernel of order $e$.
\begin{df}
$L/K$ is unramified if $e=1$.
\end{df}
\begin{pr}
For each integer $m\ge 1$,
\begin{enumerate}
\item
$k$ has a unique extension of degree $m$, and
\item
$K$ has a unique unramified extension of degree $m$.
\end{enumerate}
Moreover, these extensions are Galois with cyclic Galois group.
\end{pr}
The takeaway is that given any extension of residue fields, you can find an unramified extension with that residue field.
\begin{thm}\llabel{thm:cam8-10}
Let $[K:\Q_p]<\iy$. Suppose $E/K$ has good reduction and $p\nmid n$. If $P\in E(K)$ then $K([n]^{-1}P)/K$ is unramified.
\end{thm}
(We use the notation $[n]^{-1}P=\set{Q\in E(\ol K)}{nQ=P}$, $K(\{P_1,\ldots, P_r\})=K(x_1,\ldots, x_r,y_1,\ldots, y_r)$ where $P_i=(x_i,y_i)$.)
\begin{proof}
%size $n^2$ because the size of the fiber is the degree (we are over characteristic 0 so everything is separable).
$[n]:\wt E\to \wt E$ is a separable isogeny, since $p\nmid n$. Thus $|[n]^{-1}\wt P|=\deg[n]=n^2$. Here $[n]^{-1}\wt P=\set{Q'\in \wt E(\ol k)}{nQ'=\wt P}$. 
%
Consider the extension of residue fields $k'=k([n]^{-1}\wt P)$. 
Let $m=[k':k]$. let $L/K$ be the unramified extension of degree $m$, so $L$ has residue field $k'$. We claim that each $Q'\in \wt E(k')$ with $nQ'=\wt P$ is the reduction of some $Q\in E(L)$ with $nQ=P$. 
%really working with fact complete fields

By Proposition~\ref{pr:cam8-6}, there exists $Q_0\in E(L)$ reducing to $Q'$. Then $nQ_0-P\in E_1(L)$. 
Since $p\nmid n$, Corollary~\ref{cor:cam7-5} tells us multiplication by $n$ on $E_1(L)$ is an isomorphism, so there exists $Q_1\in E_1(L)$ such that $nQ_0-P=nQ_1$.
Then $P=n(Q_0+Q_1)$. Taking $Q=Q_0+Q_1$ proves the claim.

We found $n^2$ points just by finding points defined over $L$. Thus all $n^2$ points in $[n]^{-1}P$ are defined over $L$.

Thus $K([n]^{-1}P)= L$ and $K([n]^{-1}P)/K$ is unramified.
\end{proof}
%2:15 tues.
%visiting nottingham
\chapter{Elliptic curves over global fields}
\chapter{Computing the Mordell-Weil group}


\section*{Introduction}

Let $K$ be a global field. Our end goal is to understand the group $E(K)$, specifically what its rank is. 
\begin{enumerate}
\item
Computation of the rank:
\begin{enumerate}
\item
We can reduce the computation of the rank to the computation of $\af{E'(K)}{\phi E(K)}$. 
\item
To compute this, we map $\al_{E'}:\af{E'(K)}{\phi E(K)}\hra K^{\times}/K^{\times m}$ (for a 2-isogeny $\phi$ this turns out to just be taking the $x$-coordinate); we will find a point $b_1$ will be in the image (i.e., the $x$-coordinate of a solution) precisely when a certain curve $C_{b_1}$, called a homogeneous space, has a rational point.
\end{enumerate}
We can get a lower bound for the rank this way. In order to get a precise value, in each case where the curve does not have a rational point, we have to prove this. This might be hard; the one major tool at our disposal is showing the nonexistence of solutions modulo $K_v$ (or by Hensel's lemma, equivalently the nonexistence of solutions modulo a sufficiently high power of $p=n_v$).

Thus we would like a local-global principle---that a point in $K_v$ for all $v$ gives a point in $K$. More precisely, we'd like to see whether the fact that $b_1\in \im\al_{E'}$ in all $K_v$ ($C_{b_1}$ has a rational point for all $K_v$) implies $b_1\in \im\al_{E'}$ ($C_{b_1}$ has a rational point for $K$). In the 2-isogeny case, this means if $b_1$ is a valid $x$-coordinate for all $K_v$, then it is a valid $x$-coordinate for $K$.

However the local-global principle fails. Hence there could be a gap between the $b_1$ we show are in the image---where we could find rational points on $C_{b_1}$---and the $b_1$ that are in the image for each $K_v$; this forms the Selmer group. The quotient between them measures the failure of the local-global principle and is measured in the Tate-Shafarevich group.
\end{enumerate}
%

\section{Selmer and Shafarevich-Tate groups}
Recall the Kummer sequence
\[
0\to \fc{E'(K)}{\phi E(K)}\xra{\de}H^1(K,E[\phi])\to H^1(K,E)[\phi_*]\to 0.
\]
Take $K$ to be a number field. 
%Write $M_K=\{\text{places of }K\}$. This consists of the 
%\begin{enumerate}
%\item
%finite places (prime ideals in $\sO_K$), and
%\item
%infinite places (real and complex conjugate pairs of embeddings of $K\hra \C$). 
%stick K inside, abs on C, on K. Reals of complexes. Finite extension of $\Q_p$.
%\end{enumerate}
We fix an embedding $\ol K\subeq \ol K_v$. Then by restriction $\Gal(\ol K_v/K_v)\subeq \Gal(\ol K/K)$. 
For $v\in M_K$, $K\subeq K_v$, take the completion with respect to the $v$-adic topology. We now get maps
\[
\xymatrix{
0\ar[r] &\fc{E'(K)}{\phi E(K)}\ar[r]\ar[d]&
H^1(K,E[\phi])\ar[r]\ar[d]_{\text{res}_v}\ar@{.>}[rd]& H^1(K,E)[\phi_*]\ar[r]\ar[d]_{\text{res}_v}& 0\\
0\ar[r] &\fc{E'(K_v)}{\phi E(K_v)}
\ar[r]^{\de_v} & H^1(K_v,E[\phi]) \ar[r] & H^1(K_v,E)[\phi_*]\ar[r] &0
}
\]
We want to bound the rank on $\fc{E'(K)}{\phi E(K)}$. It is isomorphic to its image under $\de$. When you restrict it it must be in the image of $\de_v$. Thus we can bound the rank by bounding the rank of the image under $\de_v$.
%bound on rank
\begin{df}
The \textbf{$\phi$-Selmer group} is 
\bal
S^{(\phi)}(E/K)&=\ker\pa{H^1(K),E[\phi]\to \prod_{v\in M_K}H^1(K_v,E)}\\
&=\set{\al\in H^1(K,E[\phi])}{\text{res}_v(\al)\in \im(\de_v)\text{ for all }v\in M_K}
\end{align*}
\end{df}
They are points which ``on the basis of local information look like it might be in the image of $\de$."
%Which on the basis of local information looks like it might be in the image of $\de$.
\begin{df}
The \textbf{Tate-Shafarevich group} is
\[
\Sh(E/K)=\ker(H^1(K,E)\to \prod_{v\in M_K} H^1(K_v,E)).
\]
%by night prove better and better upper bounds. becomes sharp %day search for bounds of curve. eventually terminate.
\end{df}
The role of the Tate-Shafarevich group is to capture the failure to compute ranks of elliptic curves following the proof of Weak Mordell-Weil. Compare to how we capture the failure of unique factorization in number fields by defining the class group. We've ``named our problem."
%not sharp. TS get in problem.  name for group. Role of TS failure to compute ranks of elliptic curves following the proof of WMW.
%failure of UF, class group
%We get 
%\[
%0\to \fc{E'(K)}{\phi E(K)} \to S^{(\phi)}(E/K) \to \Sh(E/K)[\phi_*]\to 0
%\]
%where $S^{(\phi)}$ is finite and effectively computable.
%%%%%
\begin{thm}[Selmer and Tate-Shafarevich exact sequence]
If $\phi=[n]:E\to E$, then the following is exact:
\[
0\to \fc{E(K)}{nE(K)}\to S^{(n)}(E/K) \to \Sh (E/K)[n]\to 0.
\]
\end{thm}
\begin{proof}
%They fit in this picture. The Selmer group fits in the dotted exact sequence.
%\[
%\xymatrix{
%&&0\ard{d} & 0\ar[d] &\\
%0\ar[r]& \fc{E'(K)}{\phi E(K)} \ar[r] & S^{(\phi)}(E/K) \ard{d}\ar[r] & \Sh(E/K)\ar[d]&\\
%0\ar[r] & \fc{E'(K)}{\phi E(K)} \ar[r]^{\de} \ar[d]& H^1(K,E[\phi]) \ar[r]\ar[d]\ard{rd} & H^1(K,E)[\phi_*]\ar[r]\ar[d] & 0\\
%0\ar[r] & \prod_v \fc{E'(K_v)}{\phi E(K_v)}\ar[r] & \prod_v H^1(K_v, E[\phi]) \ar[r] & \prod_v H^1(K_v,E)[\phi_*]\ar[r] & 0.
%}
%\]
The horizontal and vertical sequences in the following are exact. The top row is exact by the Nine Lemma (or it can be shown more easily, in a direct fashion).
\[
\xymatrix{
&0\ar[d]&0\ar[d] & 0\ar[d] &\\
0\ar[r]& \fc{E'(K)}{\phi E(K)} \ar[r]\ar[d] & S^{(\phi)}(E/K) \ar[d]\ar[r] & \Sh(E/K)\ar[d]\ar[r]&0\\
0\ar[r] & \fc{E'(K)}{\phi E(K)} \ar[r]^{\de} \ar[d]& H^1(K,E[\phi]) \ar[r]\ar[d] & H^1(K,E)[\phi_*]\ar[r]\ar[d] & 0\\
0\ar[r] & 0\ar[r] & \prod_v H^1(K_v,E)[\phi_*] \ar[r]^{\id} & \prod_v H^1(K_v,E)[\phi_*]\ar[r] & 0.
}
\]
\end{proof}
Q: How do we come up with the map? We somehow reduce the calculation of $E(K)/mE(K)$ to the existence of a rational point, reduce a rank calculation question to a point-finding question. 
A: We have three pairings, let's see how to combine them. (The ``pp" is to remind us the pairing is perfect.)

\begin{center}
\begin{tabular}{cccccc}
(1) $\kappa:$ & $E'(K)/\phi E(K)$ & $\stackrel{pp}{\times}$ & $G(L_1/K)$ & $\to$ & $E[\phi]$\tabularnewline
(2) $e_{\phi}:$ & $E[\phi]$ & $\times$ & $E'[\wh{\phi}]$ & $\to$ & $\mu_{m}$\tabularnewline
(0) Kummer: & $K^{\times}/K^{\times m}$ & $\stackrel{pp}{\times}$ & $G(L_2/K)$ & $\to$ & $\mu_{m}$\tabularnewline
\end{tabular}
\end{center}
where 
\[
L_1=K(\phi^{-1}(E(K)))\qquad L_2=K(\sqrt[m]{K});
\]
$L_2$ is the maximal abelian extension of exponent $m$. 
Alternatively, we can restrict (0) to $L\subeq L_2$ to get $K^{\times}\cal L^{\times m}/K^{\times m}\stackrel{pp}{\times} G(L/K)\to \mu_m$. 
Thinking in a ``computer science" way, how can we combine pairings to get what we want? We can ``compose" pairings to get ``triplings."

\begin{tabular}{ccccc}
$A$ & $\times$ & $B$ & $\to$ & $C$\tabularnewline
$C$ & $\times$ & $D$ & $\to$ & $E$\tabularnewline
\hline
\multicolumn{3}{c}{$A\times B\times D$} & $\to$ & $E$\tabularnewline
\end{tabular}

We can switch between pairings and maps:
\[
A\times B\to C\quad \iff \quad A\to (B\to C).
\]
where we write $B\to C$ for $\Hom(B,C)$ (in this context, as groups) for visual effect. If the pairing is perfect, then the map $A\xrc (B\to C)$ map is a bijection.

We construct the map.
\begin{enumerate}
\item

\end{enumerate}•

(1)+(2) gives by rule $A$
\[
\fixme{E'(K)/\phi E(K)}\times G(L/K)\times \blu{E'[\wh{\phi}]}\to \blu{\mu_m}.
\]
which gives by rule $B$
\[
\fixme{E'(K)/\phi E(K)}\times \blu{E'[\wh{\phi}]}\to (G\to \blu{\mu_m})
\]
which gives by rule $B$ again
\[
\fixme{E(K)/\phi E(K)}\times \blu{E'[\wh{\phi}]}\to \blu{K^{\times}/K^{\times m}}.
\]
Careful: the $L$'s are not the same. Was it justified?

To motivate this more, look at what was tractable:
\begin{center}
\begin{tabular}{cccccc}
(1) $\kappa:$ & \fixme{$E(K)/\phi E(K)$} & $\stackrel{pp}{\times}$ & {$G(L_1/K)$} & $\to$ & $E[\phi]$\tabularnewline
(2) $e_{\phi}:$ & \blu{$E[\phi]$} & $\times$ & \blu{$E'[\wh{\phi}]$} & $\to$ & $\mu_{m}$\tabularnewline
(0) Kummer: & \blu{$K^{\times}/K^{\times m}$} & $\stackrel{pp}{\times}$ & $G(L_2/K)$ & $\to$ & $\mu_{m}$\tabularnewline
\end{tabular}
\end{center}
And note how we moved from what we wanted information about to what we had information about.

ADD: why the different $L$'s wasn't a problem.

%Now trace through to find what the map is. 
For $\phi$ a 2-isogeny, there's only one nonzero element of $E'[\wh{\phi}]$, and it corresponds to an element in
\[
\fixme{E(K)/\phi E(K)}\to \blu{K^{\times}/K^{\times m}}.
\]
So if we can find the preimage, we're done! Let's trace through to find what the map is: \fixme{Add}

Q: (lemma 15.3) How can we calculate $\rank E(K)$ from our info? We want to calculate $E(K)/2E(K)$. However, it's more convenient to deal with $\phi$ than $[2]$ (the reason being we just get 1 map we have to worry about above). We know this is going to be an exact sequence of some sort; first we want to change the $[2]$ into $\phi$, so we start
\[
\fc{E'(K)}{\phi E(K)}\xra{\wh{\phi}} \fc{E(K)}{2E(K)}\to \fc{E(K)}{\wh{\phi}E'(K)}\to 0.
\]
Find the kernel of the first map:
\[
0\to \fc{E'[\wh{\phi}]}{\phi E(K)} \to \fc{E'(K)}{\phi E(K)}\xra{\wh{\phi}} \fc{E(K)}{2E(K)}\to \fc{E(K)}{\wh{\phi}E'(K)}\to 0.
\]
Expand the first:
\[
0\to E[\phi]\to E[2]\xra{\phi} {E'[\wh{\phi}]} \to \fc{E'(K)}{\phi E(K)}\xra{\wh{\phi}} \fc{E(K)}{2E(K)}\to \fc{E(K)}{\wh{\phi}E'(K)}\to 0.
\]
Idea: it ``tells us what residue classes of $E(K)$ to look for independent generators."

\fixme{TODO: add Silverman's stuff. Algorithm for computing.}

\chapter{Integral points of elliptic curves}
%\chapter{Theta and elliptic functions}
\section{Theta functions}
\begin{df}
A \textbf{theta function} of degree $n$ on $[\om_1,\om_2]$ 
with parameter $b\ne0$ is an entire function $f(z)$ such that
\[
f(z+\om_1)=f(z),\quad f(z+\om_2)=be^{-\frac{2\pi inz}{\om_1}}f(z).
\]
\end{df}
We aim to classify all such functions. For simplicity assume $\om_1=1$ and $\om_2=\tau$, with $\Im \tau>0$. (Rescale.)
\begin{pr}
The space of theta functions of degree $n$ and parameter $b$ forms a $n$-dimensional space. They are in the form
\[
\sum_{k=0}^{\iy} a_k q^k
\]
where $q=e^{2\pi iz}$, $a_0,\ldots, a_{n-1}$ can be freely chosen, and the coefficients satisfy the recursive relation
\[
a_{m+pn}=b^{-p}q_0^{mp+\frac{np(p-1)}2}a_m,\quad q_0=e^{-2\pi i \tau}.
\]
\fixme{DARN there are so many different definitions of the theta function.} 
In particular, the following is a theta function of degree $1$ and parameter $b$:
\[
\theta(z)=\sum_{k\in \Z} (-1)^k q^{\frac{k(k-1)}2}e^{2\pi i kz}=C(q_0)\prod_{n=0}^{\iy} (1-q_0^n q)(1-q_0^{n+1}q^{-1}). 
\]
\end{pr}
We have the following analogue of the fundamental theorem of algebra.
\begin{thm}
Any theta function of degree $n$ is in the form
\[
f(z)=K\theta(z-z_1)\cdots \theta(z-z_n)q^r
\]
for some $z_1,\ldots, z_n\in \C$ and $r\in \Z$.
\end{thm}
\subsection{Transformation law}


\section{Elliptic functions}
\begin{df}
An \textbf{elliptic function} on the lattice $\La$ is a meromorphic function $f(z)$ on $\C$ such that
\[
f(z+\om)=f(z)\quad \text{for all }\om \in \La, z\in \C.
\]
%We call $\La$ the period.
Denote the space of all such functions by $\C(\La)$.
\end{df}

There are nice relationships involving the zeroes and poles of elliptic functions.
\begin{thm}Let $f$ be an elliptic function on $\La$.
\begin{enumerate}
\item $\sum_{w\in \C/\La} \Res_w(f)=0$.
\item $\sum_{w\in \C/\La} \ord_w(f)=0$, i.e. in a fundamental parallelogram there are as many zeros as poles, counting multiplicities.
\item $\sum_{w\in \C/\La} \ord_w(f)w\in \La$.
\end{enumerate}
\begin{proof}
\begin{enumerate}
\item
\item 
\item Label the edges of the fundamental parallelogram as follows.
\[
\xymatrix{
& \al+\omega_2 \ar[rr]^{C_2} & & \al+\omega_1+\omega_2\ar[ld]^{C_3} \\
\al\ar[ur]^{C_1} & &\al+\omega_1 \ar[ll]^{C_4}&
}
\]
We calculate $\int_{\partial P} \frac{zf'(z)}{f(z)}\,dz$ in two ways.

\textbf{Way 1:} 
\[
\int_{\partial P} \frac{zf'(z)}{f(z)} \,dz=\ba{
\int_{C_1} \frac{zf'(z)}{f(z)}\,dz
+\int_{C_3} \frac{zf'(z)}{f(z)}\,dz
}
+\ba{
\int_{C_2} \frac{zf'(z)}{f(z)}\,dz
+\int_{C_4} \frac{zf'(z)}{f(z)}\,dz
}.
\]
Noting that $C_3$ is just $C_1$ shifted by $\omega_1$ and reversed, and that $C_2$ is just $C_4$ shifted by $\omega_2$ and reversed, this equals
\[
\int_{\partial P} \frac{zf'(z)}{f(z)} \,dz=
\int_{C_1} \ba{\frac{zf'(z)}{f(z)}
-\frac{(z+\omega_1)f'(z+\omega_1)}{f(z+\omega_1 )}
}\,dz
+
\int_{C_4} \ba{\frac{zf'(z)}{f(z)}- \frac{(z+\omega_2)f'(z+\omega_2)}{f(z+\omega_2)}
}\,dz.
\]
Since $f$ is elliptic, $f(z)=f(z+\omega_1)=f(z+\omega_2)$, giving
\[
\int_{\partial P} \frac{zf'(z)}{f(z)} \,dz=
-\omega_1\int_{C_1} \frac{f'(z)}{f(z)} \,dz-\omega_2\int_{C_4} \frac{f'(z)}{f(z)}\,dz.
\]
Now $\ln(f(z))$ can be defined in a neighborhood around $C_1$ and $C_4$, since $f$ has no poles or zeros on $\partial P$. Since $f(\al)=f(\al+\omega_1)=f(\al+\omega_2)$, we have $\ln(f(\al+\omega_1))-\ln(f(\al))=2\pi i c_1$ and $\ln(f(\al))-\ln(f(\al+\omega_2))=2\pi i c_2$ for some integers $c_1$ and $c_2$. But these equal the above integrals by definition of $\ln f(z)$, so
\begin{equation}\label{p1-1-1}
\int_{\partial P} \frac{zf'(z)}{f(z)} \,dz=
-2\pi i(\omega_1 c_1+\omega_2 c_2).
\end{equation}

\textbf{Way 2:}
Note $\Res_a \frac{f'(z)}{f(z)}=\ord_a f$ so $\Res_a \frac{zf'(z)}{f(z)}=a \ord_a f$. Letting $a_k$ be the poles and zeros of $f$ in $P$, we get by Cauchy's Theorem that
\begin{equation}\label{p1-1-2}
\int_{\partial P} \frac{zf'(z)}{f(z)}=2\pi i\sum_{k} \Res_{a_k} \frac{f'(z)}{f(z)}=2\pi i \sum_{k}m_k a_k.
\end{equation}
Equating~(\ref{p1-1-1}) and~(\ref{p1-1-2}) give
\[
\sum_{k}m_ka_k=-\omega_1c_1-\omega_2c_2\equiv 0\pmod{\La}.
\]
\end{enumerate}
\end{proof}
\end{thm}
%\begin
\begin{df}
The \textbf{order} of an elliptic function is the number of poles in a fundamental parallelogram.
\end{df}
It turns out that elliptic functions can be expressed as quotients of theta functions.
\begin{thm}
\[
f(z)=K\frac{\theta(z-a_1)\cdots \theta(z-a_k)}{\theta(z-b_1)\cdots \theta(z-b_k)}, \quad \sum_{i=1}^k a_i=\sum_{i=1}^k b_i.
\]
\end{thm}
\section{Weierstrass $\wp$-function}
Our basic example of an elliptic function is the following.
\begin{df}
Define the Weierstrass $\wp$-function for the lattice $\La$ by
\[
\wp(z)=\rc{z^2}+\sum_{\la\in \La\bs\{0\}}\ba{
\rc{(z-\la)^2}-\rc{\la^2}
}.
\]
\end{df}
\begin{pr}
The series defining $\wp$ converges absolutely and locally uniformly on $\C-\{\La\}$. 
$\wp$ is an even elliptic function with period $\La$, analytic except for a double pole at each point of $\La$, 
\end{pr}
In fact, we will see that it is the building block for all elliptic functions.
\begin{proof}

\end{proof}
\begin{thm}
Every even elliptic function can be written as a polynomial in $\wp$. 
Every elliptic function can be written as a polynomial in $\wp$ and $\wp'$. 
\end{thm}
\begin{thm}
\[
\wp(z)-\wp(a)=\frac{\theta(z+a)\theta(z-a)}{\theta(z)^2} \cdot \frac{\theta'(0)^2}{\theta(a)\theta(-a)}.
\]
\end{thm}
\begin{thm}[Weierstrass differential equation]
\[
\wp'(z)^2=4(\wp(z)-e_1)(\wp(z)-e_2)(\wp(z)-e_3)
=4\wp(z)^3-\underbrace{60G_4}_{g_2}\wp(z)- \underbrace{140G_6}_{g_3}(z)
\]
\end{thm}
This says that for every $z$, the point $(\wp(z),\wp'(z))$ lies on the elliptic curve $y^2=4x^3-60G_4-140G_6$. Together with surjectivity and the Uniformization Theorem~\ref{uniformization} this implies that all elliptic curves can be parameterized in this way. (NONZERO DISC.)
\begin{thm}[Unifomization theorem]\llabel{uniformization}
Let $A,B\in \C$ satisfy $A^3-27B^2\ne 0$. Then there exists a unique lattice $\La\sub \C$ such that $g_2(\La)=A$ and $g_3(\La)=B$.
\end{thm}
%%%%%%%%%
\subsection{$\wp$ and lattices}
\begin{thm}
%cox, 10.14
Let $L$ be the lattice corresponding to $\wp(z)$. For $\al\in \C\bs \Z$, the following are equivalent.
\begin{enumerate}
\item $\wp(\al z)$ is  rational function in $\wp(z)$.
\item $\al L\subeq L$.
\item There is an order $\sO$ in an imaginary quadratic field $K$ such that $\al\in \sO$ and $L$ is homothetic to a proper $\sO$-ideal.
\end{enumerate}
Then 
\[
\wp(\al z)=\frac{A(\wp(z))}{B(\wp(z))}
\]
for relatively prime polynomials $A$ and $B$ such that 
\[
\deg(A)=\deg(B+1)=[L:\al L]=\N\al.
\]
\end{thm}
\begin{proof}
$(1)\implies (2)$: 
Suppose that $\wp(\al z)=\frac{A(\wp(z))}{B(\wp(z))}$ with $A$ and $B$ relatively prime. Then
\begin{equation}
B(\wp(z))\wp(\al z)=A(\wp(z)).
\end{equation}
%Note that each linear factor $\wp(z)+r$ of $A(\wp(z))$ and $B(\wp(z))$ has the same poles with the same orders as $\wp(z)$. 
For any $\om\in L$, %both 
$\wp(\om)$ %and $\wp(\al z)$ 
has a pole of order 2, and each linear factor $\wp(z)+r$ of $A(\wp(z))$ and $B(\wp(z))$ has a pole of order 2. 
In particular, for $\om=0$, we get that the order is
\[
2\deg(B)+2=2\deg(A)
\]
showing that $\deg(A)=\deg(B)+1$. Now take any $\om\in L$. Counting the order of $\om$ on both sides, we find that $\wp(\al z)$ has a pole of order 2 at $\om$. Thus $\al \om\in L$. This shows $\al L\subeq L$.

$(2)\implies (1)$:
For any $w\in L$, since $\al L\subeq L$we have
\[
\wp(\al (z+w))=\wp(\al z+\underbrace{\al w}_{\in L})=\wp(\al z).
\] 
Hence $\wp(z)$ is elliptic with $L$ as a lattice of periods. Since it is even, by (?) it is a rational function in $\wp$.

$(2)\implies (3)$: %Assume $\al$ acts on $L=\an{1,\tau}$ by $\smatt abcd$. Then
By a homothety we may suppose $L=\an{1,\tau}$. Since $L$ has rank 2 as a $\Z$-module, $\tau$ must be of degree 2 over $\Q$. Now take
\[
\sO=\set{\be \in \Q(\tau)}{\be L\subeq L},
\]  
i.e. the ``codifferent."

$(3)\implies (2)$: Easy.

Now, supposing (1) is true, rearrange $\wp(\al z)=\frac{A(\wp(z))}{B(\wp(z))}$ to get
\begin{equation}\label{awpb}
A(x)=\wp(\al z) B(x)=0.
\end{equation}
Fix $z$ so that $2z\nin \rc{\al}L$ and such that $A(x)-\wp(\al z)B(x)$ has distinct zeros. (Claim: Given polynomials $A$, $B$, there are only a finite nmber of values of $c$ so that $A-cB$ has multiple roots.) 
Let $\{w_i\}$ be a set of coset representatives for $L$ in $\rc{\al}L$. We claim that the roots of~(\ref{awpb}) are exactly $z+w_i$.

We have
\[
A(\wp(z+w_i))-\wp(\al z)B(\wp(z+w_i))=A(\wp(z+w_i))-\wp(\al (z+w_i))B(\wp(z+w_i))=0
\]
by blah, so $\wp(z+w_i)$ are roots of~(\ref{awpb}).

Now if $\wp(z+w_i)=\wp(z+w_j)$ then by BLAH, $(z+w_i)=\pm(z+w_j)\pmod{L}$, giving either $2z\equiv w_i-w_j\in \rc{\al}L$ and $2z\in \rc{\al}$, or $w_i\equiv w_j\pmod L$. The first is impossible by assumption on $z$, so $i=j$. This shows the roots are distinct. 

Finally, given any root of~(\ref{awpb}), by surjectivity of $\wp$ we can write it in the form $\wp(y)$. We have
\[
\wp(\al y)=\frac{A(\wp(y))}{B(\wp(y))}=\wp(\al z),
\]
where the first equality is by definition of $A$ and $B$ and the second is because $\wp(y)$ is a root of~(\ref{awpb}). Then by BLAH, $\al y\pm \al z\equiv 0\pmod{L}$. Since $\wp$ is even, we may replace $y$ by $-y$ as necessary, to get $\al(y-z)\equiv 0\pmod{\rc{\al} L}$. Thus  $y\in z+\rc{\al}L$ and $\wp(y)=\wp(z+w_i)$ for some $i$, as needed.

Since~(\ref{awpb}) has $[L:\rc{\al}L]=[\al L:L]$ roots,~(\ref{awpb}) and hence $A$ has degree $[\al L:L]$.
\end{proof}

Note the equivalence $(2)\iff (3)$ (which incidentally has nothing to do with elliptic functions) gives that a lattice is a proper fractional ideal of $\sO$ iff it has $\sO$ as its ring of complex multiplication. Nonzero fractional ideals are homothetic iff they determine the same element in the ideal class group. Hence there is a correspondence between IDEAL CLASS GRP and homothety classes of lattices with $\sO$ as full ring of complex multiplication.
%%%%%%%%%%%%%%%%%%%%%

\chapter{Modular forms on $\SL_2(\Z)$}
\section{$\SL_2(\Z)$ and congruence subgroups}
\begin{df}
$\SL_2(\Z)$ is the group of $2\times 2$ integer matrices with determinant 1.
\[
\SL_2(\Z):=\set{\matt abcd}{a,b,c,d\in \Z, \, ad-bc=1}.
\]
Define $\PSL_2(\Z)=\SL_2(\Z)/\{\pm 1\}$.
Define the following subgroups:
\begin{align*}
\Ga(N)&=\set{M\in \SL_2(\Z)}{M\equiv \matt 1001\pmod{N}}\\
\Ga_1(N)&=\set{M\in \SL_2(\Z)}{M\equiv \matt 1*01\pmod{N}}\\
\Ga_0(N)&=\set{M\in \SL_2(\Z)}{M\equiv \matt **0* \pmod{N}}.
\end{align*}
Any subgroup of $\SL_2(\Z)$ containing $\Ga(N)$ for some $N$ is called a \textbf{congruence subgroup}.
\end{df}
\begin{df}
%The fractional linear transformation transformation associated to $\matt abcd$ is
%\[
%z\mapsto \frac{az+b}{cz+d}.
%\]
%\end{df}
%Note that the map $\matt abcd \mapsto \frac{az+b}{cz+d}$ is a group isomorphism between $\PSL_2(\Z)$ and the group of fractional linear transformations. We often use a matrix to denote the fractional linear transformation.
$\SL_2(\Z)$ acts on the upper half plane $\cal H$ by
\[
\matt abcd z=\frac{az+b}{cz+d}.
\]
\end{df}

We now collect some facts about $\SL_2(\Z)$ and its congruence subgroups.
\begin{pr}
The matrices $S=\smatt 01{-1}0$ and $T=\smatt 1101$ generate $\SL_2(\Z)$.
\end{pr}

\subsection{Cosets}
\begin{pr}
We have the following:
\begin{align*}
[\SL_2(\Z):\Ga_0(N)]&=N\prod_{p\mid N}\pa{1+\rc p}\\
[\Ga_0(N):\Ga_1(N)]&=N\prod_{p\mid N}\pa{1-\rc p}\\
[\Ga_1(N):\Ga(N)]&=N.
\end{align*}
Moreover,
\begin{enumerate}
\item
Set of coset reps for $\Ga_0(N)$ in $\SL_2(\Z)$?
\item 
Let $S=\set{(a,b)\in (\Z/N\Z)^2}{\gcd(a,b)=1}$. For each
\[(z,t)\in P:=\frac{S-\{(0,0)\}}{(\Z/N\Z)^{\times}}\]
take an integer matrix of the form
$\smatt xyzt$. These matrices
form a set of right coset representatives for ${\Ga_0(N)}$ in $\SL_2(\Z)$. 
\end{enumerate}
\end{pr}
\begin{proof}
\begin{enumerate}
\item
Let $G$ be the group
\[
\{
(a,y)|a\in (\Z/N\Z)^{\times},y\in \Z/N\Z
\}/\{\pm(1,0)\}
\]
with the operation
\[
(a,y)(a',y')=(aa',ay'+a'^{-1}y).
\]
The fact that $G$ is a group can be shown directly, or by noting that the group structure on $G$ is the ``pushforward" of the group structure on $\Ga_0(N)$ by $\pi$ below.
We claim that
\[
1\to \overline{\Ga(N)}\to \overline{\Ga_0(N)} \xra{\pi}G\to 1
\]
is a short exact sequence, where 
\[
\pi\pa{\matt ab{Nc}d}=(a,b)\bmod N.
\]
We verify:
\begin{enumerate}
\item
$\pi$ is surjective: Given $(\ol{a},\ol{b})\in G$, we can choose $b$ so that $a\equiv \ol{a}\pmod N,b\equiv \ol{b}\pmod N$ so that $\gcd(a,b)=1$.
Let $d$ be an integer such that $ad\equiv 1\pmod N$. By B\'ezout's Theorem we can find $k,l$ so that $ak-lb=\frac{1-ad}{N}$. Then $a(d+kN)-Nlb=1$, and the following  matrix is in $\SL_2(\Z)$.
\[
\pi\pa{\matt ab{Nl}{d+kN}}=(a,b).
\]
\item
$\ker(\pi)=\overline{\Ga(N)}$: The inclusion $\overline{\Ga(N)}\subeq \ker (\pi)$ is clear.
Conversely, if $A=\matt ab{Nc}d\in \Ga_0(N)$, $\pi(A)=(1,0)$, then $a\equiv 1\pmod N$ and $b\equiv 0\pmod N$; moreover $ad-(Nc)d=1$ and $a\equiv 1\pmod N$ imply $b\equiv 1 \pmod N$. 
\end{enumerate}
%$G$ is a group because
%\begin{align*}
%[(a,b,y)(a',b',y')](a'',b'',y'')=(aa'a'',bb'b'',aa'y+ab''y+b'b''y)=(a,b,y)(
%\end{align*}
First suppose $N\neq 2$.
Then $|G|=\rc{2}\ph(N)N$, so
\[
[\PSL_2(\Z):\ol{\Ga_0(N)}]=\frac{[\PSL_2(\Z):\ol{\Ga(N)}]}{|G|}=\frac{\frac{N^3}{2}\prod_{p|N}\pa{1-\rc{p^2}}}{N\prod_{p|N}\pa{1-\rc p}}=N\prod_{p|N}\pa{1+\rc p}.
\]
For $N=2$, $[\PSL_2(\Z),\ol{\Ga(N)}]=6$ and $|G|=2$, so $[\PSL_2(\Z):\ol{\Ga_0(N)}]=3$ (and the above formula works as well).
\end{enumerate}
\end{proof}
\subsection{Useful decompositions}
Bruhat
\subsection{Fundamental domains}
\begin{df}
Let $H$ be a subgroup of $\SL_2(\Z)$. A \textbf{fundamental domain} for $H$ is a subset of $\cal H$ such that the following hold.
\begin{enumerate}
\item 
\end{enumerate}
\end{df}

\section{Modular forms}
\begin{df}
A \textbf{modular function} on $\SL_2(\Z)$ is a function $f:\cal H\to \C$ such that
\begin{enumerate}
\item $f$ is meromorphic on $\cal H$.
\item $f$ satisfies the following transformation property.
\[
f\pa{\matt abcd z}=(cz+d)^k f(z)\text{ for all }\matt abcd\in \SL_2.
\]
\end{enumerate}
If moreover $f$ is holomorphic on $\cal H$ we say $f$ is a 
\textbf{weakly holomorphic modular form}, and if $f$ is holomorphic on 
$\cal H^*=\cal H\cup \{\iy\}$, we say that $f$ is a \textbf{modular form}. ($f$ is ``holomorphic at $\iy$" if $f$ has a Fourier expansion with nonnegative exponents
\[
f(z)=\sum_{n\ge 0} a_nq^n,\quad q=e^{2\pi iz}.)
\]
We say $f$ is a \textbf{cusp form} is $a_0=0$ above. We denote
\begin{align*}
M_k^!&=\text{weakly holomorphic modular forms of weight }k\\
M_k&=\text{modular forms of weight }k\\
S_k&=\text{cusp forms of weight }k.
\end{align*}
\end{df}
Note we will generalize this definition several times (add references when I put them in)
\begin{thm}[Weight formula]
Let $f$ be a modular form of weight $k$. Then 
\[
k=6\ord_i(f)+4\ord_{\om}(f)+12\ord_{i\iy}(f)
+12\sum_{z\in R_{\Ga}^{\circ}}\ord_{z}(f).
\]
\end{thm}
\begin{proof}
Don't feel like writing... will be vastly generalized using Riemann-Roch anyway.
\end{proof}
\section{Eisenstein series}
The following will be our most important source of modular forms.
\begin{df}
Let $k\ge 4$ be even. 
Define the \textbf{Eisenstein series} of weight $k$ as a function on lattices to be 
\[
G_{k}(\La)=\sum_{\om \in \La\bs\{0\}} \rc{\om^{2k}}.
\]
Define the Eisenstein series as a function on $\cal H$ to be
\[
G_k(z)=G_k((1,z))=\sum_{(a,b)\in \Z^2\bs \{0\}} \rc{(a+bz)^{2k}}.
\]
Define the normalized Eisenstein series as $E_k=?G_k$. 
\end{df}
Note that if $k$ is odd, $G_k$ as defined above will be 0.
\begin{pr}
$G_k$ is absolutely convergent, and is a modular form of weight $k$.
\end{pr}
\begin{thm}
The Fourier expansion of $E_k$ is
\[
E_k(z)=1-\frac{2k}{B_k}\sum_{n=1}^{\iy} \si_{k-1}(n)q^n
\]
where $B_k$ is the $k$th Bernoulli number: $\frac{t}{e^t-1}=1+\sum_{n\ge 1}\frac{B_n}{n!}t^n$.
\end{thm}
\begin{df}
Define 
\begin{align*}
\De&=\frac{E_4^3-E_6^2}{1728}
\end{align*}
as a function either on lattices or on $\cal H$.
\end{df}
$\De$ is a cusp form of weight 12, normalized so its first term is $z$. As we will see, it spans the space of cusp forms of weight 12.

%%%
The functions $G_4,G_6$ parameterize elliptic curves over $\C$. (See...) The following will be important in establishing a connection between elliptic curves and lattices.
\begin{thm}[Uniformization theorem]
The map $\Ga\to \C^2\bs\{\De=0\}$ defined by
\[
\Ga\mapsto (G_4,G_6)
\]
is surjective (bijection?).
\end{thm}
\section{The spaces $M_k$}
\begin{thm}
The set
\[
%\set{E_4^aE_6^b}{4a+6b=k}
\set{E_{k-12r}\De^r}{0\le r\le \fl{\frac{k}{12}}, k-12r\ne 2}
\]
is a basis for $M_k$. Thus
\[
\dim(M_k)=
\begin{cases}
\fl{\frac{k}{12}},&k\equiv 2\pmod{12},\\
\fl{\frac{k}{12}}+1,&k\nequiv 2\pmod{12},
\end{cases}
\qquad
\dim(S_k)=
\begin{cases}
\fl{\frac{k}{12}}-1,&k\equiv 2\pmod{12},\\
\fl{\frac{k}{12}},&k\nequiv 2\pmod{12}.
\end{cases}
\]
\end{thm}
\section{Dedekind eta function}
\begin{thm}[Transformation properties of $\eta$]\label{eta-transforms}
The function $\eta(\tau)=q^{\rc{24}}\prod_{n=1}^{\iy} (1-q^n)$ satisfies
\begin{align*}
\eta(\tau+1)&=e^{\frac{2\pi i}{24}}\eta(\tau)\\
\eta\pf{-1}{\tau}&=\sqrt{\frac{\tau}{i}}\eta(\tau).
\end{align*}
\end{thm}
There are two main ingredients to the proof.
\begin{enumerate}
\item Derive transformation properties for twisted theta functions $\te_{\chi}$ using the Poisson summation formula.
\item Write $\eta$ in terms of theta functions using the Pentagonal Number Theorem~\ref{pentagonal}.
\end{enumerate}
\begin{proof}
For the first part, note
\[
\eta(\tau+1)=e^{\frac{2\pi i (\tau+1)}{24}}\prod_{n=1}^{\iy}(1-e^{2\pi i (\tau+1)})=e^{\frac{\pi i}{12}}\prod_{n=1}^{\iy}(1-e^{2\pi i \tau})=\eta(\tau).
\]

For the second part, recall the transformation formula for the theta function (Proposition~\ref{l-func-dirichlet}.\ref{theta-transforms})
\begin{equation}\label{eta-theta-transformation}
\te_{\chi}(\tau)=\frac{G(\chi,e^{\frac{2\pi i \bullet}{r}})}{q\sqrt{\tau}} \te_{\ol{\chi}}\prc{q^2u}
\end{equation}
where $\chi$ is a primitive multiplicative character modulo $r$.

By the Pentagonal Number Theorem,
\begin{align}
\nonumber
\eta(\tau)&=q^{\rc{24}} \prod_{n=1}^{\iy} (1-q^n)\\
\nonumber
&=q^{\rc{24}} \sum_{n\in \Z} (-1)^n q^{\frac{3n^2+n}{2}}\\
\nonumber
&=\sum_{n\in \Z} (-1)^n q^{\frac{36n^2+12n+1}{24}}\\
\nonumber
&=\sum_{n\in \Z} (-1)^n e^{-\pi(6n+1)^2 \pf{-\tau}{24}}\\
\nonumber
&=\rc{2}\pa{\sum_{n\in \Z} (-1)^n e^{-\pi(6n+1)^2 \pf{-\tau}{24}}+\sum_{n\in \Z} (-1)^n e^{-\pi(-6n-1)^2 \pf{-\tau}{24}}}\\
\label{eta-in-terms-of-theta}
&=\theta_{\chi}\pf{-\tau}{24}
\end{align}
where $\chi(n)$ is the character modulo 12 taking values $1,-1,-1,1$ at $1, 5, 7, 11$, respectively.

%Now we apply~(\ref{eta-theta-transformation}) to get 
First note $G(\chi,e^{\frac{2\pi i \bullet}{r}})=e^{\frac{\pi i}{6}}-e^{\frac{5\pi i}{6}}-e^{\frac{7\pi i}{6}}+e^{\frac{11\pi i}{6}}=2\sqrt 3$. Hence
\begin{align*}
\eta\pa{-\rc{\tau}} &=\te_{\chi}\pf{i}{12\tau}&\by{eta-in-terms-of-theta}\\
&=\frac{G(\chi,e^{\frac{2\pi i \bullet}{r}})}{12\sqrt{i/(12\tau)}}\te_{\chi}\pf{12\tau}{144i}&\by{eta-theta-transformation}\\
&=\sqrt{\frac{-i\tau}{\cancel{12}}}\cancel{2\sqrt 3}\te_{\chi}\pf{12\tau}{144i}\\
&=\sqrt{-i\tau}\eta(\tau).&\by{eta-in-terms-of-theta}
\end{align*}
\end{proof}
%
\section{Derivatives of modular forms}
Let $f$ be a modular form of weight $k$. Is $f'$ (derivative with respect to $\tau$) a modular form? Differentiating the transformation law gives
\begin{align}
\nonumber f\pf{a\tau+b}{c\tau+d}&=(c\tau+d)^k f(\tau)\\
\nonumber f'\pf{a\tau+b}{c\tau+d}(c\tau+d)^{-2}&=k(c\tau+d)^{k-1} cf(\tau)+(c\tau+d)^k f'(\tau)\\
f'\pf{a\tau+b}{c\tau+d}&=\underbrace{k(c\tau+d)^{k+1} cf(\tau)}_{\text{Uh-oh.}}+(c\tau+d)^{k+2} f'(\tau).\label{fdifftrans}
\end{align}
Unfortunately, $f'$ isn't quite modular. So we need to construct a modified notion of derivative (which we'll call $\theta$) that takes $M_k$ to $M_{k+2}$. To do this, we will use the derivative and the $P$ function, defined below in terms of the $\eta$ function. 
\begin{df}
Define
%\begin{align*}
\[P(\tau)=\frac{24}{2\pi i} \frac{\eta'(\tau)}{\eta(\tau)}.\]
%E_2(\tau)&=1-\underbrace{\frac{4}{B_2}}_{24}\su \si_1(n)q^n.
%\end{align*}
\end{df}
\begin{thm}$\,$
\begin{enumerate}
\item
$P=E_2$, i.e.
\[
P=1-\underbrace{\frac{4}{B_2}}_{24}\suo \si_1(n)q^n.
\]
\item
$P$ satisfies the transformation law
\begin{equation}\label{Ptrans}
P(\ga \tau)=(c\tau+d)^2P(\tau)+\underbrace{\frac{12c}{2\pi i}(c\tau+d)}_{\text{``nonmodular" part}}.
\end{equation}
\end{enumerate}
\end{thm}
\begin{proof}
For item 1, note that $\frac{d}{d\tau}=2\pi i q\frac{d}{dq}$ by the chain rule so
\begin{align*}
\frac{d}{d\tau}\ln \eta(\tau)&=2\pi i q\pa{\su \frac{d}{dq}\ln(1-q^n)+\frac{d}{dq} \ln q^{\rc{24}}
}\\
\frac{\eta'(\tau)}{\eta(\tau)}&=2\pi i\pa{\su \frac{nq^n}{1-q^n}+\rc{24}}\\
&=2\pi i\pa{\su \sum_{m>0,n|m}q^{m}+\rc{24}}\\
&=2\pi i\pa{\sum_{m\ge 1}\si_1(m)q^m+\rc{24}}.
\end{align*}

For item 2, note $\an{S,T}=\GL_2(\Z)$, so $\ga$ can be written as a product of $S=\matt0{-1}10,T=\matt1101,T^{-1}=\matt1101$. The base case is trivial.
For the induction step, first differentiate the transformation laws for $\eta$ to get
\begin{align*}
\rc{\tau^2}\eta'(S\tau)&=\frac{\tau^{-\rc2}}{2\sqrt i}\eta(\tau)+\frac{\tau^{\rc2}}{\sqrt i} \eta'(\tau)\\
\eta'(T\tau)&=e^{\frac{2\pi i}{24}}\eta(\tau).
\end{align*}
Using this we can calculate how $\frac{24}{2\pi i}\frac{\eta'}{\eta}$ transforms under $\eta$. The induction step comes from checking that if $\ga=\matt abcd$ then
\begin{align*}
P(S\ga\tau)&=(a\tau+b)^2P(\tau)+\frac{12a}{2\pi i}(a\tau+b)\\
P(T^{\pm 1}\ga\tau)&=P(\ga\tau).
\end{align*}
\end{proof}

Now we are ready to define our differential operator. 
\begin{df}
For $f$ a weight $k$ modular form, define
\[
\partial_k(f)=(12\theta-kP)f
\]
where
\[
\theta=q\frac{d}{dq}=\rc{2\pi i}\frac{d}{d\tau}.
\]
\end{df}

\begin{thm}$\,$
\begin{enumerate}
\item
$\partial_k$ is a map from $M_k$ to $M_{k+2}$.
\item
$\partial$ is a derivation, i.e. for $f\in M_m,g\in M_n$, we have
\[
\partial_{m+n}(fg)=(\partial_m f)g+f(\partial_n g).
\]
\item The following hold ($P=E_2,Q=E_4,R=E_6$):
\begin{align*}
\partial_2P&=-Q&\theta P&=\rc{12}(P^2-Q)\\
\partial_4Q&=-4R&\theta Q&=\rc{3}(PQ-R)\\
\partial_6R&=-6Q^2&\theta R&=\rc{2}(PR-Q^2).
\end{align*}
\end{enumerate}
\end{thm}
\begin{proof}
For part 1, calculate $(\partial f)(A\tau)$ using~(\ref{fdifftrans}) and~(\ref{Ptrans}).

For part 2,
\[
\partial_{m+n}(fg)=\rc{2\pi i} (fg)'-(m+n)Pfg=\rc{2\pi i}f'g-m(Pf)g+\rc{2\pi i}fg'-nf(Pg)=(\partial_mf)g+f(\partial_n g).
\]

For part 3, 
more calculations show that $\partial_2P+P^2$ is a modular form. The equalities follow from using $\dim(M_4)=\dim(M_6)=\dim(M_8)=1$ and matching constant terms of the $q$-series.
\end{proof}
\begin{rem}
Since $Q,R$ generate the space of modular forms, this completely describes the action of $\partial$ on modular forms. The fact that it is a derivation means that we can calculate its action on a polynomial in $P,Q,R$ as if it were actually a derivative, taking note what $\partial_2P,\partial_4Q,\partial_6R$ are. This is since for polynomials, stuff like the chain rule can be derived from the product rule, which we have.
\end{rem}
%
\section{The $j$-function}
\begin{df}
Define the $j$-function (on lattices or $\cal H$) by
\[
j=\frac{E_4^3}{\De}.\]
\end{df}
Since $E_4^3$ and $\De$ are modular forms of weight $12$, $j$ is a modular function of weight $0$. The function $j$ has some very nice properties.

\begin{thm}
$j$ takes on every value in $\C$ exactly once in its fundamental domain. \fixme{in/excluding boundaries the right way}
\end{thm}
\begin{thm}
A function on $\cal H$ is a modular function of weight 0 if and only if it is a rational function of $j$.
\end{thm}
%As an application we prove the following
%\begin{thm}[Picard]
%
%\end{thm}
\index{modular polynomial}
\subsection{The modular polynomial $\Phi_m$}
\begin{df}
Define $\Phi_m(X,Y)$ so that $\Phi_m(j, Y)$ is the minimal polynomial of $j(Nz)$ over $\C(j)$. 
\end{df}
Note this is well-defined because $\C(j)\cong \C(X)$.

This will be important when we define the moduli space of an elliptic curve, because $(j(z),j(Nz))$ will map the moduli space to an algebraic curve whose associated function field is $\C(j(z), j(Nz))$.
\begin{pr}\label{phim}
The following are true.
\begin{enumerate}
\item
$\Phi_m(X,Y)\in \Z$.
\item
$\Phi_m(X,Y)$ is symmetric for $m>1$.
\item (Kronecker's congruence)
If $p$ is prime, then 
\[
\Phi_p(X,Y)=(X^p-X)(Y^p-Y)\pmod{p}.
\]
\item If $m$ is squarefree then $\Phi_m(X,X)$ has leading coefficient $\pm1$.
\end{enumerate}
\end{pr}
\begin{proof}
\begin{enumerate}
\item
\item $F(X,Y)=F(Y,X)$:
Replacing $z$ with $-\rc{Nz}$ in 
\[
F(j(z),j(Nz))=0
\]
gives
\begin{align*}
F\pa{
j\pa{-\rc{Nz}},j\pa{-\rc z}
}=0.
\end{align*}
Note that $j$ is invariant under $\ga=\smatt0{1}{-1}0\in \SL_2(\Z)$ which sends $z$ to $-\rc{z}$. Hence $j\pa{-\rc{Nz}}=j(Nz)$, $j\pa{-\rc z}=j(z)$, and we get
\[
F(j(Nz),j(z))=0.
\]
Since $F(X,Y)$ is irreducible in $\C[X,Y]$, so is $F(Y,X)$. 
Then $F(Y,j)$ is also the irreducible polynomial of $Y$ over $\C(j)$, so replacing $j$ with $X$, this says that $F(Y,X)|F(X,Y)$. The only way for this to happen is if $F(X,Y)=cF(Y,X)$. We have $F(X,Y)=cF(Y,X)=c^2F(X,Y)$, so $c=\pm 1$. If $c=-1$, then $F(X,Y)=-F(Y,X)$, and putting $X=Y$ gives $F(X,X)=0$. This shows $X-Y|F(X,Y)$, which is impossible since $F(X,Y)$ is irreducible with degree $[\Ga(1):\Ga_0(N)]>1$. Thus $F(X,Y)=F(Y,X)$.
\item\begin{lem}
Let $\ga_1,\ldots, \ga_{p+1}$ be coset representatives for $[\Ga(1):\Ga_0(p)]$. Then 
\[
\{j(p\ga_1z),\ldots,j(p\ga_{p+1}z)\}=\{j(pz)\}\cup\bc{j\pf{z+k}{p}:0\le k<p}.
\]
\end{lem}
\begin{proof}
There are indeed $p+1$ coset representatives because $\mu=N\prod_{\text{prime } q|N}\pa{1+\frac{1}{q}}=p+1$ in this case.
%We can take $\ga_{p+1}=I$; this gives $
%
Given $\ga=\smatt abcd$, we have $p\ga z=\smatt{pa}{pb}cdz$. 
For any $\ga'\in \Ga(1)$, we have $j(\ga'p\ga z)=j(p\ga z)$ since $j$ is invariant under $\Ga(1)$. By Lemma 6.3.1 we can multiply $\smatt{pa}{pb}cd$ on the left by some matrix in $\Ga(1)$ to get some $\smatt {a'}{b'}0{d'}$ with $a'd'=\det\smatt{pa}{pb}cd=p$ and $0\le b'<d'$. The $p+1$ possible matrices are $\smatt p001$ and $\smatt 1{k}0p$ for $0\leq k<p$. We claim that all these are in fact attained. Let $M$ be one of these matrices. Then by the Elementary Divisors Theorem there exist $A,B\in \Ga(1)$ such that $AMB=\smatt p001$. But then $M=A^{-1}NB$, so $j(Mz)=j(A^{-1}NBz)$, and we could have picked $B$ as a coset representative (the choice doesn't matter anyways). The lemma follows upon noting that $\smatt p001z=pz$ and $\smatt 1{k}0pz=\frac{z+k}{p}$.
\end{proof}
Let $\zeta_p$ be a $p$th root of unity. %and let $\mfp=\an{1-\zeta_p}$. 
We have that $1-\zeta_p|p$: indeed 
\[
p=x^{p-1}+\cdots +1|_{x=1}=(1-\zeta_p)\cdots (1-\zeta^{p-1}).
\]
When we expand $j\pf{z+k}{p}$, its coefficients are roots of unity times the coefficients of $j(z)$. However, roots are unity are congruent to $1$ modulo $\mfp$, since $\zeta_p^k-1=(\zeta_p-1)(\zeta_p^{k-1}+\cdots +1)$. Then
\begin{align*}
%F(j(z),Y)&=\prod_{i=1}^{p+1}(Y-j(
%F(X,j(pz))%&=F(j(pz),X)\\
F(j(z),Y)%&=F(j(\\
&=\prod_{i=1}^{p+1}(Y-j(\ga_ipz))\\
&=(Y-j(pz))\prod_{k=1}^{p}\pa{Y-j\pf{z+k}{p}}\\
&\equiv (Y-j(pz))\pa{Y-j\pa{\frac zp}}^p\pmod{1-\zeta_p} \\
&\equiv (Y-j(z)^p)\pa{Y^p-j(z)}\pmod{1-\zeta_p},
\end{align*}
the last equation following because raising the $j$ function to the $p$th power is the same, modulo $p$, as raising each term to the $p$th power, and the coefficients (which are integers) are not affected modulo $p$, while the exponents are multiplied by $p$. 
Replacing $j(z)$ by $X$ we get
\[F(X,Y)\equiv (Y-X^p)(Y^p-X)\equiv X^{p+1}+Y^{p+1}-X^pY^p-XY \pmod{1-\zeta_p}.\]
However, $\an{1-\zeta_p}\cap \Z=\an{p}$ (it contains $\an{p}$, and $\an{p}$ is maximal in $\Z$), and we know $F(X,Y)$ has integer coefficients, so congruence holds modulo $p$.
\end{enumerate}
\end{proof}
\section{$j$ and Hilbert class fields}
Our main theorem in this section (Theorem~\ref{j-generates-hilbert}) is that values of the $j$-function at quadratic integers (or equivalently quadratic ideals) generate Hilbert class fields of quadratic extensions. To prove this we first need a result on $j$ in terms of lattices.
\begin{df}
A \textbf{cyclic sublattice} $L'\subeq L$ is a lattice such that $L/L'$ is a cyclic group.
\end{df}
\begin{thm}[Correspondence between roots of $\Phi$ and cyclic sublattices]\label{cyclic-roots-phi}
Let $m\in \N$. The following are equivalent.
\begin{enumerate}
\item $\Phi_m(u,v)=0$.
\item There is a lattice $L$ with cyclic sublattice $L'\subeq L$ of index $m$ such that $u=j(L')$ and $v=j(L)$.
\end{enumerate}
\end{thm}
We first characterize cyclic sublattices.
\begin{lem}\label{char-cyclic-latt}
The cyclic lattices of $\an{1,\tau}$ are exactly those given by
\begin{equation}\label{cyclic-lattices}
L'=\an{d, a+b\tau},\quad \matt ab0d\in C(m),
\end{equation}
where 
\[
C(m)=\set{\matt ab0d}{ad=m, a>0, 0\le b<d, \gcd(a,b,d)=1}.
\]
Moreover, these give rise to distinct lattices.
\end{lem}
\begin{proof}
Suppose $L'=\an{d,a\tau+b}$. Then the presentation of the $\Z$-module $L/L'$ is  given by $\smatt ab0d$. By the structure theorem for modules, we have $ \smatt ab0d \in \SL_2(\Z)\matt {d_1}00{d_2}\SL_2(\Z)$ for some $d_1\mid d_2$ and that $L/L'\cong \Z/d_1\Z\times \Z/d_2\Z$. Note that multiplying by a matrix in $\SL_2(\Z)$ preserves the gcd of the entries. Hence we find that $d_1=\gcd(a,b,d)$. Hence 
\begin{equation}\label{cycliciff}
L'\text{ is cyclic }\iff\gcd(a,b,d)=1.
\end{equation}

This shows that all lattices in the form~(\ref{cyclic-lattices}) are cyclic.

Now given a cyclic sublattice $L'$, let $d\in \N$ be the smallest integer in $L'$, and $a+b\tau$ be such that $L'=\an{d, a\tau+b}$. 
%be a nonreal element of smallest absolute value, so that of those elements it makes the smallest angle with the real axis. 
We may change $b$ by a multiple of $d$ so that $0\le b<d$.  Since $m=[L:L']=\sdetm ab0d=ad$ and $\gcd(a,b,d)=1$ by~(\ref{cycliciff}), $\smatt ab0d\in C(m)$.

Uniqueness follows since $d$ is the least positive integer in $L'=\an{d, a\tau+b}$, and once $d$ is determined, $a=\frac{m}d$ and $b$ are determined.
\end{proof}
\begin{proof}[Proof of Theorem~\ref{cyclic-roots-phi}]
By Lemma~\ref{char-cyclic-latt}, when $L'=[d, a+b\tau]$, letting $\si=\smatt ab0d$, we have
\[
j(L')=j(d[1,\si \tau])=j([1,\si\tau]).
\]
Then
\[
\Phi_m(X,j(\tau))=\prod_{\si\in C(m)} (X-j(\si\tau))=\prod_{L'\text{ cyclic in }L, \,[L:L']=m} (X-j(L')).
\]

Hence any pair $(j(L),j(L'))$ is a solution; conversely, given a solution $(X,Y)$, we have $Y=j(L)$ for some $L$, and the above gives $X=j(L')$.
\end{proof}

\begin{thm}
Let $\sO$ be an order in an imaginary quadratic field and $\ma$ a $\sO$-ideal. Then $j(\ma)$ is an algebraic integer and
$K(j(\ma))$ is the ring class field of $\sO$.
\end{thm}
\begin{proof}
Let $M=K(j(\ma))$ and $L$ be the ring class field of $\sO$. 

\noindent{\underline{Step 1:}} 
%Suppose $\al\in \sO$ is primitive. 
Suppose $\al\ma$ is a cyclic sublattice of $\ma$; let $m=\N(\al)$. We have
\begin{equation}
\label{phi-ja}
\Phi_m(j(\ma),j(\ma))=\Phi_m(j(\al\ma),j(\ma))=0,
\end{equation}
where the first equality is by Theorem~\ref{cyclic-roots-phi} and the second is because $\ma$ and $\al\ma$ are similar lattices. Hence $j(\ma)$ is a root of $\Phi_m(X,X)$.

Pick $\al$ so that $\N\al$ is squarefree. 
To do this we note that by Theorem~??.\ref{p=x2+ny2}
\begin{equation}\label{spl-is-norm}
\Spl(L/\Q)\approx \set{p\text{ prime}}{p=N(\al)\text{ for some }\al\in \sO}.
\end{equation}
%Take $\mfp \in \Spl(L/\Q)$. 
Choosing such $\al$, we have $[\ma:\al\ma]=N(\al)=p$, so $\al\ma$ must be cyclic. Then the leading coefficient of $\Phi_m(X,X)$ is $\pm 1$ by Proposition~(\ref{phim}), so $j(\ma)$ is an algebraic integer.\\

\noindent\underline{Step 2:} 
We show $M= L$ by examining how primes split in $L$ and $M$, i.e. we show $\Spl(M/K)\approx \Spl(L/K)$ and use Theorem~??.\ref{split-chebotarev}. First we show $\Spl(M/K)\stackrel{\supset}{\sim} \Spl(L/K)$. 
Take $\mfp\subeq \Spl(L/\Q)$. The idea is to use Kronecker's congruence: We know that we have 
\begin{equation}\label{fermat-iff-p}
a^p\equiv a\pmod{p}\text{ for every }a\in \F\iff  \F=\F_p.
\end{equation}
When we have $X$, $Y$ equal to values of $j$ in a field extension $M/K$ and $\Phi_p(X,Y)=0$, then this congruence gives us information about the residue field of $M$. %We can use this to show that the residue field $K(j(\mfp))$ 
We will find that it equals $\F_p$, so $M/K$ is unramified, giving that $\mfp$ splits completely in $L$.

By~(\ref{spl-is-norm}), for all but finitely many $p\in \Spl(L/\Q)$, $p=N(\al)$ for some $\al\in \sO$. As in~(\ref{phi-ja}), we get $0=\Phi_p(j(\ma),j(\ma))$. 
By Kronecker's congruence, $0=-(j(\ma)^p-j(\ma))^2\pmod{p}$, so
\begin{equation}\label{kronecker-ja}
j(\ma)^p\equiv j(\ma)\pmod{p};
\end{equation}
{\it a fortiori} this holds modulo $\mP$.

Next note $\sO_K[j(\ma)]$ has finite index in $\sO_M$, because the fact that $M=K(j(\ma))$ gives it is a full lattice in $\sO_M$ (considering them as $\Z$-modules).

Now assume $p\nmid [\sO_M:\sO_K[j(\ma)]]$; we will show that~(\ref{kronecker-ja}) implies the congruence
\begin{equation}\label{alpa}
\al^p\equiv \al\pmod{\mP}
\end{equation}
for $\al\in \sO_M$. First, take $\mfp=\mP\cap K$, and note that $p\in \Spl(M/\Q)$ implies that the residue degree of $\mP$ is $p$, and hence $\al^p\equiv \al\pmod{\mfp}$ and {\it a fortiori} modulo $\mP$ for $\al\in \sO_K$. So~(\ref{alpa}) holds for $\al\in \sO[j(\ma)]$. Now for arbitrary $\al\in \sO_M$, letting $N=[\sO_M:\sO_K[j(\ma)]]$ we know 
\begin{align*}
(N\al)^p&\equiv N\al\pmod{\mP};\\
\text{in particular, }N^p&\equiv N\pmod{\mP};
\end{align*}
But $p\nmid N$ means $N$ is invertible in $m:=\sO_M/\mP$, so dividing these two equations gives the desired answer.

Now by~(\ref{fermat-iff-p}),~(\ref{alpa}) gives that $|m|=p$, i.e. $f(\mP/p)=1$ and $\mfp\in \Spl(M/\Q)$.

From this step we obtain $M\subeq L$.\\

\noindent\underline{Step 3:}
Next we show $\wt{\Spl}(M/\Q)\stackrel{\sub}{\sim} \Spl(L/Q)$. Take $p\in \wt{\Spl}(M/\Q)$; assume $p$ unramified in $M$ and relatively prime to
\[
\De=\prod_{\{\ma,\mb\}\in C_K}(j(\ma)-j(\mb)).
\]
(Note this is in $\sO_L$ by step 2.)
Using the description of $\Spl(L/\Q)$ given in step 1, it suffices to show $p=N(\al)$ for some $\al$. 

We have $f(\mP/p)=1$ for some $\mP$ in $M$ above $p$. Let $\mP'$ lie above $\mP$ in $L$. Let $\mfp=\mP\cap \sO_K$; we see $f(\mfp/p)=1$ so $(p)$ splits as $\mfp\ol{\mfp}$ in $K$ and $\N\mfp=p$. Hence $\mfp\ma$ is cyclic in $\ma$. Theorem~(\ref{cyclic-roots-phi}) and Kronecker's congruence give
\[
0\equiv \Phi_p(j(\mfp\ma),j(\ma))\equiv (j(\ma)-j(\mfp\ma)^p)(j(\mfp\ma)^p-j(\ma))\pmod{p};
\]
this holds modulo $\mP'$ as well. 
Hence we have
\[
j(\ma)\equiv j(\mfp\ma)^p\pmod{\mP'}\qquad\text{ or }\qquad
j(\mfp\ma)^p\equiv j(\ma)\pmod{\mP'}.
\]
%Since $j(\ma)\equiv j(\ma)^p\pmod{\mP}$ from~(\ref{kronecker-jma}), we get $j(\ma)\equiv j(\mfp\ma)\pmod{\mP}$. (In the first case we can take $p$th roots because $p\perp |\sO_L/\mP'|$.) By assumption $f(\mP'/p)=1$, giving %no that was the converse, silly
By assumption, $f(\mP/\mfp)=1$, so $\sO_L/\mP\cong \F_p$ and $j(\ma)^p\equiv j(\ma)\pmod{\mP'}$. Together with the above we find that\footnote{In the first case we can take $p$th roots because $p\perp |\sO_L/\mP'|$.}
\[
j(\mfp\ma)\equiv j(\ma)\pmod{\mP'}.
\]
If $\ma,\mfp\ma$ are in distinct ideal classes, then $\mP'\mid j(\mfp\ma)-j(\ma)\mid \De$, contradicting the fact that $p$ and $\De$ are relatively prime. Thus they are in the same ideal class, and $\mfp=(\al)$ is a principal ideal. This means $p=\N\al$ is in~\ref{spl-is-norm}, as needed.

Combining steps 2 and 3 gives $L=M$.
\end{proof}
\section{Hecke operators}
Hecke operators give a map on modular forms. We first define their action on lattices.
\begin{df}
Let $\cal L$ denote the set of full lattices in $\C$, and $\cal K=\Z^{\opl \cal L}$ denote the free abelian group generated by the elements of $\cal L$. Define the \textbf{Hecke operator} on $\cal K$ by setting
\[
T(n)[\La]=\sum_{\La'\in \cal L,\,[\La:\La']=n}[\La']
\]
and extending linearly.
\end{df}
The sum is finite because any $\La'$ in the sum must contain $n\La$, and $\La/n\La$ is finite. We may think of modular forms as functions on lattices $f(z)=F((1,\tau))$, hence $T(n)$ induces a map on the space of modular forms of dimension $k$, $M_k$:
\[
T(n)\cdot f(\tau)=n^{k-1}F(T(n)\Ga(1,\tau)).
\]
Note the constant $n^{k-1}$ is just to make formulas nicer.

\begin{pr}
$T(n)$ is a map $M_k\to M_k$, and restricts to a map on cusp forms $S_k\to S_k$.
\end{pr}
\begin{proof}
Let $A=\matt abcd\in \SL_2(\Z)$. We have
\begin{align*}
T(n)\cdot f(A\tau)&=n^{k-1}F(T(n)\Ga(A\tau,1))\\
&=n^{k-1}F[T(n)(c\tau+d)^{-1}\Ga(a\tau+b,c\tau+d)]\\
&=n^{k-1}(c\tau+d)^{-k}F[T(n)\Ga(a\tau+b,c\tau+d)]&F\text{ homogeneous},\\
&=(c\tau+d)^{-k}n^{k-1}F[T(n) \Ga(\tau,1)]&(\tau, 1)=(a\tau+b,c\tau+d)\\
&=(c\tau+d)^{-k} T(n)\cdot f(\tau).
\end{align*}
\end{proof}
In the following subsections, we prove several key properties of the Hecke operator, and the Hecke algebra (the algebra generated by the $T(n)$).
\begin{itemize}
\item
The operators $T(n)$ are multiplicative.
\item
The Hecke algebra is commutative.
\item
The Hecke operators (on modular forms) are self-adjoint with respect to the Petersson inner product.
\end{itemize}
We will prove the last two items more generally, for a generalization of the Hecke operators, $T_{\al}$ where $\al$ is a {\it matrix}. We will then compute the explicit action of $T(n)$ on the Fourier coefficients of modular forms. The main application of Hecke operators is that we can diagonalize $M_k$ with respect to the Hecke algebra; thus we can speak of {\it eigenfunctions} in $M_k$. Using the multiplicativity of $T(n)$, we how that the coefficients of these eigenfunctions are multiplicative.
%In particular, many functions we are familiar with will be eigh
\subsection{Hecke operators on lattices}
\begin{df}
Define $R(n):\cal K\to \cal K$ by
\[
R(n)[\La]=[n\La].
\]
\end{df}
\begin{thm}[Multiplicativity of Hecke operators, I]
For any $m,n$,
\[
T(m)T(n)=\sum_{d\mid \gcd(m,n),\,d>0} dR(d)T\pf{mn}{d^2}.
\]
In particular, the following hold.
\begin{enumerate}
\item
If $m\perp n$, then
\[
T(m)T(n)=T(mn)
\]
\item If $p$ is prime and $r\ge 1$ then
\[
T(p^r)T(p)=T(p^{r+1})+pR(p)T(p^{r-1}).
\]
\end{enumerate}
\end{thm}
Translating these properties to modular forms we get the following.
\begin{thm}[Multiplicativity of Hecke operators, II]
For any $m,n$,
\[
T(m)T(n)f=\sum_{d\mid \gcd(m,n),\,d>0} d^{k-1} T\pf{mn}{d^2}f.
\]
In particular, the following hold.
\begin{enumerate}
\item
If $m\perp n$ then 
\[
T(m)T(n)f=T(mn)f.
\]
\item 
If $p$ is prime and $r\ge 1$,
\[
T(p)T(p^r)=T(p^{r+1})f+p^{k-1}T(p^{r-1})f.
\]
\end{enumerate}
\end{thm}
\section{Simultaneous Eigenforms}
\begin{df}
A \textbf{simultaneous eigenform} is a modular form $f$ that is an eigenfunction for every Hecke operator $T_n$. 
%We let $\la(n)$ denote the eigenvalue corresponding to $T_n$.
\end{df}
Write
\[
f(\tau)=\sum_{m\ge 0} c(m)q^m.
\]
We know that
\[
(T_nf)(\tau)=\sum_{m\ge 0} \ga_n(m)q^m
\]
where
\[
\ga_n(m)=\sum_{d\mid \gcd(m,n)} d^{k-1} c\pf{mn}{d^2}.
\]
%If $f$ is an eigenfunction of $T_n$ with eigenvalue $\la(n)$ then we have
%Compare
To find properties/criteria for eigenfunctions $f$, we compare:
\begin{align}
f(\tau)&=c(0)+c(1)q+\cdots\label{ef1}\\
%\la(n)f(\tau)=
(T_nf)(\tau)&=\si_{k-1}(n)c(0)+c(n)q+\cdots.\label{ef2}
\end{align}

First, we consider the nonvanishing of $c(1)$. Keep the above notation.
\begin{thm}[Apostol, 6.14]
Suppose $k\ge 4$ is even, and $f\in M_k$ is a simultaneous eigenform. Then
\[
c(1)\neq 0.
\]
\end{thm}
\begin{proof}
Let $\la(n)$ denote the eigenvalue corresponding to $f$ for $T_n$. 
From~(\ref{ef1}) and~(\ref{ef2}) we get
\[
c(n)=\la(n)c(1).
\]
If $c(1)=0$ then $c(n)=0$ for all $n$, so $f$ is a constant, contradiction.
\end{proof}

The previous theorem allows us to normalize a simultaneous eigenform so $c(1)=1$.
\begin{thm}[Simultaneous eigenforms have multiplicative coefficients]\label{multthm}
Suppose 
\[f(\tau)=\sum_{n\ge 1} c(n)q^n\in S_k\]
with $k\ge 12$ even. Then the following are equivalent.
\begin{enumerate}
\item
$f$ is a simultaneous normalized eigenform.
\item For all $m\ge n$,
\[
c(m)c(n)=\sum_{d\mid \gcd(m,n)} d^{k-1} c\pf{mn}{d}.
\]
\end{enumerate}
Moreover, \[\la(n)=c(n).\]
\end{thm}
\begin{proof}
Again from~(\ref{ef1}) and~(\ref{ef2}), if $f$ is a simultaneous eigenform we have
\[
\la(n)=c(n).
\]
Now $\la(n)f(\tau)=(T_nf)(\tau)$  is equivalent to
\[
c(n)c(m)=\la(n)c(m)=\ga_n(m)=\sum_{d\mid\gcd(m,n)} d^{k-1} c\pf{mn}{d}.
\]
for all $m,n\ge 1$.
\end{proof}
\subsection{Examples}
We can use Theorem~\ref{multthm} to conclude the multiplicativity of the coefficients $\tau(n)$ of $\Delta$.
\begin{cor}
Write $\De(\tau)=\su \tau(n)q^n$. Then
\[
\tau(m)\tau(n)=\sum_{d\mid \gcd(m,n)} d^{11} \tau\pf{mn}{d^2}.
\]
In particular,
\begin{align*}
\tau(mn)&=\tau(m)\tau(n)&\text{when }m\perp n\\
\tau(p^{n+1})&=\tau(p^n)\tau(p)-p^{11}\tau(p^{n-1}).
\end{align*}
\end{cor}
\begin{thm}[Noncuspidal eigenforms]
The only normalized simultaneous eigenform in $M_{2k}-S_{2k}$ is $\frac{-B_{2k}}{4k}E_{2k}$.
\end{thm}
\begin{proof}
The fact that $\frac{-B_{2k}}{4k}E_{2k}$ is a normalized simultaneous eigenform follows from Theorem~(\ref{multthm}). (The conditions there hold by simple calculation.)

Suppose $f(\tau)=\sum_{m\ge 0} c(m)q^m$ is a normalized simultaneous eigenform. 
Use~(\ref{ef1}) and~(\ref{ef2}) to match coefficients in $\la(n)f(\tau)=(T_nf)(\tau)$. We get 
\begin{align*}
\la(n)\cancel{c(0)}&=\si_{k-1}(n)\cancel{c(0)}\\%\cancelto{c(0)}{1}%
\la(n)c(1)&=c(n)%=\pf{2k}{B_k}^2
\end{align*}
So the only possibility is $\la(n)=\si_{k-1}(n)$, and this completely determines all the $c(n)$ by the second equation above. (Then only one value of $c(0)$ will work.)
\end{proof}
\section{Existence}
\begin{thm}
There exists a basis of simultaneous eigenforms for $M_{2k}$.
\end{thm}
\begin{proof}
Since we already have a simultaneous eigenform in $M_{2k}-S_{2k}$ and $\dim(M_{2k})-\dim(S_{2k})=1$, it suffices to show that there is a basis of simulatenous eigenforms for $S_{2k}$.

We proceed in three steps.
\begin{enumerate}
\item
Define the \textbf{Petersson inner product} on $S_{2k}$ by
\[
\an{f,g}=\int_{R_{\Ga}} f(\tau)\overline{g(\tau)} y^k\frac{dxdy}{y^2}.
\]
(Here $\tau=x+yi$.) It's clear that this is positive definite.
Note the following:
\begin{enumerate}
\item
$\frac{dxdy}{y^2}$ is the Haar measure with respect to $\SL_2(\Z)$ (it is invariant under the action of $\SL_2(\Z)$).
\item
$f(\tau)\overline{g(\tau)} y^k$ is invariant under transformation by $\SL_2(\Z)$: Using
\[
\Im(A\tau)=\frac{\Im(\tau)}{|c\tau+d|^2}
\] 
we get 
\[f(A\tau)\overline{g(A\tau)}(\Im A\tau)^k=f(\tau)(c\tau+d)^{-k} g(\tau)\overline{(c\tau+d)^{-k}}\frac{y^k}{|c\tau+d|^{2k}}=
f(\tau)\overline{g(\tau)}y^k.\]
\item The integral converges. Since $f$ is cuspidal, $f(\tau)=O(e^{-|\tau|})=O(e^{-y})$. Thus the integral is dominated by
\[
\int_{-\rc 2}^{\rc2}\int_c^{\iy} Ce^{-y}{y^{k-2}}\,dx\,dy<\iy.
\]
\end{enumerate}
\item The Hecke operators $T_n$ are self-adjoint under this inner product, i.e.
\[
\an{T_nf,g}=\an{f,T_ng}.
\]
(See pg. 82-86 of Brubaker's notes~\url{http://math.mit.edu/~brubaker/785notes.pdf}.)
\item We use the following linear algebra theorems.
\begin{thm}[Spectral theorem]
A self-adjoint linear operator on a finite-dimensional $\C$-vector space has an orthogonal basis of eigenvectors (so is diagonalizable).
\end{thm}
\begin{thm}
Let $\cal F$ be a commuting family of diagonalizable linear operators on a finite-dimensional vector space. Then $\cal F$ is simultaneously diagonalizable.
\end{thm}
Since the Hecke operators commute, the two theorems, combined with item 2, give the desired result.
\end{enumerate}
\end{proof}
%\section{Partition congruences}

%not integrated yet
\begin{comment}
\section{Derivatives of modular forms}
Let $f$ be a modular form of weight $k$. Is $f'$ (derivative with respect to $\tau$) a modular form? Differentiating the transformation law gives
\begin{align}
\nonumber f\pf{a\tau+b}{c\tau+d}&=(c\tau+d)^k f(\tau)\\
\nonumber f'\pf{a\tau+b}{c\tau+d}(c\tau+d)^{-2}&=k(c\tau+d)^{k-1} cf(\tau)+(c\tau+d)^k f'(\tau)\\
f'\pf{a\tau+b}{c\tau+d}&=\underbrace{k(c\tau+d)^{k+1} cf(\tau)}_{\text{Uh-oh.}}+(c\tau+d)^{k+2} f'(\tau).\label{fdifftrans}
\end{align}
Unfortunately, $f'$ isn't quite modular. So we need to construct a modified notion of derivative (which we'll call $\theta$) that takes $M_k$ to $M_{k+2}$. To do this, we will use the derivative and the $P$ function, defined below in terms of the $\eta$ function. First, we need to the the transformation properties in Theorem~\ref{eta-transforms}.
\begin{df}
Define
%\begin{align*}
\[P(\tau)=\frac{24}{2\pi i} \frac{\eta'(\tau)}{\eta(\tau)}.\]
%E_2(\tau)&=1-\underbrace{\frac{4}{B_2}}_{24}\su \si_1(n)q^n.
%\end{align*}
\end{df}
\begin{thm}$\,$
\begin{enumerate}
\item
$P=E_2$, i.e.
\[
P=1-\underbrace{\frac{4}{B_2}}_{24}\suo \si_1(n)q^n.
\]
\item
$P$ satisfies the transformation law
\begin{equation}\label{Ptrans}
P(\ga \tau)=(c\tau+d)^2P(\tau)+\underbrace{\frac{12c}{2\pi i}(c\tau+d)}_{\text{``nonmodular" part}}.
\end{equation}
\end{enumerate}
\end{thm}
\begin{proof}
For item 1, note that $\frac{d}{d\tau}=2\pi i q\frac{d}{dq}$ by the chain rule so
\begin{align*}
\frac{d}{d\tau}\ln \eta(\tau)&=2\pi i q\pa{\su \frac{d}{dq}\ln(1-q^n)+\frac{d}{dq} \ln q^{\rc{24}}
}\\
\frac{\eta'(\tau)}{\eta(\tau)}&=2\pi i\pa{\su \frac{nq^n}{1-q^n}+\rc{24}}\\
&=2\pi i\pa{\su \sum_{m>0,n|m}q^{m}+\rc{24}}\\
&=2\pi i\pa{\sum_{m\ge 1}\si_1(m)q^m+\rc{24}}.
\end{align*}

For item 2, note $\an{S,T}=\GL_2(\Z)$, so $\ga$ can be written as a product of $S=\matt0{-1}10,T=\matt1101,T^{-1}=\matt1101$. The base case is trivial.
For the induction step, first differentiate the transformation laws for $\eta$ to get
\begin{align*}
\rc{\tau^2}\eta'(S\tau)&=\frac{\tau^{-\rc2}}{2\sqrt i}\eta(\tau)+\frac{\tau^{\rc2}}{\sqrt i} \eta'(\tau)\\
\eta'(T\tau)&=e^{\frac{2\pi i}{24}}\eta(\tau).
\end{align*}
Using this we can calculate how $\frac{24}{2\pi i}\frac{\eta'}{\eta}$ transforms under $\eta$. The induction step comes from checking that if $\ga=\matt abcd$ then
\begin{align*}
P(S\ga\tau)&=(a\tau+b)^2P(\tau)+\frac{12a}{2\pi i}(a\tau+b)\\
P(T^{\pm 1}\ga\tau)&=P(\ga\tau).
\end{align*}
\end{proof}

Now we are ready to define our differential operator. 
\begin{df}
For $f$ a weight $k$ modular form, define
\[
\partial_k(f)=(12\theta-kP)f
\]
where
\[
\theta=q\frac{d}{dq}=\rc{2\pi i}\frac{d}{d\tau}.
\]
\end{df}

\begin{thm}$\,$
\begin{enumerate}
\item
$\partial_k$ is a map from $M_k$ to $M_{k+2}$.
\item
$\partial$ is a derivation, i.e. for $f\in M_m,g\in M_n$, we have
\[
\partial_{m+n}(fg)=(\partial_m f)g+f(\partial_n g).
\]
\item The following hold ($P=E_2,Q=E_4,R=E_6$):
\begin{align*}
\partial_2P&=-Q&\theta P&=\rc{12}(P^2-Q)\\
\partial_4Q&=-4R&\theta Q&=\rc{3}(PQ-R)\\
\partial_6R&=-6Q^2&\theta R&=\rc{2}(PR-Q^2).
\end{align*}
\end{enumerate}
\end{thm}
\begin{proof}
For part 1, calculate $(\partial f)(A\tau)$ using~(\ref{fdifftrans}) and~(\ref{Ptrans}).

For part 2,
\[
\partial_{m+n}(fg)=\rc{2\pi i} (fg)'-(m+n)Pfg=\rc{2\pi i}f'g-m(Pf)g+\rc{2\pi i}fg'-nf(Pg)=(\partial_mf)g+f(\partial_n g).
\]

For part 3, 
more calculations show that $\partial_2P+P^2$ is a modular form. The equalities follow from using $\dim(M_4)=\dim(M_6)=\dim(M_8)=1$ and matching constant terms of the $q$-series.
\end{proof}
\begin{rem}
Since $Q,R$ generate the space of modular forms, this completely describes the action of $\partial$ on modular forms. The fact that it is a derivation means that we can calculate its action on a polynomial in $P,Q,R$ as if it were actually a derivative, taking note what $\partial_2P,\partial_4Q,\partial_6R$ are. This is since for polynomials, stuff like the chain rule can be derived from the product rule, which we have.
\end{rem}
\end{comment}
%1- theta and elliptic functions
%2- modular forms on SL_2(Z)
%3- modular forms on congruence subgroups
%4- modular forms of half-integral weight
%5- modular forms modulo p
%6- more general automorphic forms (Fuschian groups, all that good 18785 stuff), harmonic Maass forms
\chapter{Complex multiplication}\llabel{ch:CM}
\index{complex multiplication}
In this chapter, we combine class field theory with the theory of elliptic curves, first to characterize the maximal abelian extension of $K$, then to illustrate the relationships in Section~\ref{ch:cft-app}.\ref{sec:intro-langlands} for CM elliptic curves. We will assume basic facts about elliptic curves (for an introduction see Silverman~\cite[Chapter III]{Si86}).

We know that every elliptic curve over $\C$ has endomorphism ring either equal to $\Z$ or a quadratic order. In the second case, the elliptic curve is said to have \textbf{complex multiplication}. This gives the elliptic curve a lot more structure. On one hand, it is useful algebraically---as we will see, torsion points of a CM elliptic curve give abelian extensions of imaginary quadratic fields. In general, because of the added structure, much more is known about CM elliptic curves than other elliptic curves, and they can act as a kind of ``testing ground" or ``first case" of general conjectures.

On the other hand, CM elliptic curves have practical uses---for instance, if we take an CM elliptic curve corresponding to a specific endomorphism ring, we can easily compute its order. Hence we can generate an elliptic curve with near-prime order, useful in cryptography. This is much more efficient than generating random elliptic curves and using Schoof's algorithm to find their orders. 

There are several big theorems about complex multiplication. In Section~\ref{sec:cm-C}, we specialize our knowledge about the relationship between elliptic curves over $\C$ and complex tori to CM elliptic curves and build a toolbox of basic facts. However, since we are interested in number theory, we want to take curves defined over $\C$ and define them over $\ol{\Q}$ instead---which we do in Section~\ref{sec:cm-Q}. Once we have these basics, we can then prove the big theorems.

We suppose $E$ has CM by a quadratic order $\sO\sub K$ (i.e. $\End(E)\cong \sO$), where $K$ is a quadratic extension of $\Q$. Then the following hold.
\begin{enumerate}
\item
The $j$-invariant $j(E)$ generates the {\it ring class field} of $\sO$ over $K$. In particular, if $\sO=\sO_K$, then $j(E)$ generates the {\it Hilbert class field} of $K$, the maximal unramified abelian extension (Theorem~\ref{thm:j-generates-hilbert}):
\[
K(j(E))=H_K.
\]
\item
If $E$ is defined over $H_K$, and we adjoin certain functions of torsion points of $E$, then we get the {\it maximal abelian extension} of $K$ (Theorem~\ref{thm:max-abe-ext-K}):
\[
K(j(E),h(E\tors))=K\abe.
\]
Compare this with the Kronecker-Weber Theorem, which says the maximal abelian extension of $\Q$ is generated by roots of unity (torsion points of $\ol{\Q}^{\times}$).
\item
$j(E)$ is moreover an {\it algebraic integer} (We omit this; see Silverman AT,~\cite[II.6]{Si94}.) %(Theorem~\ref{thm:j-alg-int}). 
\item
The action of the idele class group sending $K/\ma$ to $K/\mx^{-1}\ma$ corresponds to the Galois action on the corresponding elliptic curves, where the Galois action is given by the Frobenius element of $\si$. This is the Main Theorem of Complex Multiplication~\ref{thm:mt-cm}, and plays an important part in taking moduli spaces initially defined only over $\C$ and defining them over algebraic number fields.
%There is an {\it analytic} isomorphism between $K/\ma$ (a ``lattice" in $K$) and the corresponding elliptic curve
\item
The $L$-series of a CM elliptic curve is particularly easy to understand, because it is a product of 2 Hecke $L$-series (Theorem~\ref{thm:cmec-l}). 
%(I'm not sure I like this "reason.") This is because the associated Galois representation is abelian so ``decomposes" into two abelian subrepresentations.
\end{enumerate}

Two ``big ideas" we'll consistently see are the following.
\begin{enumerate}
\item We expect abelian extensions because for CM elliptic curves (with endomorphism ring $\sO_K$, say), the image of the map $G(L/H_K)\hra \Aut(E[m])$ commutes with $\sO_K$, not just $\Z$ and hence must be abelian, with appropriate $L$.
\item We can use torsion points $E[m]$ to ``keep book" on the action of Frobenius, in the same way that we used the roots of unity $\mu_m$ to keep book on the action of Frobenius on $G(\Q(\mu_m)/\Q)$.
\end{enumerate}
\section{Elliptic curves over $\C$}
The following theorem helps us understand elliptic curves over $\C$.
\begin{thm}\llabel{thm:lattice-ec-eoc}
Let $g_2(\La)=60G_4(\La)$ and $g_3(\La)=140G_6(\La)$, where $G_n$ is the Eisenstein series.
Let $\La$ be a lattice in $\C$ and $\wp$ be the associated Weierstrass $\wp$-function.

There is a complex analytic isomorphism between the complex torus $\C/\La$ and the elliptic curve over $\C$,
\[
y^2=4x^3-g_2(\La)x-g_3(\La)
\]
given by
\[
\Phi(z)=(\wp(z),\wp'(z)).
\]

The map $\Phi$ gives an equivalence of categories between the following.
\begin{enumerate}
\item
Objects: Complex tori $\C/\La$, where $\La$ is a lattice in $\C$.\\
Maps: Multiplication-by-$\al$ $\C/\La_1\to \C/\La_2$ where $\al\La_1\subeq \La_2$.
\item
Objects: Elliptic curves over $\C$.\\
Maps: Isogenies.
\end{enumerate}
\end{thm}
\begin{proof}
Silverman~\cite[VI.5.1.1, 5.3]{Si86}
\end{proof}
The endomorphism ring of a lattice $\La\sub \C$ is either $\Z$ or an imaginary quadratic order, so the same is true of an elliptic curve $E$ over $\C$. If the endomorphism ring is a quadratic order $\sO$, we say $E$ has \textbf{complex multiplication} by $\sO$.
\section{Complex multiplication over $\C$}\llabel{sec:cm-C}
%\circlearrowright=\cir
%We've associated each elliptic curve with a lattice, and shown that that isogenies correspond to maps on the lattice. 
\subsection{Embedding the endomorphism ring}
We know the endomorphism ring $\End(E)$ of a CM elliptic curve corresponds to a quadratic order $\sO$ but since any quadratic order has conjugation as an isomorphism, we need to specify a way to embed $\End(E)$ into $\C$. 
\begin{ex}\llabel{ex:which-i}
Consider the curve $E:y^2=x^3+x$. We note that the endomorphisms
\begin{align*}
\phi_1(x,y)&=(-x,iy)\\
\phi_2(x,y)&=(-x,-iy)
\end{align*}
both square to $-1$. Which one should we call $[i]$, multiplication by $i$?
\end{ex}
Fortunately, we have a way of embedding $\End(\La)$ into $\C$, where $\La$ is the lattice corresponding to $E$, because $\La$ itself is in $\C$. This to give a canonical way of embedding $\End(E)$ into $\C$.

%Restrict our attention to elliptic curves with specified endomorphism ring congruent to the quadratic order $\sO$. There are two possible isomorphisms $R\cong \End(E)$; we need to fix one so that we get commutativity. This we do with the invariant differential.
\begin{pr}\llabel{pr:normalize-cmec}
Let $E/\C$ be a CM elliptic curve with complex multiplication by $\sO$. There is a unique isomorphism $[\cdot]:\sO\xra{\cong}\End(E)$ 
satisfying either of the following equivalent conditions.
\begin{enumerate}
\item
$[\al]$ is the unique morphism making the following diagram commute, where the top map is multiplication by $\al$.
\[\xymatrix{
\C/\La\ar[r]^{m_{\al}}\ar[d]^{\Phi} & \C/\La\ar[d]^{\Phi}\\
E_{\La}\ar[r]^{[\al]}& E_{\La}
}\]
\item For any invariant differential $\om\in \Om_E$, $[\al]^*\om=\al\om$.
\end{enumerate}
Moreover, we have the following.
\begin{enumerate}
\item[3.]
Define $[\cdot]_1$ and $[\cdot]_2$ for elliptic curves $E_1$ and $E_2$. For any morphism $\phi:E_1\to E_2$,
\[
\phi\circ [\al]_{1}=[\al]_{2}\circ \phi.
\]
In other words, multiplication by $\al$ commutes with all morphisms.
\item[4.] For any $\si\in \Aut(\C)$,
\[
[\al]_E^{\si}=[\si(\al)]_{\si(E)},
\]
i.e. it commutes with Galois action.
\end{enumerate}
\end{pr}
The pair $(E,[\cdot])$ is called a \textbf{normalized} elliptic curve. After we prove this proposition, we will assume all CM elliptic curves are normalized.
\begin{proof}
The uniqueness and existence of $[\al]$ satisfying item 1 follows directly from the equivalence of categories (Theorem~\ref{thm:lattice-ec-eoc}).

Define $[\al]$ as in item 1.
%By Proposition REF, the space of invariant differentials is 1-dimensional. 
For any invariant differential $\om$ on $E_{\La}$, since $\Phi$ is an analytic isomorphism, we can consider its pullback to $\C/\La$; it will be $c\,dz$ for some $c$ (The space of invariant differentials on $\C/\La$ is 1-dimensional.) Clearly, $m_{\al}^*(c\,dz)=c\,d(\al z)=\al c\,dz$. Transferring this to the bottom row of the commutative diagram gives $[\al]^*\om=\al\om$. 
For uniqueness, note the map
\begin{align}
\Hom(E_1,E_2)&\hra \Hom(\Om_{E_2},\Om_{E_1})\llabel{eq:ec-diff-inj}
\\
\nonumber\phi&\to \phi^*
\end{align}
is injective when all isogenies $E_1\to E_2$ are separable (in particular, in characteristic 0), i.e. the action of an isogeny of elliptic curves on an invariant differential completely determines the morphism. Taking $E_1=E_2$ and considering the preimage of multiplication-by-$\al$ gives uniqueness in item 2.

A simple diagram chase shows that $(\phi \circ [\al]_1)^*$ and $([\al]_2\circ \phi)^*$ act the same way on $\om\in \Om_{E_2}$. Then~(\ref{eq:ec-diff-inj}) gives item 3.

The proof of item 4 is similar.
\end{proof}
\begin{ex}
The definition using differentials is useful for calculations. 
Revisiting the above Example~\ref{ex:which-i}, we see that we should let
\begin{align*}
[i](x,y)&=(-x,iy).
\end{align*}
Indeed, defining $[i]$ in this way, we check that
\[
[i]^*\fc{dx}y=\fc{d(-x)}{iy}=i\fc{dx}y.
\]
\end{ex}
\subsection{The class group parameterizes elliptic curves}
Let $K$ be an imaginary quadratic field and $\sO$ an order inside $K$.
\begin{df}
Let $L$ be a field. Define
\begin{align*}
\text{Ell}_L(\sO)&=\{\text{elliptic curves $E/L$ with $\End(E)\cong \sO$}\}\\
\Ell_{L}(\sO)&=\fc{\{\text{elliptic curves $E/L$ with $\End(E)\cong \sO$}\}}{\text{isomorphism over }L},
\end{align*}
i.e. $\Ell_L(\sO)$ is the set of elliptic curves over $L$ whose endomorphism ring is $\sO$. If we omit $L$, we assume $L=\C$.
\end{df}
If $E\in \text{Ell}(\sO)$, then its corresponding lattice $\La$ must be homothetic to a fractional ideal of $\sO$: indeed, we can scale the lattice so that $1\in \La$; then $\sO\subeq \La$ so $\La\subeq K$; since it is a lattice it must be a fractional $\sO$-ideal. %We may assume
%, so correspond to an ideal of $\sO$. 
Now note an $\sO$-ideal $\ma$ has endomorphism ring $\sO$ iff $\ma$ is a {\it proper} ideal (see Definition~\ref{quadratic-forms}.\ref{df:proper-ideal}).\footnote{When $R=\sO_K$, all ideals are proper, so this distinction is not important. The reader unfamiliar with non-maximal orders can take $R=\sO_K$ throughout.} Hence we get a correspondence between isomorphism classes of elliptic curves $[E]\in \Ell(\sO)$ and proper $\sO$-ideals up to homothety.
%Two elliptic curves are isomorphic iff they correspond to homothetic fractional ideals, and 
However, two fractional ideals $\ma$ and $\mb$ are homothetic iff $\la\ma=\mb$ for some $\la$, i.e. iff they are equivalent in the class group. Thus the class group of $\sO$ parameterizes all isomorphism classes of elliptic curves with endomorphism ring $\sO$. This is summarized in the following.
\[
\Ell(\sO)=\fc{\{\text{elliptic curves $E/\C$ with $\End(E)\cong \sO$}\}}{\text{isomorphism over }\C}
=\fc{\{\text{proper fractional $\sO$-ideal}\}}{\text{principal $\sO$-ideals}}=\Cl(\sO).
\]
We state this as a theorem.
\begin{thm}\llabel{thm:ell=cl}
We have a bijection
\[
\Ell(\sO)\cong \Cl(\sO)
\]
where $[E]\in \Ell(\sO)$ is sent to a $[\ma]$, where $\ma$ is a fractional ideal homothetic to the lattice corresponding to $E$. 
\end{thm}
We get much more than this, however. $\Ell(\sO)$ is a priori just a set; however, $\Cl(\sO)$ is a {\it group}. 
We can define the action of $I(\sO)$ on $\text{Ell}(\sO)$ since $I(\sO)$ acts on lattices. This action will descend to an action of $\Cl(\sO)$ on $\Ell(\sO)$, since isomorphic elliptic curves correspond to equivalent ideals.
%: because the class group acts naturally on itself, it acts on isomorphism classes of elliptic curves.
%The class group acts naturally on itself, it acts on isomorphism classes of elliptic curves.
%Define
%\[
%Ell(R)=\fc{\{\text{elliptic curves $E/\C$ with $\End(E)\cong R$}\}}{\text{isomorphism over }\C}.
%\]
%Silverman 2.1.2.
\begin{thm}
There is a group action of $\Id(\sO)$ on $\text{Ell}(\sO)$ given by
\[
\ma E_{\La}=E_{\ma^{-1}\La}
\]
where $E_{\La}$ denotes the elliptic curve corresponding to the lattice $\La$.

This descends to a simply transitive group action of $\Cl(\sO)$ on $\Ell(\sO)$.
\end{thm}
\begin{proof}
Just check that if $\La$ has endomorphism ring $\sO$, then so does the lattice $\ma^{-1}\La$. (Note that $\mb L$ is defined by $\set{s\al}{s\in \mb, \al\in L}$.)

For the second part, note that $E_{\La}\cong \ma E=E_{\ma^{-1}\La}$ iff $\La$ and $\ma^{-1}\La$ are homothetic, i.e. $\ma$ is principal.
\end{proof}
\index{torsor}
\begin{rem}
Another way of saying that $\Cl(\sO)$ acts simply transitively on $\Ell(\sO)$ is that $\Ell(\sO)$ is a \textbf{torsor} or \textbf{principal homogeneous space} for $\Cl(\sO)$.
\end{rem}
This action will be fundamental to our understanding of CM elliptic curves. Later on we will relate this to the Galois action. The interplay between these two actions is the source for much of the richness of CM theory.

\subsection{Ideals define maps}
For any $n\in \Z$ and any elliptic curve $E$, $n$ defines the multiplication by $n$ map $[n]:E\to E$. When $E$ has CM, we saw in Theorem~\ref{pr:normalize-cmec} that $\al\in \sO$ defines (canonically) the multiplication by $\al$ map $[\al]:E\to E$. We now extend this to {\it ideals}: if $\ma$ is a proper $\sO$-ideal, $\ma$ determines a ``multiplication by $\ma$" map. The only difference is that $[\ma]$ is now a map $E\to \ma E$.
\begin{df}
Let $E\in \text{Ell}(\sO)$ correspond to the lattice $\La$. Let $\ma$ be a proper integral ideal of $\sO$. We have $\ma R\subeq R$, so $\ma$ determines a map $\C/\La\to \C/\ma^{-1}\La$, sending $z\mapsto z$. 
Define the multiplication by $\ma$-map as the corresponding map on elliptic curves 
\[[\ma]:E\to E_{\ma^{-1}\La}=\ma E.\]
\end{df}
\begin{pr}\llabel{pr:E[a]}
%(Does this need to be over $\sO_K$?)
\fixme{Do we need $R=\sO_K$?}
Let $E\in \text{Ell}(\sO_K)$. We have the following.
\begin{enumerate}
\item
The kernel of $[\ma]$ (the ``$\ma$-torsion points") is
\[
E[\ma]:=\set{P\in E}{[\al]P=0\text{ for all }\al\in \ma}\cong \sO_K/\ma.
\]
\item
The degree of $[\ma]$ is
\[\deg([\ma])=|E[\ma]|=\fN(\ma),\]
and in particular, $\deg([\al])=|E[\al]|=\nm_{K/\Q}(\al)$.
\end{enumerate}
\end{pr}
\begin{proof}
Silverman AT~\cite[pg. 102-3]{Si94}.
\end{proof}
%Define the group of $\ma$-torsion points of $E$ by
%\[
%E[\ma]=\set{P\in E}{[\al]P=0\text{ for all }\al\in \ma}.
%\]
\section{Defining CM elliptic curves over $\ol{\Q}$}
\llabel{sec:cm-Q}
We show that we do not lose anything if we just consider elliptic curves over $\ol{\Q}$ instead of over $\C$. To do this, we look at the $j$-invariants.
%$\Ell_{\ol{\Q}}(R)$ instead
\begin{pr}\llabel{pr:j-alg}
Suppose $E$ is an elliptic curve with CM by a quadratic order $\sO$.
Then $j(E)\in \ol{\Q}$, i.e. $j(E)$ is algebraic.
\end{pr}
\begin{proof}
Let $\si$ be any automorphism of $\C$ over $\Q$. We look at how $\si$ acts on $j(E)$.

Note that $E^{\si}$ is defined by taking any equation for $E$ and operating on all the coefficients of $E$ by $\si$, so $\si(j(E))=j(E^{\si})$.

First note that $ \End(E)\cong \End(E^{\si})$ by the map $\phi\mapsto \phi^{\si}$. Hence $\End(\si(E))=\sO$ as well. But $\Cl(\sO)$ is finite, and as $|\Cl(\sO)|=|\Ell(\sO)|$ (Theorem~\ref{thm:ell=cl}) we see that the $E^{\si}$ lie in finitely many isomorphism classes. Because isomorphic elliptic curves have the same $j$-invariant, there are a finite number of possibilities for $j(E^{\si})$.

As $\set{\si(j(E))}{\si\in \Aut(\C)}$ is finite, $j(E)$ must be algebraic.
\end{proof}
This allows us to prove the following.
\begin{thm}\llabel{thm:ell-c-q}
We have
\[
\Ell_{\C}(\sO)\cong \Ell_{\ol{\Q}}(\sO).
\]
\end{thm}
\begin{proof}
We use the following properties of the $j$-invariant. (\cite[III.1.4]{Si86})
\begin{enumerate}
\item
For every $j\in K$, there exists an elliptic curve $E/K$ with $j(E)=j$.
\item
Let $K$ be an algebraically closed field and $E_1$, $E_2$ be elliptic curves defined over $K$. Then $E_1\cong E_2$ over $K$ iff $j(E_1)=j(E_2)$. (The backwards direction does not necessarily hold if $K$ is not algebraically closed.)
\end{enumerate}
We show that the map
\begin{equation}\llabel{eq:ellqc}
\Ell_{\ol{\Q}}(\sO)\to \Ell_{\C}(\sO)
\end{equation}
is an isomorphism (of sets, in fact, of $\Cl(\sO)$-modules). The map is well-defined, because any automorphism over $\ol{\Q}$ is an automorphism over $\C$.

By Lemma~\ref{pr:j-alg}, if $[E]\in \Ell_{\C}(\sO)$ then $j(E)\in \ol{\Q}$.
By item 1, there exists an elliptic curve $E'$ defined over $\ol{\Q}$ with $j(E')=j(E)$. Then $E'$ is isomorphic to $E$ over $\C$. Thus the map~(\ref{eq:ellqc}) above is surjective. It is injective because if $E,E'$ are defined over $\ol{\Q}$ and isomorphic over $\C$, then item 2 says $j(E)=j(E')$; and the other direction of item 2 says that $E\cong E'$ over $\ol{\Q}$.
\end{proof}
It is also important to know what fields we can define elliptic curves and isogenies over.
\begin{pr}
Suppose $E$ is an elliptic curve with CM by $\sO\sub K$, where $K$ is an imaginary quadratic field.
\begin{enumerate}
\item
If $E$ is defined over $L$ then endomorphisms of $E$ can be defined over $LK$.
\item
If $E_1,E_2$ are defined over $L$ then there exists a finite extension $M/L$, so that every isogeny $E_1\to E_2$ is defined over $M$.
\end{enumerate}
\end{pr}
\begin{proof}
For item 1, note that all endomorphisms are in the form $[\al]$ and use Proposition~\ref{pr:normalize-cmec}(4). 

For item 2, first we claim that any isogeny $\phi$ is defined over a finite extension of $L$. For any $\si\in \Aut(\C)$ fixing $L$, $\phi^{\si}$ is a map $E_1\to E_2$ having the same degree as $\phi$.
Any isogeny is determined by its kernel, up to automorphism of $E_1$ and $E_2$. As $E_1$ has a finite number of subgroups of given index and $\deg(\phi)=\ker(\phi)$, there are finitely many isogenies of a given degree. Hence $\set{\phi^{\si}}{\si \in G(\C/L)}$ is finite, showing $\phi$ is defined over a finite extension of $L$.

Now $\Hom(E_1,E_2)$ is a finitely generated group, so we can take the field of definition for a finite set of generators.
\end{proof}
\section{Hilbert class field}
\subsection{Motivation: Class field theory for $\Q(\ze_n)$ and Kronecker-Weber}\llabel{sec:kw-cm}
\subsubsection{The case of $\Q$}
First we give some motivation for the next two sections by making an analogy with class field theory for $\Q(\ze_n)$.
We can think of $\mu_n$, the $n$th roots of unity, as the analogue of $E[n]$: $\mu_n$ are the $n$-torsion points of the group variety $\ol{\Q}^{\times}$ under multiplication, and $E[n]$ are the $n$-torsion points of an elliptic curve. To emphasize this analogy, we write $K^{\times}[n]$ to denote the $n$th roots of unity in $\ol K$.

Recall how we established class field theory for $\Q(\ze_n)$: given a prime $p$, we want to find $(p,\Q(\ze_n)/\Q)$. To do this we looked at the action of $(p,\Q(\ze_n)/\Q)$ on $\Q^{\times}[n]=\mu_n$, by taking everything modulo $p$. We know by definition of $(p,\Q(\ze_n)/\Q)$ how it must act on the residue field extension $l/\F_p$ and hence on $\F_p^{\times}[n]$. Suppose $p\nmid n$. Because the maps
\begin{align}
\nonumber\Q^{\times}[n]&\hra \F_p^{\times}[n]\\
\End(\Q^{\times}[n])&\hra  \End(\F_p^{\times}[n])\llabel{eq:end-inj}
\end{align}
are injective (the first is because $p\nmid n$ and the second is a direct consequence of the first), once we know how $(p,\Q(\ze_n)/\Q)$ acts on $\F_p^{\times}[n]$, we know it acts on $\Q_p^{\times}[n]$, so we know exactly what automorphism it is:
\[
(p,\Q(\ze_n)/\Q)(\ze_n)=\ze_n^p.
\]
In particular, since $\ze_n$ is a $n$-torsion point (i.e. $\ze_n^n=1$) this only depends on $p\pmod n$. Hence we get the Artin map $\psi_{\Q(\ze_n)/\Q}$ factoring through the modulus $\iy n$:\footnote{The $\iy$ is a technical detail coming from the fact that $\Q$ is totally real.}
\[
\psi_{\Q(\ze_n)/\Q}:I_{\Q}/I_{\Q}(1,n\iy)\xrc G(\Q(\ze_n)/\Q).
\]
Finally, since every modulus divides $\iy n$ for some $n$, we get the Kronecker-Weber Theorem
\[
\Q\abe=\Q(\ze_{\iy})=\Q(\Q^{\times}[\iy]).
\]
%(Also mention how parameterized by exponential.)

In summary, we found the ray class groups and thus the maximal abelian extension by looking at how $(p,\Q(\ze_n)/\Q)$ acted on $\Q^{\times}[n]$:
\begin{equation}\llabel{thm:max-ab-Q}
\xymatrix{
&\Q^{\times}[n] \ar@{.>}[r]^{\wt{\bullet}}_{\text{reduction}}
\ar@{}[d]|-{\textstyle{\cir}} & \F_p^{\times}[n]\ar@{}[d]|-{\textstyle\cir}\\
I_{\Q}/P_{\Q}(1,n\iy)\ar[r]^{\psi_{\Q(\ze_n)/\Q}} & G(\Q(\ze_n)/\Q)\ar@{.>}[r]^{\wt{\bullet}} & G(\F_p(\ze_n)/\F_p).
}
\end{equation}
\subsubsection{The case of $K$}
One big difference when we're working over an imaginary quadratic field $K$ is that while we had $\Cl_{\Q}=1$, we have $\Cl_K$ is nontrivial in general. This corresponds to the fact that there is only 1 nonisomorphic ``version" of $\G_m(\Q)=\Q^{\times}$, but multiple elliptic curves with endomorphism ring by the same order $\sO$. Hence $G(K\abe /K)$ no longer operates on the same elliptic curve. Instead we have to analyze it in two steps.
\begin{enumerate}
\item
Consider the action of $G(H_K/K)$ on $\Ell_{\ol{\Q}}(\sO)$, i.e. equivalence classes of elliptic curves with CM by $\sO$.
\item
Consider the action of $G(K\abe/H_K)$ on the torsion points $E\tors$ of a single elliptic curve. 
\end{enumerate}
In both cases, we will understand the action by looking at how the Frobenius elements of the Galois groups act.

\subsubsection{The case of $K$: Part 1}
We have two natural actions on the set of elliptic curve $\Ell_{\ol{\Q}}(\sO_K)$, namely the action of $G(\ol K/K)$ and $\Cl(\sO_K)$. Our first task is to relate these, i.e. find a dotted map that preserves the action on $\Ell_{\ol{\Q}}(\sO_K)$:
\begin{equation}\llabel{eq:G-cl-compat}
\xymatrix{
&\Ell_{\ol{\Q}}(\sO_K)\ar@{}[ld]|-{\textstyle{\cir}}\ar@{}[rd]|-{\textstyle{\cir}}&\\
G(\ol K/K)\ar@{.>}[rr] &&\Cl(\sO_K).
}
\end{equation}
We'll see that this map factors through $G(L/K)$ where $L=K(j(E))$. 
We have a map $\psi_{L/K}:I_{K}^{\mf}/P_{K}(1,\mf)\to G(L/K)$; we show that $\mf=1$ and the composition of the two maps is an isomorphism, and that in fact we have
\beq{eq:G-cl-compat2}
\xymatrix{
&&\Ell_{\ol{\Q}}(\sO_K)\ar@{}[ld]|-{\textstyle{\cir}}\ar@{}[rd]|-{\textstyle{\cir}}&\\
I_K/P_K\ar[r]^{\Psi_{L/K}}\ar@/_1pc/@{.>}[rrr]_{\ma\mapsto [\ma]}&G(L/K)\ar[rr] &&\Cl(\sO_K).
}
\eeq
We establish~\eqref{eq:G-cl-compat2} by looking at the reduction of the elliptic curves modulo some $\mP$.

Since $G(H_K/K)\cong \Cl(\sO_K)$ this will show that $L=H_K$, the Hilbert class field of $K$.
\subsubsection{The case of $K$: Part 2}
We can now do the same thing we did with $\Q$, use the torsion points of elliptic curves to find the ray class fields and the maximal abelian extensions. We can't work directly over $K$ because $\Cl_K$ is nonzero, but if we imitate the argument (with some modifications) over $\Q$ for $H_K$ we will get the ray class fields of $K$. We let $L_n=K(j(E),h(E[n]))$ where $h$ is a Weber function (to be defined).

Let $l_n$, $l$ be the residue fields of $L_n$ and $H_K$ modulo some prime. 
We show $L_n$ is the ray class field for $(n)$ by constructing the diagram
%We are going to do the same thing for torsion points of elliptic curves and find that
\beq{eq:rcf-k}
\xymatrix{
&E[n] \ar@{.>}[r]^{\wt{\bullet}}_{\text{reduction}}
\ar@{}[d]|-{\textstyle{\cir}} & \wt{E}[n]\ar@{}[d]|-{\textstyle\cir}\\
\nm_{H_K/K}(I_{H_K}^n)/P_{K}(1,n)\ar[r]^-{\psi_{L_n/K}}_{\cong} & G(L_n/H_K)\ar@{.>}[r]^{\wt{\bullet}} & G(l_n/l).
}
\eeq
%{thm:max-ab-Q}

%Class field theory then tells us that the ray class group of $G(\Q(\ze_n)/\Q)$ is 
%we get that the Artin map $\psi_{\Q(\ze_n)/\Q}$ factors
We now carry out these two parts.
\subsection{The Galois group and class group act compatibly}
We establish the map in~(\ref{eq:G-cl-compat}).
\begin{thm}\llabel{thm:map-G-cl}
There exists a map $F:G(\ol K/K)\to \Cl(\sO_K)$ such that for {\it any} elliptic curve $E$,
\[
[E^{\si}]=F(\si)E.
\]
This map factors through $G(K\abe/K)$.
\end{thm}
As a reminder, the action of $\Cl(\sO_K)$ on $\Ell_{\ol{\Q}}(\sO_K)$ is such that if $E=E_{\La}$, then $F(\si)E=E_{F(\si)^{-1}\La}$. Theorem~\ref{thm:map-G-cl} expresses a deep relationship because the left-hand side expresses an algebraic action, while the right-hand side expresses an analytic action, as it is defined on lattices and the map between $E$ and $\C/\La$ is inherently analytic.

Proving this theorem essentially boils down to showing the Galois action commutes with the action on $\Cl(\sO_K)$.
\begin{pr}
For all $E$,
\[
\si([\ma] [E])=[\si(\ma)][\si(E)].
\]
\end{pr}
%\fixme{MOTIVATION??}
\begin{proof}
Suppose $E$ corresponds to $\La$, i.e. $E\cong \C/\La\C$. Then we have the exact sequence
\[
0\to \La\to \C\to E\to 0.
\]
Then $\ma E$ corresponds to $\ma^{-1}\La$. Take a resolution for $\ma$:
\[
R^m\xra{A} R^n\to \ma\to 0.
\]
Take a ``Hom product" and use the Snake Lemma. See~\cite[II.2.5]{Si94}.
%this CRAZY HOM PRODUCT and use SNAKE LEMMA.
\end{proof}
\begin{proof}[Proof of Theorem~\ref{thm:map-G-cl}]
See~\cite[II.2.4]{Si94}.
\end{proof}
\index{Hilbert class field of imaginary quadratic field}
\subsection{Hilbert class field}
Before we proceed with finding the Hilbert class field, we need to show injectivity of the reduction map like in~(\ref{eq:end-inj}).
\begin{thm}\llabel{thm:ec-hom-red}
Suppose $E_1$ and $E_2$ are elliptic curves defined over $L$ with good reduction at $\mP$. Then the reduction map
\[
\Hom(E_1,E_2)\to \Hom(\wt{E_1},\wt{E_2})
\]
is injective and preserves degrees.
\end{thm}
\begin{proof}
See Silverman AT \cite[pg. 124]{Si94} (Also see Silverman's errata).
\end{proof}

\fixme{Eventually rewrite this to work for all orders.}
The main theorem of this section is the following.
\begin{thm}[$j(E)$ generates the Hilbert class field]\llabel{thm:j-generates-hilbert}
Let $E$ be an elliptic curve with CM by $\sO_K$. Then
\begin{enumerate}
\item $K(j(E))=H_K$, the Hilbert class field of $K$.
\item $G(\ol K/K)$ acts transitively on the isomorphism classes of 
%$j$ invariants of 
curves in $\Ell(\sO_K)$.
%if $\{E_i\}$ are a set of representatives for the elements in $\Ell(\sO_K)$, then 
%\[
%\{j(E_i)\}=\set{\si(j(E_1))}{\si\in G(\ol K/K)}.
%\]
\item For any ideal $\ma\in I_K$,
\[
%j(E)^{\psi_{H_K/K}(\ma)}=j([\ma]E).
[E^{\psi_{H_K/K}(\ma)}]=[\ma][E].
\]
In particular, the action of Frobenius on the $j$-invariant is given by operating by $[\mfp]$ on the elliptic curve:
\[
%j(E)^{\Frob_{L/K}(\mfp)}=j(\mfp E).
[E^{(\mfp,H_K/K)}]=[\mfp][E].
\]
\end{enumerate}
\end{thm}
%Remark about $\ma^{-1}$ in action.
\begin{proof}
\noindent{\underline{Step 1:}}
First we show the following: There exists a finite set of primes $S$ of $\Z$ such that for any $p\nin S$ that splits completely in $K$, $p=\mfp\ol{\mfp}$, we have
\[
F((\mfp,L/K))=[\mfp]\in \Cl(\sO_K).
\]
This will show the dotted map in~(\ref{eq:G-cl-compat2}) is the identity for a large number of primes $\mfp$.

We have the map $[\mfp]:E\to \mfp E$. We show that this is ``like" the $p$th power Frobenius map. To do this, we show that it is inseparable of degree $p$ (this is why we needed $p$ to be split)\footnote{If $\mfp$ is not split, one can still show the map is inseparable of degree $p^2$, with some more work.}, %(according to Serre in the short CM article in CF.
and then look at the $j$-invariants of the reduced maps modulo $\mfp$.

As $\Ell_{\ol{\Q}}(\sO_K)=\Ell_{\C}(\sO_K)$ is finite, we can find a finite extension $L/K$ and representatives $E_1,\ldots, E_h$ of classes in $\Ell_{\C}(\sO_K)$, that are defined over $L$. Let $S$ be a set of primes containing the primes that satisfy one of the following conditions.
\begin{enumerate}
\item
$p$ ramifies in $L$. (Primes that ramify always cause trouble.)
\item
$E$ or some $E_i$ has bad reduction at some prime of $L$ lying over $p$.
\item
$v_p(\nm_{L/\Q}(j(E_i)-j(E_k)))\ne 0$ for some $i\ne k$. (This allows us to know what equivalence class an elliptic curve lies in, just by looking at its reduction modulo $p$.)
\end{enumerate}
Let $\La$ be the lattice such that $E(\C)\cong \C/\La$, and let $\ma$ be an integral ideal relatively prime to $\mfp$ such that $\ma\mfp=(\al)$ is principal (This exists by Corollary~\ref{factorization}.\ref{uf-dedekind-cor}). By the equivalence of categories~\ref{thm:lattice-ec-eoc}, the following maps on complex tori correspond to isogenies of elliptic curves:
\[
\xymatrix{
\C/\La \ar[r]^-i \ar[d]^{\Phi}_{\cong} &
\C/\mfp^{-1}\La \ar[r]^-i \ar[d]^{\Phi}_{\cong} &
\C/\mfp^{-1}\ma^{-1}\La \ar[r]^-{[\al]}_-{\cong} \ar[d]^{\Phi}_{\cong}&
\C/\La \ar[d]^{\Phi}_{\cong}\\
E\ar[r]^{\phi_1}&
\mfp E\ar[r]^{\phi_2}&
\ma \mfp E\ar[r]^-{\phi_3}_-{\cong}&
E
}
\]
Let the composition of the top maps be $f$ and the composition of the bottom maps be $g$.

Let $\om$ be an invariant differential on $E$. Then $\om'=\Phi^*\om$ is an invariant differential on $\C/\La$. It is in the form $c\,dz$. The composition of the top maps is just multiplication by $\al$, so $f^*\om'=\al\om'$. By commutativity, we get $g^*\om=\al\om$ as well.

Let $p\nin S$ and $\mP\mid \mfp\mid p$ in $L$, $K$, $\Q$,  respectively. 
Since $E$ has good reduction at $\mP$, we can reduce the elliptic curves and maps modulo $\mP$ to get
\[
\wt{g}^* \wt{\om}=\wt{\al}\wt{\om}=0
\]
since $\mP\mid \al$. By a criterion for separability ($g$ is separable iff $g^*$ does not act as 0 on $\Om_E$), $\wt g$ is inseparable. Now
\begin{align*}
\deg(\phi_1)&=\fN\mfp=p,\\
\deg(\phi_2)&=\fN\ma\perp p,\\
\deg(\phi_3)&=1.
\end{align*}
An inseparable map must have degree divisible by $p$, and the composition of separable maps is separable, so $\wt{\phi_1}$ must be inseparable.

Any inseparable map factors through the Frobenius map:
\beq{pE-factors-frob}
\xymatrix{
\wt E\ar[r] \ar[rd]_{\wt{\phi_1}}\ar[r]_{\phi_{p}}& \wt E^{(p)}\ar[d]_{\cong}^{\ep}\\
&\wt {\mfp E}.
}
\eeq
We have $p\deg(\ep)=\deg(\phi_p)\deg(\ep)=\deg(\wt{\phi_1})=p$ so $\deg(\ep)=1$. This shows $\ep$ is an isomorphism.

Thus we have 
\[
\wt{\mfp E}\cong \wt{E}^{(p)}.
\]
Now by definition of the Frobenius element (it is the $p$th power map modulo $\mP$), we have $j(\wt E^{(p)})=j(\wt E)^p=j(E)^{(\mfp,L/K)}$ modulo $\mP$. Putting everything together,
\[
j(\mfp E)\equiv j(\wt E^{(p)})\equiv j(E^{(\mfp,L/K)})\pmod{\mP}.
\]
But we chose $p$ so that nonisomorphic curves have $j$-invariants that are not congruent modulo $p$ (item 3). Therefore, $\mfp E\cong E^{(\mfp,L/K)}$. This shows that the action of $\mfp$ is the same as the action of $(\mfp,L/K)$, i.e. $F((\mfp,L/K))=[\mfp]$.\\

\step{2}
We show that $F:G(\ol K/K)\to \Cl(\sO_K)$ has kernel equal to $G(\ol K/K(j(E)))$, and so factors through $G(K(j(E))/K)\hra \Cl(\sO_K)$. Indeed,
\begin{align*}
\ker(F)&=\set{\si}{F(\si)E=E}\\
&=\set{\si}{E^{\si}=E}&\text{definition of }\si\\
&=\set{\si}{j(E)^{\si}=j(E)}&j \text{ parameterizes isomorphism classes}\\
&=G(\ol K/K(j(E))).
\end{align*}
We let $L=K(j(E))$.\\

\step{3} Let $\mf$ be the conductor of $L/K$. 
We extend Step 1 to all ideals $\ma$: for all $\ma$ we have
\[
F((\ma,L/K))=[\ma]\in \Cl(\sO_K);
\]
in other words $\mf=1$ and the following composition is the identity map.
\beq{eq:F(a)=[a]}
\xymatrix{
I_K/P_K\ar[r]^{\psi_{L/K}}\ar@/_2pc/@{.>}[rr]_{\Id}^{\cong}&G(L/K)\ar@{^(->}[r]^F &\Cl(\sO_K).
}
\eeq
Given $\ma\in I_K^{\mf}$, there are infinitely many $\mfp\in I_K^{\mf}$ in the same class as $\ma$ with degree 1 by Corollary~\ref{ch:cft-app}.\ref{cor:chebotarev-resfield1}. 
%Indeed, by the proof of Dirichlet's theorem, $\sum_{\mfp\sim\ma} \rc{\fN \mfp}$ diverges, while $\sum_{f(\mfp)>1}\rc{\fN\mfp}$ converges, so there must be infinitely many primes $\mfp\sim \ma$ with $f(\mfp)=1$. 
Choose such a prime $\mfp$, that does not divide a prime in $S$. 
%This argument due to http://math.stackexchange.com/questions/27308/does-every-ideal-class-contains-a-prime-ideal-that-splits
Note $\ma,\mfp$ differ by an ideal in $P_K(1,\mf)$ so they have the same image by the Artin symbol. Step 1 shows that
\[
F((\ma,L/K))=F((\mfp,L/K))\stackrel{\text{Step 1}}=[\mfp]=[\ma].
\]
In particular, for any principal ideal $(\al)\in I_K^{\mf}$, we have $F(((\al),L/K))=1$. However, by definition the conductor is the smallest $\mfp$ such that $\al\equiv 1\pmod{\mf}$ implies $((\al),L/K)=1$, so we must have $\mf=(1)$.\footnote{
Technically, we only have $((\al),L/K)=1$ for $(\al)\perp\mf$, and a priori $((\al),L/K)$ is not defined for $(\al)\perp \mf$. (We don't know $\mf=1$ yet.) The proper way to conclude $\mf=(1)$ is transfer the problem over to ideles: We know $\psi_{L/K}(P_K^{\mf})=1$, so $\phi_{L/K}(K^{\times}\mathbb U_K^{\mf})=1$. By $\I_K^{\mf}/K(1,\mf)\mathbb U_K(1,\mf)\cong \I_K/K^{\times}\mathbb U_K(1,\mf)$ we conclude that $\phi_{L/K}(K^{\times}\mathbb U_K)=1$. Hence $\mf =1$.
%So technically we should transfer the problem over to ideles, where we don't have this problem, and then conclude from there that the group corresponding to $L/K$ is exactly that of the Hilbert class field. \fixme{Fix this! Probably add a lemma on the conductor in the intro to cft chapter.}
}
Thus the map $F:I_K^{\mf}/P_K(1,\mf)\to G(L/K)$ we had originally is actually just $F:I_K/P_K\to G(L/K)$, and we get~(\ref{eq:F(a)=[a]}).\\

\step{4} 
Since the conductor is divisible by exactly the ramifying primes, $L/K$ is unramified, and $L\subeq H_K$. On the other hand, the map $F\circ \psi_{L/K}:I_K/P_K\to \Cl(\sO_K)$ is an isomorphism because $F\circ \psi_{L/K}$ is just the identity map. This gives $[L:K]=|\Cl(\sO_K)|=[H_K:K]$. Hence $L=H_K$. This shows item 1.\\

\step{5}
Item 3 now follows immediately, since we already showed $E^{\psi_{L/K}(\ma)}=[\ma]E$ and we now know $L=H_K$. Item 2 follows since the fact that the composition in~(\ref{eq:F(a)=[a]}) is an isomorphism means the map $F:G(L/K)\to \Cl(\sO_K)$ is surjective. Since $F$ transfers the action of $G(L/K)$ on $\Ell_{\ol{\Q}}(\sO_K)$ to $\Cl(\sO_K)$, and $\Cl(\sO_K)$ acts simply transitively on $\Ell_{\ol{\Q}}(\sO_K)$, we get that the same is true for $G(L/K)$. %We can of course transfer this statement to the $j$-invariants.
\end{proof}
%\begin{rem}
%In step 1, we used the fact that for $\mfp$ satisfying certain conditions, $\wt{\mfp E}\cong \wt{E}^(p)$. 
%Consider the special case where $\mfp E\cong \wt E^{(p)}$.
%\end{rem}
\section{Maximal abelian extension}
\index{maximal abelian extension of imaginary quadratic field}
We next carry out part 2 of our outline in Section~\ref{sec:kw-cm}. We construct the ray class fields for $K$, then take their compositum to get the maximal abelian extension.
\begin{df}
Suppose $E$ has CM by an order in $K$, and $E$ is defined over $H_K$.
A \textbf{Weber function} is an isomorphism $h:E/\Aut(1)\to \Pj^1$ defined over $H_K$. (So if $f:E\to E'$ is an automorphism, then $h(P)=h(f(P))$.)
\end{df}
We can always fix a concrete Weber function.
\begin{ex}
The simplest Weber function is the following. If $E$ has the form
\[
y^2=x^3+Ax+B,\quad A,B\in H_K,
\]
then take \[h(P)=\begin{cases}
x,&AB\ne 0\\
x^2,&B=0\\
x^3,&C=0.
\end{cases}
\]
In the 3 cases, respectively, $\Aut(E)$ is $1$, $\Z/2$ or $\Z/4$, and $\Z/3$ or $\Z/6$. %need to check this.

We can define a Weber function that is ``model independent," i.e. doesn't change under if we change to an isomorphic elliptic curve, by
\[
h(f(z))=\begin{cases}
\fc{g_2(\La)g_3(\La)}{\De(\La)}\wp(z,\La),&j(E)\ne 0,1728\\
\fc{g_2(\La)^2}{\De(\La)}\wp(z,\La)^2,&j(E)=1728\\
\fc{g_3(\La)}{\De(\La)}\wp(z,\La)^3,&j(E)=0.
\end{cases}
\]
This is because the expressions have ``weight 0."
%\fixme{mention this is because the expressions have ``weight 0."}
\end{ex}
The importance of the Weber function is given below. It would not be true if $h(P)$ were just defined as $h(x,y)=x$.
\begin{lem}\llabel{lem:K-ab-ext}
Let $E$ be an elliptic curve with CM by $\sO$. 
\begin{enumerate}
\item
The extension $K(j(E),E\tors)/K(j(E))$ is abelian.
\item
The extension $K(j(E),h(E\tors))/K$ is abelian.
\end{enumerate}
\end{lem}
The first statement is important because it tells us $G(\ol K/K(j(E)))$ acts in an abelian way on $E\tors$. Thus the ``Galois representation" of the Galois group on $E\tors$ is abelian. Thus, as we will see, it will decompose into two Gr\"ossencharacters.
\begin{proof}
%Since $E[m]=\Z/m\times \Z/m$, 
We have an injective map $G(K(j(E),E[m])/K(j(E)))\hra \Aut(E[m])$.\footnote{Since $E[m]=\Z/m\times \Z/m$, if we choose a basis for $E[m]$, we have $\Aut(E[m])\cong \GL_2(\Z/m)$, so we have a Galois representation.} Now, the image of $G$ in $\Aut(E[m])$ commutes with $\sO_K$, so is contained in 
\[\Aut_{\sO_K/m\sO_K}(E[m])\cong \Aut_{\sO_K/m\sO_K}(\sO_K/m\sO_K)\cong (\sO_K/m\sO_K)^{\times}
\]
which is abelian.

%For the second, we know $K(j(E),h(E\tors))/K(j(E))$ is abelian. We need t
For the second, suppose $\si,\tau\in G(K(j(E),h(E\tors))/K)$. We show that $\si\tau=\tau\si$. Since $K(j(E))/K$ is abelian, $\si\tau\si^{-1}\tau^{-1}$ fixes $j(E)$. %Suppose $P\in E\tors$. Then Now $\si(\tau(P))\in \si(\tau(E))$ and $\tau(\si(P))\in \tau(\si(E))$. However, $E':=\si(\tau(E))=\tau(\si(E))$ because the Galois action factors through $G(K\abe/K)$ (Theorem~\ref{thm:map-G-cl}). 
Now 
$\si\tau\si^{-1}\tau^{-1}$ gives an automorphism of $E'=\tau\si(E)$ because 
\[(\si\tau\si^{-1}\tau^{-1})\tau\si(E)=\si\tau(E)\cong \tau\si(E),\]
as the Galois action factors through $G(K\abe/K)$ and hence is abelian (Theorem~\ref{thm:map-G-cl}) (alternatively, because $\si\tau\si^{-1}\tau^{-1}$ fixes $j(E)$). As $E$ is defined over $H_K$, we actually have equality.

Since $h$ is invariant under automorphism, for any $P\in E\tors$,
\[
h(P)=h(\si\tau\si^{-1}\tau^{-1} P)=\si\tau\si^{-1}\tau^{-1} h(P).
\]
(We know $h$ is defined over $H_K$ and $\si\tau\si^{-1}\tau^{-1}$ fixes $H_K=K(j(E))$.) Hence $\si\tau\si^{-1}\tau^{-1}$ fixes $h(E\tors)$ as well, and $\si\tau\si^{-1}\tau^{-1}=1$.
%As the map $G(K(j(E),h(E\tors))/K)\to G(K(j(E))/K)\times G(K(h(E\tors))/K)$ is injective, we get $\si\tau\si^{-1}\tau^{-1}=1$, as needed.
\end{proof}
\begin{thm}\llabel{thm:max-abe-ext-K}
Suppose $K$ is a quadratic imaginary field and $E$ has CM by $\sO_K$.
\begin{enumerate}
\item
For an integral ideal $\ma$ of $\sO_K$, $L_{\ma}:=H_K(h(E[\ma]))=K(j(E),h(E[\ma]))$ is the ray class field of $K$ modulo $\ma$. 
\item
The maximal abelian extension of $K$ is
\[
K(j(E),h(E\tors)).
\]
\end{enumerate}
\end{thm}
\begin{proof}
\step{1}
We need the following lemma.
\begin{lem}\llabel{lem:comm-in-image}
Suppose $E$ is an elliptic curve defined over $L$ with CM by $\sO_K$, and has good reduction at $\mP$. Let $\wt{E}$ be the reduction modulo $\mP$. Let $\theta:\End(E)\to \End(\wt E)$ be the reduction map on endomorphisms. Then for any $\ga\in \End(\wt{E})$,
\[
\ga\in \im(\te)\iff \ga\text{ commutes with every element in }\im(\te).
\]
\end{lem}
\begin{proof}
Since $E$ has good reduction, the map $\End(E)\hra \End(\wt E)$ in injective. Consider 2 cases.
\begin{enumerate}
\item
$\End(\wt E)$ is a quadratic order. Then $\End(E)=\End(\wt E)$ (as $\End(E)$ is a maximal order) so this case is clear.
\item
$\End(\wt E)$ is an order in a quaternion algebra. Then $\End(E)\ot \Q$ is its own centralizer in the quaternion algebra $\End(\wt E)\ot \Q$, by the Double Centralizer Theorem~\ref{galois-cohomology-ch}.\ref{thm:dct-gen}.
%following general theorem from noncommutative algebra.
%\begin{thm}[Double centralizer theorem]
%Let $B$ be a simple $k$-subalgebra of a central simple $k$-algebra $A$. Then the centralizer $C=C(B)$ is simple, 
%\[[B:k][C:k]=[A:k],\]
%and $C(C(B))=C$
%\end{thm}
%\begin{proof}
%See Milne~\cite{Mi08}, Theorem IV.3.1 (pg. 129).
%\end{proof}
\end{enumerate}
\end{proof}

\step{2} We show that in general, we can lift the Frobenius map.
\begin{pr}\llabel{pr:ec-lift-frob}
Suppose $E$ has CM by $\sO_K$ and is defined over $H_K$. Let $\mP\mid \mfp\mid p$ in $H_K$, $K$, $\Q$, respectively, with $\mfp$ having degree 1 and $p\nin S$, $S$ being defined as in the proof of Theorem~\ref{thm:j-generates-hilbert}. Then the $p$th power Frobenius map can be lifted to a map on $E$, i.e. there is $\la$ making the following commute:
\[
\xymatrix{
E\ar[r]^-{\la} \ar[d] &E^{(\mfp,H_K/K)}\ar[d]\\
\wt E\ar[r]^{\wt{\la}=\phi_p} & \wt E^{(p)}.
}
\]
Moreover, if $E$ corresponds to the complex torus $\C/\La$, then up to isomorphism, $\la$ corresponds to the map $\C/\La\to \C/\mfp^{-1}\La$. (Recall that $E^{(\mfp,H_K/K)}\cong \mfp E$ by Theorem~\ref{thm:j-generates-hilbert}.)
\end{pr}
\begin{proof}
We need to show $\phi_p$ is the reduction of some map; we do this by first reducing the problem to showing a certain endomorphism is in the image of $\te$ and then showing the conditions of the previous lemma hold.

Again we use~(\ref{pE-factors-frob}): $\wt{\phi_1}:\wt E\to \wt{\mfp E}$ is ``like" the Frobenius map. We know $\wt{\phi_1}$ is the reduction of a map, namely the map $\phi_1:E\to \mfp E$. Now note $\wt{\mfp E}\cong \wt{E^{(\mfp,L/K)}}=\wt{E}^{(p)}$, the first from Thm~\ref{thm:j-generates-hilbert} and the second from definition of the Frobenius element.

Let $\si=(\mfp,L/K)$. It remains to show that $\ep:\wt{E^{\si}}\to \wt{\mfp E}\cong \wt{E^{\si}}$ is the reduction of a map $\ep'$, because then ${\ep'}^{-1}\circ \phi_1$ will be the desired map. Let $\wt{[\al]}\in \Aut(\wt{E^{\si}})$ be the reduction of a map $[\al]$. 
To show $\ep$ commutes with $[\al]$, we consider $\wt{\phi_1}=\ep\circ \phi_p$, and consider how $[\al]$ ``commutes" with $\wt{\phi_1}$ and $\phi_p$.
\begin{enumerate}
\item
$\wt{\phi_1}$: By normalization (Proposition~\ref{pr:normalize-cmec}(3)), we know
\[
\phi_1\circ [\al]_E=[\al]_{E^{\si}}\circ \phi_1.
\]
\item
$\phi_p$: Note that for any morphism of varieties $f:V\to W$ over a field of characteristic $p$, the following commutes, where $\phi_V,\phi_W$ are the $p$th power Frobenius maps on $V$ and $W$:
\[
\xymatrix{
V\ar[r]^f \ar[d]^{\phi_V}& W\ar[d]\ar[d]^{\phi_W}\\
V^{(p)}\ar[r]^{f^{\si}} & W^{(p)}
}\quad \phi_W\circ f=f^{\si}\circ \phi_V.
\]
Applying this to $[\al]_E$,
\[
\phi_p\circ \wt{[\al]_E}=\wt{[\al]_E^{\si}}\circ \phi_p=\wt{[\al]_{E^{\si}}}\circ \phi_p,
\]
where in the last step we used Theorem~\ref{pr:normalize-cmec}(4), noting $\si(\al)=\al$ since $\al \in K$ and $\si\in G(H_K/K)$.
\end{enumerate}
Hence
\[
\wt{[\al]_{E^{\si}}}\circ \underbrace{\ep\circ \phi_p}_{\phi_1}\stackrel{1}=
\ep\circ \phi_p\circ \wt{[\al]_E}
\stackrel{2}=
\ep\circ \wt{[\al]_{E^{\si}}}\circ \phi_p.
\]
Cancelling $\phi_p$ gives $\wt{[\al]_{E^{\si}}}\circ \ep=\ep\circ \wt{[\al]_{E^{\si}}}$, so Lemma~\ref{lem:comm-in-image} shows $\ep$ is the reduction of some $\ep'$, as needed.

To finish, note that $\phi_1$ does indeed correspond to $\C/\La\to \C/\mfp^{-1}\La$. Hence $\la$ corresponds to $\C/\La\to \C/\mfp^{-1}\La$, up to some automorphism.
\end{proof}

\step{3} When $(\mfp,H_K/K)=1$, $\la$ is just an endomorphism of $E$, hence equals $[\al]$ for some $\al$. In fact, the following proposition shows it is $[\pi]$ for some $\pi$ generating $\mfp$, so that multiplication by $\pi$ corresponds to the $p$th power Frobenius in the reduction.
\begin{pr}\llabel{pr:pi-is-frob}
Suppose $E$ has CM by $\sO_K$ and is defined over $H_K$. For all but finitely many degree 1 prime ideals $\mfp$ with $(\mfp,H_K/K)=1$ (equivalently, such that $\mfp$ is principal), there exists a unique $\pi$ such that $\mfp=(\pi)$ and the following commutes.
\[
\xymatrix{
E\ar[r]^{[\pi]} \ar[d] &E \ar[d]\\
\wt E\ar[r]^{\phi_p} & \wt E.
}
\]
\end{pr}
\begin{proof}
Since $(\mfp,H_K/K)=1$, Proposition~\ref{pr:ec-lift-frob} gives a diagram
\[
\xymatrix{
E\ar[r]^{\la} \ar[d] &E \ar[d]\\
\wt E\ar[r]^{\phi_p} & \wt E.
}
\]
for some $\la$. We know $\la$ is in the form $[\pi]$, and show $\pi$ satisfies the desired conditions. We have by Proposition~\ref{pr:E[a]} that
\[
\nm_{K/\Q}(\pi)=\deg([\pi])=\deg(\phi)=p=\fN\mfp
\]
so either $(\pi)=\mfp$ or $(\pi)=\ol{\mfp}$. As always, when we're deciding between conjugates, normalization comes to the rescue. Take $\om\in \Om_E$ whose reduction modulo $\mP$ is nonzero. Normalization says that $[\pi]^*\om=\pi \om$ so
\[
\wt{\pi}\wt{\om}=\wt{[\pi]}^*\wt{\om}=\phi_p^*\wt{\om}=0,
\]
the last step since the Frobenius map is inseparable. We get $\mP\mid \pi$, forcing $(\pi)=\mfp$.

For uniqueness, note the map
\[
\xymatrix{
\sO_K\ar[r]^-{[\cdot]}_-{\cong}& \End(E)\ar[r]^{\wt{E}} &\End(\wt E)
}
\]
is injective for $E$ having good reduction at $\mP$ (Theorem~\ref{thm:ec-hom-red}).
\end{proof}

\step{4} Consider~(\ref{eq:rcf-k}). We need to show that $P_K(1,\ma)$ is exactly the kernel of the Artin map $\psi_{L_{\ma}/K}$. 
Note that $P_K(1,\ma)$ and $\ker(\psi_{L_{\ma}/K})$ are both subgroups of $P_K^{\ma}=\ker(\psi_{H_K/K})=\ker(\psi_{L_{\ma}/K}(\bullet)|_{H_K})$. It suffices to show that for all but finitely many primes $\mfp$ of degree 1 such that $(\mfp,H_K/K)=1$, we have $\mfp\in P_K(1,\ma)$ iff $\mfp\in \ker(\psi_{L_{\ma}/K})$.

Let $\mfp$ satisfy the conditions of Proposition~\ref{pr:pi-is-frob}. Since the reduction of $\psi_{L/K}(\mfp)$ is the Frobenius map, we get that $\psi_{L/K}(\mfp)=[\pi]$, for some $\pi$ such that $(\pi)=\mfp$.\footnote{Note the analogy with the cyclotomic case. $\psi_{L/K}(\mfp)$ acts on torsion points as $[\pi]$, just as in the cyclotomic case it acted as the $p$th power map, that corresponds to $[p]$ if we consider the natural map $\Z\to \End(\Q(\ze_n))$.} Since $(\mfp,H_K/K)=1$, we have the commutative diagram
\beq{eq:pi-is-frob}
\xymatrix{
E\ar[r]^{\psi_{L/K}(\mfp)=[\pi]} \ar[d] &E \ar[d]\\
\wt E\ar[r]^{\phi_p} & \wt E.
}
\eeq

We have the following string of equivalences, for all but finitely many degree 1 primes $\mfp$ with $(\mfp,H_K/K)=1$,
\begin{enumerate}
\item
$\mfp\in P_K(1,\ma)$.
\item
$\mfp=(\pi)$ where $\pi=u\al$ where $u$ is a unit and $\al\equiv 1\pmod{\ma}$.
\item For all $\ma$-torsion points $P\in E[\ma]$, 
$h([\pi]P)=h(P)$.
\item[$3'$.] For all $\ma$-torsion points $P\in \wt E[\ma]$, 
$\wt h(\wt{[\pi]}\wt P)=\wt h(\wt P)$.
\item
$(\mfp,L_{\ma}/K)$ fixes $h(E[\ma])$.
\item
$\mfp\in \ker(\psi_{L_{\ma}/K})$.
\end{enumerate}
(1)$\iff$(2) is clear.

For $(2)\implies (3)$, note that for all $\ma$ torsion points $P\in E[\ma]$, %since $h$ is $\Aut(E)$-invariant and $[u]$ is an automorphism,
\begin{align*}
%h(P)^{(\mfp,L/K)}&= h(P^{(\mfp,L/K)})&\text{$(\mfp,L/K)$ fixes $H_K$ and $E$ defined over $H_K$}\\
%&=
h([\pi]P)
&=h([u][\al]P)\\
&=h([\al]P)&\text{$h$ is $\Aut(E)$-invariant}\\
&=h(P)&\al\equiv 1\pmod{\ma}\text{ and }P\in E[\ma].
\end{align*}
Note it is important that $h$ be $\Aut(E)$-invariant.

For $(3')\implies (2)$, let $P\in E[\ma]$ be a torsion point. By~\cite[VII.3.1b]{Si86}, $E[\ma]\hra \wt{E}[\ma]$ is injective for $\mfp\nmid \ma$ and $E$ with good reduction at $\mfp$. Since $h$ is an isomorphism (in particular, an injection) $E/\Aut(E)\to \Pj^1$, we get that $[\pi]P= [u]P$ for some $[u]\in \Aut(E)$. 
But $E[\ma]\cong \sO_K/\ma$, so we can choose $u$ such that $\pi\equiv u\pmod{\ma}$. Then there exists $\al$ such that $\pi=u\al$, with $\al\equiv 1\pmod{\ma}$.

For $(3)\implies (4)$, we calculate the action of $(\mfp,L/K)$ on a torsion point $P\in E[\ma]$, in the reduced curve:
\[
\wt{P^{(\mfp,L/K)}}=\phi_p(\wt P)=\wt{[\pi]P},
\]
the second equality from Proposition~\ref{pr:pi-is-frob}. This allows us to understand the action on the nonreduced curve, since $E[\ma]\hra\wt E[\ma]$ is injective for $\mfp\nmid \ma$ and $\mfp$ of good reduction. We get
\[
P^{(\mfp,L/K)}=[\pi]P.
\]
Thus (3) implies 
\begin{align*}
h(P)^{(\mfp,L/K)}&= h(P^{(\mfp,L/K)})&\text{$(\mfp,L/K)$ fixes $H_K$ and $E$ defined over $H_K$}\\
&=h([\pi]T)\\
&=h(T)&\text{by (3)}.
\end{align*}

Now we prove $(4)\implies(3')$. Let $\si\in G(\ol K/K)$ be an automorphism such that $\si|_{K\abe}=(\mfp,K\abe/K)$. Then for any $P\in E[\ma]$,
\[
\wt{h}(\wt{[\pi]}\wt P)\stackrel{\eqref{eq:pi-is-frob}}{=}\wt{h}(\phi(\wt{P}))=\wt{h(P^{\si})}=\wt{h(P)}^{\si}=\wt h(\wt P),
\]
the last two equalities since $\si|_H=1$, $h$ is defined over $H$, and $\si|_{L_{\ma}}$ fixes $h(E[\ma])$ by assumption. Thus $(3')$ holds. %\fixme{Reduced vs. nonreduced} 

Now (4)$\iff$(5) comes from the fact that $(\mfp,L_{\ma},K)$ already fixes $K(j(E))$, so to fix $L_{\ma}$ it only needs to fix $h(E[\ma])$.\\
%
%%We know $\mfp\in P_K(1,\ma)\subeq P_K$ so $(\mfp,H_K/K)=(\mfp,L_n/K)|_{H_K}=1$. 
%It remains show that $(\mfp,L_n/K)$ fixes $h(E\tors)$; then it will fix $H_K(h(E\tors))$, as needed.
%
%Let $\mfp=(\al)\in P_K(1,\ma)$. We want to get information about $\psi_{L/K}(\mfp)$ from knowing that its reduction is the Frobenius map. This is~(\ref{pr:pi-is-frob})! \fixme{$\psi_{L/K}(\mfp)$ acts on torsion points as $[\pi]$, just as in the cyclotomic case it acted as the $p$th power map, that corresponds to $(p)$ if we consider the natural map $\Z\to \End(\Q(\ze_n))$.} We get, for all but finitely many $\mfp$,
%\beq{eq:pi-is-frob}
%\xymatrix{
%E\ar[r]^{[\pi]} \ar[d] &E \ar[d]\\
%\wt E\ar[r]^{\phi_p} & \wt E.
%}
%\eeq
%where $(\pi)=(\al)$, i.e. $\pi=u\al$ for some unit $u$. 
%
%We calculate the action of $(\mfp,L/K)$ on a torsion point $P\in E[\ma]$, in the reduced curve:
%\[
%\wt{P^{(\mfp,L/K)}}=\phi_p(\wt P)=\wt{[\pi]P},
%\]
%the last from Proposition~\ref{pr:pi-is-frob}. This allows us to understand the action on the nonreduced curve, since $E[\ma]\hra\wt E[\ma]$ is injective (conditions?). We get
%\[
%P^{(\mfp,L/K)}=[\pi]P
%\]
%We get that
%\begin{align*}
%h(P)^{(\mfp,L/K)}&= h(P^{(\mfp,L/K)})&\text{$(\mfp,L/K)$ fixes $H_K$ and $E$ defined over $H_K$}\\
%&=h([\pi]T)\\
%&=h([u][\al]T)&\text{$h$ is $\Aut(E)$-invariant}\\
%&=h(T)&\al\equiv 1\pmod{\ma}\text{ and }P\in E[\ma].
%\end{align*}
%Note it is important that $h$ be $\Aut(E)$-invariant.\\
%\step{6}
%We show the other direction, $P_K(1,\ma)\supeq \ker(\psi_{L_n/K})$. This is basically the backwards argument. Take $\mfp\in \ker(\psi_{L_n/K})$ of degree 1...; we get $(\ma,H_K/K)=(\ma,L_n/K)|_{H_K}=1$. 
%As in~(\ref{eq:pi-is-frob}), we get $\pi$ such that $\wt{[\pi]}=\phi_p$.
%
%Pretty darn straightforward; See Silv. p.137.\\

\step{7}
The maximal abelian extension is the union of the all ray class fields. Note every $\mc$ divides $n$ for some $n$ so we can just restrict to ray class fields corresponding to $(n)$ for some $n\in \N$:
\[
K\abe=\bigcup_{n}K(j(E),h(E[n]))=K(j(E),h(E\tors)).
\]
\end{proof}
%\section{$j(E)$ is an algebraic integer}
%Omitted for now.
\section{The Main Theorem of Complex Multiplication}
\index{main theorem of complex multiplication}
Given $\si\in \Aut(\C/K)$, consider the map $\si: E(\C)\to E^{\si}(\C)$. We would like to know how this map acts on torsion points. This is since to get Galois representations of elliptic curves, we look at how $\si$ acts on torsion points---often specializing to torsion points that are a power of a prime.

Because we are considering CM elliptic curves, we can identify the torsion points with $K/\ma$, for some ideal $\ma$. Namely, given an analytic isomorphism $f:\C/\ma\xra{\cong} E(\C)$, we can restrict it to $K/\ma$ to get
\[
f|_{K/\ma}:K/\ma\xrc E\tors \hra E(\C).
\]

The main theorem of complex multiplication tells us we can transfer the map $\si:E(\C)\to E^{\si}(\C)$ via an {\it analytic isomorphism} to a multiplication-by-an-idele map $[\mathbf x^{-1}]:K/\ma\to K/\mathbf x^{-1}\ma$, where $\mathbf x$ and $\si$ are related in terms of the Artin map (to be made precise).

%In some sense, we have been building up to this theorem: 
%%The map $E\to E^{(\mfp,
%From Theorem~\ref{j-generates-hilbert}, we know that $(\mfp,L/K)$ sends an elliptic curve associated with to lattice $\La$, to a an elliptic curve associated to the lattice $\mfp^{-1}\La$; we can think of this as the case where $\si=(\mfp,L/K)$ and $\mathbf x$ corresponds to the ideal $\mfp$. 
%Thinking of $K\ma\cong E\tors$, we have a commutative diagram
%\[
%\xymatrix{
%K/\ma\ar[r]^i\ar[d]^{\Phi} & K/\mfp^{-1}\ma\\
%E\ar[r]^{(\mfp,L/K)} &E^{\si}
%}
%\]
%where the top map is just the inclusion.
%
%More generally, we want to consider the map $K/\ma\to K/\mathbf a^{-1}\ma$ for 
%
%of ideles on lattices, not just the action of ideals.
\begin{df}
Let $\mathbf x=\prod_{\mfp\in V_K^0} \mfp^{m(\mfp)}\prod_{v\in V_K^{\iy}} v^{m(v)}\in \I_K$ be an idele.
Let $\ma$ be an ideal, and define $\mathbf x\ma$ by
\[
\mathbf x\ma=p(\mathbf x)\ma = \pa{\prod_{\mfp\in V_K} \mfp^{m(\mfp)}}\ma.
\]
Define the map
\beq{eq:mult-idele-map}
[\mathbf x]:K/\ma\to K/\mathbf x\ma
\eeq
as follows. Note $K/\ma\cong \prod_{\mfp} K_{\mfp}/\ma K_{\mfp}$ by the Chinese Remainder Theorem, %or an easy generalization of
where $x$ is just identified with its images in the $K_{\mfp}/\ma K_{\mfp}$: $(x_{\mfp})_{\mfp\in V_K^0}$. Then~(\ref{eq:mult-idele-map}) sends
\beq{eq:mult-by-idele}
(a_{\mfp})\mapsto (x_{\mfp}a_{\mfp}) \text{ where }\mathbf x=(x_{\mfp}).
\eeq
\end{df}
\begin{thm}[Main Theorem of Complex Multiplication]\llabel{thm:mt-cm}
Suppose $E$ is an elliptic curve with CM by $\sO_K$. Let $\si\in \Aut(\C/K)$ and $\mathbf x\in \I_K$ be such that
\[
\si|_{K\abe}=\phi_{K}(\mathbf x).
\]
Fix an analytic isomorphism $f:\C/\ma\xra{\cong} E(\C)$. Then there exists a unique analytic isomorphism $f':K/\mx^{-1}\ma\to E^{\si}(\C)$ such that the following commutes:
\[
\xymatrix{
K/\ma \ar[r]^{\mx^{-1}}\ar[d]^f & K/\mx^{-1}\ma\ar@{->}[d]^{f'}\\
E(\C)\ar[r]^{\si} & E^{\si}(\C).
}
\]
\end{thm}
\begin{rem}
The map~(\ref{eq:mult-by-idele}) can be a bit weird to think about: For instance, consider the simpler case $K=\Q$, $\ma=\Z$. Take the idele $\mx$ with 1's everywhere except $x_5=2$. Then $[\mx]$ sends $\rc 2\mapsto \rc 2,\rc3\to \rc 3, \rc 7\to \rc 7$ and so forth but sends $\rc 5\to \fc 25$. So it is surprising that $\mx^{-1}:K/\ma\to K/\mx^{-1}\ma$ can be related analytically to $E(\C)\to E^{\si}(\C)$. %can be related to an map of varieties over $\C$
\end{rem}
Compare this theorem to Proposition~\ref{pr:pi-is-frob}. Rather tan just dealing with the Frobenius element of a prime, we deal with the Artin map of an idele.
\begin{proof}
Note uniqueness follows from the fact that topologically, the closure of $K/\mx^{-1}\ma$ is $\C/\mx^{-1}\ma$, and any continuous function is determined by its values on a dense set.\\

First we prove this for $E$ defined over $\Q(j(E))$ and $\ma$ integral. We do this in 2 steps.
\step{1} Approximate $\si$ by a field automorphism $\la$ that is the Frobenius element of a prime $\mfp$. (The Frobenius element is something much more concrete to work with than the abstract Artin map of an idele.) We will take better and better approximations, which determine the action on $E[m]$ for larger and larger $m$, and take an inverse limit.

So let $L'_m$ be the Galois closure of $K(j(E),E[m])/K$. By Corollary~\ref{ch:cht-app}.\ref{chebotarev-resfield1}, there are infinitely many primes with $\mP\mid \mfp$ in $K$ and $L$ such that
\[
(\mP,L/K)=\si|_{L'_m},\quad \fN(\mfp)=1.
\]
%Each $\mfp$ is divisible by $|\an{\si|_{L'_m}}|$ primes, ranging over exactly the elements of the conjugacy class (Lemma~\ref{intro-cft}.\ref{lem:frob-lem}), so we can take $\mP\mid \mfp$ with $(\mP,L'_m/K)=\si|_{L'_m}$.  
We can furthermore choose $\mfp$ satisfying the following, because each condition excludes only finitely many primes.
\begin{enumerate}
\item
$\mfp$ is unramified in $L'_m$.
\item
$\mfp\nin S$, where $S$ is defined as in the proof of Theorem~\ref{thm:j-generates-hilbert}.
\item
$\mfp\nmid m$.
\end{enumerate}
By Proposition~\ref{pr:ec-lift-frob}, there exists a map $\la:E\to E^{\si}$ that reduces to $\phi_p$ modulo $\mP$. On $\wt{E}[m]$, both $\la$ and $\si$ act as $\phi_p$. 
Because $\mP\nmid m$ by item 3, the reduction map modulo $\mP$,  $E[m]\to \wt E[m]$, is injective. Hence $\la$ and $\si$ act the same on $E[m]$:
%, and the following commutes: 
%Now $\la|_{L_m}=(\mP,L_m/K)=\si|_{L_m}$ acts as the $p$th power Frobenius map on $\wt E[m]$, so $\la
%\[
%\xymatrix{
%E[m]\ar[r]^{\la}\ar@{^(->}[d] & E^{\si}[m]\ar@{^(->}[d]\\
%E(\C)\ar[r]^{\si} E^{\si}(\C).
%}
%\]
\beq{eq:la=si}
\la|_{E[m]}=\si|_{E[m]}:E[m]\to E^{\si}[m].
\eeq
But we know how the map $\la$ acts: 
Proposition~\ref{pr:ec-lift-frob} tells us that 
%by Theorem~\ref{thm:j-generates-hilbert}, $[E^{\si}]=[\mfp]*[E]$, and 
the map $\la: E\to E^{\si}$ corresponds to the map on complex tori $i:\C/\ma\to \C/\mfp^{-1}\ma$.\footnote{The map $\si$ and $\mx^{-1}$ appearing in the theorem statement are bijections, while $\la$ and $i$ are not. This is okay, though, because we only use $\la,i$ to approximate $\si$ on $m$-torsion, and $\la,i$ are injective on $m$-torsion, since $\mP\nmid m$.} Hence we have the commutative diagram
\beq{eq:mt-cm-1}
\xymatrix{
\C/\ma \ar[r]^{i}\ar[d]^{f} & \C/\mfp^{-1}\ma\ar[d]^{f''}\\
E(\C)\ar[r]^{\la} &E^{\si}(\C)
}
\eeq
for some analytic isomorphism $f''$.

%But we're given a map $\C/\ma\to \C/\mx^{-1}\ma$. Note we can relate $\mx$ and $\mfp$ as follows. Let $\pi$ be the uniformizer of $K_{\mfp}$, let $i_{\mfp}:K_{\mfp}\to \I_K$ be the inclusion map.
\step{2}
%Note that
%\[
%\C/\mfp^{-1}\ma \cong E^{\si}(\C)\cong \C/\mx^{-1}\ma,
%\]
%already tells $\mfp^{-1}\ma$ and $\mx^{-1}\ma$ must be homothetic ideals, in other words, $(\mx)=(\al)\mfp$ for some principal ideal $(\al)$. We get an even stronger condition, though, from the fact that
By Theorem~\ref{thm:max-abe-ext-K}, the ray class group modulo $m$ is $K_m=K(j(E),h(E[m]))$. Note $K_m\subeq L_m'$. 
Now by assumption, $\mfp$ was chosen so that the images of $\mfp$ and $\mx$ under the Artin map both project to $\si|_{K_m}$:
\[
%\psi_{L'_m/K}((\mx))=
\phi_{K_m/K}(\mx)=\si|_{K_m} =\psi_{K_m/K}(\mfp)=\phi_{K_m/K}(i_{\mfp}(\pi))
\]
where $\psi$, $\phi$ denote the Artin map on ideals and on ideles, respectively, and $\pi$ is the uniformizer of $\mfp$ in $K_{\mfp}$. We have
\[
\ker\psi_{K_m/K}=K^{\times}\mathbb U_K(1,m).
%P_K(1,m).
\]
(See Definition~\ref{intro-cft}.\ref{df:more-idele-dfs} for notation.) This follows from the definition of the ray class field and from the correspondence between ray class groups in Definition~\ref{intro-cft}.\ref{df:ray-class-field} and idele class groups in Example~\ref{intro-cft}.\ref{ex:class-group-idele-quotient}. 
We have $\mx\in i_{\mfp}(\pi)\ker \phi_{K_m/K}$, giving
%Hence, since the kernel of $(\bullet):\I_K\to I_K$ is exactly the units $\prod_{v}U_v$, we can write
\[
\mx=\al\cdot  i_{\mfp}(\pi)\cdot  \mathbf u,\quad \al\in K^{\times},\quad \mathbf u\in \mathbb U_K(1,m).
\]

%where $\al\in K^{\times}$ and $u\in \I_K(1,m)$ is a unit at every component.
% $i_{\mfp}(\pi)$ denote the element of $\I_K$ with $\pi$ at 

We now compose~(\ref{eq:mt-cm-1}) with the homothety $\al^{-1}$, and note $(\mx)=(\al)\mfp$, to get the desired map $\C/\mx^{-1}\ma\to E^{\si}(\C)$:
\beq{eq:mt-cm-2}
\xymatrix{
\C/\ma \ar[r]^i\ar[d]^f & \C/\mfp^{-1}\ma\ar[r]^{\al^{-1}} \ar[d]^{f''} & \C/\mx^{-1}\ma  \ar[ld]^{f'_m}\\
E(\C) \ar[r]^{\la} & E^{\si}(\C) & 
}
\eeq
Here, $f'_m(z):=f''(\al z)$.

This isn't quite what we want yet, though, because the top row is the map $\al^{-1}$ rather than the map $\mx^{-1}$. We need to show that for $m$-torsion points, $\al^{-1}$ acts the same as $\mx^{-1}$. Then we would have
\[
\si(f(t))=\la(f(t))=f'_m(\al^{-1}t)=f'_m(\mx^{-1}t),\quad t\in m^{-1}\ma/\ma.
\]
The first equality is since $\si$, $\la$ were by construction the same on $E[m]$ (\ref{eq:la=si}), so $\si\circ f$ and $\la\circ f$ are the same on $m^{-1}\ma/\ma$. The second is by commutativity of~(\ref{eq:mt-cm-2}).

To show the third equality, we note that
\begin{align*}
&f'_m(\al^{-1}t)=f'_m(\mx^{-1}t)&\text{for all } t\in m^{-1}\ma/\ma\\
(f'_m\text{ bijective})\quad \iff& \al^{-1}t-\mx^{-1}t\in \ma &\text{for all } t\in m^{-1}\ma\\
\iff&\al^{-1}t_{\mq}-x_q^{-1}t_{\mq} \in \ma_{\mq}&\text{for all } t\in m^{-1}\ma,\,\mq\\
\text{(multiplying by $x_{\mq}=\al [i_{\mfp}(\pi)]_{\mq} u_{\mq}$)} \quad \iff 
& [i_{\mfp}(\pi)]_{\mq} u_{\mq}t-t\in \ma_{\mq} &\text{for all }t\in m^{-1}\ma_{\mq}\\
\iff& ([i_{\mfp}(\pi)]_{\mq} u_{\mq}-1)\ma_{\mq}\subeq m\ma_{\mq}\\
u_{\mq}\in \mathbb U_K(1,m)  \quad \iff &
([i_{\mfp}(\pi)]_{\mq}-1)\ma_{\mq}\subeq m\ma_{\mq}.
\end{align*}
Consider 2 cases.
\begin{enumerate}
\item $\mq\ne \mfp$. In this case, $[i_{\mfp}(\pi)]_{\mq}=1$, so this is trivial.
\item $\mq= \mfp$: $[i_{\mfp}(\pi)]_{\mfp}=\pi$, and $\pi-1$ is a unit. By assumption $\mfp\nmid m$. hence $(\pi-1)\ma=\ma=m\ma$.
\end{enumerate}

\step{3} 
We now show that the maps $f'_m$ are all actually the same for $m\ge 3$. Indeed, $f'_m|_{E[m]}=f'_{mn}|_{E[m]}$ by construction, so $f'_m,f'_{mn}$ differ by an automorphism that fixes $E[m]$. This automorphism must be $[\ze]$ for some element of norm 1 in $K$, and $f'_m=[\ze]\circ f'_{mn}$. Since $f'_m,f'_{mn}$ are isomorphisms, this says
\[
E[m]\subeq \ker [1-\ze]
\]
The only possibilities are $\ze$ a 4th or 6th root of unity, and if $\ze\ne 1$, then $[1-\ze]$ has norm at most 4. So for $m\ge 3$, $\ze=1$, and $f'_m=f'_{mn}$.\\

\step{4}
Finally, we show the theorem holds for general $E/L$. Any elliptic curve $E$ has a model $E'$ defined over $M'=\Q(j(E))$, corresponding to a complex torus $\C/\ma'$ with $\ma'$ an integral ideal (see the left face below). Let $E\to E'$ be an isomorphism and $K/\ma\to K/\ma'$ be the corresponding map on torsion. Then the existence of $f'_{E'}$ for $E'/L$ gives the existence of $f'_{E}$ for $E/L$, by choosing $f'_{E}$ to make the below diagram commute.
\[
\xymatrix{
%line 1
K/\ma\ar[rr]^{\mx^{-1}}\ar[rd]^{\cong}
\ar[dd]_{f_E} & & K/\mx^{-1}\ma \ar@{.>}[dd]^{f_E'}\ar[rd]^{\cong} & \\
%line 2
& K/\ma' \ar[rr]^{\mx^{-1}} \ar[dd]^{f_{E'}} & & K/\mx^{-1}\ma'\ar[dd]^{f'_{E'}}\\
%line 3
E(\C) \ar[rr]^{\si} \ar[rd]& & 
E^{\si}(\C)\ar[rd]^{\cong} &\\
%line 4
& E'(\C)\ar[rr] & & E'^{\si}(\C).
}
\]
%We now show the maps $f'_m|_{L_m}:m^{-1}\ma/\ma\to m^{-1}\mx^{-1}\ma/\ma$ fit together compatibly, so that we can define $f':K/\ma\to K/\mx^{-1}\ma$ by
%\[
%f':=\varprojlim_m f'_m.
%\]
%By construction, $f'_m$ and $f'_{nm}$ act the same on 
\end{proof}

\subsection{The associated Gr\"ossencharacter}
\index{Gr\"ossencharacter of elliptic curve}
The Main Theorem involved 2 different elliptic curves, and 2 different analytic isomorphisms.
%analytic isomorphisms. 
In the special case that $\si$ fixes $E$, the curves will be the same, and by nudging the map upstairs by a constant depending on $\mathbf x$, we can restate the theorem using a consistent choice of $f$. (Compare to how we specialized from Proposition~\ref{pr:ec-lift-frob} to~\ref{pr:pi-is-frob}.) 
%We restate it in a way that involves a consistent choice of $f$ throughout. Then 
The action of $\phi_L(\mathbf x)$ on the elliptic curve will ``essentially" correspond to multiplication by $\chi_{E/L}$ on $K/\ma$. 
%The frobenius endomorphism should be a Hecke character
\begin{thm}[Gr\"ossencharacter of an elliptic curve]\llabel{thm:grossen-ec}
Let $E/L$ be an elliptic curve with complex multiplication by $\sO_K$, and suppose $K\subeq L$. Let $\mathbf x\in \I_L$ and $\mathbf y=\nm_{L/K}(\mathbf x)\in \I_K$. Then there exists a unique $\al=\al_{E/L}(x)\in K^{\times}$ with the following properties.
\begin{enumerate}
\item
$\al\sO_K=(\mathbf y)$. %, where $p:\I_K\to I_K$ sends an idele to its corresponding ideal (see~(\ref{eq:idele-to-ideal})).
\item
For any fractional ideal $\ma\subeq K$ and any analytic isomorphism $f:\C/\ma\to E(\C)$, the following commutes.
\[
\xymatrix{
K/\ma \ar[r]^{\al \mathbf y^{-1}}\ar[d]^f & K/\ma \ar[d]^f\\
E(L\abe)\ar[r]^{\phi_L(\mathbf x)} & E(L\abe).
}
\]
\end{enumerate}

Moreover, defining $\chi_{E/L}:\I_L\to \C^{\times}$ by
\[
\chi_{E/L}(\mathbf x):=\al_{E/L}(\mathbf x)[\nm_{L/K}(\mx^{-1})]_{\iy},
\]
$\chi_{E/L}$ is a Gr\"ossencharacter of $K$, and $\chi_{E/L}$ is ramified at $\mP$ (i.e. $\chi_{E/L}(U_{\mP})$ is not identically 1) iff $E$ has bad reduction at $\mP$.
\end{thm}
%\begin{rem}
%Note that the image of $f$ is just the torsion points of the elliptic curve:
%\[
%f:K/\ma 
%\]
%
%We can think of this theorem as telling how to transfer the action of the Frobenius 
%\end{rem}
\begin{proof}
\noindent\underline{Part 1:} 
Since $f$ is an isomorphism, uniqueness is clear. To construct $\al$, choose any $\si\in \Aut(\C/L)$ such that $\si|_{L\abe}=\phi_L(\mx)$. We use Theorem~\ref{thm:mt-cm} with $\si$ and $\mathbf y\in \I_K$, noting the following points.
\begin{enumerate}
\item
$E^{\si}=E$ since $E$ is defined over $L$ and $\si$ fixes $L$.
\item
The image of $f$ is contained in $E(L\abe)$ as $E\tors\in E(L\abe)$ by Lemma~\ref{lem:K-ab-ext}.
\item
By compatibility of the Artin map,
$\phi_L(\mx)|_{K\abe}=\phi_K(\nm_{L/K}\mathbf x)=\phi_K(\mathbf y)$.
\end{enumerate}
%Note that , and note that  Thus Theorem~\ref{thm:mt-cm} gives
We obtain an analytic map $f'$ making the following commute.
\[
\xymatrix{
K/\ma \ar[r]^{\mathbf y^{-1}}\ar[d]^f & K/\mathbf y^{-1}\ma\ar@{->}[d]^{f'}\\
E(L\abe)\ar[r]^{\phi_L(\mx)} & E(L\abe).
}
\]
%Because $\C/\my^{-1}
Because
\[
\C/\mathbf y^{-1} \ma\cong E^{\si}(\C)\cong E(\C)\cong \C/\ma,
\]
we have that $\mathbf y^{-1}\ma$ is homothetic to $\ma$, i.e. there exists $\be$ so that $\be$ takes $K/\mathbf y^{-1}\ma$ back to $K/\ma$. Defining $f''(x)=f'(\be^{-1}x)$, we have that it differs from $f$ by some automorphism $[\ze]$: $f\circ [\ze]=f''$. Let $\al=\be\ze$. Then we can extend the above diagram as follows.
\[
\xymatrix{
K/\ma \ar[r]^{\mathbf y^{-1}}\ar[d]^f & K/\mathbf y^{-1}\ma\ar@{->}[d]^{f'}\ar[r]^{\al}  & K/\ma \ar[ld]^{f}\\
E(L\abe)\ar[r]^{\phi_L(\mx)} & E(L\abe) &
}
\]
As $\al\mathbf y^{-1}\ma=\ma$, we get $(\al)=(\mathbf y)$.

To see that $\al$ is independent of $f$ and the ideal $\ma$, let $f'$ be another analytic isomorphism $K/\ma'\to E(L\abe)$. Let the map $K/\ma'\to K/\ma$ be multiplication-by-$\ga$. Then $f(\ga x)$ is also an analytic isomorphism $K/\ma'\to E(L\abe)$. Hence $\ga^{-1} f^{-1}\circ f'$ is an automorphism $[\ze]$ of $K/\ma'$, i.e. $f'(x)=f([\ze]\ga x)$. Thus $\phi_L(\mathbf x)[f'(x)]=f'(\al \mathbf y^{-1} x)$ as well.\\
%Showing that the map is independent of $f$ is another exercise in drawing a commutative cube, and left to the reader.%. COMMUTATIVE CUBE!

\noindent\underline{Part 2:} 
$\al_{E/L}$ and hence $\chi_{E/L}$ is a homomorphism since it's clear that $\phi_L(\mathbf x\mathbf x')\circ f=f\circ \al\al'\mathbf y\mathbf y'^{-1}$, and $\phi_L(\mathbf x^{-1})\circ f=f\circ \al^{-1}\mathbf y$.

We need to check that $\chi_{E/L}(L^{\times})=1$ and that $\chi_{E/L}$ factors through a modulus. 

For the first point, note $\phi_L(L^{\times})=1$, the identity element of $G(L\abe/L)$. Let $i:K^{\times}\to \I_K$, $L^{\times}\to \I_L$ be the diagonal maps, and suppose $\mathbf x=i(x)$. We have $\mathbf y=\nm_{L/K}(i(x))=i(\nm_{L/K}(x))$. Then $\al$ is just the element such that $\al \nm_{L/K}(x)^{-1}$ induces the identity map, i.e. $\al=\nm_{L/K}(x)=[\nm_{L/K} \mathbf x]_{\iy}$, so $\chi_{E/L}(\mathbf x)=1$. 
%First, the fact that $\al\nm_{L/K}(\mx)^{-1}\ma=\ma$ gives $\al\nm_{L/K}(\mx^{-1})

For the second point, fix $m\ge 3$ ($m=3$ works fine). We'll show that for any idele $\mathbf x$ in a small enough open subset of finite index, $\phi_{L}(\mathbf x)$ acts just like multiplication by $\al_{E/L}(\mathbf x)$ {\it and} fixes $E[m]$, without the extra $\nm_{L/K}(\mathbf x)_{\iy}$ factor, so that $\al$ will actually be 1.

Let $B_m$ be the kernel of the Artin map $\I_L\to G(L(E[m])/L)$ (abelian by Lemma~\ref{lem:K-ab-ext}), so that it induces an isomorphism
\beq{eq:LEm}
\phi_{L(E[m])/L}:\I_L/B_m\xra{\cong} G(L(E[m])/L).
\eeq
We show that 
\[
U_m:=B_m \cap L^{\times}\pa{\nm_{L/K}^{-1}\mathbb U_K(1,m)
}\subeq \ker \chi_{E/L}.
\]
This is of finite index in $\I_L$ since $B_m$ is open of finite index in $\I_L$ and $K^{\times}\mathbb U_K(1,m)$ is open of finite index in $\I_K$.

Fixing an analytic isomorphism $f:\C/\ma\xra{\cong}E(\C)$, we get that for any $t\in m^{-1}\ma/\ma$ and any $\mathbf x\in U_m$, $f(t)\in E[m]$ so 
%RSince $\phi_L(\mathbf x)$ fixes $L$, we get
\bal
f(t)&=f(t)^{\phi_L(\mathbf x)}&\text{by~(\ref{eq:LEm}) and }\mathbf x\in B_m\\
&=f(\al\nm_{L/K}(\mathbf x)^{-1}t)&\text{by the Main Theorem~\ref{thm:mt-cm}}\\
&=f(\al t)&t\in m^{-1}\ma/\ma\text{ and }\nm_{L/K}(\mathbf x)_{\mfp}\equiv 1\pmod{m\sO_{K_{\mfp}}}\text{ for all }\mfp.
\end{align*}
Thus multiplication by $\al$ fixes $m^{-1}\ma/\ma$, i.e. $\al\equiv 1\pmod{m\sO_K}$. Note $\nm_{L/K}(\mathbf x)^{-1}\in \mathbb U_K(1,m)$, so
\[
(\al)=(\mathbf y)=(\nm_{L/K}(\mathbf x))=\sO_K
\]
and $\al$ is a unit. Together with $\al\equiv 1\pmod{m\sO_K}$, we get $\al=1$.\footnote{Any number in the form $m\tau+1$,  $\tau\in \sO_K$ with norm 1 has norm at least $(\nm_{K/\Q} (m)-1)^2-1$, by the triangle inequality. In order for it to have norm 1, $\tau=0$.}\\

\noindent{\underline{Part 3:}} 
The relationship between ramification and bad reduction hinges on the N\'eron-Ogg-Shafarevich Criterion. See~\cite[pg. 169-170]{Si94}.
%unfinished
%We have the following chain of equivalences.
%\begin{enumerate}
%\item
%$E$ has good reduction at a prime $\mP$.
%\item 
%The inertia group $I_{\mP}(K\abe/K)$ acts trivially on $E[m]$ for infinitely many $m\perp \mP$. ((1)$\iff$(2) is N\'eron-Ogg-Shafarevich.)
%\item For all $x\in U_{\mP}$ and $t\in m^{-1} \ma/\ma$,
%\[
%f(t)^{\phi_L(\mx)}=f(t).
%\]
%Note (2)$\iff$(3) follows from the fact from local class field theory that
%\[
%\phi_K(i_{\mP}(U_{\mP}))=\phi_{K_{\mP}}(U_{\mP})=I_{\mP}\abe
%\]
%\fixme{add REF.}
%Here we think of $I_{\mP}$ inside $G(\ol K/K)$ by the map $I_{\mP}\hra D_{\mP}\hra G(\ol K/K)$. %%%%write out
%\item For all $\mathbf x\in U_{\mP}$ and $t\in m^{-1}\ma/\ma$,
%\[
%f(\al_{E/L}(\mx) \nm_{L/K}(\mx)^{-1}t)=f(t).
%\]
%\item
%$\chi_{E/L}(\mathbf x)\equiv 1\pmod{m\sO_K}$ for all $\mathbf x\in U_{\mP}$. ((4)$\iff$(5)) is because \fixme{blah blah blah}.)
%\end{enumerate}
\end{proof}
Note that if $\chi_{E/L}$ is unramified at $\mP$, then $\chi_{E/L}(i_{\mP}(U_{\mP}))=1$, so it makes sense to talk about $\chi_{E/L}(\mP)$ (defined as $\chi_{E/L}(i_{\mP}(\pi))$ for any uniformizer $\pi$).
\begin{pr}
Let $E/L$ be an elliptic curve with CM by $\sO_K$, with $K\subeq L$. Let $\mP$ be a prime of $L$ of good reduction, let $\wt{E}$ be the reduction of $E$ modulo $\mP$. Let $\phi_{\mP}$ be the Frobenius on $\wt{E}$. 
Then the following commutes.
\[
\xymatrixcolsep{5pc}
\xymatrix{
E\ar[r]^{[\chi_{E/L}(\mP)]}\ar[d]
\ar[d] & E\ar[d]\\
\wt E\ar[r]^{\phi_{\mP}} & \wt E
}
\]
\end{pr}
%Compare to Proposition~\ref{pr:pi-is-frob}.
\begin{proof}
Let $\pi$ be a uniformizer of $L_{\mP}$, and let $\varpi
=i_{\mP}(\pi)$. 
Note that $\varpi_{\iy}=1$. Hence $\nm_{L/K}(\varpi)_{\iy}=1$, giving
\[
\chi_{E/L}(\mP)=\chi_{E/L}(\varpi)=\al_{E/L}(\varpi).
\]
If $m$ is an integer such that $\mP\nmid m$, then $\nm_{L/K}(\varpi)$ fixes $m^{-1}\ma/\ma$ (since it is 1 at all $\mQ$ with  $\mQ\mid m$). Then
\bal
f(t)^{\phi_L(\varpi)}&=f([\al_{E/L}(\varpi)]\nm_{L/K}(\varpi)^{-1}t)&\text{definition of }\al_{E/L}\\
&=f([\chi_{E/L}(\mP)]\nm_{L/K}(\varpi)^{-1}t)\\
&=[\chi_{E/L}(\mP)]f(\nm_{L/K}(\varpi)^{-1}t)&f\text{ preserves the action of }\sO_K\\
&=[\chi_{E/L}(\mP)]f(t)&\nm_{L/K}(\varpi) \text{ fixes } m^{-1}\ma/\ma.
\end{align*}
Modulo $\mP$, $\phi_L(\varpi)$ is just the $q$th power Frobenius map, so we get
\[
\phi_{\mP}|_{\wt E[m]}=\wt{[\chi_{E/L}(\mP)]}|_{E[m]}.
\]
Since an isogeny is determined by its action on $E[m]$ for $m\to \iy$ (the kernel of a nonzero isogeny is finite), we get that this is true for $E$, not just $E[m]$, as needed.
\end{proof}
%$E$ has good reduction at a prime $\mP$, iff the inertia group $I_{\mP}\abe$ acts trivially on $E[m]$ for infinitely many $m\perp \mP$. From local class field theory, we know
%\[
%\phi_K(i_{\mP}(U_{\mP}))=\phi_{K_{\mP}}(U_{\mP})=I_{\mP}\abe
%\]
%where we think of $I_{\mP}$ inside $G(\ol K/K)$ by the map $I_{\mP}\hra D_{\mP}\hra G(\ol K/K)$. Hence, $I_{\mP}\abe$ acts trivially on $E[m]$ iff
%\[
%f(t)^{\phi_L(\mx)}=f(t)\text{ for all }x\in U_{\mP},\, t\in m^{-1} \ma/\ma.
%\]
%This becomes
%\bal
%f(\al_{E/L}(\mx) \nm_{L/K} \mx^{-1}t)=f(t)\text{ for all }\mathbf x\in U_{\mP}, t\in m^{-1}\ma/\ma
%\eal
%which is true iff $\psi_{E/L}(x)\equiv 1\pmod{m\sO_K}$ for all $\mathbf x\in U_{\mP}$.

%$\chi_{E/L}(\mathbb U_K(1,\mm))=1$ (\fixme{change other notation!}). 

To study the Galois representation $G(\ol K/H_K)\to \Aut E\tors$ of $E$, we reduce modulo a prime $\mP$ of $L$, and show that on this reduced curve, the $q$th power Frobenius acts exactly as multiplication by the Gr\"ossencharacter. In particular, the $q$th power Frobenius is represented by multiplication by $\chi_{E/L}(\mP)$ when we think of $E\tors$ as $K/\ma$. Thinking of $E\tors$ as a 2-dimensional space $\Q^2$, this says exactly that the eigenvalues of the Frobenius acting on $E\tors$ is exactly $\chi_{E/L}(\mP)$ and $\ol{\chi}_{E/L}(\mP)$. Typically we just restrict our attention to $\ell$-power torsion points for some $\ell$.
%Thus thinking of this as giving a representation $G(\ol K/K)\to \Aut(E\tors)$, an element $\phi_L(\mP)$ gets sent to 

\section{$L$-series of CM elliptic curve}
\llabel{sec:l-series-cmec}
\subsection{Defining the $L$-function}
We define the $L$-series of an elliptic curve as the $L$-series of the corresponding Galois representation.
\index{L-series of elliptic curve}
\begin{df}%See def of GReps!
\llabel{df:E-L1}
Let $E$ be an elliptic curve defined over $K$, and $\rh_{\ell}$ the associated Galois representation $G(\ol K/K)\to \Aut V_{\ell}E\cong \GL_2(\Q_{\ell})$.

Define the \textbf{local $L$-factor} of $E$ at a prime $\mfp$ of $K$ as follows. Choose $\ell$ such that $\mfp\nmid \ell$, and let 
\[
L_{\mfp}(E,s):=L_{\mfp}(\rh_{\ell},s)=\det(1-q^{-s}\Frob(\mfp)| (V_{\ell}E)^{I_{\mfp}})^{-1},
\]
where $q=\fN\mfp$ and $I_{\mfp}$ is the inertia subgroup of $G(\ol K/K)$. (Choose an embedding $\Q_{\ell}\hra \C$.)
The \textbf{$L$-series} of $E$ is the product of local factors
\[
L_{\mfp}(E/K,s):=\prod_{\mfp} L_{\mfp}(E,s).
\]
\end{df}
\begin{rem}
This is (almost) the same as saying: fix a prime $\ell$ and let  $L(E/K,s):=L(\rho_{\ell},s)$. The only difference is that we run into trouble with the local factor $L_{\mfp}(\rh_{\ell},s)$ on the right hand side, so we have to choose a different $\ell'$ and let this local factor be $L_{\mfp}(\rh_{\ell'},s)$ instead.
\end{rem}
The following is an equivalent definition (that is more concrete).
\begin{df}\llabel{df:E-L2}
Let $N$ be the conductor\footnote{$N$ is divisible by exactly the primes of bad reduction} of the elliptic curve $E$. Define the local $L$-factor by
\[
L_{\mP}(E,s)=1-a_qq^{-s}+\chi(q)qq^{-2s},\quad a_q=q+1-|E(\F_q)|,\quad \chi(q)=\begin{cases}
1,&m\perp N\\
0,&\text{else}
\end{cases}
\]
where $q=\fN \mfp$. 
Thus
\[
L_v(E,s)=\begin{cases}
1-a_qq^{-s}+qq^{-2s},&\text{good reduction}\\
1-q^{-s},&\text{split multiplicative reduction}\\
1+q^{-s},&\text{non-split multiplicative reduction}\\
1,&\text{additive reduction.}
\end{cases}
\]
\end{df}
Note that $a_q$, the ``trace of Frobenius," is related to the number of points of $E$ over $\F_q$. Hence the $L$-function contains information about the number of points of $E$ over each $\F_q$.

Showing that these two definitions are equivalent requires us to show that $(V_{\ell}E)^{I_{\mfp}}$ is 2, 1, or 0-dimensional when $E$ has good, multiplicative, and additive reduction, respectively. The general idea is that the action of $I_{\mfp}$ on $V_{\ell}E$ contains exactly the information lost by looking at the reduced elliptic curve, since $I_{\mfp}$ is exactly the kernel of $D_{\mfp}(\ol K/K)\to G(\ol k/k)$, so nontrivial action of $I_{\mfp}$ corresponds to bad reduction.

In the CM case, we cannot have multiplicative reduction, so the $L$-series is particularly simple. We will show that the two definitions are equivalent in this case.
\begin{thm}\llabel{thm:cmec-no-mr}
Let $E/K$ be a CM elliptic curve. Then $E$ cannot have multiplicative reduction at any prime.
\end{thm}
\begin{proof}
An elliptic curve $E$ has potential good reduction iff its $j$-invariant is integral~\cite[VII.5.5]{Si86}. CM have integral $j$-invariants, so have potential good reduction, i.e. have good or multiplicative reduction.  
%is preserved by finite field extension, so $E$ must have good reduction over $K$ already.
\end{proof}
\begin{proof}[Proof that Definitions~\ref{df:E-L1} and~\ref{df:E-L2} are equivalent in the CM case]
Suppose $E$ has CM by an order $\sO$ in $K$, and $E$ is defined over $L$. 
By N\'eron-Ogg-Shafarevich, $I_{\mfp}$ acts trivially on $V_{\ell}E$ iff $E$ has good reduction at $\mfp$. Let $q=\fN\mfp$.

In the case of good reduction we need to show $\det(1-q^{-s}\Frob(\mfp)|V_{\ell}E)=1-a_qa^{-s}+qq^{-2s}$.
Every endomorphism $\phi$ on $E$ satisfies $\phi^2-\tr(\phi)\phi+\deg(\phi)=0$, where $\tr(\phi)=1+\deg(\phi)-\deg(1-\phi)$. Since $\Frob(\mfp)$ acts as the Frobenius morphism $\phi_{\mfp}$, its characteristic polynomial is
\[
\det(\la-\Frob(\mfp))= \la^2-\tr(\phi_{\mfp})\la+\deg(\phi_{\mfp}).
\]
But
\begin{align*}
\deg(\phi_{\mfp})&=q\\
\tr(\phi_{\mfp})&=1+\deg(\phi_{\mfp})-\deg(1-\phi_{\mfp})\\
&=q+1-\ker(1-\phi_{\mfp})\\
&=q+1-|E(\F_q)|.
\end{align*}
(This part of the proof doesn't use the fact that $E$ has CM.)

Since $E$ has no multiplicative reduction by Theorem~\ref{thm:cmec-no-mr}, it remains to prove that $W:=(V_{\ell}E)^{I_{\mfp}}=0$ when $E$ has multiplicative reduction. We know by N\'eron-Ogg-Shafarevich that $\dim(W)\le 1$. But because $E$ is CM, $V_{\ell}E\cong (\varprojlim_n \ell^{-n}\ma/\ma)\ot \Q$ has the structure of a $\sO_K\ot \Q_{\ell}$-vector space. If $a\in W$, then for any $\al\in K$, $\al a\in W$ because $[\al]$ commutes with the Galois action. Hence $W$ is not just a $\Q_{\ell}$-subspace of $V$, but also a $\sO_K\ot \Q_{\ell}$-subspace. Hence its dimension over $\Q_{\ell}$ is even, and must be 0.
\end{proof}
\subsection{Analytic continuation}
\begin{thm}[Deuring]\llabel{thm:cmec-l}
Let $E/L$ be an elliptic curve with CM by $\sO_K$ with $K\subeq L$.Then
\[
L(E/L,s)=L(s,\psi_{E/L})L(s,\ol{\psi_{E/L}}).
\]
\end{thm}
\begin{cor}[Analytic continuation of $L$-function for CM elliptic curves]\llabel{cor:cmec-l-ac}
Let $E/L$ be an elliptic curve with CM by $\sO_K$. Then $L$ admits an analytic continuation to $\C$ and satisfies a functional equation relating its values at $s$ and $2-s$.
\end{cor}
This theorem for general elliptic curves is very deep (it follows from the Modularity Theorem and the analytic properties of $L$-functions associated to modular forms).
\begin{proof}[Proof of Theorem~\ref{thm:cmec-l}]
By Theorem~\ref{thm:cmec-no-mr}, $E$ has no multiplicative reduction. Let $\mP$ be a prime, and consider 2 cases.
\begin{enumerate}
\item
$E$ has good reduction at $\mP$. Choose any $\ell$ not dividing $\mP$. The characteristic polynomial of the action of $\phi_{\mP}$ on $V_{\ell}E$ is 
$\det(\la- \phi_{\mP}|V_{\ell}E)$. However, if we make the identification $E\tors\cong K/\ma$, we have
\[
V_{\ell}E=\varprojlim \ell^{-n}\ma/\ma,
\]
and we know that $\phi_{\mP}$ acts on $E\tors\cong K/\ma$ as multiplication by $\chi_{E/L}(\mP)$. Therefore, the eigenvalues of the action of $\phi_{\mP}$ on $V_{\ell}E$ are just $\chi_{E/L}(\mP)$ and $\ol{\chi}_{E/L}(\mP)$, and 
\[
\det(\la- \phi_{\mP}|V_{\ell}E)=(\la-\chi_{E/L}(\mP))(\la-\ol{\chi}_{E/L}(\mP)).
\]
Taking $\la=p^{s}$ and dividing by $p^{2s}$ gives
\[
L_{\mP}(E/L,s)=\det(1- p^{-s}\phi_{\mP}|V_{\ell}E)
=L_{\mP}(s,\chi_{E/L})L(s,\ol{\chi}_{E/L}).
\]
\item
$E$ has bad reduction at $\mP$. Then $\chi_{E/L}(\mP)=0$ by definition, and $L_{\mP}(E/L,s)=1=(1-\chi_{E/L}(\mP))(1-\ol{\chi}_{E/L}(\mP))=L_{\mP}(s,\chi_{E/L})L(s,\ol{\chi}_{E/L})$.
\end{enumerate}
Multiplying together all the local factors gives the result.
\end{proof}
\begin{proof}[Proof of Corollary~\ref{cor:cmec-l-ac}]
The $L$-functions of Gr\"ossencharacters have analytic continuation (Theorem~\ref{ch:cft-app}.\ref{thm:l-analytic-cont}, which works for Gr\"ossencharacters as well). Thus the result follows directly from Theorem~\ref{thm:cmec-l}.
\end{proof}
Thus we have carried out the program in Section~\ref{ch:cft-app}.\ref{sec:intro-langlands} for CM elliptic curves, to get the correspondences.
\[
\text{(CM Elliptic curves)}\rightarrow \text{(Galois representation)}
\rightarrow \text{(2 Gr\"ossencharacters)}\\
\]
Remember Gr\"ossencharacters are 1-dimensional automorphic representations. If we wanted a modular form, we can use the technique of {\it automorphic induction} to construct a modular form from 2 Gr\"ossencharacters.

\chapter{Modular curves}
\bibliographystyle{plain}
\bibliography{\filepath/refs}
\printindex
\printnomenclature
\end{document}
