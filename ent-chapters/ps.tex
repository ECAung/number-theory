%Partial summation
\section{Finite calculus}
\begin{thm}[Summation by parts, Abel summation]\llabel{sum-parts}
Suppose that $u$ is an arithmetic function, and let
\[
U(x)=\sum_{n\le x} u(x).
\]
Then for $m,n\in \N$ %(or $\Z$, if $u$ is defined over $\Z$),
\[
\sum_{x=m}^n u(x)v(x)=U(n)v(n)-U(m-1)v(m-1) -\sum_{x=m}^n
U(x-1)(v(x)-v(x-1)).
\]
%\end{thm}
%\begin{thm}[Abel summation]\llabel{abel-sum}
%Keep the above notation and suppose 
If $0\le a<b$ and $v$ has continuous derivative on $a<x<b$, then
\[
\sum_{a\le x\le b} u(x)v(x)=U(b)v(b)-U(a)v(a) -\int_{a}^b U(x)v'(x).
\]
\end{thm}
\begin{proof}
We imitate the proof of integration by parts. For a function $f$ define the function
\[
\De_-(f)=f(x)-f(x-1).
\]
This is the discrete analogue of differentiation. It is the inverse of summation in the sense that by telescoping,
\begin{equation}\label{telescope}
\sum_{x=m}^n\De_-(f)=f(n)-f(m-1).
\end{equation}
Note that $\De_-(U)=u$. 
We have the ``product rule"
\begin{align*}
\De_-(uv)&=u(x)v(x)-u(x-1)v(x-1)\\
&=(u(x)-u(x-1))v(x)+u(x-1)(v(x)-v(x-1))\\
&=\De_-(u)v+E_-u\De_-(v)
\end{align*}
where $E_-$ is the left shift operator $(E_-f)(x)=f(x-1)$. 
Replacing $u$ by $U$ and rearranging gives
\[
uv=\De_-(Uv)-E_-U\De_-(v).
\]
Summing over $m\le x\le n$ and telescoping using~(\ref{telescope}) gives
\[
\sum_{x=m}^n u(x)v(x)=U(n)v(n)-U(m-1)v(m-1) -\sum_{x=m}^n
U(x-1)(v(x)-v(x-1)).
\]
When $v$ has continuous derivative, noting $U(t)=U(\fl{t})$, we have
\begin{align*}
\sum_{x=m}^n
U(x-1)(v(x)-v(x-1))
&=\sum_{x=m}^n \int_{x-1}^{x}U(t)v'(t)\,dt\\
&=\int_{m-1}^n U(t)v'(t)\,dt.
\end{align*}
For general $a,b$, since $U$ is constant on $(\fl{b},b)$ and $(a,\fl{a}+1)$,
\begin{align*}
\sum_{a<x\le b} u(x)v(x)
&=\sum_{x=\fl{a}+1}^{\fl{b}} u(x)v(x)\\
&=U(\fl{b})v(\fl{b}) - U(\fl{a}) v(\fl a) +\int_{\fl{a}}^{\fl{b}} U(t)v'(t)\,dt\\
&=U(b)v(b) - U(a) v(a) +\int_{a}^{b} U(t)v'(t)\,dt\qedhere
\end{align*}
\end{proof}
