\chapter{Unique factorization}\label{uf}
\section{Unique factorization domains}
%We prove that when $K$ be a field, unique factorization holds in $K[x]$. It is the same strategy used to prove that integers can be factored uniquely. (For definitions of ring, field, and integral domain, see Lecture 8.)
MOTIVATION

First, we define what exactly unique factorization means. Let $R$ be an integral domain.
\begin{df}
An element $a\in R$ is \textbf{irreducible} if it is not a unit, and its only factors are units and associates. A unit is an invertible element in $R$, while an associate of $a$ is a unit times $a$.
\end{df}
For the positive integers we often just say $a$ is irreducible if $a\neq 1$, and its only factors are 1 and itself. However, if we work with the integers, then there will also be the factors $-1$ and $-a$, and we don't want to view these as different. For example, 5 is irreducible over the integers because its only factors are units, $\pm1$, and associates, $\pm 5$.

\begin{df}
A \textbf{unique factorization domain (UFD)} is a integral domain where factoring terminates and every nonzero, nonunit element factors uniquely into irreducible elements. That is, if 
\[
a=p_1\ldots p_m=q_1\ldots q_n,
\]
%where $u$ and $v$ are units, 
and $p_1,\ldots, p_m,q_1,\ldots, q_n$ are irreducible elements, then $m=n$ and we can reorder the $q_i$'s so that $p_i$ is an associate of $q_i$, for each $i$.
\end{df}
For example, we regard $6=2\cdot 3=-2\cdot -3$ as the same factorization.

%Now what could hamper unique factorization? If we can't factor a number completely, it will fail, but in all the rings we'll be working with, it is clear that we will factor into irreducible elements in a finite number of steps. 
Unique factorization doesn't hold for all domains---for example, consider $\Z[\sqrt{-5}]$, that is, numbers of the form $a+b\sqrt{-5}$. Then
\[
(1+\sqrt{-5})(1-\sqrt{-5})=6=2\cdot 3
\]
are two factorizations of 6 into irreducible elements. 

The notion of a prime is related to that of an irreducible element. People use them as synonyms in elementary math---because they coincide for the integers---but the distinction between them will be quite important for us.
\begin{df}
A \text{prime} in $R$ is an element $p$, not a unit, such that if $p|ab$ then $p|a$ or $p|b$.
\end{df}
This tells us that if a prime $p$ divides $a$, then {\it no matter how we factor }$a$, we can't avoid $p$ dividing one element of $a$. The connection between primes, irreducibles, and unique factorization is given by the following.
\begin{lem}
If $R$ is a ring where factoring terminates, and every irreducible element is prime, then $R$ is a UFD. Conversely, in a UFD, every irreducible element is prime.
\end{lem}
\begin{proof}
Suppose $a=p_1\ldots p_m=q_1\ldots q_n$ are two factorizations into irreducible elements. Since $p_1$ is irreducible, it is prime, and hence must divide one of the $q_i$. Since $q_i$ is irreducible, its only factors are units and associates, so $p_1$ must be associated with $q_i$. Then we can cancel them, leaving a unit. Repeating this process, every factor in the left factorization is paired with one in the right factorization.

For the converse, suppose $p$ is irreducible and $p|ab$. Then $pd=ab$ for some $d$. Factoring $a$, $b$, and $d$ shows that $p$ must divide one of the factors of $a$ or $b$ by unique factorizaton.
\end{proof}
(Note that primes are always irreducible, because if $p=ab$ were a proper factorization, then $p\nmid a$ and $p\nmid b$.)

The main strategy for proving unique factorization is the following.
\begin{enumerate}
\item Show that the ring $R$ in question (here, $K[x]$) admits \textbf{division with remainder}, with some measure of size so that the remainder is smaller than the quotient.
\item Show that if we have division of remainder, then \textbf{greatest common divisors} exist, and moreover that they have the nice property given by B\'ezout's Theorem.
\item Show that this implies that all irreducible elements are prime, and hence $R$ is a UFD.
\end{enumerate}
The advantage of such an abstract approach lies in the fact that it works for a variety of different number systems. In particular, once we've shown items 2 and 3, then given any ring, we only have to show that we can have division with remainder, and it will follow that it is a UFD. This simultaneously shows unique factorization for $\Z$, $K[x]$, and even $\Z[i]=\{a+bi|a,b\in \Z\}$.\footnote{The converse is not true; a UFD is not necessarily a PID or Euclidean domain. For example $\Z[\frac{1+\sqrt{-163}}{2}]$ is a UFD but not an Euclidean domain.}

In the language of abstract algebra, the above steps are phrased as follows:
\begin{enumerate}
\item $R$ is an Euclidean domain.
\item An Euclidean domain is a principal ideal domain.
\item A principal ideal domain is a unique factorization domain.
\end{enumerate}
We now carry out this program.

\subsection{Step 1: Euclidean domains}
%First we formalize several concepts from Josh's lecture (Introduction to Number Theory).
\begin{df}
An integral domain $R$ is an \textbf{Euclidean domain} if there is a function $|\cdot |:R\to \mathbb N_0$ (called the norm) such that the following hold.
\begin{enumerate}
\item $|a|=0$ iff $a=0$.
\item For any nonzero $a,b\in R$ there exist $q,r\in R$ such that $b=aq+r$ and $|r|<|a|$.
\end{enumerate}
\end{df}
Note that both the integers $\Z$ and $K[x]$ are Euclidean domains. The norm on $\Z$ is simply the absolute value, while the norm on $K[x]$ is the degree of the polynomial. Theorem~\ref{dwr} shows that $K[x]$ is an Euclidean domain.

\subsection{Step 2: Euclidean domain$\implies$PID}
We'd like to prove B\'ezout's Theorem for an Euclidean domain, that given $a,b$ in $R$ there exists a greatest common divisor $g$ and $s,t$ so that $as+bt=g$. Rather than thinking of this as an equation in variables $s,t$, we can think of it as an equation in sets $(a)$ and $(b)$, where $(x)$ denotes the set of multiples of $x$. For two sets $S,T$ we define $S+T=\{s+t|s\in S,t\in T\}$; then it turns out what we want is
\[
(a)+(b)=(g).
\]
(See Lemma~\ref{pidbez} below.)

\begin{df}
An \textbf{ideal} in a ring $R$ is a subset $I$ such that if $a,b\in I$ then $ra,a+b\in I$ for any $r\in R$. A principal ideal is an ideal generated by one element, that is, there is a $a$ such that $I=\{ra|r\in R\}$. We write $I=(a)$.

A \textbf{principal ideal domain} (PID) is a integral domain where every ideal is principal.
\end{df}

\begin{thm}
An Euclidean domain is a PID.
\end{thm}
\begin{proof}
Let $R$ be an Euclidean domain, $I\subseteq R$ and ideal, and $b$ be the nonzero element of smallest norm in $I$.
Suppose $ a\in I$. Then we can write $ a = qb + r$ with $ 0\leq r < |b|$, but since $ b$ has minimal nonzero norm, $ r = 0$ and $ b|a$. Thus $ I=(b)$ is principal.
\end{proof}

\begin{lem}\label{pidbez}
A PID satisfies B\'ezout's Theorem.
\end{lem}
\begin{proof}
Let $R$ be a PID. Since every ideal in $R$ is principal, for every $a,b$ (not both 0) we have $(a)+(b)=(d)$
for some $d\in R$. (Note the sum of two ideals is an ideal---check this for yourself.) This says there exist $ s,t\in R$ such that
\[as + bt = d.\]
From this, any divisor of $a,b$ must divide $ d$. Furthermore, $d$ must divide both $a$ and $b$ since $a=a+0$ and $b=0+b$ are both in $(a)+(b)= (d)$. In other words, $d$ is the greatest common divisor of $ a,b$.
\end{proof}
\subsection{Step 3: PID$\implies$ UFD}
\begin{thm}
A PID is a UFD.
\end{thm}
\begin{proof}
Suppose $ p$ is irreducible; we show $ p$ is prime. Suppose $ p|ab$ but $ p$ does not divide $ a$. Then using Bezout's Theorem and the fact that $ a$ and $p$ are relatively prime, we get $ as + pt =1$ for some $ s,t$. Multiply by $ b$ to get
\[ abs + ptb = b.\]
Since $ p|ab|abs, p|ptb$, we have $ p|b$. This shows that irreducible elements are prime in $ \mathbb{Z}$.

It remains to show factoring terminates.\footnote{
This argument is not needed for our purposes: Both $\Z$ and $K[x]$ are Euclidean domains, and factoring must terminate for them because factors always have smaller norm (absolute value and degree, respectively).
}
Otherwise, there would be an infinite sequence of nonassociated elements $a_1,a_2,\ldots \in R$ such that $a_{i+1}|a_i$. Then $(a_1)\subset (a_2)\subset\cdots$. However, $\bigcup_{i\geq 1} (a_i)$ is an ideal, so it is principal, say generated by $b$. Then $b\in (a_i)$ for some $i$; this implies that $(b)=(a_i)$. Hence $(a_i)=(a_{i+1})=\cdots$, a contradiction.

Since irreducible elements are prime and every nonzero element of $R$ factors into irreducibles, $R$ is a UFD.
\end{proof}

\begin{cor}
$\mathbb Z$ and $K[x]$ are UFDs.
\end{cor}
\section{Example: $x^2+y^2=n$}
\begin{thm}\label{number-sum-of-squares}
Let $n$ be a positive integer. Then the equation 
\[x^2+y^2=n\]
has a solution in integers iff every prime $p\equiv 3\pmod 4$ appears in $n$ with even exponent.

If $ n=2^ap_1^{b_1}\cdots p_k^{b_k}q_1^{c_1}\cdots q_m^{c_m}$ where $ p_j$ and $ q_j$ are primes congruent to 1, 3 modulo 4, then the equation $x^2+y^2=n$ has
\[4(b_1+1)\cdots (b_k+1)\]
solutions in integers.
\end{thm}
\begin{proof}
Each solution to $ x^2+y^2=n$ corresponds to a factoring $ (x+yi)(x-yi)=n$ over the Gaussian integers $ \mathbb Z[i]$. Thus the number of solutions is the number of $ z$ such that $ z\bar{z}=n$, or 4 times the number of nonassociated $ z\in \mathbb Z[i]$ such that $ z\bar{z}=n$. (Two Gaussian numbers are associated if they differ by a unit $ \pm 1, \pm i$, so $ x+yi, -y+xi, -x-yi, y-xi$ are considered the same.)

Now factor $ n=2^ap_1^{b_1}\cdots p_k^{b_k}q_1^{c_1}\cdots q_m^{c_m}$ where $ p_j$ and $ q_j$ are primes congruent to 1, 3 modulo 4, respectively. From knowledge of factoring in $ \mathbb{Z}[i]$ we know that
\begin{enumerate}
\item 2 ramifies in $ \mathbb Z[i]$, that is, it is the product of two associated primes $ 1+i,1-i$.
\item The $ p_j\equiv 1\pmod 4$ split, that is, $ p_j=z_j\bar{z_j}$ where $ z$ is prime in $ \mathbb{Z}[i]$ and not associated to $ \bar z$.
\item The $ q_j\equiv 3\pmod 4$ remain prime.
\end{enumerate}
Now if $ z\bar z=n$ and a Gaussian prime divides $ z$, then its conjugate must divide $ \bar z$. Thus, since we have unique factorization in $ \mathbb{Z}[i]$, each such $ z$, up to multiplication by associates, corresponds to a way of splitting the prime factors of $ n$ into complex conjugate pairs. We note the following:
\begin{enumerate}
\item The factors $ q_j$ are their own conjugates, so $ z$ and $ \bar z$ must each get $ q_j^{c_j/2}$. If one of the $ c_j$ is odd there is no solution. So we suppose they are all even.
\item It doesn't matter how the prime factors of $ 2^a$ are split since they are all associates.
\item There are $ b_j+1$ ways to split the factors of $ q_j^{b_j}$, since we can have either $ z_j^{b_j}$, or $ z_j^{b_j-1}\bar{z_j}$,... or $ \bar z_j^{b_j}$ divide $ z$. Thus there are $ (b_1+1)\cdots (b_k+1)$ solutions to $ z\bar z=n$ up to associates.
\end{enumerate}
\end{proof}
A similar argument works for the equations $x^2+2y^2=n$ and $x^2+xy+y^2=n$.
\section{Problems}
\begin{enumerate}
\item \label{ideal-problems}(The power of ideals) We rephrase some earlier results that used the ``division with remainder arguement" in terms of ideals.
\begin{enumerate}
\item Let $n>1$ and $a$ be relatively prime to $n$. Show that
\[
\set{m}{a^m\equiv 1\pmod n}
\]
is an ideal.
\item Conclude Proposition~\ref{multiplicative}.\ref{order-pid}.
\end{enumerate}
\item For which $n$ does $x^2+2y^2=n$ have a solution? How many solutions are there? How about $x^2+xy+y^2=n$?
\end{enumerate}