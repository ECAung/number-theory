\chapter{Field Theory}\llabel{field}
An \textbf{extension} of a field $F$ is a field containing $F$.
\begin{enumerate}
\item
A \textbf{number field} is a subfield of $\C$.
\item
A \textbf{finite field} has finitely many element.
\item
A \textbf{function field} is an extension of $\C(t)$.
\end{enumerate}

\section{Algebraic elements}
\begin{df}
Let $L$ be an extension of $K$ and $\al$ be an element of $L$. $\al$ is \textbf{algebraic} over $K$ if it is the zero of a polynomial in $K[x]$, and \textbf{transcendental} otherwise.
\end{df}
Note $\al$ is transcendental if and only if the substitution homomorphism $\ph:K[x]\to L$ is injective.

\section{Degree of a field extension}
\begin{df}
The degree $[L:K]$ is the dimension of $L$ as a $K$-vector space.
\end{df}
\section{Fundamental theorem of algebra}
\begin{thm}
$\C$ is algebraically closed.
\end{thm}
In other words, every nonconstant polynomial with coefficients in $\C$ has a zero. Equivalently, every nonconstant polynomial with coefficients in $\C$ splits completely.
\begin{proof}
We first show that all polynomials with real coefficients are reducible over the complex numbers, by induction on the highest power of 2 dividing the degree. For odd degree, the the statement follows since the polynomial has different signs near at $ \pm \infty$. Now assuming the induction hypothesis, suppose $ \deg(f)=2^km$ where $ k$ is odd. Choose a splitting field $ L$ of $ f$, and write $ P(x)=(x-r_1)\cdots (x-r_n)$. Consider the polynomial
\[P_t(x)=\prod_{i\leq i<j\leq n}(x-r_i-r_j-tr_ir_j).\]
Its degree is $ \frac{n(n-1)}{2}=2^{k-1}m(n-1)$. Since its coefficients are symmetric polynomials in the $ r_i$, by hypothesis it has a complex zero, i.e. $ r_i+r_j+tr_ir_j$ is real for some $ i,j$. Since this is true for infinitely many values of $ t$, we must have that $ r_i+r_j+tr_ir_j$ is real for all $ t$ some $ i,j$. This means $ r_i+r_j$ and $ r_ir_j$ are both real. Then $ r_i,r_j$ are roots of the quadratic $ x^2-(r_i+r_j)x+r_ir_j$ so they are complex roots of $ P(x)$. This concludes the induction.

Next for an arbitrary polynomial $ P(x)$, consider the real polynomial $ P(x) \overline{P(x)}$. (We take the conjugate of the coefficients, not $ x$.) By the above, it factors entirely into linear factors. $ P(x)$ divides $ P(x) \overline{P(x)}$, so it splits as well.
\end{proof}
\section{Constructions}
\chapter{Finite fields}\label{finite-fields}
\index{finite fields}
\section{Finite fields}
A finite field is a vector space over $\F_p$ for some prime $p$, so has order $q=p^r$. The (unique) field of order $q$ is denoted by $\F_q$.
\begin{pr}\label{finite-field-pr}$\,$
\begin{enumerate}
\item
The elements in a field of order $q$ are roots of $x^q-x=0$ (everything is modulo $p$).
\item%%
$\F_q^{\times}$ is a cyclic group of order $q-1$. 
\item There exists a unique field of order $q$ up to isomorphism.
\item A field of order $p^r$ contains a subfield of order $p^k$ iff $k\mid r$. (Note this is a relation between the exponents, not the orders.)
\item The irreducible factors of $x^{q}-x=0$ over $\F_p$ are the irreducible polynomials $g\in \F_p[X]$ whose degrees divide $r$.
\item For every $r$ there is an irreducible polynomial of degree $r$ over $\F_p$.
\end{enumerate}
\end{pr}
\begin{proof}
\begin{enumerate}
\item
 The multiplicative group $\F_q$ of nonzero elements has order $q-1$. The order of any element divides $q-1$ so $\al^{q-1}=1$ for any $\al\in \F_q$.
\item
By the Structure Theorem for Abelian Groups, $\F_q^{\times}$ is a direct product of cyclic subgroups of orders $d_1\mid \cdots \mid d_k$, and the group has exponent $d_k$. Since $x^{d_k}-1=0$ has at most $d_k$ roots, $k=1$ and $d_1=q-1$.
\item
Existence: Take a field extension where $x^q-x$ splits completely. If $\al,\be$ are roots of $x^q-x=0$ then $(\al+\be)^q=\al+\be$. Since $-1$ is a root, $-\al$ is a root. The roots form a field.

Uniqueness: Suppose $K,K'$ have order $q$. Let $\al$ be a generator of $K^{\times}$; then $K=F(\al)$. The irreducible polynomial $f\in K[X]$ with root $\al$ divides $x^q-x$. $x^q-x$ splits completely in both $K, K'$, so $f$ has a root $\al'\in K'$. Then $F(\al)\cong F[x]/(f)\cong F(\al')=K'$.
\item $\F_{p^k}\subeq \F_{p^r}\implies k\mid r$: Multiplicative property of the degree.

$\F_{p^k}\subeq \F_{p^r}\Leftarrow k\mid r$: $p^r-1\mid p^k-1$. Cyclic $\F_{p^r}^{\times}$ contains a cyclic group of order $p^k$. Including 0, they are the roots of $x^{p^k}-x=0$ and thus form a field by 3a.
\item $\implies$: Multiplicative property.

$\Leftarrow$: Let $\be$ be a root of $g$. If $k\mid r$, by 4, $\F_q$ contains a subfield isomorphic to $F(\be)$. $g$ has a root in $\F_q$ so divides $x^q-x$.
\item $\F_q$ ($q=p^r$) has degree $r$ over $\F_p$ and has a cyclic multiplicative group generated by an element $\al$. $\F_p(\al)$ has degree $r$ over $\F_p$.
\end{enumerate}
\end{proof}
To compute in $\F_q$, take a root $\be$ of the irreducible factor of $x^q-x$ of degree $r$; $(1,\be,\ldots, \be^{r-1})$ is a basis.

Let $W_p(d)$ be the number of irreducible monic polynomials of degree $d$ in $\F_p$. Then by 2,
\[
p^n=\sum_{d\mid n} dW_p(d).
\]
By M\"obius inversion,
\[
W_p(n)=\rc{n} \sum_{d\mid n} \mu\pf nd p^d.
\]
\begin{thm}
The Galois group $G(\F_{q^r}/\F_q)$ is cyclic of order $r$ generated by the Frobenius automorphism
\[
\phi(x)=x^q.
\]
\end{thm}

%%%%%%%%%%%%%
\index{primitive elements}
\begin{df}
Let $L$ be a field extension of $K$. An element $\al\in K$ such that $L=K(\al)$ is a \textbf{primitive element} for the extension.
\end{df}
\begin{thm}[Primitive element theorem]
Every finite extension of a field $K$ contains a primitive element.
\end{thm}
\begin{proof}
Need a general proof!
\end{proof}

\section{Quadratic reciprocity via finite fields}
We work in $\F_p$.
Since $\pf{p}{q}= p^{\frac{q-1}2}$, we will explicitly find an element $\al$ such that $\al^2=\pm p$. Then $\pf{p}{q}=\al^{q-1}$.

Let $\zeta_q$ be a primitive $q$th root of unity and consider the Gauss sum
\[
\al=\sum_{j=1}^{q-1} \pf{j}{q}\ze_q^j.
\]
All inverses below are modulo $q$. We calculate $\al^{q-1}$ in two different ways.\\

\noindent{\underline{Step 1:}} We calculate
\begin{align}
\nonumber
\al^2 &= \sum_{j=1}^{q-1} \sum_{k=1}^{q-1}\pf{j}{q}\pf{k}{q}\ze_q^{j+k}\\
\nonumber
&= \sum_{j=1}^{q-1} \sum_{k=1}^{q-1}\pf{jk}{q}\ze_q^{j+k}&\pf{\bullet}{q}\text{ is group homomorphism}\\
&= \sum_{s=0}^{q-1}\pa{ \ze_q^s \sum_{j=1}^{q-1} \pf{j(s-j)}{q}}\label{al-sq-qr}
\end{align}
(When $s=j$ the terms are 0.) 
\begin{enumerate}
\item When $s\ne 0$, noting $1-sj^{-1}$ ranges over $\F_q-\{1\}$ when $s$ ranges over $\F_q-\{0\}$, we have
\begin{align*}
\sum_{j=1}^{q-1} \pf{j(s-j)}{q}&=\sum_{j=1}^{q-1} \pf{-1}{q}\pf{j^2}{q}\pf{1-sj^{-1}}{q}\\
&=\sum_{j=1}^{q-1} (-1)^{\frac{q-1}2} \pf{1-sj^{-1}}{q}\\
%&=(-1)^{\frac{q-1}2}\sum_{j=1}^{q-1}  \pf{1-j}{q}\\
&=(-1)^{\frac{q-1}2} \pa{\pa{\sum_{j=0}^{q-1} \pf{j}{q}}-\pf{1}{q}}\\
&=-(-1)^{\frac{q-1}2}.
\end{align*}
The last step comes from noting that there are as many quadratic residues as nonresidues.
\item When $s=0$, we have 
\[
\sum_{j=1}^{q-1} \pf{j(s-j)}{q}=\pf{-1}{q}\pf{j^2}{q}(q-1)=(-1)^{\frac{q-1}2}(q-1).
\]
\end{enumerate}
Hence the sum~(\ref{al-sq-qr}) equals
\[
(-1)^{\frac{q-1}2}\pa{(q-1)-\sum_{j=1}^{q-1}\ze^q}
=(-1)^{\frac{q-1}2}q
\]
and
\[
\al^{p-1}=(\al^2)^{\frac{p-1}2} =[(-1)^{\frac{q-1}2}q ]^{\frac{p-1}2}=(-1)^{\frac{p-1}2\cdot \frac{q-1}2}q^{\frac{p-1}2}=(-1)^{\frac{p-1}2\cdot \frac{q-1}2}\pf qp.
\]

\noindent{\underline{Step 2:}}Since the Frobenius map is an endomorphism, we have that
\begin{align*}
\al^p&=\sum_{j=1}^{q-1} \pf{j}{q}^p\cancelto{1}{\ze_q^{pj}}\\
&=\sum_{j=1}^{q-1} \pf{jp^{-1}}{q}\ze_q^p&\text{since }p\equiv 1\pmod 2\\
&=\pf pq \sum_{j=1}^{q-1} \pf{j}{q}\ze_q^p\\
&=\pf pq \al
\end{align*}
so $\al^{p-1}=\pf pq$.

Equating the results of stpes 1 and 2 gives the result.
\section{Chevalley-Warning}\label{chevalley-warning}
\index{Chevalley-Warning Theorem}
\begin{lem}\label{ff-power-sum}
\[
\sum_{\al\in \F_q} \al^n=\begin{cases}
0&\text{ if }q-1\nmid n\\
1&\text{ if }q-1\mid n.
\end{cases}
\]
\end{lem}
\begin{proof}
If $q-1\mid n$ then $\al^n=1$ for all $\al\in \F_q$, so the sum is 0.

If $q-1\nmid n$ then (since $\F_q^{\times}\cong \Z/(q-1)\Z$) there exists $\be\in \F_q^{\times}$ such that $\be^n\ne 1$. Multiplication by $\be$ is a bijection on $\F_q$ so
\[
\sum_{\al\in \F_q}\al^n=\sum_{\al\in \F_q}(\al\be)^n
\be^n\sum_{\al\in \F_q}\al^n.
\]
Thus the sum must be 0.
\end{proof}
\begin{thm}[Chevalley-Warning]\label{chevalley-warning}
Let $f_1,\ldots f_k\in \F_q[X_1,\ldots, X_n]$ be polynomials with
\[
\sum_{j=1}^k \deg(f_j)<n.
\]
Let $V(f_1,\ldots, f_k)=\set{(x_1,\ldots, x_n)}{f_j(x_1,\ldots, x_n)=0\text{ for all }n}$. Then 
\[|V(f_1,\ldots, f_k)|\equiv 0\pmod{p}.\]
In particular, there is a nontrivial point in $V(f_1,\ldots, f_k)$.\footnote{This result says that finite fields are $C_1$ fields.}
\end{thm}
\begin{proof}
We engineer a polynomial that is 1 when $x\in V(f_1,\ldots, f_k)$ and 0 otherwise:
\[
P(X_1,\ldots, X_n):=\prod_{j=1}^k (1-f_j(X_1,\ldots,X_n)^{q-1}).
\]
Indeed, 
\[
f_j(x_1,\ldots, x_n)^{q-1}=\begin{cases} 
1,&f_j(x_1,\ldots, x_n)\ne 0\\
0,&f_j(x_1,\ldots, x_n)=0
\end{cases},
\]
so
\[
1-f_j(x_1,\ldots, x_n)^{q-1}=\begin{cases} 
0,&f_j(x_1,\ldots, x_n)\ne 0\\
1,&f_j(x_1,\ldots, x_n)=0
\end{cases},
\]
and multiplying gives the desired conclusion.

Hence we can count the number of points in $V(f_1,\ldots, f_n)$ as follows:
\begin{equation}\label{count-points}
|V(f_1,\ldots, f_n)|=
\sum_{(x_1,\ldots, x_n)\in \F_q^n} P(x_1,\ldots, x_n).
\end{equation}
Note
\[
\deg P=(q-1)\sum_{j=1}^k \deg(f_j)<(q-1)n
\]
so each term in $P(X_1,\ldots, X_n)$ is in the form
\[
X_1^{a_1}\cdots X_n^{a_n}
\]
with $a_1+\cdots + a_n<(q-1)n$; this means $a_j<q-1$ for some $j$. Then $\sum_{x_j\in \F_q} x_1^{a_1}\cdots x_n^{a_n}\equiv 0\pmod{p}$ by Lemma~\ref{ff-power-sum} so (after summing over the other $x_i$) this term contributes 0 modulo $p$ to the sum in~\ref{count-points}. Summing over all terms gives the result.
\end{proof}
\index{Erdos Ginzburg Ziv Theorem@Erd\H os-Ginzburg-Ziv Theorem}
\begin{thm}[Erd\H os-Ginzburg-Ziv]\label{erdos-ginzburg-ziv}
From any set of $2n-1$ integers there exist $n$ whose sum is divisible by $n$.
\end{thm}
\begin{proof}
We first prove the result for $n=p$ prime.

Let $S=\{a_1,\ldots, a_{2p-1}\}$. 
Associate a subset $T$ to any $(2p-1)$-tuple $(x_1,\ldots, x_{2p-1})\in \F_p$ where $x_k\ne 0$ iff $a_k\in T$. We will translate the condition on $T$ into equations in the $x_k$. 

Consider
\begin{align*}
f_1(x)&:=x_1^{p-1}+\cdots +x_{2p-1}^{p-1}\\
f_2(x)&:=a_1x_1^{p-1}+\cdots +a_{2p-1}x_{2p-1}^{p-1}
\end{align*}
in $\F_p$. 
The first equation is 0 iff $|T|\equiv 0\pmod{p}$, while the second is 0 iff $\sum_{a\in T}a\equiv 0\pmod p$. We have $\deg f_1+\deg f_2=2(p-1)<2p-1$ so by Chevalley-Warning the number of solutions is a multiple of $p$. Since $(0,\ldots, 0)$ is a solution, there must be another one. That solution must correspond to a subset of size $p$ and hence satisfies the required conditions.

Next suppose that the theorem holds for $m,n$ relatively prime; we show it holds for $mn$. Given $r>2m-1$ elements, by assumption there will be a subset $T$ of $m$ elements whose sum is divisible by $m$. We start with a set $S$ of $2mn-1$ integers; continue to pick subsets of size $m$ as described. After $k$ steps we will have $m(2n-k)-1$ elements, so we will be able to carry out $2n-1$ steps and get
\[
T_1,\ldots, T_{2n-1}.
\]
Let the sums of elements of these sets be $t_1,\ldots, t_{2n-1}$. By the hypothesis for $n$, we can find a subset of $n$ elements, say $t_{j_1},\ldots, t_{j_n}$ with sum divisible by $n$. Then
\[
T_{j_1}\cup\cdots \cup T_{j_n}
\]
has $mn$ elements and sum divisible both by $m$ and $n$, hence by $mn$.
\end{proof}
\section*{Problems}
\begin{enumerate}
\item
If $k$ is infinite and $P$ is a nonzero polynomial in $k[x_1,\ldots,x_{n}]$, then there exist $t_1,\ldots, t_n$ such that $P(t_1,\ldots, t_n)\neq 0$.\\

\textbf{Solution: }
Induct on $n$. For $n=1$, the polynomial can have at most $n$ roots so the assertion holds. Suppose it's proved for $n-1$ and $P\in k[t_1,\ldots,t_{n}]$. Since $k[t_1,\ldots,t_{n-1}]$ has infinitely many elements, thinking of $P$ as a polynomial of $t_n$ with coefficients in $k[t_1,\ldots,t_{n-1}]$, some element in $k[t_1,\ldots,t_{n-1}]$ is not a zero of $P$. Set $t_n$ to be this element to get a nonzero element of $k[t_1,\ldots,t_{n-1}]$. By the induction hypothesis we can find values for $t_1,\ldots, t_{n-1}$ so that the polynomial does not evaluate to 0; substitute these values into the polynomial for $t_n$ to get $t_n$.
\end{enumerate}

\section{Problems}%\label{chevalley-warning}
\begin{enumerate}
\item
(Harvard Quals, 2013/2.4) 
\begin{enumerate}
\item
Let $K/F$ be a field extension of degree $2n+1$ generated by $t$. Prove that for every $c\in K$ there exists a unique rational function $f\in F[T]$ such that $\deg(f)\le n$ and $c=f(t)$. (The degree of a rational function $f$ is the smallest $d$ such that $f=\fc PQ$ for polynomials $P$, $Q$ each of degree at most $d$.)
\item
Deduce that if $[K:F]=3$ then $\text{PGL}_2(F)$ acts simply transitively by fractional linear transformations on $K\bs F$ (the complement of $F$ in $K$). If $|F|=q<\iy$, compute $\text{PGL}_2(F)$ directly, and verify that it equals $|K|-|F|$.
\end{enumerate}•
\end{enumerate}•